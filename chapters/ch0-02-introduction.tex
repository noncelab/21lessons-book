\chapter*{들어가며}
\label{ch:introduction}

\begin{chapquote}{루이스 캐롤, \textit{이상한 나라의 앨리스}}
	\enquote{나는 이상한 세상에 들어가고 싶지 않아.}  앨리스가 말했다. 
	\enquote{너는 여기서 빠져나올 수 없어.}  고양이가 말했다. 
	\enquote{우리는 모두 미쳤어. 나도 미쳤어. 너도 미쳤어.} 
	\enquote{내가 미친것을 네가 어찌 알아?}  앨리스가 물었다. 
	\enquote{그래야만해.} 고양이가 대답했다, 
	\enquote{아니면 여기 오지 않았을 거야.}
\end{chapquote}

%In October 2018, Arjun Balaji asked the innocuous question,
%\textit{What have you learned from Bitcoin?} After trying to answer this
%question in a short tweet, and failing miserably, I realized that the things
%I've learned are far too numerous to answer quickly, if at all.
2018년 10월에 아준 발라지(Ajun Balaji)는 나에게 물었다.
\enquote{비트코인으로부터 배운 것이 뭐야?} 나는 트윗을 통해 이 질문에 대답하기 위해 노력했지만 할 수 없었다.
짧은 트윗으로 대답하기에는 너무 많은 것을 배웠다.

%The things I've learned are, obviously, about Bitcoin - or at least related to
%it. However, while some of the inner workings of Bitcoin are explained, the
%%following lessons are not an explanation of how Bitcoin works or what it is,
%they might, however, help to explore some of the things Bitcoin touches:
%philosophical questions, economic realities, and technological innovations.
내가 배운 것은 명백하게 비트코인에 관한 것이다. 하지만 비트코인의 작동 원리는 비트코인을 설명하기엔 부족하다.
이 책에서 다루는 내용은 비트코인의 작동 원리가 아니다. 우리는 비트코인의 작동 원리로부터 철학적
질문, 경제적 현실, 기술 혁신을 배울 수 있다.


\begin{center}
	\includegraphics[width=7cm]{assets/images/the-tweet.png}
\end{center}

%The \textit{21 Lessons} are structured in bundles of seven, resulting in three
%chapters. Each chapter looks at Bitcoin through a different lens, extracting
%what lessons can be learned by inspecting this strange network from a different
%angle.
스물 한가지 교훈은 일곱 개씩 묶어서 세 개의 챕터로 나누어진다. 각각 챕터에서는 비트코인을 다른 관점에서
바라본다. 이 이상한 네트워크를 다른 관점에서 바라보며 어떤 교훈이 있는지 살펴보자.

%\paragraph{\hyperref[ch:philosophy]{Chapter 1}}{explores the philosophical
	%teachings of Bitcoin. The interplay of immutability and change, the concept of
	%true scarcity, Bitcoin's immaculate conception, the problem of identity, the
	%contradiction of replication and locality, the power of free speech, and the
	%limits of knowledge.
	%}

\paragraph{\hyperref[ch:philosophy]{Chapter 1}}{에서는 비트코인의 철학적 가르침을 알아본다.
	\begin{itemize}
		\item 불변성과 변화
		\item 희소성의 개념
		\item 비트코인의 무결한 개념
		\item 비트코인의 정체성 문제
		\item 복제와 지역성의 모순
		\item 언론 자유의 힘
		\item 지식의 한계
\end{itemize}}

%explores the philosophical
%teachings of Bitcoin. The interplay of immutability and change, the concept of
%true scarcity, Bitcoin's immaculate conception, the problem of identity, the
%contradiction of replication and locality, the power of free speech, and the
%limits of knowledge.
%}

%\paragraph{\hyperref[ch:economics]{Chapter 2}}{explores the economic teachings
%of Bitcoin. Lessons about financial ignorance, inflation, value, money and the
%history of money, fractional reserve banking, and how Bitcoin is re-introducing
%sound money in a sly, roundabout way.}
\paragraph{\hyperref[ch:economics]{Chapter 2}}{에서는 비트코인의 경제적 가르침을 알아본다.
\begin{itemize}
	\item 금융 무지
	\item 인플레이션
	\item 가치
	\item 돈과 돈의 역사
	\item 지급 준비 제도
	\item 비트코인이 우회적인 방법으로 건전화폐 사회를 만드는 방법
	\end{itemize}}
	
	%\paragraph{\hyperref[ch:technology]{Chapter 3}}{explores some of the lessons
%learned by examining the technology of Bitcoin.  Why there is strength in
%numbers, reflections on trust, why telling time takes work, how moving slowly
%and not breaking things is a feature and not a bug, what Bitcoin's creation can
%tell us about privacy, why cypherpunks write code (and not laws), and what
%metaphors might be useful to explore Bitcoin's future.}
\paragraph{\hyperref[ch:technology]{Chapter 3}}{에서는 비트코인의 기술을 통한 가르침을 알아본다.
\begin{itemize}
	\item 왜 숫자들이 중요한지
	\item 신뢰에 대한 고찰
	\item 작업시간이 오래 걸리는 이유
	\item 느린 움직임과 깨지지 않는 것이 버그가 아닌 이유
	\item 비트코인이 프라이버시에 대해 말해주는 것
	\item 싸이퍼 펑크가 (법률이 아닌)코드를 작성하는 이유
	\item 미래를 탐험하는데에 유용할지 모르는 비트코인의 비유
	\end{itemize}}
	
	~
	
	%Each lesson contains several quotes and links throughout the text. If an idea is
	%worth exploring in more detail, you can follow the links to related works in the
	%footnotes or in the bibliography.
	각 교훈에는 인용문과 링크가 포함되어 있다. 만약 더 자세한 설명이 필요하다면 링크를 통해 자료를 확인해 보기
	바란다.
	
	%Even though some prior knowledge about Bitcoin is beneficial, I hope that these
	%lessons can be digested by any curious reader. While some relate to each other,
	%each lesson should be able to stand on its own and can be read independently. I
	%did my best to shy away from technical jargon, even though some domain-specific
	%vocabulary is unavoidable.
	비트코인에 대해 궁금한 독자가 이 책의 내용을 소화할 수 있기를 바란다. 각 교훈은 독립적으로 설명되기 때문에
	발췌독해도 괜찮다. 쉬운 이해를 위해 일부 피할 수 없는 경우를 제외하고는 최대한 기술적인 용어를 사용하지 않으려고 노력하였다.
	
	%I hope that my writing serves as inspiration for others to dig beneath the
	%surface and examine some of the deeper questions Bitcoin raises. My own
	%inspiration came from a multitude of authors and content creators to all of whom
	%I am eternally grateful.
	이 글을 통해 비트코인에 대해 더 탐구할 수 있는 계기가 되기를 바란다. 이 책의 영감은 다수 저자와
	콘텐츠 제작자들에게서 나왔다. 도움을 준 모든 이들에게 감사의 말을 전한다.
	
	%Last but not least: my goal in writing this is not to convince you of anything.
	%My goal is to make you think, and show you that there is way more to Bitcoin
	%than meets the eye. I can’t even tell you what Bitcoin is or what Bitcoin will
	%teach you. You will have to find that out for yourself.
	마지막으로, 이 글은 당신을 설득하기 위해 쓴 글이 아니다. 이 글의 목표는 비트코인이 눈에 보이는 것이
	전부가 아니라는 점을 알려주는 것이다. 나는 비트코인이 무엇이고 무엇을 가르치는지 말할 수 없다. 당신이 스스로
	찾아야 한다.
	
	\begin{quotation}\begin{samepage}
	\enquote{이 이후는 되돌릴 수 없다네. 당신이 파란 약을 먹는 순간 이야기는 끝나고 당신이 침대에서 일어나는 순간
		당신이 믿는 것을 볼 수 있네. 만약 빨간 약을 먹는다면\footnote{the \textit{orange} pill} 당신은 이상한 나라에 가게 될걸세, 
		그리고 이 토끼 굴이 얼마나 깊은지를 보여주지.}
	\begin{flushright} -- 모피어스
	\end{flushright}\end{samepage}\end{quotation}
	
	\begin{figure}
\includegraphics{assets/images/bitcoin-orange-pill.jpg}
\caption*{기억하라. 너에게 알려주고자 하는 것은 과장이 없는 진실이다.}
\label{fig:bitcoin-orange-pill}
\end{figure}

%
% [Morpheus]: https://en.wikipedia.org/wiki/Red_pill_and_blue_pill#The_Matrix_(1999)
% [this question]: https://twitter.com/arjunblj/status/1050073234719293440
%
% <!-- Internal -->
% [chapter1]: {{ 'bitcoin/lessons/ch1-00-philosophy' | absolute_url }}
% [chapter2]: {{ 'bitcoin/lessons/ch2-00-economics' | absolute_url }}
% [chapter3]: {{ 'bitcoin/lessons/ch3-00-technology' | absolute_url }}
%
% <!-- Wikipedia -->
% [alice]: https://en.wikipedia.org/wiki/Alice%27s_Adventures_in_Wonderland
% [carroll]: https://en.wikipedia.org/wiki/Lewis_Carroll
