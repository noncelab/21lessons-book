\chapter*{들어가며}
\label{ch:introduction}

\begin{chapquote}{루이스 캐롤, \textit{이상한 나라의 앨리스}}
	\enquote{전 미친 사람들이랑 어울리고 싶지 않아요.} 앨리스가 말했다. 
	\enquote{오, 그건 어쩔 수 없는데.}  고양이가 말했다. 
	\enquote{우리 모두 미쳤어. 나도 미쳤고, 너도 미쳤어.} 
	\enquote{제가 미쳤는지 어떻게 알아요?} 앨리스가 물었다. 
	\enquote{넌 미친게 틀림없어.} 고양이가 대답했다, 
	\enquote{아니라면 여기 오지도 않았을테니까.}
\end{chapquote}


%In October 2018, Arjun Balaji asked the innocuous question,
%\textit{What have you learned from Bitcoin?} After trying to answer this
%question in a short tweet, and failing miserably, I realized that the things
%I've learned are far too numerous to answer quickly, if at all.
\paragraph{}
2018년 10월, 아준 발라지(Ajun Balaji)는 나에게 물었다.
\enquote{당신은 비트코인으로부터 무엇을 배웠나요?} 이 질문에 짧은 트윗으로 대답하려다 처참하게 실패하고야 깨달았다.
배운 것이 너무 많아 단번에 대답하기 어렵다는 것을 말이다.

%The things I've learned are, obviously, about Bitcoin - or at least related to
%it. However, while some of the inner workings of Bitcoin are explained, the
%%following lessons are not an explanation of how Bitcoin works or what it is,
%they might, however, help to explore some of the things Bitcoin touches:
%philosophical questions, economic realities, and technological innovations.
\paragraph{}
내가 배운 것들은 분명히 비트코인에 대한 것이거나 적어도 비트코인과 관련된 것이다. 
몇몇 교훈은 비트코인의 내부 동작을 설명하지만 나머지 교훈들은 그런 것들이 아니다.
다만 비트코인이 다루는 철학적 질문, 경제적 현실, 그리고 기술 혁신 같은 주제를 탐구하는 데 도움이 될 만한 것들이다. 

\begin{center}
	\includegraphics[width=7cm]{assets/images/the-tweet.png}
\end{center}

%The \textit{21 Lessons} are structured in bundles of seven, resulting in three
%chapters. Each chapter looks at Bitcoin through a different lens, extracting
%what lessons can be learned by inspecting this strange network from a different
%angle.
\paragraph{}
스물 한가지 교훈은 일곱 개씩 묶인 세 개의 장으로 나누어진다. 
각 장에서는 비트코인을 각기 다른 시각으로 조명하고 이 낯선 네트워크를 다른 각도에서 살펴봄으로써 우리가 배울 수 있는 교훈을 이끌어낸다.

%\paragraph{\hyperref[ch:philosophy]{Chapter 1}}{explores the philosophical
	%teachings of Bitcoin. The interplay of immutability and change, the concept of
	%true scarcity, Bitcoin's immaculate conception, the problem of identity, the
	%contradiction of replication and locality, the power of free speech, and the
	%limits of knowledge.
	%}

\paragraph{\hyperref[ch:philosophy]{제 1 장}}{에서는 비트코인의 철학적 가르침을 알아본다.
	\begin{itemize}
		\item 불변성과 변화
		\item 희소성의 개념
		\item 비트코인의 무결성
		\item 비트코인의 정체성 문제
		\item 복제와 국소성의 모순
		\item 언론 자유의 힘
		\item 지식의 한계
\end{itemize}}

%explores the philosophical
%teachings of Bitcoin. The interplay of immutability and change, the concept of
%true scarcity, Bitcoin's immaculate conception, the problem of identity, the
%contradiction of replication and locality, the power of free speech, and the
%limits of knowledge.
%}

%\paragraph{\hyperref[ch:economics]{Chapter 2}}{explores the economic teachings
%of Bitcoin. Lessons about financial ignorance, inflation, value, money and the
%history of money, fractional reserve banking, and how Bitcoin is re-introducing
%sound money in a sly, roundabout way.}
\paragraph{\hyperref[ch:economics]{제 2 장}}{에서는 비트코인의 경제적 가르침을 알아본다.
\begin{itemize}
	\item 금융적 무지
	\item 인플레이션
	\item 가치
	\item 돈과 돈의 역사
	\item 지급 준비 제도
	\item 비트코인이 우회적인 방법으로 건전 화폐 사회를 만드는 방법
	\end{itemize}}
	
%\paragraph{\hyperref[ch:technology]{Chapter 3}}{explores some of the lessons
%learned by examining the technology of Bitcoin.  Why there is strength in
%numbers, reflections on trust, why telling time takes work, how moving slowly
%and not breaking things is a feature and not a bug, what Bitcoin's creation can
%tell us about privacy, why cypherpunks write code (and not laws), and what
%metaphors might be useful to explore Bitcoin's future.}
\paragraph{\hyperref[ch:technology]{제 3 장}}{에서는 기술을 통한 비트코인의 가르침을 알아본다.
\begin{itemize}
	\item 숫자들이 중요한 이유
	\item 신뢰에 대한 고찰
	\item 시간이 오래 걸리는 이유
	\item 천천히 동작하고 깨지지 않는 것이 버그가 아닌 기능이라는 점
	\item 비트코인 탄생이 프라이버시에 대해 말해주는 것
	\item 사이퍼펑크가 법이 아닌 코드를 작성하는 이유
	\item 비트코인의 미래를 탐험하는데 유용할지 모르는 비유들
	\end{itemize}}
	

%Each lesson contains several quotes and links throughout the text. If an idea is
%worth exploring in more detail, you can follow the links to related works in the
%footnotes or in the bibliography.
\paragraph{}
각 교훈에는 인용문과 링크가 포함되어 있다. 더 자세한 설명이 필요하다면 링크를 통해 자료를 확인해 보기
바란다.

%Even though some prior knowledge about Bitcoin is beneficial, I hope that these
%lessons can be digested by any curious reader. While some relate to each other,
%each lesson should be able to stand on its own and can be read independently. I
%did my best to shy away from technical jargon, even though some domain-specific
%vocabulary is unavoidable.
\paragraph{}
비트코인에 대한 사전 지식이 있다면 도움이 되겠지만, 
그렇지 않더라도 호기심 많은 독자라면 누구나 이 내용을 소화할 수 있기를 기대한다.
서로 연관된 교훈들도 있지만, 각 교훈은 독립적이기 때문에 발췌하여 읽을 수 있다. 
일부 어쩔 수 없는 전문 용어를 제외하고 기술적인 용어는 최대한 사용하지 않으려고 노력했다. 

%I hope that my writing serves as inspiration for others to dig beneath the
%surface and examine some of the deeper questions Bitcoin raises. My own
%inspiration came from a multitude of authors and content creators to all of whom
%I am eternally grateful.
\paragraph{}
이 글이 다른 이들로 하여금 비트코인을 더 깊이 탐구하고 비트코인이 제시하는 깊이 있는 질문을 들여다 볼 수 있는 영감을 선사하길 바란다. 
나 또한 수많은 저자와 콘텐츠 제작자로부터 영감을 받았다. 이 글을 빌어 도움을 준 모든 분들에게 깊은 감사의 마음을 전한다.

%Last but not least: my goal in writing this is not to convince you of anything.
%My goal is to make you think, and show you that there is way more to Bitcoin
%than meets the eye. I can’t even tell you what Bitcoin is or what Bitcoin will
%teach you. You will have to find that out for yourself.

\paragraph{}
마지막으로, 이 글을 쓰는 나의 목표는 여러분을 설득하는 것이 아니다. 
나의 목표는 여러분을 생각하게 하고 눈에 보이는 것이 비트코인의 전부가 아니라는 점을 알려주려는 것이다.
비트코인이 무엇인지, 비트코인이 여러분에게 무엇을 가르쳐줄지 알려줄 수 없다. 
그것이 무엇인지는 여러분 스스로 알아내야 한다.

\begin{quotation}\begin{samepage}
\enquote{이 이후엔 되돌릴 수 없다네. 파란 약을 먹는 순간 이야기는 끝나고 침대에서 일어나 자네가 믿고 싶은 것을 믿게 되지.
	만약 빨간 약을 먹는다면\footnote{the \textit{orange} pill} 자네는 이상한 나라에 남게 될걸세. 
	그땐 이 토끼굴이 얼마나 깊은지 보여주지.}
\begin{flushright} -- 모피어스
\end{flushright}\end{samepage}\end{quotation}
	
\begin{figure}
\includegraphics{assets/images/bitcoin-orange-pill.jpg}
\caption*{기억하세요. 제가 보여드리는 건 진실 뿐입니다. 그 이상은 없습니다.}
\label{fig:bitcoin-orange-pill}
\end{figure}

%
% [Morpheus]: https://en.wikipedia.org/wiki/Red_pill_and_blue_pill#The_Matrix_(1999)
% [this question]: https://twitter.com/arjunblj/status/1050073234719293440
%
% <!-- Internal -->
% [chapter1]: {{ 'bitcoin/lessons/ch1-00-philosophy' | absolute_url }}
% [chapter2]: {{ 'bitcoin/lessons/ch2-00-economics' | absolute_url }}
% [chapter3]: {{ 'bitcoin/lessons/ch3-00-technology' | absolute_url }}
%
% <!-- Wikipedia -->
% [alice]: https://en.wikipedia.org/wiki/Alice%27s_Adventures_in_Wonderland
% [carroll]: https://en.wikipedia.org/wiki/Lewis_Carroll
