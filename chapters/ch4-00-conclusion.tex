\addpart{마지막 생각}
\pdfbookmark{Conclusion}{결론}
\label{ch:conclusion}

\chapter*{결론}

\begin{chapquote}{루이스 캐롤, \textit{이상한 나라의 앨리스}}
	\enquote{처음부터 읽어라.} 왕은 엄중하게 말했다. \enquote{그리고 끝까지 읽고, 끝내라.}
\end{chapquote}

\begin{comment}
	As mentioned in the beginning, I think that any answer to the
	question \textit{“What have you learned from Bitcoin?”} will always be incomplete. The
	symbiosis of what can be seen as multiple living systems -- Bitcoin, the
	technosphere, and economics -- is too intertwined, the topics too numerous, and
	things are moving too fast to ever be fully understood by a single person.
\end{comment}
서두에서도 언급했지만 \enquote{비트코인으로부터 무엇을 배웠는가?}에 대한 대답은 늘 불완전하다.
비트코인은 기술과 경제학같은 여러 살아있는 시스템들의 공생 관계가 얽혀있어서 다루어야 할 주제가 방대하고, 
비트코인을 완전히 이해하기엔 모든 것이 너무 빠르게 변화하고 있다. 

\begin{comment}
	Even without understanding it fully, and even with all its quirks and seeming
	shortcomings, Bitcoin undoubtedly works. It keeps producing blocks roughly every
	ten minutes and does so beautifully. The longer Bitcoin continues to work, the
	more people will opt-in to use it.
\end{comment}
비트코인을 완전히 이해하지 못하고 단점과 결점이 있다 하더라도, 비트코인은 의심할 여지 없이 작동하고 있다.
그것도 약 10분마다 계속 블록을 생성하며 매우 아름답게 작동하고 있다.
비트코인이 더 오래 작동할 수록 더 많은 사람이 비트코인을 선택하게 될 것이다.

\begin{quotation}\begin{samepage}
		%\enquote{It's true that things are beautiful when they work. Art is function.}
		\enquote{어떤 것이 작동할 때 아름답다는 것은 사실이다. 예술은 기능이다.}
		\begin{flushright} -- 지안니나 브라스키\footnote{Giannina Braschi, \textit{Empire of Dreams} \cite{braschi2011empire}}
\end{flushright}\end{samepage}\end{quotation}

\paragraph{} 
\begin{comment}Bitcoin is a child of the internet. It is growing exponentially,
	blurring the lines between disciplines. It isn’t clear, for example, where the
	realm of pure technology ends and where another realm begins. Even though
	Bitcoin requires computers to function efficiently, computer science is not
	sufficient to understand it. Bitcoin is not only borderless in regards to its
	inner workings but also boundaryless in respect to academic disciplines.
\end{comment}
비트코인은 인터넷의 산물이다. 
비트코인은 기하급수적으로 성장하고 있으며 분야의 경계를 무너뜨리고 있다.
예를 들어 순수 기술의 영역이 어디에서 끝나고 다른 영역이 어디서부터 시작되는지 명확하지 않다.
비트코인이 효율적으로 작동하려면 컴퓨터가 필요하지만, 컴퓨터 과학만으로는 비트코인을 이해하기에 충분하지 않다.
비트코인의 내부 작동 측면에서도 경계가 없을 뿐 아니라 학문 분야에서도 경계가 없다.

\begin{comment}
	Economics, politics, game theory, monetary history, network theory, finance,
	cryptography, information theory, censorship, law and regulation, human
	organization, psychology -- all these and more are areas of expertise which might
	help in the quest of understanding how Bitcoin works and what Bitcoin is.
\end{comment}
비트코인이 무엇이고 비트코인이 어떻게 작동하는지 이해하기 위해서는 
경제학, 정치학, 게임 이론, 화폐의 역사, 네트워크 이론, 금융, 암호학, 정보 이론, 검열, 법률과 규제,
인간 조직, 심리학 등 이 모든 전문 분야의 도움이 필요하다.


\begin{comment}
	No single invention is responsible for its success. It is the combination of
	multiple, previously unrelated pieces, glued together by game theoretical
	incentives, which make up the revolution that is Bitcoin. The beautiful blend of
	many disciplines is what makes Satoshi a genius.
\end{comment}
비트코인의 성공은 하나의 발명품 때문이 아니다.
이전에는 서로 관련 없던 여러 가지가 게임 이론적 인센티브에 의해 결합된 것이 바로 비트코인이라는 혁명을 만들어낸 것이다. 
여러 분야를 망라하는 아름다운 조화가 사토시를 천재로 만들었다. 

\paragraph{} 
\begin{comment}Like every complex system, Bitcoin has to make tradeoffs in terms
	of efficiency, cost, security, and many other properties. Just like there is no
	perfect solution to deriving a square from a circle, any solution to the
	problems which Bitcoin tries to solve will always be imperfect as well.
\end{comment}
모든 복잡한 시스템이 그러하듯 비트코인도 효율성, 비용, 보안 및 여러 측면에서 절충이 필요하다.
원에서 사각형을 도출하는 데 완벽한 방법이 없는 것처럼, 비트코인이 해결하려는 해결책도 항상 불완전할 수밖에 없다.

\begin{quotation}\begin{samepage}
		%\enquote{I don’t believe we shall ever have a good money again before we take the
			%thing out of the hands of government, that is, we can’t take it violently
			%out of the hands of government, all we can do is by some sly roundabout way
			%introduce something that they can’t stop.}
		\enquote{정부의 손에서 빼앗기 전에는 다시는 좋은 돈을 가질 수 없다고 생각합니다.
		즉, 정부의 손에서 폭력적으로 빼앗을 수는 없으며, 우리가 할 수있는 일은 교활한 우회로를 통해
		그들이 막을 수없는 무언가를 도입하는 것 뿐입니다.}
		\begin{flushright} -- 프리드리히 하이에크\footnote{Friedrich Hayek on Monetary Policy, the Gold Standard, Deficits, Inflation, and John Maynard Keynes \url{https://youtu.be/EYhEDxFwFRU}}
\end{flushright}\end{samepage}\end{quotation}

\begin{comment}
	Bitcoin is the sly, roundabout way to re-introduce good money to the world. It
	does so by placing a sovereign individual behind each node, just like Da Vinci
	tried to solve the intractable problem of squaring a circle by placing the
	Vitruvian Man in its center. Nodes effectively remove any concept of a center,
	creating a system which is astonishingly antifragile and extremely hard to shut
	down. Bitcoin lives, and its heartbeat will probably outlast all of ours.
\end{comment}
비트코인은 건전 화폐를 세상에 다시 도입하는 교묘하고 우회적인 방법이다.
다빈치가 비트루비안 맨(Vitruvian Man)을 중앙에 배치하여 원을 제곱하는 까다로운 문제를 풀려고했던 것처럼,
비트코인은 각 노드 뒤에 자주적 개인을 배치하여 문제를 해결하려고 한다.
노드는 효과적으로 중앙의 개념을 제거하여 놀라울 정도로 공격에 취약하지 않고 중단시키기 어려운 시스템을 만든다.
비트코인은 살아있고, 그 심장 박동은 아마도 우리의 심장박동보다 오랫동안 뛸 것이다.

\begin{comment}
	I hope you have enjoyed these twenty-one lessons. Maybe the most important
	lesson is that Bitcoin should be examined holistically, from multiple angles, if
	one would like to have something approximating a complete picture. Just like
	removing one part from a complex system destroys the whole, examining parts of
	Bitcoin in isolation seems to taint the understanding of it. If only one person
	strikes \enquote{blockchain} from her vocabulary and replaces it with \enquote{a
		chain of blocks} I will die a happy man.
\end{comment}
스물한 가지 교훈이 도움이 되었길 바란다. 
아마도 가장 중요한 교훈은 비트코인의 완전한 그림을 그리려면 전체적으로 여러 각도에서 살펴봐야 한다는 점일 것이다. 
복잡한 시스템의 한 부분을 제거하면 전체가 파괴되는 것처럼, 
비트코인의 일부를 따로따로 공부하면 비트코인에 대한 이해가 흐려진다.
누군가 단어장에서 \enquote{blockchain}이란 단어를 \enquote{a chain of blocks}로 바꾼다면, 난 정말 행복하게 죽을 수 있을 것 같다.

\begin{comment}
	In any case, my journey continues. I plan to venture further down into the
	depths of this rabbit hole, and I invite you to tag
	along for the ride.\footnote{\url{https://twitter.com/dergigi}}
\end{comment}
어쨌든 내 여정은 계속된다. 
나는 이 토끼굴에 더 깊이 들어갈 계획이며, 여러분도 함께 동행해 주길 바란다.\footnote{\url{https://twitter.com/dergigi}}

% <!-- Twitter -->
% [dergigi]: https://twitter.com/dergigi
%
% <!-- Internal -->
% [sly roundabout way]: https://youtu.be/EYhEDxFwFRU?t=1124
% [Giannina Braschi]: https://en.wikipedia.org/wiki/Braschi%27s_Empire_of_Dreams
