\part{경제학}
\label{ch:economics}
\chapter*{경제학}

\begin{chapquote}{루이스 캐롤, \textit{이상한 나라의 앨리스}}
	\enquote{정원 입구 근처에 커다란 장미나무가 서 있었는데, 나무에 달린 흰색 장미를 정원사 세 명이 바쁘게 빨간색으로 칠하고 있었다.
		앨리스는 참 신기한 일이라고 생각했다\ldots}
\end{chapquote}

%Money doesn’t grow on trees. To believe that it does is foolish, and our
%parents make sure that we know about that by repeating this saying like a
%mantra. We are encouraged to use money wisely, to not spend it frivolously,
%and to save it in good times to help us through the bad. Money, after all,
%does not grow on trees.
땅을 판다고 돈이 생기지 않는다. 그렇다고 믿는 것은 어리석은 일이며, 부모님은 이 진리를 되풀이하여 가르친다.
우리는 돈을 현명하게 사용하고, 경솔하게 써버려선 안되며, 좋은 시절에 저축하여 위기를 극복하라고 배운다. 
돈은 결코 땅을 판다고 저절로 생기지 않는다.

%Bitcoin taught me more about money than I ever thought I would need to know.
%Through it, I was forced to explore the history of money, banking, various
%schools of economic thought, and many other things. The quest to understand
%Bitcoin lead me down a plethora of paths, some of which I try to explore in
%this chapter.
비트코인은 내가 돈에 대해 알아야 한다고 생각했던 것보다 더 많은 것을 가르쳐주었다. 
나는 비트코인을 통해 돈의 역사, 은행, 다양한 경제학파 등 많은 것을 탐구할 수 밖에 없었다.
비트코인을 이해하기 위한 공부는 나를 수많은 길로 이끌었고, 그 중 일부를 이 장에서 살펴보고자 한다. 

%In the first seven lessons some of the philosophical questions Bitcoin touches
%on were discussed. The next seven lessons will take a closer look at money and
%economics.
앞서 살펴본 첫 일곱개의 교훈에서는 비트코인이 다루는 몇 가지 철학적 질문에 대해 논하였다.
이제 다음 일곱개의 교훈에서는 돈과 경제에 대해 자세히 살펴볼 것이다. 

~

\begin{samepage}
	Part~\ref{ch:economics} -- 경제학
	
	\begin{enumerate}
		\setcounter{enumi}{7}
		\item 금융적 무지
		\item 인플레이션
		\item 가치
		\item 돈
		\item 역사와 돈의 몰락
		\item 부분 지급준비금의 광기
		\item 건전 화폐
\end{enumerate}
\end{samepage}

%Again, I will only be able to scratch the surface. Bitcoin is not only
%ambitious, but also broad and deep in scope, making it impossible to cover all
%relevant topics in a single lesson, essay, article, or book. I doubt if it is
%even possible at all.
다시 강조하지만 나는 표면적인 내용만 다룰 수 있을 것이다.  
비트코인은 범위가 넓고 깊어서 하나의 강의, 에세이, 기사 또는 책에서 모든 주제를 다루는 것이 불가능하다.
과연 그렇게 하는게 가능할지 의문이다.  


%Bitcoin is a new form of money, which makes learning about
%economics paramount to understanding it. Dealing with the nature of human action
%and the interactions of economic agents, economics is probably one of the
%largest and fuzziest pieces of the Bitcoin puzzle.
비트코인은 새로운 형태의 돈으로 비트코인을 이해하기 위해서는 무엇보다 경제에 대해 배우는 것이 중요하다.
인간 행동의 본질과 경제 주체들의 상호 작용을 다루는 경제학은 아마도 비트코인 퍼즐에서 가장 크고 모호한 조각 중 하나일 것이다.

%Again, these lessons are an exploration of the various things I have learned
%from Bitcoin. They are a personal reflection of my journey down the rabbit hole.
%Having no background in economics, I am definitely out of my comfort zone and
%especially aware that any understanding I might have is incomplete. I will do my
%best to outline what I have learned, even at the risk of making a fool out of
%myself. After all, I am still trying to answer the question:
다시 한번 말하지만, 이 글은 내가 비트코인에서 배운 다양한 것들을 탐구한 것이다. 
비트코인 토끼굴로 내려가는 나의 여정을 개인적으로 회고한 것이기도 하다.  
경제학이 나의 전문 영역이 아니고, 나에게는 배경 지식도 없기 때문에 나의 이해가 불완전하다는 것을 알고 있다. 
하지만, 바보로 보일 부담을 감수하고서라도 내가 배운 것을 설명하기 위해 최선을 다할 것이다.
여전히 나는 같은 질문에 답하기 위해 노력 중이다.\enquote{비트코인으로부터 무엇을 배웠는가?}

%After seven lessons examined through the lens of philosophy, let’s use the lens
%of economics to look at seven more. Economy class is all I can offer this time.
%Final destination: \textit{sound money}.
철학적 관점에서 일곱 가지 교훈을 살펴보았으니, 이번에는 경제학적 관점에서 일곱 가지 교훈을 더 살펴보자.
이번에는 경제 수업이고 종착지는 건전화폐(sound money)이다.

% [the question]: https://twitter.com/arjunblj/status/1050073234719293440
