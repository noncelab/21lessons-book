\part{철학}
\label{ch:philosophy}
\chapter*{철학}

\begin{chapquote}{루이스 캐롤, \textit{이상한 나라의 앨리스}}
	생쥐는 호기심 어린 눈으로 그녀를 바라보았고, 작은 눈으로 그녀에게 윙크하는 것 같았지만 아무 말도 하지 않았다.
\end{chapquote}

%Looking at Bitcoin superficially, one might conclude that it is slow, wasteful,
%unnecessarily redundant, and overly paranoid. Looking at Bitcoin inquisitively,
%one might find out that things are not as they seem at first glance.
비트코인을 겉핥기로 바라보는 것은 비트코인을 이해하는 데 도움이 되지 않는다. 조금만 더 호기심을 가지고 비트코인을 살펴보면, 보이는 것이 
전부가 아니라는 것을 알 수 있을 것이다.

%Bitcoin has a way of taking your assumptions and turning them on their heads.
%After a while, just when you were about to get comfortable again, Bitcoin will
%smash through the wall like a bull in a china shop and shatter your assumptions
%once more.
당신이 나름의 비트코인의 가정을 받아들이고 머릿속에서 오랫동안 고민하여 결론을 낼 수도 있다. 하지만,
당신이 이해했다고 생각할 때쯤에 비트코인은 당신의 가정을 깨트리고 다시 한번 머리를 복잡하게 만든다.

\begin{figure}
	\includegraphics{assets/images/blind-monks.jpg}
	\caption{비트코인 황소를 조사하는 맹인 승려}
	\label{fig:blind-monks}
\end{figure}

%Bitcoin is a child of many disciplines. Like blind monks examining an elephant,
%everyone who approaches this novel technology does so from a different angle.
%And everyone will come to different conclusions about the nature of the beast.
비트코인은 다양한 분야로부터 영향을 받아 탄생하였다. 
이 때문에 맹인 승려가 코끼리를 조사하는 것처럼 이 기술에 접근하는 모든 사람이 다른 각도에서 바라보게 된다. 
그리고 모두가 이 짐승의 본성에 대해 각각의 다른 결론을 낸다.

%The following lessons are about some of my assumptions which Bitcoin shattered,
%and the conclusions I arrived at. Philosophical questions of immutability,
%scarcity, locality, and identity are explored in the first four lessons.  Every
%part consists of seven lessons.
아래의 교훈들은 비트코인의 여러 가정과 내가 도달한 결론에 대한 내용이다. 첫 네 개의 교훈은
불변성, 희소성, 지역성, 정체성에 대한 철학적 질문을 다룰 예정이다.

~

\begin{samepage}
	Part~\ref{ch:philosophy} -- 철학:
	
	\begin{enumerate}
		\item 불변성과 변화
		\item 진정한 희소성
		\item 복제와 지역성
		\item 정체성의 문제
		\item 무결점의 개념
		\item 언론 자유의 힘
		\item 지식의 한계
	\end{enumerate}
\end{samepage}

%Lesson \ref{les:5} explores how Bitcoin's origin story is not only fascinating but
%absolutely essential for a leaderless system. The last two lessons of this
%chapter explore the power of free speech and the limits of our individual
%knowledge, reflected by the surprising depth of the Bitcoin rabbit hole.
다섯번째 교훈은 비트코인의 기원에 대한 이야기가 매력적인지, 그리고 리더의 부재가 왜 절대적으로 필요한지
알아본다. 마지막 두 개의 교훈은 깊은 토끼굴로 들어가 언론 자유에 대해 탐구하고 개인 지식의 한계에 관해 이야기하고자 한다. 

%I hope that you will find the world of Bitcoin as educational, fascinating and
%entertaining as I did and still do. I invite you to follow the white rabbit and
%explore the depths of this rabbit hole. Now hold on to your pocket watch, pop
%down, and enjoy the fall.
당신도 비트코인의 세계가 교육적이고 매력적이라는 사실을 깨닫길 바란다. 흰토끼를 따라 토끼 굴의 깊이를 탐험해보라.
지금부터 회중시계를 들고 토끼 굴로 떨어져보자.

