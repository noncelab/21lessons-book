\part{철학}
\label{ch:philosophy}
\chapter*{철학}

\begin{chapquote}{루이스 캐롤, \textit{이상한 나라의 앨리스}}
	생쥐는 호기심 어린 눈으로 앨리스를 바라보며 작은 눈으로 윙크하는 것 같았지만, 아무 말도 하지 않았다.
\end{chapquote}

%Looking at Bitcoin superficially, one might conclude that it is slow, wasteful,
%unnecessarily redundant, and overly paranoid. Looking at Bitcoin inquisitively,
%one might find out that things are not as they seem at first glance.
\paragraph{}
비트코인을 표면적으로만 보면 느리고, 낭비스럽고, 불필요하게 중복적이고, 지나치게 편집증적이라고 결론 내릴 수 있다. 
하지만 비트코인을 자세히 들여다보면 언뜻 보이는 것과는 다르다는 것을 알 수 있다.

%Bitcoin has a way of taking your assumptions and turning them on their heads.
%After a while, just when you were about to get comfortable again, Bitcoin will
%smash through the wall like a bull in a china shop and shatter your assumptions
%once more.
\paragraph{}
비트코인은 여러분의 가정(assumptions)을 받아들이고 여러분의 머리를 뒤집어 놓을 것이다.
잠시 후 여러분이 다시 편안해질 때 쯤 다시 한번 고삐 풀린 망아지마냥 벽을 부수고 여러분의 가정을 깨뜨릴 것이다.

\begin{figure}
	\includegraphics{assets/images/blind-monks.jpg}
	\caption{비트코인 황소를 조사하는 눈먼 승려}
	\label{fig:blind-monks}
\end{figure}

%Bitcoin is a child of many disciplines. Like blind monks examining an elephant,
%everyone who approaches this novel technology does so from a different angle.
%And everyone will come to different conclusions about the nature of the beast.
\paragraph{}
비트코인은 다양한 분야로부터 영향을 받은 산물이다.
때문에 코끼리를 관찰하는 눈먼 승려처럼 이 새로운 기술에 접근하는 사람은 모두 다른 각도에서 접근한다.
그리고 모두 이 짐승의 본성에 관해 서로 다른 결론을 내리게 될 것이다.

%The following lessons are about some of my assumptions which Bitcoin shattered,
%and the conclusions I arrived at. Philosophical questions of immutability,
%scarcity, locality, and identity are explored in the first four lessons.  Every
%part consists of seven lessons.
\paragraph{}
다음 교훈들은 비트코인이 깨뜨린 몇 가지 가정과 내가 도달한 결론에 대한 내용이다. 
첫 네 개의 교훈은 불변성, 희소성, 지역성, 정체성에 대한 철학적 질문을 탐구한다. 
모든 장은 총 7개의 교훈으로 구성되어 있다.

~

\begin{samepage}
	Part~\ref{ch:philosophy} -- 철학
	
	\begin{enumerate}
		\item 불변성과 변화
		\item 진정한 희소성
		\item 복제와 지역성
		\item 정체성의 문제
		\item 무결점의 개념
		\item 언론 자유의 힘
		\item 지식의 한계
	\end{enumerate}
\end{samepage}

%Lesson \ref{les:5} explores how Bitcoin's origin story is not only fascinating but
%absolutely essential for a leaderless system. The last two lessons of this
%chapter explore the power of free speech and the limits of our individual
%knowledge, reflected by the surprising depth of the Bitcoin rabbit hole.
\paragraph{}
다섯번째 교훈에서는 흥미로운 비트코인의 기원 이야기 뿐 아니라 리더가 없는 시스템이 절대적으로 필요한 이유에 대해 살펴본다. 
이 장의 마지막 두 교훈에서는 언론의 자유와 비트코인 토끼굴의 놀라운 깊이에 비친 개인 지식의 한계에 관해 이야기하고자 한다. 
 
%I hope that you will find the world of Bitcoin as educational, fascinating and
%entertaining as I did and still do. I invite you to follow the white rabbit and
%explore the depths of this rabbit hole. Now hold on to your pocket watch, pop
%down, and enjoy the fall.
\paragraph{}
과거에 내가 그랬고 지금도 그렇듯이 여러분도 비트코인의 세계가 교육적이고 매력적이며 재미있다는 것을 알게 되기를 바란다. 
흰 토끼를 따라 이 토끼굴 깊숙한 곳을 탐험해보라. 지금부터 회중시계를 손에 쥐고 토끼굴로 빠져보자.
