\part{기술}
\label{ch:technology}
\chapter*{기술}

\begin{comment}
	\begin{chapquote}{Lewis Carroll, \textit{Alice in Wonderland}}
		\enquote{Now, I'll manage better this time} she said to herself, and began by taking
		the little golden key, and unlocking the door that led into the garden
	\end{chapquote}
\end{comment}
\begin{chapquote}{루이스 캐롤, \textit{이상한 나라의 앨리스}}
	\enquote{이번에는 더 잘할 수 있어.} 앨리스는 혼자 중얼거리며 작은 황금 열쇠를 꺼내 정원으로 이어지는 문을 열었다.
\end{chapquote}

\begin{comment}
	Golden keys, clocks which only work by chance, races to solve
	strange riddles, and builders that don't have faces or names. What sounds like
	fairy tales from Wonderland is daily business in the world of Bitcoin.
\end{comment}
황금 열쇠, 어쩌다 우연히 작동되는 시계, 이상한 수수께끼를 풀기 위한 경쟁, 얼굴도 이름도 없는 건축가.
이상한 나라의 동화처럼 들리는 이것들은 비트코인 세계에서 일상적인 것들이다.

\begin{comment}
	As we explored in Chapter~\ref{ch:economics}, large parts of the current financial system are systematically broken. 
	Like Alice, we can only hope to manage better this time. 
	But, thanks to a pseudonymous inventor, we have incredibly sophisticated technology to support us this time around: Bitcoin.
\end{comment}
경제학~\ref{ch:economics}에서 살펴보았듯, 우리는 현재 금융 시스템이 상당히 손상되었다는 것을 안다.
앨리스처럼 우리도 이번에는 더 잘 헤쳐 나갈 수 있길 바랄 뿐이다.
그러나 이번에 다른 점은 익명의 발명가 덕분에 우리를 도와줄 수 있는 놀랍도록 정교한 기술인 비트코인을 갖게 되었다는 것이다. 

\begin{comment}
	Solving problems in a radically decentralized and adversarial environment
	requires unique solutions. What would otherwise be trivial problems to solve
	are everything but in this strange world of nodes. Bitcoin relies on strong
	cryptography for most solutions, at least if looked at through the lens of
	technology. Just how strong this cryptography is will be explored in one of the
	following lessons.
\end{comment}
철저히 탈중앙화되고 적대적인 환경에서 발생하는 문제를 해결하려면 뾰족한 수가 필요하다.
노드들로 이루어진 이상한 나라에서 사소한 문제들을 해결하지 못한다면 모든 것이 문제가 되고 만다.
기술적 관점에서 볼 때 비트코인의 대부분 해결책은 강력한 암호학에 의존하고 있다. 
암호학이 얼마나 강력한지 차차 살펴볼 예정이다.

\begin{comment}
	Cryptography is what Bitcoin uses to remove trust in authorities.
	Instead of relying on centralized institutions, the system relies on the final
	authority of our universe: physics. Some grains of trust still remain, however.
	We will examine these grains in the second lesson of this chapter.
\end{comment}
비트코인은 기관에 대한 신뢰를 제거하기 위해 암호학 기술을 사용한다.
비트코인은 중앙 기관에 의존하는 대신 만물의 법칙인 물리학에 의존한다.
그럼에도 여전히 일부 신뢰의 문제가 남아있다.
이에 대해서는 이 장의 두 번째 교훈에서 살펴볼 것이다.

~

\begin{samepage}
	Part~\ref{ch:technology} -- Technology:
	
	\begin{enumerate}
		\setcounter{enumi}{14}
		\item 숫자의 강력함
		\item \enquote{신뢰하지 말고 검증하라}에 대한 고찰
		\item 시간을 알려주는 데는 노력이 필요하다.
		\item 천천히 움직여라, 아무것도 깨뜨리지 않도록.
		\item 프라이버시는 죽지 않았다.
		\item 사이퍼펑크는 코드를 작성한다.
		\item 비트코인의 미래에 대한 비유
	\end{enumerate}
\end{samepage}

\begin{comment}
	The last couple of lessons explore the ethos of technological development in
	Bitcoin, which is arguably as important as the technology itself. Bitcoin is not
	the next shiny app on your phone. It is the foundation of a new economic
	reality, which is why Bitcoin should be treated as nuclear-grade financial
	software.
\end{comment}
마지막 두 개의 교훈에서는 기술 자체만큼이나 중요한 비트코인의 개발 정신을 살펴본다.
비트코인은 당신의 휴대전화에서 구동될 빛나는 차세대 앱이 아니다.
비트코인은 새로운 경제 현상의 토대이다. 
그렇기 때문에, 비트코인을 핵폭탄급 금융 소프트웨어로 취급해야 한다.

\begin{comment}
	Where are we in this financial, societal, and technological revolution? 
	
	Networks and technologies of the past may serve as metaphors for Bitcoin's future, which
	are explored in the last lesson of this chapter.
\end{comment}
우리는 이 금융 혁명 혹은 사회 혁명이자 기술 혁명의 어디쯤 와 있을까?
과거의 네트워크와 기술들은 비트코인의 미래를 예측할 수 있는 실마리가 될 수 있다. 이에 대해서는 마지막 교훈에서 살펴본다. 

\begin{comment}
	Once more, strap in and enjoy the ride. Like all exponential technologies, we
	are about to go parabolic.
\end{comment}
다시 한번 안전띠를 매고 즐겨보자. 다른 모든 기하급수적으로 성장하는 기술들처럼, 우리도 곧 포물선을 그리게 될 것이다.
