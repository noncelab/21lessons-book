
\def\bitcoinB{\leavevmode
	{\setbox0=\hbox{\textsf{B}}%
		\dimen0\ht0 \advance\dimen0 0.2ex
		\ooalign{\hfil \box0\hfil\cr
			\hfil\vrule height \dimen0 depth.2ex\hfil\cr
		}%
	}%
}

\chapter*{이 책에 대하여 \\ (... 그리고 저자에 대하여)}
\pdfbookmark{About This Book (... and About the Author)}{about}

%This is a bit of an unusual book. But hey, Bitcoin is a bit of an unusual
%technology, so an unusual book about Bitcoin might be fitting. I'm not sure if
%I'm an unusual guy (I like to think of myself as a \textit{regular} guy) but the
%story of how this book came to be, and how I came to be an author, is worth
%telling.

\paragraph{}
이 책은 조금 특이하다. 
비트코인은 약간 특이한 기술이기 때문에 비트코인에 대한 책으론 특이한 것이 걸맞을 수 있다. 
내가 특이한 사람인진 잘 모르겠지만(전 제가 평범한 사람이라고 생각하고 싶어요.) 
이 책이 어떻게 탄생하게 됐는지, 그리고 왜 내가 작가가 되었는지 이야기해볼 필요가 있다고 생각한다.

%First of all, I'm not an author. I'm an engineer. I didn't study writing. I
%studied code and coding. Second of all, I never intended to write a book, let
%alone a book about Bitcoin. Hell, I'm not even a native speaker.\footnote{The
	%reason why I'm writing these words in English is that my brain works in
	%mysterious ways. Whenever something technical comes up, it switches to English
	%mode.} I'm just a guy who caught the Bitcoin bug. Hard.

\paragraph{}
우선, 나는 작가가 아니다. 엔지니어다. 글쓰기 공부를 한 적이 없다. 나는 코드를 배웠고 코딩을 전공했다. 
둘째, 비트코인에 대한 책은 말할 것도 없고 내가 책을 쓸거라 생각해 본 적이 없다. 난 심지어 원어민도 아니다. 
\footnote{제가 영어로 이 글을 쓰는 이유는 제 뇌가 신비한 방식으로 작동하기 때문입니다. 기술적인 내용이 떠오를 때마다 영어 모드로 전환됩니다.}
단지 비트코인 시스템의 버그를 잡던 사람이다. 열심히.


%Who am \textit{I} to write a book about Bitcoin? That's a good question. The
%short answer is easy: I'm Gigi, and I'm a bitcoiner.

\paragraph{}
비트코인에 대한 책을 쓰는 나는 누구인가? 좋은 질문이다. 
짧게 대답하면 쉽다. 내 이름은 지지(Gigi)이고, 비트코이너이다.

%The long answer is a bit more nuanced.

\paragraph{}
길게 대답하자면 약간 미묘해진다.

\paragraph{}
%My background is in computer science and software development. In a
%previous life, I was part of a research group that tried to make computers think
%and reason, among other things. In yet another previous life I wrote software
%for automated passport processing and related stuff which is even scarier. I
%know a thing or two about computers and our networked world, so I guess I have a
%bit of a head-start to understand the technical side of Bitcoin. However, as I
%try to outline in this book, the tech side of things is only a tiny sliver of
%the beast which is Bitcoin. And every single one of these slivers is important.
내 전공은 컴퓨터 과학과 소프트웨어 개발이다. 
이전에 나는 컴퓨터가 사고하고 추론할 수 있도록 연구하는 조직에 속해 있었다. 
자동화 여권을 개발하고 관련 작업을 하기도 했는데 그건 훨씬 무시무시한 일이었다. 
나는 컴퓨터와 네트워크 분야의 한두 가지 정도는 알고 있기 때문에 비트코인의 기술적 측면을 이해하는 데 어느 정도는 앞서 있다고 생각한다.
하지만 이 책에서 설명하고자 하는 것처럼 기술적인 측면은 비트코인이라는 거대한 야수를 이루는 단 하나의 작은 조각에 불과하다. 
그리고 이런 작은 조각 하나하나가 모두 중요하다. 

%This book came to be because of one simple question: \textit{\enquote{What have
		%you learned from Bitcoin?}} I tried to answer this question in a single tweet.
%Then the tweet turned into a tweetstorm. The tweetstorm turned into an article.
%The article turned into three articles. Three articles turned into 21 Lessons.
%And 21 Lessons turned into this book. So I guess I'm just really bad at
%condensing my thoughts into a single tweet.런런

\paragraph{}
이 책은 아주 간단한 질문으로부터 탄생했다.  
\enquote{당신은 비트코인에서 무엇을 배웠습니까?}
나는 이 질문에 대한 답을 트윗 한 줄에 담으려고 노력했다.
그러자 그 트윗은 트윗 폭풍이 되어 돌아왔다. 트윗 폭풍은 기사로 바뀌었고, 이내 기사는 세 개가 되었다. 
세 개의 기사는 21개의 교훈으로 바뀌었다. 그리고 21개의 교훈이 이 책이 되었다. 내 생각을 트윗 한 줄로 압축하는 것에 정말 서툴렀던 것 같다.


%you might
%ask. Again, there is a short and a long answer. The short answer is that I
%simply had to. I was (and still am) \textit{possessed} by Bitcoin. I find it to
%be endlessly fascinating. I can't seem to stop thinking about it and the
%implications it will have on our global society. The long answer is that I
%believe that Bitcoin is the single most important invention of our time, and
%more people need to understand the nature of this invention. Bitcoin is
%still one of the most misunderstood phenomena of our modern world, and it took
%me years to fully realize the gravitas of this alien technology. Realizing what
%Bitcoin is and how it will transform our society is a profound experience. I%
%hope to plant the seeds which might lead to this realization in your head.
\paragraph{}
\enquote{왜 이 책을 썼는가?} 라고 물어볼 수 있겠다.
다시 말하지만 이 질문에는 짧은 대답과 긴 대답이 있다.  
짧게 대답하자면, 단순히 그래야만 했기 때문이다. 나는 비트코인에 매료됐고 지금도 그렇다. 비트코인의 매력은 끝이 없다고 생각한다. 
비트코인과 비트코인이 글로벌 사회에 미칠 영향에 대한 생각을 멈출 수가 없다.
길게 대답하자면, 비트코인은 우리 시대의 가장 중요한 발명품이며, 더 많은 사람들이 이 발명품의 본질을 이해해야 한다고 믿기 때문이다. 
여전히 비트코인은 현대 사회에서 가장 오해받는 현상 중 하나이고, 나 역시 이 낯선 기술의 중요함을 깨닫기까지 몇 년이 걸렸다. 
비트코인이 무엇인지, 그리고 비트코인이 우리 사회를 어떻게 변화시킬지 깨닫는 것은 심오한 경험이다. 
나는 여러분의 머릿속에 이러한 깨달음으로 이어질 수 있는 씨앗을 심고 싶다.


%While this section is titled \enquote{\textit{About This Book (... and About the
		%Author)}}, in the grand scheme of things, this book, who I am, and what I did
%doesn't really matter. I am just a node in the network, both literally
%\textit{and} figuratively. Plus, you shouldn't trust what I'm saying anyway. As
%we bitcoiners like to say: do your own research, and most importantly: don't
%trust, verify.
\paragraph{}
이 섹션의 제목이 \enquote{이 책에 대하여 (... 그리고 저자에 대하여)}이지만, 큰 틀에서 보면, 이 책, 내가 누군지, 내가 무엇을 했던 사람인지는 조금도 중요하지 않다. 
말 그대로 나는 네트워크 속 노드 중 하나에 불과하다. 
무엇보다 여러분은 내가 하는 말을 그대로 믿어선 안 된다. 
비트코이너들이 자주하는 말처럼, 스스로 조사하고 공부해야 한다. 가장 중요한 것은 믿지 말고 검증하라(Don't trust, verify.)는 것이다.

%I did my best to do my homework and provide plenty of sources for you, dear
%reader, to dive into. In addition to the footnotes and citations in this book, I
%try to keep an updated list of resources at
%\href{https://21lessons.com/rabbithole}{21lessons.com/rabbithole} and on
%\href{https://bitcoin-resources.com}{bitcoin-resources.com}, which also lists
%plenty of other curated resources, books, and podcasts that will help you to
%understand what Bitcoin is.
\paragraph{}
나는 독자 여러분이 공부하고 더 깊이 있게 살펴볼 수 있도록 다양한 자료를 제공하기 위해 최선을 다했다.
이 책의 각주 및 인용문 외에도 리소스 목록을 최신 상태로 유지하려고 노력하고 있다. 
\href{https://21lessons.com/rabbithole}{21lessons.com/rabbithole}과 \href{https://bitcoin-resources.com}{bitcoin-resources.com}
에는 비트코인을 이해하는 데 도움이 될 만한 엄선된 리소스, 서적, 팟캐스트도 많이 소개되어 있다.

\paragraph{}
%In short, this is simply a book about Bitcoin, written by a bitcoiner.
%Bitcoin doesn't need this book, and you probably don't need this book to
%understand Bitcoin. I believe that Bitcoin will be understood by you as soon as
%\textit{you} are ready, and I also believe that the first fractions of a bitcoin
%will find you as soon as you are ready to receive them. In essence, everyone
%will get \bitcoinB{}itcoin at exactly the right time. In the meanwhile, Bitcoin
%simply is, and that is enough.\footnote{Beautyon, \textit{Bitcoin is. And that
		%is enough.}~\cite{bitcoin-is}}
요약하면, 이 책은 비트코이너가 쓴 비트코인에 관한 책이다. 
비트코인에게 이 책이 필요하지 않으며, 당신도 비트코인을 이해하기 위해 이 책이 필요 없을지도 모른다. 
당신이 준비가 되었다면 즉시 비트코인을 이해할 수 있을 것이고, 
비트코인을 이해했다면 받아들일 수 있을 것이다. 
본질적으로 모든 사람이 아주 적절한 시기에 \bitcoinB{}itcoin을 받아들이게 될 것이다. 
그때까지 비트코인은 그 자체로 충분하다.\footnote{뷰티온(Beautyon, 역자: 필명), \textit{비트코인이 존재한다. 그리고 그 자체로 충분하다. (Bitcoin is. And that is enough.)} ~\cite{bitcoin-is}}.