
\def\bitcoinB{\leavevmode
	{\setbox0=\hbox{\textsf{B}}%
		\dimen0\ht0 \advance\dimen0 0.2ex
		\ooalign{\hfil \box0\hfil\cr
			\hfil\vrule height \dimen0 depth.2ex\hfil\cr
		}%
	}%
}

\chapter*{이 책에 대하여 \\ (... 그리고 저자에 대하여)}
\pdfbookmark{About This Book (... and About the Author)}{about}

%This is a bit of an unusual book. But hey, Bitcoin is a bit of an unusual
%technology, so an unusual book about Bitcoin might be fitting. I'm not sure if
%I'm an unusual guy (I like to think of myself as a \textit{regular} guy) but the
%story of how this book came to be, and how I came to be an author, is worth
%telling.

이 책은 조금 특이하다. 
비트코인은 원래 기술적으로 특이한 것이므로 이 책이 특이한 것은 당연하다.
나 또한 특이한 사람이다(하지만 나는 그렇게 생각하지 않는다). 
그러나 이 책이 어떻게 탄생하였고 내가 왜 작가가 되었는지를 이야기하는 것은 의미가 있다고 생각한다.

%First of all, I'm not an author. I'm an engineer. I didn't study writing. I
%studied code and coding. Second of all, I never intended to write a book, let
%alone a book about Bitcoin. Hell, I'm not even a native speaker.\footnote{The
	%reason why I'm writing these words in English is that my brain works in
	%mysterious ways. Whenever something technical comes up, it switches to English
	%mode.} I'm just a guy who caught the Bitcoin bug. Hard.

첫째, 나는 작가가 아니다. 나는 엔지니어다. 나는 글쓰기를 공부한 적이 없다. 나는 코딩을 전공하였다. 
둘째, 나는 내가 책을 쓰리라고 상상해 본 적이 없다. 나는 심지어 네이티브 스피커도 아니다. 
나는 단지 비트코인 버그를 잡는 사람이다. 열심히.


%Who am \textit{I} to write a book about Bitcoin? That's a good question. The
%short answer is easy: I'm Gigi, and I'm a bitcoiner.

비트코인에 관련된 책을 쓰는 나는 누구인가? 좋은 질문이다. 
짧게 대답하자면, 나는 Gigi이고, 나는 비트코이너이다.

%The long answer is a bit more nuanced.
길게 대답하자면 약간 미묘해진다.

\paragraph{}
%My background is in computer science and software development. In a
%previous life, I was part of a research group that tried to make computers think
%and reason, among other things. In yet another previous life I wrote software
%for automated passport processing and related stuff which is even scarier. I
%know a thing or two about computers and our networked world, so I guess I have a
%bit of a head-start to understand the technical side of Bitcoin. However, as I
%try to outline in this book, the tech side of things is only a tiny sliver of
%the beast which is Bitcoin. And every single one of these slivers is important.
나는 컴퓨터과학을 공부하였고 소프트웨어 개발자이다. 
나는 컴퓨터가 생각하고 추론하게 만드는 연구를 하였다.
나는 자동화된 여권 처리 등 무시무시한 작업을 위한 소프트웨어를 개발한 적도 있다. 
나는 컴퓨터와 네트워크에 대해 알고 있었기 때문에 비트코인의 기술적 측면을 이해하는 데에 도움이 되었다고 생각한다. 
하지만 이 책에서 설명하는 것처럼 비트코인의 기술적인 측면은 아주 작은 부분에 불과하다. 
그리고 이 조각들 하나하나를 모두 이해하는 것이 중요하다.


%This book came to be because of one simple question: \textit{\enquote{What have
		%you learned from Bitcoin?}} I tried to answer this question in a single tweet.
%Then the tweet turned into a tweetstorm. The tweetstorm turned into an article.
%The article turned into three articles. Three articles turned into 21 Lessons.
%And 21 Lessons turned into this book. So I guess I'm just really bad at
%condensing my thoughts into a single tweet.
이 책은 아주 단순한 질문에서 시작되었다. 
\enquote{비트코인으로부터 배운 것이 뭐야?} 라는 질문이다.
나는 정답을 찾기 위해 간단한 트윗을 작성하였다. 
그러자 이 트윗은 트윗 폭풍으로 돌아왔다. 나는 이 트윗 폭풍을 잘 정리하였다.
그리고 트윗에서 언급한 스물 한가지 교훈은 이 책이 되었다. 
내 생각을 하나의 트윗으로 압축하는 것은 쉽지 않았다.

\paragraph{}
\enquote{왜 이 책을 썼는가?},
%you might
%ask. Again, there is a short and a long answer. The short answer is that I
%simply had to. I was (and still am) \textit{possessed} by Bitcoin. I find it to
%be endlessly fascinating. I can't seem to stop thinking about it and the
%implications it will have on our global society. The long answer is that I
%believe that Bitcoin is the single most important invention of our time, and
%more people need to understand the nature of this invention. Bitcoin is
%still one of the most misunderstood phenomena of our modern world, and it took
%me years to fully realize the gravitas of this alien technology. Realizing what
%Bitcoin is and how it will transform our society is a profound experience. I%
%hope to plant the seeds which might lead to this realization in your head.
그 질문에는 짧은 대답과 긴 대답이 있다. 
짧은 대답은 나는 비트코인에 홀렸기 때문이다.
나는 아직도 비트코인의 끝없는 매력을 발견하고 있다. 
나는 비트코인이 글로벌 사회에 미칠 영향에 대해 생각하는 것을 멈출 수 없다.
긴 대답은 비트코인은 사람들이 이해해야 할 가장 중요한 발명품이라 믿기 때문이다. 
세상이 바라보는 비트코인은 오해로 가득 차있다.
나는 이 외계 기술을 완전히 깨닫는 데에 몇 년이 걸렸다. 
비트코인이 무엇이며 비트코인이 우리 사회를 어떻게 변화시킬 것인지를 깨닫는 것은 심오한 경험이다.
이 깨달음의 씨앗을 여러분 머리에도 심기를 바란다.

%While this section is titled \enquote{\textit{About This Book (... and About the
		%Author)}}, in the grand scheme of things, this book, who I am, and what I did
%doesn't really matter. I am just a node in the network, both literally
%\textit{and} figuratively. Plus, you shouldn't trust what I'm saying anyway. As
%we bitcoiners like to say: do your own research, and most importantly: don't
%trust, verify.
이 섹션의 제목이 \enquote{이 책에 대하여 (... 그리고 저자에 대하여)} 인 것은 중요하지 않다. 
비유하자면, 나는 단지 네트워크에서 하나의 노드일 뿐이다. 
당신은 내 말을 그대로 믿어서는 안 된다. 
비트코이너들이 좋아하는 그 말처럼 스스로 조사하고 공부하길 바란다. 
믿지 말고 검증하라.(Don't trust, verify.)



%I did my best to do my homework and provide plenty of sources for you, dear
%reader, to dive into. In addition to the footnotes and citations in this book, I
%try to keep an updated list of resources at
%\href{https://21lessons.com/rabbithole}{21lessons.com/rabbithole} and on
%\href{https://bitcoin-resources.com}{bitcoin-resources.com}, which also lists
%plenty of other curated resources, books, and podcasts that will help you to
%understand what Bitcoin is.
나는 당신이 더 깊이 있게 공부할 수 있는 다양한 출처를 제공하기 위해 최선을 다하였다. 
비트코인을 이해하는 데에 도움이 될 만한  참고서적, 정리된 자료, 팟캐스트 등 여러 
참고자료들은 \href{https://21lessons.com/rabbithole}{21lessons.com/rabbithole}과 \href{https://bitcoin-resources.com}{bitcoin-resources.com}
에 주기적으로 업데이트할 예정이다. 


\paragraph{}
%In short, this is simply a book about Bitcoin, written by a bitcoiner.
%Bitcoin doesn't need this book, and you probably don't need this book to
%understand Bitcoin. I believe that Bitcoin will be understood by you as soon as
%\textit{you} are ready, and I also believe that the first fractions of a bitcoin
%will find you as soon as you are ready to receive them. In essence, everyone
%will get \bitcoinB{}itcoin at exactly the right time. In the meanwhile, Bitcoin
%simply is, and that is enough.\footnote{Beautyon, \textit{Bitcoin is. And that
		%is enough.}~\cite{bitcoin-is}}
요컨대, 이 책은 비트코인에 관한 책이다. 
비트코인은 이 책이 필요하지 않으며, 비트코인을 이해하는 데 이 책이 필요 없을지도 모른다. 
당신이 준비다면 비트코인을 이해할 수 있을 것이고, 
비트코인을 이해하게 되면 비트코인을 받아들일 수 있을 것이다. 
때가 되면 모든 사람이 비트코인을 받아들이게 될 것이다. 
그때가 오기까지 비트코인은 비트코인으로서 충분하다
\footnote{Beautyon, \textit{Bitcoin is. And that is enough.}~\cite{bitcoin-is}}.

