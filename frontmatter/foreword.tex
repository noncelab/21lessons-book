\chapter*{서문}
\pdfbookmark{Foreword}{foreword}

%Some call it a religious experience. Others call it Bitcoin.
누군가는 이를 두고 종교적 경험이라 부르고  
누군가는 비트코인이라 부른다.

%I first met Gigi in one of my spiritual homes -- Riga, Latvia -- the home of
%\textit{The Baltic Honeybadger} Conference, where the most fervent of the
%Bitcoin faithful make a yearly pilgrimage. After a deep lunchtime conversation,
%the bond Gigi and I forged was as set in stone as a Bitcoin transaction that was
%processed when we first shook hands a few hours prior.

나의 정신적 고향 중 하나인 라트비아 리가에서 기기(Gigi)를 처음 만났다. 리가는 발틱 벌꿀오소리 컨퍼런스(\textit{The Baltic Honeybadger Conference - 유럽에서 열리는 비트코인 컨퍼런스})의 본거지로 열렬한 비트코인 지지자들이 매년 방문하는 곳이다. 
기기와 점심시간 동안 나눈 깊은 대화 후, 그와 내가 맺은 유대감은 우리가 몇 시간 전에 처음 악수했을 당시 처리된 비트코인 거래 만큼이나 확고했다. 


%My other spiritual home, Christ Church, Oxford, where I had the privilege to
%study for my MBA, was where I had my \enquote{Rabbit Hole} moment. Like Gigi, I
%transcended the economic, technical and social realms, and was spiritually
%enveloped by Bitcoin. After \enquote{buying high} in the November 2013 bubble,
%there were several extremely hard-learned lessons to be had in the relentlessly
%crushing and seemingly never-ending 3-year bear market. These 21 Lessons would
%indeed have served me very well in that time. Many of these lessons are simply
%natural truths that, to the uninitiated, are obscured by an opaque, fragile
%film. By the end of this book however, the fa\c{c}ade will fragment fiercely.

나의 또 다른 정신적 고향인 옥스포드 크라이스트 처치(Christ Church)는 내가 MBA를 공부했던 특권을 가진 곳이자 \enquote{토끼굴}의 순간을 가졌던 곳이다.
기기처럼 나는 경제적, 기술적, 사회적 영역을 초월했고 영적으로 비트코인에 둘러싸여 있었다. 
2013년 11월 버블 속에서 \enquote{고점 구매}를 한 후, 끊임없이 파괴되고 결코 끝나지 않을 것처럼 보이는 3년의 약세장을 견디며 극도로 힘들게 얻은 몇가지 교훈이 있었다.
이 스물한 가지의 교훈은 그 당시 나에게 큰 도움이 되었을 것이다. 
이 교훈들 중 많은 것들은, 경험이 없는 사람들에게, 불투명하고 깨지기 쉬운 필름으로 가려진 지극히도 자연스러운 진실이다.  그러나 이 책이 끝날 무렵 그 필름은 산산조각이 날 것이다.


%On a crystal-clear night in Oxford in late-August 2016, just a few weeks after
%the knife twisted in my heart again when the Bitfinex Exchange was hacked, I sat
%in quiet contemplation at Christ Church’s Master’s Garden. Times were tough, and
%I was at my mental and emotional breaking-point after what seemed to be a
%lifetime of torture; not because of financial loss, but of the crushing
%spiritual loss I felt being isolated in my world view. If only there were
%resources like this one at the time to see that I was not alone. The Master’s
%Garden is a very special place to me and many who came before me over the
%centuries. It was there where one Charles Dodgson, a Math Tutor at Christ
%Church, observed one of his young pupils, Alice Liddell, the daughter of the
%Dean of Christ Church. Dodgson, better known by his pen-name, Lewis Carroll,
%used Alice and The Garden as his inspiration, and in the magic of that hallowed
%turf, I stared deeply into the crypto-chasm, and it stared blazingly back,
%annihilating my arrogance, and slapping my self-pride square in the face. I was
%finally at peace.

2016년 8월 말 청명한 밤에, 비트파이넥스 해킹으로 또 한번의 심장을 후벼파는 아픔을 겪고 난 몇 주 후, 나는 크라이스트 처치의 마스터스 가든에 조용히 앉아 생각에 잠겼다. 
힘든 시간을 보내며 평생 고문을 당한 것처럼 정신적, 감정적 한계에 다다랐다. 그것은 경제적 손실 때문이 아니라, 내 세계관에 고립된 느낌을 받았던 참담한 정신적 손실 때문이었다. 
당시에 내가 혼자가 아니라는 것을 알 수 있는 이 스물한 가지 교훈 같은 자료들이 있었으면 참 좋았을 것이다. 
마스터스 가든은 수 세기에 걸쳐 나와 내 이전에 존재한 많은 이들에게 매우 특별한 장소이다.
그리스도 교회의 수학 교사인 찰스 도지슨(Charles Dodgson)이 그의 어린 학생 중 한 명인 그리스도 교회 학장의 딸 앨리스 리델을 관찰했던 의미 있는 곳이다.
그의 필명인 루이스 캐롤로 더 잘 알려진 도지슨은 앨리스와 정원에서 영감을 얻었고, 그 신성한 정원의 마법으로 내가 암호화폐 구멍을 깊이 응시하자
그것은 맹렬하게 나를 쏘아보면서 나의 오만함을 말살하고 자만하던 나의 뺨을 때렸다. 그리고 나는 마침내 평화를 얻었다. 

%21 Lessons takes you on a true Bitcoin journey; not just a journey of
%philosophy, technology and economics, but of the soul.
스물 한가지 교훈은 철학, 기술, 경제학의 관점을 넘어 영혼에 이르는 진정한 비트코인 여행을 경험하게 해준다.

%As you dive deeper into the philosophy tersely laid out in 7 of the 21 Lessons,
%one can go as far as to understand the origin of all beings with enough time and
%contemplation. His 7 lessons on economics captures, in simple terms, how we are
%at the financial mercy of a small group of Mad Hatters, and how they have
%successfully managed to put blinders on our minds, hearts and souls. The 7
%lessons on technology lay out the beauty and technologically-Darwinian
%perfection of Bitcoin. Being a non-technical Bitcoiner, the lessons provide a
%salient review of the underlying technological nature of Bitcoin, and indeed,
%the nature of technology itself.

21개의 교훈 중 7개에 간결하게 제시된 철학에 더 깊이 파고들면 충분한 시간과 숙고를 통해 모든 존재의 기원을 이해하는 데까지 나아갈 수 있다. 
경제학에 대한 7개의 교훈은 간단한 용어로 매드 해터(Mad Hatters - 이상한 나라의 앨리스 등장 인물인 모자장수로 앨리스와 이상한 다과회를 갖는다.) 소수 집단이 어떻게 우리의 마음과 영혼을 성공적으로 가릴 수 있었는지 알려준다. 
기술에 관한 일곱 가지 교훈은 비트코인의 아름다움과 기술적으로 다윈적인(진화론적인) 완벽함을 설명한다. 
기술을 잘 모르는 비트코이너로서 이 교훈은 비트코인의 근본적인 기술적 특성과 기술 자체의 특성에 대한 핵심을 이해할 수 있게 해준다.

%In this transient experience we call life, we live, love and learn. But what is
%life but a timestamped order of events?

이 일시적인 경험 속에서 우리는 삶이라 부르고, 살아가고, 사랑하고 배운다. 
하지만 삶이란 시간이 지정된(timestamped) 사건의 연속일 뿐이 아닐까? 

%Conquering the Bitcoin mountain is not easy. False summits are rife, rocks are
%rough, and cracks and crevices are ubiquitously lying in wait to swallow you up.
%After reading this book, you will see that Gigi is the ultimate Bitcoin Sherpa,
%and I will appreciate him forever.

비트코인 산을 정복하는 것은 쉽지 않다. 거짓된 정상은 만연하고 바위는 거칠며 당신을 집어삼키려는 균열과 틈새는 어디에나 기다리고 있다.
이 책을 읽고 나면 기기가 궁극의 비트코인 셰르파(네팔에서 히말라야 산맥을 등산하는 사람을 가이드하는 현지인 또는 네팔 특정 민족을 칭한다.)임을 알게 될 것이다. 
그리고 나는 그를 영원히 존경할 것이다. 


\begin{flushright}
	%Hass McCook \\의
	%November 29, 2019
	2019년 11월 29일 \\
	하스 맥쿡(Hass McCook)
\end{flushright}
