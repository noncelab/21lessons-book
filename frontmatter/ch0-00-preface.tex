\chapter*{서문}

%Falling down the Bitcoin rabbit hole is a strange experience. Like many others,
%I feel like I have learned more in the last couple of years studying Bitcoin
%than I have during two decades of formal education.
비트코인 토끼 굴에 빠지는 것은 이상한 경험이다. 
다른 사람들과 마찬가지로 나는 지난 20년 동안의 받은 정식 교육보다 비트코인을 공부하며
더 많은 것을 배웠다는 사실을 깨달을 수 있었다.

%The following lessons are a distillation of what I’ve learned. First published
%as an article series titled \textit{“What I’ve Learned From Bitcoin,”} what follows
%can be seen as a third edition of the original series.
이 가르침들은 내가 배워온 것들을 무색하게 한다. 
이 글은 “내가 비트코인으로부터 배운 것들”에서 시작된 글의 세 번째 에디션이다.

%Like Bitcoin, these lessons aren't a static thing. I plan to work on them
%periodically, releasing updated versions and additional material in the future.
비트코인 그러하듯, 이 교훈들은 멈추지 않는다. 
나는 주기적으로 글을 업데이트하기로 했다.

%Unlike Bitcoin, future versions of this project do not have to be backward
%compatible. Some lessons might be extended, others might be reworked or
%replaced.
비트코인이 그렇지 않듯이, 이 글은 하위 호환성을 제공할 필요가 없다. 
어떤 가르침은 확장될 것이며, 어떤 것들은 대체될지도 모르기 때문이다.

%Bitcoin is an inexhaustible teacher, which is why I do not claim that these
%lessons are all-encompassing or complete. They are a reflection of my personal
%journey down the rabbit hole. There are many more lessons to be learned, and
%every person will learn something different from entering the world of Bitcoin.
비트코인은 포괄적이거나 완벽한 교훈을 주장하지 않기 때문에 지칠 줄 모르는 선생님과 같다. 
그저 토끼 굴을 따라 들어갈 뿐이다.
그곳에는 더 많은 가르침이 있고 모든 사람은 비트코인 세계에 들어가고 나서 각각 다른 것을 배운다.

%I hope that you will find these lessons useful and that the process of learning
%them by reading won’t be as arduous and painful as learning them firsthand.
이 교훈들이 당신에게 유용하기를 바라며, 그 과정이 힘들고 고통스럽지 않았으면 좋겠다.

% <!-- Internal -->
% [I]: 
%
% <!-- Twitter -->
% [dergigi]: https://twitter.com/dergigi
%
% <!-- Wikipedia -->
% [alice]: https://en.wikipedia.org/wiki/Alice%27s_Adventures_in_Wonderland
% [carroll]: https://en.wikipedia.org/wiki/Lewis_Carroll
