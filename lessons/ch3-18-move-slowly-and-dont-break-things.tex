\chapter{천천히 움직여라 아무것도 깨뜨리지 않도록}
\label{les:18}

\begin{chapquote}{루이스 캐롤, \textit{이상한 나라의 앨리스}}
	%So the boat wound slowly along, beneath the bright summer-day, with its merry crew and its music of voices and laughter\ldots
	그리하여 배는 밝은 여름날 아래 천천히 나아갔다. 즐거운 선원들과 음악의 선율, 그리고 웃음 소리와 함께\ldots
\end{chapquote}

\begin{comment}
	It might be a dead mantra, but \enquote{move fast and break things} is still how
	much of the tech world operates. The idea that it doesn't matter if you
	get things right the first time is a basic pillar of the \textit{fail early,
		fail often} mentality. Success is measured in growth, so as long as you
	are growing everything is fine. If something doesn't work at first you
	simply pivot and iterate. In other words: throw enough shit against the
	wall and see what sticks.
\end{comment}
오래된 진리일지 모르겠지만, \enquote{빠르게 움직여라 무언가 깨뜨릴 정도로\footnote{역주: 페이스북의 모토}}는 기술 세계에서 여전히 통하는 방식이다.
처음부터 제대로 해내는 것이 중요하지 않다는 생각은 \textit{일찌감치 실패하고 자주 실패하라}는 사고방식의 기조이다.
성공은 성장으로 측정되기 때문에 성장하고 있는 한 모든 것이 괜찮다. 
처음에 무언가가 잘 작동하지 않으면 방향을 전환하고 반복하면 된다. 
다른 말로, 똥인지 된장인지는 먹어봐야 안다는 뜻이다.

\begin{comment}
	Bitcoin is very different. It is different by design. It is different
	out of necessity. As Satoshi pointed out, e-currency has been tried
	many times before, and all previous attempts have failed because there
	was a head which could be cut off. The novelty of Bitcoin is that it is
	a beast without heads.
\end{comment}
비트코인은 사뭇 다르다. 설계가 다르다. 필요성부터 다르다.
사토시가 말한 것처럼 전자 화폐는 이전에도 여러 번 시도되었으나, 머리가 날라갈 수도 있었기에 모두 실패하고 말았다.
비트코인의 참신함은 이 머리가 없다는 점이다.

\begin{quotation}\begin{samepage}
	\begin{comment}
		\enquote{A lot of people automatically dismiss e-currency as a lost cause
			because of all the companies that failed since the 1990's. I hope it's
			obvious it was only the centrally controlled nature of those systems
			that doomed them.}
		\begin{flushright} -- Satoshi Nakamoto\footnote{Satoshi Nakamoto, in a reply to Sepp Hasslberger \cite{satoshi-centralized-nature}}
		\end{comment}
		\enquote{많은 사람들이 1990년대부터 시도된 전자 화폐의 실패 원인을 회사가 망했기 때문이라고 생각합니다.
			분명히 말하지만 전자 화폐 시스템이 망한 이유는 중앙에서 제어되는 특성을 가졌기 때문이었습니다.}
		\begin{flushright} -- 사토시 나카모토\footnote{Satoshi Nakamoto, in a reply to Sepp Hasslberger \cite{satoshi-centralized-nature}}
\end{flushright}\end{samepage}\end{quotation}
	
\begin{comment}
	One consequence of this radical decentralization is an inherent
	resistance to change. \enquote{Move fast and break things} does not and will
	never work on the Bitcoin base layer. Even if it would be desirable, it
	wouldn't be possible without convincing \textit{everyone} to change their ways.
	That's distributed consensus. That's the nature of Bitcoin.
\end{comment}
이러한 급진적인 탈중앙화의 결과 중 하나는 내재되어있는 변화에 대한 저항이다.
\enquote{빠르게 움직여라 무언가 깨뜨릴 정도로}의 방식은 비트코인에서는 통하지 않으며, 앞으로도 그럴 것이다.
설사 이것이 바람직하다 하더라도, 모든 사람들이 자신의 방식을 바꾼다고 설득되지 않는 한 실행 불가능하다.
이것이 바로 분산 합의이자 비트코인의 본질이다.

\begin{quotation}\begin{samepage}
	\begin{comment}
		\enquote{The nature of Bitcoin is such that once version 0.1 was released, the
			core design was set in stone for the rest of its lifetime.}
		\begin{flushright} -- Satoshi Nakamoto\footnote{Satoshi Nakamoto, in a reply to Gavin Andresen \cite{satoshi-centralized-nature}}
		\end{comment}
		\enquote{비트코인의 특성상 0.1 버전이 출시되고 나면 핵심 설계는 비트코인이 사라질 때까지 확정된 것이나 다름없습니다.}
		\begin{flushright} -- 사토시 나카모토\footnote{Satoshi Nakamoto, in a reply to Gavin Andresen \cite{satoshi-centralized-nature}}
\end{flushright}\end{samepage}\end{quotation}

\begin{comment}
	This is one of the many paradoxical properties of Bitcoin. We all came
	to believe that anything which is software can be changed easily. But
	the nature of the beast makes changing it bloody hard.
\end{comment}
이것은 비트코인의 많은 역설적 특징 중 하나이다.
우리 모두 소프트웨어는 쉽게 바꿀 수 있다고 믿게 되었다.
하지만 비트코인은 변경이 매우 어렵다. 
	
\begin{comment}
	As Hasu beautifully shows in Unpacking Bitcoin's Social
	Contract~\cite{social-contract}, changing the rules of Bitcoin is only possible
	by \textit{proposing} a change, and consequently \textit{convincing} all users
	of Bitcoin to adopt this change. This makes Bitcoin very resilient to change,
	even though it is software.
\end{comment}
하수(Hasu)는 비트코인의 사회 계약 풀기(Unpacking Bitcoin's Social Contract)\cite{social-contract}에서
비트코인의 규칙을 변경하기 위해서는 제안을 통해서만 가능하며, 모든 사용자가 이를 채택하도록 설득해야 한다고 언급했다.
이로 인해 비트코인은 소프트웨어임에도 불구하고 변화에 매우 복원력이 강하다.

\begin{comment}
	This resilience is one of the most important properties of Bitcoin.
	Critical software systems have to be antifragile, which is what the
	interplay of Bitcoin's social layer and its technical layer guarantees.
	Monetary systems are adversarial by nature, and as we have known for
	thousands of years solid foundations are essential in an adversarial
	environment.
\end{comment}
이러한 복원력은 비트코인의 주요 특성 중 하나이다.
중요한 역할을 수행하는 소프트웨어 시스템은 사회적 계층과 기술적 계층의 상호작용이 보장하는 
안티프래질\footnote{역주: 충격을 받으면 더 강해지는 특성} 특성을 갖춰야 한다.
그동안의 화폐 시스템은 본질적으로 적대적이었다.
우리가 수천 년 동안 보았듯이 이러한 적대적 환경에서는 견고한 기반이 필수이다.
	
\begin{quotation}\begin{samepage}
	\begin{comment}
		\enquote{The rain came down, the floods came, and the winds blew, and beat on
			that house; and it didn't fall, for it was founded on the rock.}
		\begin{flushright} -- Matthew 7:24--27
		\end{comment}
		\enquote{비가 내리고 창수가 나고 바람이 불어 그 집에 부딪치되 무너지지 아니하나니 이는 주추를 반석 위에 놓은 까닭이요.}
		\begin{flushright} -- 마태복음 7:24--27
\end{flushright}\end{samepage}\end{quotation}
		
\begin{comment}
	Arguably, in this parable of the wise and the foolish builders Bitcoin
	isn't the house. It is the rock. Unchangeable, unmoving, providing the
	foundation for a new financial system.
\end{comment}
성경에 등장하는 '현명한 건축가와 어리석은 건축가' 우화에 비유한다면 비트코인은 집(house)이 아니다. 반석(rock)이다.
\footnote{역주: 집을 짓되 깊이 파고 주추를 반석 위에 놓은 사람과 같으니(he is like a man which built an house, and digged deep, and laid the foundation on a rock) (성경 누가복음 6:48)}
비트코인은 변치않고, 움직이지 않으며 새로운 금융 시스템의 토대를 제공한다.

\begin{comment}
	Just like geologists, who know that rock formations are always moving
	and evolving, one can see that Bitcoin is always moving and evolving as
	well. You just have to know where to look and how to look at it.
\end{comment}
암석층이 항상 움직이고 진화하고 있다는 것을 아는 지질학자들과 마찬가지로
비트코인도 항상 움직이고 진화하고 있다는 것을 알 수 있다.
어디를 어떻게 봐야 하는지만 알면 된다.
		
\begin{comment}
	The introduction of pay to script hash\footnote{ Pay to script hash (P2SH)
		transactions were standardised in BIP 16. They allow transactions to be sent to
		a script hash (address starting with 3) instead of a public key hash (addresses
		starting with 1).~\cite{btcwiki:p2sh}} and segregated
	witness\footnote{Segregated Witness (abbreviated as SegWit) is an implemented
		protocol upgrade intended to provide protection from transaction malleability
		and increase block capacity. SegWit separates the \textit{witness} from the list
		of inputs.~\cite{btcwiki:segwit}} are proof that Bitcoin's rules can be changed
	if enough users are convinced that adopting said change is to the benefit of the
	network. The latter enabled the development of the lightning
	network\footnote{\url{https://lightning.network/}}, which is one of the houses
	being built on Bitcoin's solid foundation. Future upgrades like Schnorr
	signatures~\cite{bip:schnorr} will enhance efficiency and privacy, as well as
	scripts (read: smart contracts) which will be indistinguishable from regular
	transactions thanks to Taproot~\cite{taproot}. Wise builders do indeed build on
	solid foundations.
\end{comment}
P2SH\footnote{Pay to script hash (P2SH)
	트랜잭션 표준은 BIP 16에 정의되어 있다. 이 표준은 공개키(1로 시작하는 주소)로 지불하는 것 대신 스크립트 해시(3으로 시작하는 주소)에 지불하는 것을 허용한다.
	.~\cite{btcwiki:p2sh}}와
세그윗(SegWit)\footnote{Segregated Witness 는
	트랜잭션의 유연성으로부터 네트워크를 보호하고 블록의 용량 효율을 늘리기위해 구현된 프로토콜 업그레이드이다.
	SegWit는 입력값에서 검증 데이터를 분리한다.~\cite{btcwiki:segwit}}의 도입은
다수의 네트워크 참여자가 해당 변경을 채택하는 것이 네트워크에 이익이 된다는 확신이 있다면 규칙을 변경할 수 있다는 것을 보여준 증거이다.
세그윗은 비트코인의 단단한 반석 위에 지어진 집 중 하나인 라이트닝 네트워크\footnote{\url{https://lightning.network/}} 개발을 가능하게 했다.
향후 슈노르 서명\cite{bip:schnorr}과 같은 업그레이드를 통해 효율성과 프라이버시가 향상될 것이며, 탭루트 덕분에 일반 트랜잭션과 구별할 수 없는 스마트 컨트랙트가 등장할 것이다.
현명한 건축가는 견고한 반석 위에 집을 짓는다.
		
\begin{comment}
	Satoshi wasn't only a wise builder technologically. He also understood
	that it would be necessary to make wise decisions ideologically.
\end{comment}
사토시는 기술적으로만 현명한 건축가가 아니었다.
그는 이념적으로도 현명한 결정이 필요하다는 것을 이해하고 있었다.

\begin{quotation}\begin{samepage}
	\begin{comment}
		\enquote{Being open source means anyone can independently review the code. If
			it was closed source, nobody could verify the security. I think it's
			essential for a program of this nature to be open source.}
		\begin{flushright} -- Satoshi Nakamoto\footnote{Satoshi Nakamoto, in a reply to SmokeTooMuch \cite{satoshi-open-source}}
		\end{comment}
		\enquote{오픈소스라는 것은 누구나 독립적으로 코드를 검토할 수 있다는 것을 뜻합니다.
			비공개 소스라면 누구도 보안성을 검증할 수 없습니다. 
			나는 이런 성격의 프로그램은 당연히 오픈소스로 공개되어야 한다고 생각합니다.}
		\begin{flushright} -- 사토시 나카모토\footnote{Satoshi Nakamoto, in a reply to SmokeTooMuch \cite{satoshi-open-source}}
\end{flushright}\end{samepage}\end{quotation}
	
\begin{comment}
	Openness is paramount to security and inherent in open source and the
	free software movement. As Satoshi pointed out, secure protocols and the
	code which implements them have to be open --- there is no security
	through obscurity. Another benefit is again related to decentralization:
	code which can be run, studied, modified, copied, and distributed freely
	ensures that it is spread far and wide.
\end{comment}
개방성은 보안에 있어 가장 중요한 요소이며 오픈소스 및 자유 소프트웨어 운동에 개방성이 내재되어 있다.
사토시가 지적했듯이 보안 프로토콜과 이를 구현하는 코드는 공개되어야 하며, 모호함으로는 보안을 확보할 수 없다.
개방성의 또 다른 장점은 탈중앙성과도 관련이 있다.
자유롭게 실행되고, 연구하며, 수정, 복사 및 배포할 수 있는 코드는 널리 확산될 수 있다.
	
	
\begin{comment}
	The radically decentralized nature of Bitcoin is what makes it move
	slowly and deliberately. A network of nodes, each run by a sovereign
	individual, is inherently resistant to change --- malicious or not. With
	no way to force updates upon users the only way to introduce changes is
	by slowly convincing each and every one of those individuals to adopt a
	change. This non-central process of introducing and deploying changes is
	what makes the network incredibly resilient to malicious changes. It is
	also what makes fixing broken things more difficult than in a
	centralized environment, which is why everyone tries not to break things
	in the first place.
\end{comment}
비트코인의 극단적인 탈중앙성으로 인해 비트코인은 느리고 신중하게 움직인다.
주권자 개인의 운영으로 구성된 노드 네트워크는 악의적이든 아니든 본질적으로 변화에 저항한다.
참여자에게 업데이트를 강제할 방법이 없기 때문에 변화시킬 수 있는 유일한 방법은
모든 개인이 변경 사항을 채택하도록 천천히 설득하는 것이다.
변경 사항을 도입하고 배포하는 이 탈중앙화된 프로세스는 악의적인 변경에 대해 놀라울 정도로 탄력적으로 대응할 수 있게 해준다.
탈중앙화된 환경에서는 중앙 집중식 환경에서보다 고장난 것을 고치는 것이 더 어렵기 때문에 모든 사람들이 애초에 비트코인을 고장내지 않으려 노력하는 것도 이유이다. 
	
%\paragraph{Bitcoin taught me that moving slowly is one of its features, not a bug.}
\paragraph{비트코인은 느리게 움직이는 것이 버그가 아닌 비트코인의 특징 중 하나라는 것을 가르쳐주었다.}

% ---
%
% #### Through the Looking-Glass
%
% - [Lesson 1: Immutability and Change][lesson1]
%
% #### Down the Rabbit Hole
%
% - [Unpacking Bitcoin's Social Contract] by Hasu
% - [Schnorr signatures BIP][Schnorr signatures] by Pieter Wuille
% - [Taproot proposal][Taproot] by Gregory Maxwell
% - [P2SH][pay to script hash], [SegWit][segregated witness] on the Bitcoin Wiki
% - [Parable of the Wise and the Foolish Builders][Matthew 7:24--27] on Wikipedia
%
% <!-- Down the Rabbit Hole -->
% [lesson1]: {{ '/bitcoin/lessons/ch1-01-immutability-and-change' | absolute_url }}
%
% [Unpacking Bitcoin's Social Contract]: https://uncommoncore.co/unpacking-bitcoins-social-contract/
% [Matthew 7:24--27]: https://en.wikipedia.org/wiki/Parable_of_the_Wise_and_the_Foolish_Builders
% [pay to script hash]: https://en.bitcoin.it/wiki/Pay_to_script_hash
% [segregated witness]: https://en.bitcoin.it/wiki/Segregated_Witness
% [lightning network]: https://lightning.network/
% [Schnorr signatures]: https://github.com/sipa/bips/blob/bip-schnorr/bip-schnorr.mediawiki#cite_ref-6-0
% [Taproot]: https://lists.linuxfoundation.org/pipermail/bitcoin-dev/2018-January/015614.html
%
% <!-- Wikipedia -->
% [alice]: https://en.wikipedia.org/wiki/Alice%27s_Adventures_in_Wonderland
% [carroll]: https://en.wikipedia.org/wiki/Lewis_Carroll
