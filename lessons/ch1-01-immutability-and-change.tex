\chapter{불변성과 변화}
\label{les:1}

%\begin{chapquote}{Alice}
%\enquote{I wonder if I've been changed in the night. Let me think. Was I the same when I
	%got up this morning? I almost think I can remember feeling a little different.
	%But if I'm not the same, the next question is `Who in the world am I?' Ah,
	%that's the great puzzle!}
%\end{chapquote}$
\begin{chapquote}{루이스 캐롤, \textit{이상한 나라의 앨리스}}
	\enquote{혹시 내가 하룻밤 사이 변한건가? 어디 생각해보자.
		오늘 아침에 일어났을 때는 똑같았나? 이전과 조금 달랐던 것 같기도 한데.. 
		좀 달랐다면, 지금의 난 도대체 누구란 말이지? 아, 정말 알다가도 모를 일이네!}
\end{chapquote}

%Bitcoin is inherently hard to describe. It is a \textit{new thing}, and any
%attempt to draw a comparison to previous concepts -- be it by calling
%it digital gold or the internet of money -- is bound to fall short of
%the whole. Whatever your favorite analogy might be, two aspects of
%Bitcoin are absolutely essential: decentralization and immutability.
\paragraph{}
비트코인은 본질적으로 설명하기 어렵다. 
비트코인은 \textit{새로운 개념}이기에 디지털 금, 돈을 위한 인터넷(the internet of money) 같이 이미 있는 개념에 빗대어 설명하기엔 뭔가 부족해 보인다. 
당신이 가장 좋아하는 비유가 무엇이든 비트코인의 두 가지 측면은 반드시 필요하다. 탈중앙성과 불변성이 바로 그것이다.

\paragraph{}
%One way to think about Bitcoin is as an automated social contract\footnote{Hasu,
	%Unpacking Bitcoin's Social Contract~\cite{social-contract}}. The software is
%just one piece of the puzzle, and hoping to change Bitcoin by changing the
%software is an exercise in futility. One would have to convince the rest of the
%network to adopt the changes, which is more a psychological effort than a
%software engineering one.
비트코인을 이해하는 한 가지 방법은 자동화된 사회 계약으로 여기는 것이다.\footnote{Hasu, Unpacking Bitcoin's Social Contract~\cite{social-contract}}. 
소프트웨어는 비트코인 퍼즐의 한 조각일 뿐이기 때문에 소프트웨어를 변경하여 비트코인을 바꾸려는 시도는 헛된 것이다.
비트코인을 바꾸기 위해선 네트워크 구성원들이 변경 사항을 받아들이도록 설득해야 하는데, 이는 소프트웨어 공학적 접근이라기보다 심리적 노력에 가깝다.

\paragraph{}
%The following might sound absurd at first, like so many other things in
%this space, but I believe that it is profoundly true nonetheless: You
%won't change Bitcoin, but Bitcoin will change you.
당신이 비트코인을 바꿀 순 없지만, 비트코인은 당신을 바꿀 것이다.
(You won't change Bitcoin, but Bitcoin will change you.)
이 책에 나오는 다른 내용들처럼 이 말이 처음엔 터무니없게 들릴 수 있음에도 불구하고, 
나는 이것이 매우 중요한 진실이라 믿어 의심치 않는다.

\begin{quotation}\begin{samepage}
		\enquote{우리가 비트코인을 변하게 하는 것 보다 비트코인이 우리를 더 많이 변하게 할 것입니다.}
		\begin{flushright} -- 마티 벤트\footnote{Tales From the Crypt~\cite{tftc21}}
\end{flushright}\end{samepage}\end{quotation}

%It took me a long time to realize the profundity of this. Since Bitcoin
%is just software and all of it is open-source, you can simply change
%things at will, right? Wrong. \textit{Very} wrong. Unsurprisingly, Bitcoin's
%creator knew this all too well.
나는 그 심오함을 깨닫기까지 꽤 오랜 시간이 걸렸다.
비트코인은 소프트웨어일 뿐이고 모든 것이 오픈 소스이기 때문에 마음대로 변경할 수 있는 것 아닌가? 
틀렸다. 매우 잘못된 생각이다. 
당연히 비트코인 창시자도 이 사실을 잘 알고 있었다.

\begin{quotation}\begin{samepage}
		%\enquote{The nature of Bitcoin is such that once version 0.1 was released, the core
			%design was set in stone for the rest of its lifetime.}
		\enquote{비트코인의 특성상 0.1 버전이 출시되고 나면 핵심 설계는 평생 동안 확정된 것이나 다름없습니다.}
		\begin{flushright} -- 사토시 나카모토\footnote{BitcoinTalk forum post: `Re:
				Transactions and Scripts\ldots'~\cite{satoshi-set-in-stone}}
\end{flushright}\end{samepage}\end{quotation}

%Many people have attempted to change Bitcoin's nature. So far all of
%them have failed. While there is an endless sea of forks and altcoins,
%the Bitcoin network still does its thing, just as it did when the first
%node went online. The altcoins won't matter in the long run. The forks
%will eventually starve to death. Bitcoin is what matters. As long as our
%fundamental understanding of mathematics and/or physics doesn't change,
%the Bitcoin honeybadger will continue to not care.
많은 사람들이 비트코인의 본질을 바꾸려고 시도했었다. 그러나 지금까지 그런 시도는 모두 실패했다. 
끊임없는 포크와 알트코인의 홍수 속에서도 비트코인 네트워크는 첫 번째 노드가 구동될 때와 마찬가지로 여전히 제 역할을 하고 있다.
장기적으로 알트코인은 하찮아질 것이고 포크된 체인은 결국 사장될 것이다.
수학이나 물리학에 대한 우리의 근본적 이해가 변하지 않는 한, 이 비트코인 벌꿀오소리는 흔들리지 않을 것이다.
\footnote{역주: 비트코인을 벌꿀오소리라는 동물에 빗댄다. 벌꿀오소리는 힘과 회복력, 강인함으로 유명하다.}

\begin{quotation}\begin{samepage}
		\enquote{비트코인은 새로운 삶의 형태를 보여줍니다. 비트코인은 인터넷 속에 살아 숨 쉽니다. 
		지불하는데에 비트코인을 쓸 수 있기 때문에 비트코인은 살아있습니다. [\ldots] 비트코인은 변하지 않습니다. 논쟁의 여지가 없습니다.
		조작될 수도, 손상될 수도, 멈출 수도 없습니다. [\ldots] 핵전쟁으로 지구 절반이 파괴되어도 청렴결백하게 살아남을 것입니다. }
		\begin{flushright} -- 랄프 머클\footnote{DAOs, Democracy and Governance,~\cite{merkle-dao}}
\end{flushright}\end{samepage}\end{quotation}

\paragraph{}
%The heartbeat of the Bitcoin network will outlast all of ours.
비트코인 네트워크의 심장박동은 우리 모두의 심장박동 보다 더 오래 뛸 것이다.

\paragraph{}
%Realizing the above changed me way more than the past blocks of the Bitcoin
%blockchain ever will. It changed my time preference, my understanding of
%economics, my political views, and so much more. Hell, it is even changing
%people's diets\footnote{Inside the World of the Bitcoin
%Carnivores,~\cite{carnivores}}. If all of this sounds crazy to you, you're in
%good company. All of this is crazy, and yet it is happening.
나는 이 사실을 깨닫고 나서 그 동안 쌓인 비트코인 블록보다 훨씬 더 큰 변화를 겪었다.
시간선호도, 경제에 대한 이해, 정치적 견해 등이 바뀌었다. 
심지어 비트코인은 사람들의 식단까지 바꾸고 있다.\footnote{Inside the World of the Bitcoin Carnivores,~\cite{carnivores}} 
이게 다 정신나간 소리로 들린다해도 괜찮다. 당신만 그런게 아니다.
이 모든 것이 미친 소리 같지만, 실제로 일어나고 있는 일이다.

\paragraph{비트코인은 나에게 비트코인이 변하지 않는다는 것을 알려주었다. 내가 변할 뿐이다.}

% ---
%
% #### Through the Looking-Glass
%
% - [Bitcoin's Gravity: How idea-value feedback loops are pulling people in][gravity]
% - [Lesson 18: Move slowly and don't break things][lesson18]
%
% #### Down the Rabbit Hole
%
% - [Unpacking Bitcoin's Social Contract][automated social contract]: A framework for skeptics by Hasu
% - [DAOs, Democracy and Governance][Ralph Merkle] by Ralph C. Merkle
% - [Marty's Bent][bent]: A daily newsletter highlighting signal in Bitcoin by Marty Bent
% - [Technical Discussion on Bitcoin's Transactions and Scripts][Satoshi Nakamoto] by Satoshi Nakamoto, Gavin Andresen, and others
% - [Inside the World of the Bitcoin Carnivores][carnivores]: Why a small community of Bitcoin users is eating meat exclusively by Jordan Pearson
% - [Tales From the Crypt][tftc] hosted by Marty Bent
%
% <!-- Internal -->
% [gravity]: 
% [lesson18]: {{ 'bitcoin/lessons/ch3-18-move-slowly-and-dont-break-things' | absolute_url }}
%
% <!-- Further Reading -->
% [automated social contract]: https://medium.com/@hasufly/bitcoins-social-contract-1f8b05ee24a9
% [carnivores]: https://motherboard.vice.com/en_us/article/ne74nw/inside-the-world-of-the-bitcoin-carnivores
% [tftc]: https://tftc.io/tales-from-the-crypt/
% [bent]: https://tftc.io/martys-bent/
%
% <!-- Quotes -->
% [Ralph Merkle]: http://merkle.com/papers/DAOdemocracyDraft.pdf
% [Satoshi Nakamoto]: https://bitcointalk.org/index.php?topic=195.msg1611#msg1611
%
% <!-- Twitter People -->
% [Marty Bent]: https://twitter.com/martybent
%
% <!-- Wikipedia -->
% [alice]: https://en.wikipedia.org/wiki/Alice%27s_Adventures_in_Wonderland
% [carroll]: https://en.wikipedia.org/wiki/Lewis_Carroll
