\chapter{불변성과 변화}
\label{les:1}

%\begin{chapquote}{Alice}
%\enquote{I wonder if I've been changed in the night. Let me think. Was I the same when I
	%got up this morning? I almost think I can remember feeling a little different.
	%But if I'm not the same, the next question is `Who in the world am I?' Ah,
	%that's the great puzzle!}
%\end{chapquote}$
\begin{chapquote}{앨리스}
	\enquote{혹시 내가 변한 건가?
		왠지 다른 느낌이 들었거든. 만약에 내가 다른 모습이라면 다음 질문은 `도대체 나는 누굴까?' 야.
		아, 그것은 너무 큰 수수께끼야!}
\end{chapquote}



%Bitcoin is inherently hard to describe. It is a \textit{new thing}, and any
%attempt to draw a comparison to previous concepts -- be it by calling
%it digital gold or the internet of money -- is bound to fall short of
%the whole. Whatever your favorite analogy might be, two aspects of
%Bitcoin are absolutely essential: decentralization and immutability.
비트코인은 본질적으로 설명하기 어렵다. 
비트코인은 새로운 개념이고, 디지털 금, 인터넷 돈 등으로 설명하기엔 뭔가 부족해 보인다. 
당신이 가장 좋아하는 비유가 무엇이든 비트코인의 두 가지 측면은 절대적으로 필요하다. 
탈중앙성과 불변성이 그것이다.

\paragraph{}
%One way to think about Bitcoin is as an automated social contract\footnote{Hasu,
	%Unpacking Bitcoin's Social Contract~\cite{social-contract}}. The software is
%just one piece of the puzzle, and hoping to change Bitcoin by changing the
%software is an exercise in futility. One would have to convince the rest of the
%network to adopt the changes, which is more a psychological effort than a
%software engineering one.
비트코인을 이해하는 한 가지 방법은 자동화된 사회 계약으로 보는 것이다\footnote{Hasu, Unpacking Bitcoin's Social Contract~\cite{social-contract}}. 
비트코인 소프트웨어는 네트워크의 일부분일 뿐이다. 
소프트웨어를 변경한다고 해서 비트코인을 변경할 수 없기 때문이다. 
변경을 위해서는 네트워크의 나머지 참여자들을 설득 해야 하는데, 
이는 소프트웨어적 노력보다 심리적 노력이 더 필요하다.


\paragraph{}
%The following might sound absurd at first, like so many other things in
%this space, but I believe that it is profoundly true nonetheless: You
%won't change Bitcoin, but Bitcoin will change you.
당신은 비트코인을 변경하지 않지만, 비트코인은 당신을 변하게 할 수 있다.
조금 터무니없이 들리겠지만, 나는 이를 사실이라 믿는다.


\begin{quotation}\begin{samepage}
		\enquote{우리가 비트코인을 변하게 하는 것 보다 비트코인이 우리를 더 변하게 할 것이다.}
		\begin{flushright} -- 마티 벤트\footnote{Tales From the Crypt~\cite{tftc21}}
\end{flushright}\end{samepage}\end{quotation}

%It took me a long time to realize the profundity of this. Since Bitcoin
%is just software and all of it is open-source, you can simply change
%things at will, right? Wrong. \textit{Very} wrong. Unsurprisingly, Bitcoin's
%creator knew this all too well.
나는 이 심오함을 알아차리는 데 꽤 오랜 시간이 걸렸다.
비트코인은 단지 소프트웨어일 뿐이고 오픈 소스이기 때문에 
마음대로 변경할 수 있다고 생각했지만 매우 잘못된 생각이었다. 
비트코인 창시자는 이를 너무 잘 알고 있었다.

\begin{quotation}\begin{samepage}
		%\enquote{The nature of Bitcoin is such that once version 0.1 was released, the core
			%design was set in stone for the rest of its lifetime.}
		\enquote{비트코인의 본질은 비트코인 0.1 버전이 출시됨과 동시에 비트코인이 사라질 때까지 
			핵심 설계가 돌처럼 고정된다는 것이다.}
		\begin{flushright} -- 사토시 나카모토\footnote{BitcoinTalk forum post: `Re:
				Transactions and Scripts\ldots'~\cite{satoshi-set-in-stone}}
\end{flushright}\end{samepage}\end{quotation}

%Many people have attempted to change Bitcoin's nature. So far all of
%them have failed. While there is an endless sea of forks and altcoins,
%the Bitcoin network still does its thing, just as it did when the first
%node went online. The altcoins won't matter in the long run. The forks
%will eventually starve to death. Bitcoin is what matters. As long as our
%fundamental understanding of mathematics and/or physics doesn't change,
%the Bitcoin honeybadger will continue to not care.
많은 사람이 비트코인의 본질을 변경하고자 시도하였다. 그러나 모든 시도는 실패하였다. 알트코인이 수많은
포크를 진행하는 동안 비트코인 네트워크는 첫 번째 노드가 구동될 때와 같이 여전히 제 역할을 하고 있다.
알트코인은 장기적인 관점을 중요하지 않게 생각한다. 포크는 결국 굶어 죽을 것이다. 수학적, 물리학적 근본적인 
이해가 변하지 않는 한 비트코인 벌꿀오소리는 변하지 않을 것이다.

\begin{quotation}\begin{samepage}
		\enquote{비트코인은 삶의 새로운 형태를 보여준다. 비트코인은 인터넷에서 살아 숨 쉰다. 비트코인은
			사람들이 생존을 위해 구매를 지속하는 한 살아있을 것이다. [\ldots] 비트코인은 변하지 않는다. 논쟁의 여지가 없다.
			조작될 수도, 손상될 수도, 멈출 수도 없다. [\ldots] 핵전쟁으로 지구의 절반이 파괴되어도 
			부패하지 않고 살아남아 있을 것이다. }
		\begin{flushright} -- 랄프 머클\footnote{DAOs, Democracy and
				Governance,~\cite{merkle-dao}}
\end{flushright}\end{samepage}\end{quotation}


%The heartbeat of the Bitcoin network will outlast all of ours.
비트코인 네트워크의 심장박동은 모든 사람보다 오래 지속될 것이다.

~

%Realizing the above changed me way more than the past blocks of the Bitcoin
%blockchain ever will. It changed my time preference, my understanding of
%economics, my political views, and so much more. Hell, it is even changing
%people's diets\footnote{Inside the World of the Bitcoin
	%Carnivores,~\cite{carnivores}}. If all of this sounds crazy to you, you're in
%good company. All of this is crazy, and yet it is happening.

위의 내용을 깨닫는 것은 나를 변화시키는 것이었다. 
이 사실은 나의 시간선호도, 경제에 대한 이해, 정치적 견해 등을 바꾸었다. 
심지어 사람들의 식단도 바꾼다\footnote{Inside the World of the Bitcoin Carnivores,~\cite{carnivores}}. 
이 말들이 이상하게 들린다면 당신은 좋은 사회에 있는 것이다. 
모두가 미쳤고 미친 짓은 계속되고 있다.
~

\paragraph{비트코인은 비트코인이 변하지 않는다는 것을 알려주었다. 단지 내가 변할 뿐이다.}

% ---
%
% #### Through the Looking-Glass
%
% - [Bitcoin's Gravity: How idea-value feedback loops are pulling people in][gravity]
% - [Lesson 18: Move slowly and don't break things][lesson18]
%
% #### Down the Rabbit Hole
%
% - [Unpacking Bitcoin's Social Contract][automated social contract]: A framework for skeptics by Hasu
% - [DAOs, Democracy and Governance][Ralph Merkle] by Ralph C. Merkle
% - [Marty's Bent][bent]: A daily newsletter highlighting signal in Bitcoin by Marty Bent
% - [Technical Discussion on Bitcoin's Transactions and Scripts][Satoshi Nakamoto] by Satoshi Nakamoto, Gavin Andresen, and others
% - [Inside the World of the Bitcoin Carnivores][carnivores]: Why a small community of Bitcoin users is eating meat exclusively by Jordan Pearson
% - [Tales From the Crypt][tftc] hosted by Marty Bent
%
% <!-- Internal -->
% [gravity]: 
% [lesson18]: {{ 'bitcoin/lessons/ch3-18-move-slowly-and-dont-break-things' | absolute_url }}
%
% <!-- Further Reading -->
% [automated social contract]: https://medium.com/@hasufly/bitcoins-social-contract-1f8b05ee24a9
% [carnivores]: https://motherboard.vice.com/en_us/article/ne74nw/inside-the-world-of-the-bitcoin-carnivores
% [tftc]: https://tftc.io/tales-from-the-crypt/
% [bent]: https://tftc.io/martys-bent/
%
% <!-- Quotes -->
% [Ralph Merkle]: http://merkle.com/papers/DAOdemocracyDraft.pdf
% [Satoshi Nakamoto]: https://bitcointalk.org/index.php?topic=195.msg1611#msg1611
%
% <!-- Twitter People -->
% [Marty Bent]: https://twitter.com/martybent
%
% <!-- Wikipedia -->
% [alice]: https://en.wikipedia.org/wiki/Alice%27s_Adventures_in_Wonderland
% [carroll]: https://en.wikipedia.org/wiki/Lewis_Carroll
