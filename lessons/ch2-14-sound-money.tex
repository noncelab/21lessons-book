\chapter{건전 화폐}
\label{les:14}

\begin{chapquote}{루이스 캐롤, \textit{이상한 나라의 앨리스}}
	\begin{comment}	
		\enquote{The first thing I've got to do,} said Alice to herself, as she wandered about
		in the wood, \enquote{is to grow to my right size, and the second thing is to find my
			way into that lovely garden. I think that will be the best plan.}
	\end{comment}
	\enquote{내가 첫 번째로 해야 할 일은,} 앨리스가 숲속을 배회하며 혼잣말을 했다. \enquote{원래 크기로 돌아가는 것이고, 
		두 번째 할 일은 아름다운 정원으로 가는 방법을 찾는거야. 이게 최선이야.}
\end{chapquote}

\paragraph{}
\begin{comment}	
	The most important lesson I have learned from Bitcoin is that in the
	long run, hard money is superior to soft money. Hard money, also
	referred to as \textit{sound money}, is any globally traded currency that
	serves as a reliable store of value.
\end{comment}
내가 비트코인에서 배운 가장 중요한 교훈은 장기적으로 경화(hard money)가 연화(soft money)보다 우월하다는 것이다.
건전 화폐(sound money)라고도 불리는 경화는 신뢰할 수 있는 가치 저장소 역할을 하는 전 세계적으로 거래되는 화폐를 의미한다.

\paragraph{}
\begin{comment}	
	Granted, Bitcoin is still young and volatile. Critics will say that it
	does not store value reliably. The volatility argument is missing the
	point. Volatility is to be expected. The market will take a while to
	figure out the just price of this new money. Also, as is often jokingly
	pointed out, it is grounded in an error of measurement. If you think in
	dollars you will fail to see that one bitcoin will always be worth one
	bitcoin.
\end{comment}
물론, 비트코인은 아직 초기 단계에 있고 변동성이 크다. 
이를 두고 비평가들은 가치를 안정적으로 저장하지 못한다고 주장할 것이다.
하지만, 변동성 논쟁은 요점이 아니다. 변동성은 얼마든지 나타날 수 있다. 
시장은 이 새로운 돈의 정당한 가격을 결정하는 데 시간이 걸릴 것이다. 
또한 흔히 농담조로 지적하듯이 이러한 가치 측정 방법에는 오류가 있다. 
비트코인의 가치를 달러로 측정하면 1비트코인이 항상 1비트코인의 가치가 있다는 사실을 놓치게 된다는 점을 알아야 한다.

\begin{quotation}\begin{samepage}
		\begin{comment}	
			\enquote{A fixed money supply, or a supply altered only in accord with
				objective and calculable criteria, is a necessary condition to a
				meaningful just price of money.}
		\end{comment}
		\enquote{고정된 공급, 즉 객관적이고 계산 가능한 기준대로 변경되는 공급이 정당한 화폐 가격을 위한 필수 조건이다.}
		\begin{flushright} -- 버나드 뎀프시\footnote{Fr. Bernard W. Dempsey, S.J., \textit{Interest and Usury}~\cite[p.~210]{dempsey_interest_1943}}
\end{flushright}\end{samepage}\end{quotation}

\newpage

\paragraph{}
\begin{comment}	
	As a quick stroll through the graveyard of forgotten currencies has
	shown, money which can be printed will be printed. So far, no human in
	history was able to resist this temptation.
\end{comment}
역사적으로 잊혀진 많은 화폐를 훑어보면 알 수 있듯이 돈은 계속 인쇄될 것이다. 
지금까지 역사상 그 누구도 이 유혹을 뿌리치지 못했다.

\paragraph{}
\begin{comment}	
	Bitcoin does away with the temptation to print money in an ingenious
	way. Satoshi was aware of our greed and fallibility --- this is why he
	chose something more reliable than human restraint: mathematics.
\end{comment}
비트코인은 기발한 방법으로 돈을 인쇄하고자 하는 유혹을 물리친다. 
사토시는 우리의 탐욕 때문에 우를 범할 것임을 예상했다.
이것이 그가 인간의 통제보다 신뢰할 수 있는 것, 즉 수학을 선택한 이유이다.

\begin{figure}
	\centering
	\begin{equation}
		\sum\limits_{i=0}^{32} \frac{210000 \lfloor \frac{50*10^8}{2^i} \rfloor}{10^8}
	\end{equation}
	\caption{비트코인 공급량 공식}
	\label{fig:supply-formula-white}
\end{figure}

\paragraph{}
\begin{comment}	
	While this formula is useful to describe Bitcoin's supply, it is actually
	nowhere to be found in the code. Issuance of new bitcoin is done in an
	algorithmically controlled fashion, by reducing the reward which is paid to
	miners every four years~\cite{btcwiki:supply}. The formula above is used to
	quickly sum up what is happening under the hood. What really happens can be best
	seen by looking at the change in block reward, the reward paid out to whoever
	finds a valid block, which roughly happens every 10 minutes.
\end{comment}다
이 공식은 비트코인의 공급을 설명하는 데 유용하지만 실제 코드에서는 찾을 수 없다. 
신규 비트코인 발행은 4년마다 채굴자에게 지급되는 보상을 줄이는 알고리즘을 통해 제어된다.\cite{btcwiki:supply}
위 공식은 단지 비트코인 내부의 작동 방식을 빠르게 이해하고 요약하는 데 사용되는 것이다. 
실제로는 대략 10분마다 발생하는 유효한 블록을 찾는 채굴자에게 지급되는 보상의 변화를 보면 잘 알 수 있다.

\begin{figure}
	\includegraphics{assets/images/you-are-here.png}
	\caption{비트코인의 공급량 제한}
	\label{fig:you-are-here.png}
\end{figure}

\paragraph{}
\begin{comment}	
	Formulas, logarithmic functions and exponentials are not exactly
	intuitive to understand. The concept of \textit{soundness} might be easier to
	understand if looked at in another way. Once we know how much there is
	of something, and once we know how hard this something is to produce or
	get our hands on, we immediately understand its value. What is true for
	Picasso's paintings, Elvis Presley's guitars, and Stradivarius violins
	is also true for fiat currency, gold, and bitcoins.
\end{comment}
공식, 대수 함수, 지수는 이해하기 직관적이지 않다. 
건전성이라는 개념은 다른 방식으로 살펴보면 더 쉽게 이해할 수 있다.
무언가가 얼마나 많은지 알게 되면, 그리고 그것을 생산하거나 획득하는 것이 얼마나 어려운지 알게 되면 
우리는 즉시 그 가치를 이해하게 된다. 
피카소의 그림, 엘비스 프레슬리의 기타, 스트라디바리우스의 바이올린이 왜 가치있는지 안다면,
법정화폐, 금, 비트코인이 얼마나 가치 있는지도 알 수 있을 것이다.

\paragraph{}
\begin{comment}	
	The hardness of fiat currency depends on who is in charge of the
	respective printing presses. Some governments might be more willing to
	print large amounts of currency than others, resulting in a weaker
	currency. Other governments might be more restrictive in their money
	printing, resulting in harder currency.
\end{comment}
법정화폐의 건전성은 인쇄기를 담당하는 사람에게 달려있다. 
어떤 정부는 다른 정부보다 더 많은 양의 돈을 인쇄하여 통화 가치의 약세를 초래할 수 있다.
한편 어떤 정부는 화폐 발행을 제한하여 통화 강세를 이끌 수도 있다.


\begin{samepage}\begin{quotation}
		\begin{comment}	
			\enquote{One important aspect of this new reality is that institutions like
				the Fed cannot go bankrupt. They can print any amount of money that
				they might need for themselves at virtually zero cost.}
		\end{comment}
		\enquote{이 새로운 현실의 한 가지 중요한 측면은 연준과 같은 기관이 파산할 수 없다는 것이다. 
		그들은 사실상 비용을 전혀 들이지 않고 자신이 필요한 만큼 돈을 인쇄할 수 있다.}
		\begin{flushright} -- 요르그 귀도 홀스만\footnote{Jörg Guido Hülsmann, \textit{The
					Ethics of Money Production}~\cite{hulsmann2008ethics}}
\end{flushright}\end{quotation}\end{samepage}

\paragraph{}
\begin{comment}	
	Before we had fiat currencies, the soundness of money was determined by
	the natural properties of the stuff which we used as money. The amount
	of gold on earth is limited by the laws of physics. Gold is rare because
	supernovae and neutron star collisions are rare. The \enquote{flow} of gold is
	limited because extracting it is quite an effort. Being a heavy element
	it is mostly buried deep underground.
\end{comment}
법정화폐 이전 시대에는 화폐로 사용되는 물질의 자연적 속성이 화폐 건전성을 결정했다. 
물리 법칙에 의해 지구에 매장된 금은 한정되어 있다.
금이 새로 생기려면 초신성과 중성자성이 충돌해야 하는데, 이러한 일은 거의 발생하지 않는다.
금을 추출하는 것은 상당히 어렵기 때문에 공급이 제한적이다.
무거운 원소들은 대체로 지하 깊은 곳에 묻혀 있기 때문이다.

\paragraph{}
\begin{comment}	
	The abolishment of the gold standard gave way to a new reality: adding new money
	requires just a drop of ink. In our modern world adding a couple of zeros to the
	balance of a bank account requires even less effort: flipping a few bits in a
	bank computer is enough.
\end{comment}
금본위제 폐지는 우리를 새로운 국면으로 이끌었다.
이제 화폐 발행을 위해 잉크 한 방울만 있으면 된다.
현대 사회에서는 훨씬 쉽다. 은행 계좌 잔액에 0 몇 개를 더 추가하기만 하면 된다.
은행 컴퓨터에서 몇 개의 비트만 뒤집으면 되는 것이다.

\paragraph{}
\begin{comment}	
	The principle outlined above can be expressed more generally as the
	ratio of \enquote{stock} to \enquote{flow}. Simply put, the \textit{stock} is how much of
	something is currently there. For our purposes, the stock is a measure
	of the current money supply. The \textit{flow} is how much there is produced
	over a period of time (e.g. per year). The key to understanding sound
	money is in understanding this stock-to-flow ratio.
\end{comment}
위에 설명한 원리는 \enquote{저량(stock)} 대 \enquote{유량(flow)}을 사용하여 좀 더 일반적으로 표현할 수 있다.
간단히 말해 \enquote{저량}은 현재 존재하는 양을 말한다. 여기에서는 현재 통화 공급량을 의미한다. 
\enquote{유량}은 일정 기간 동안의 생산량을 나타낸다.
건전 화폐 이해의 핵심은 이 저량 대 유량 비율을 이해하는 것이다.

\paragraph{}
\begin{comment}	
	Calculating the stock-to-flow ratio for fiat currency is difficult, because how
	much money there is depends on how you look at it.~\cite{wiki:money-supply} You
	could count only banknotes and coins (M0), add traveler checks and check
	deposits (M1), add saving accounts and mutual funds and some other things (M2),
	and even add certificates of deposit to all of that (M3). Further, how all of
	this is defined and measured varies from country to country and since the US
	Federal Reserve stopped publishing \cite{web:fed-m3} numbers for M3, we will
	have to make do with the M2 monetary supply. I would love to verify these
	numbers, but I guess we have to trust the fed for now.
\end{comment}
정부화폐의 저량 대 유량 비율을 계산하는 것은 어렵다. 
어떤 돈을 기준으로 하느냐에 따라 달라질 수 있기 때문이다\cite{wiki:money-supply}.
지폐와 동전(M0)만 계산할 수도, 여행자 수표와 예금(M1)을 추가할 수도, 
저축 계좌와 뮤추얼 펀드 등(M2)을 추가할 수도, 심지어 예금 증서(M3)를 추가할 수도 있다.
게다가 이것들의 정의와 측정하는 방법이 국가마다 다르다.
미국 연방준비 은행이 M3 수치 공개를 중단했기 때문에 우리는 M2로 확인할 수 밖에 없다.\cite{web:fed-m3}
이 수치를 직접 검증하고 싶지만 지금은 연준을 믿을 수 밖에 없다.

\paragraph{}
\begin{comment}	
	Gold, one of the rarest metals on earth, has the highest stock-to-flow
	ratio. According to the US Geological Survey, a little more than 190,000
	tons have been mined. In the last few years, around 3100 tons of gold
	have been mined per year.~\cite{mineral-commodity-summaries}
\end{comment}
지구상에서 가장 희귀한 금속인 금은 저량 대 유량의 비율이 가장 높다.
미국 지질조사국에 따르면 금은 현재까지 약 19만 톤 조금 넘게 채굴되었다.
그 중 지난 몇 년 동안 매년 약 3,100톤이 채굴되었다.~\cite{mineral-commodity-summaries}

\paragraph{}
\begin{comment}	
	Using these numbers, we can easily calculate the stock-to-flow ratio for
	gold (see Figure~\ref{fig:stock-to-flow-gold}).
\end{comment}
이 숫자를 사용하여 우리는 금의 저량 대 유량 비율을 쉽게 계산할 수 있다. (그림 ~\ref{fig:stock-to-flow-gold})

\begin{figure}
	\centering
	\begin{equation}
		\frac{190,000 t}{3,100 t} = ~ 61
	\end{equation}
	\caption{금의 저량 대 유량 비율}
	\label{fig:stock-to-flow-gold}
\end{figure}

\paragraph{}
\begin{comment}	
	Nothing has a higher stock-to-flow ratio than gold. This is why gold, up to now,
	was the hardest, soundest money in existence. It is often said that all the gold
	mined so far would fit in two olympic-sized swimming pools. According to my
	calculations\footnote{\url{https://bit.ly/gold-pools}}, we would need four. So
	maybe this needs updating, or Olympic-sized swimming pools got smaller.
\end{comment}
금보다 저량 대 유량 비율이 더 높은 것은 없다. 
이것이 금이 지금까지, 그리고 현존하는 가장 견고하고 건전한 화폐인 이유이다. 
지금까지 채굴된 금의 양이 올림픽 규격 수영장 두 개 정도 될 것이라고들 말한다.
하지만 내 계산에 따르면 수영장 네 개가 필요하다.\footnote{\url{https://bit.ly/gold-pools}}
수치가 잘못되었거나 올림픽 규모의 수영장 크기가 작아지지 않았다면 말이다.

\paragraph{}
\begin{comment}	
	Enter Bitcoin. As you probably know, bitcoin mining was all the rage in
	the last couple of years. This is because we are still in the early
	phases of what is called the \textit{reward era}, where mining nodes are
	rewarded with \textit{a lot} of bitcoin for their computational effort. We are
	currently in reward era number 3, which began in 2016 and will end in
	early 2020, probably in May. While the bitcoin supply is predetermined,
	the inner workings of Bitcoin only allow for approximate dates.
	Nevertheless, we can predict with certainty how high Bitcoin's
	stock-to-flow ratio will be. Spoiler alert: it will be high.
\end{comment}
이제 비트코인의 저량 대 유량 비율을 보자. 
잘 알겠지만, 비트코인 채굴은 지난 몇 년간 대유행이었다.
아직 채굴 노드가 계산 노력에 비해 많은 보상을 받는 초기 보상 단계에 있기 때문이다.
우리는 현재 2016년에 시작되어 2020년 5월에 끝나는 세 번째 반감기를 지나고 있다.
반감기에 따른 비트코인의 공급량은 미리 정해져 있지만 날짜는 정확하지 않을 수 있다.
그럼에도 불구하고 비트코인의 저량 대 유량의 비율을 계산할 수 있다.
미리 스포일러를 하자면 높다.


\begin{comment}	
	How high? Well, it turns out that Bitcoin will get infinitely hard (see
	Figure~\ref{fig:stock-to-flow-white-cropped}).
\end{comment}
높지 않은가? 비트코인은 무한대가 될 때까지 높아질 것이다.(그림 ~\ref{fig:stock-to-flow-white-cropped})

\begin{figure}
	\includegraphics{assets/images/stock-to-flow-white-cropped.png}
	\caption{달러, 금, 비트코인의 저량과 유량}
	\label{fig:stock-to-flow-white-cropped}
\end{figure}

\paragraph{}
\begin{comment}	
	Due to an exponential decrease of the mining reward, the flow of new
	bitcoin will diminish resulting in a sky-rocketing stock-to-flow ratio.
	It will catch up to gold in 2020, only to surpass it four years later by
	doubling its soundness again. Such a doubling will occur 32 times in
	total. Thanks to the power of exponentials, the number of bitcoin mined
	per year will drop below 100 bitcoin in 50 years and below 1 bitcoin in
	75 years. The global faucet which is the block reward will dry up
	somewhere around the year 2140, effectively stopping the production of
	bitcoin. This is a long game. If you are reading this, you are still
	early.
\end{comment}
채굴 보상이 기하급수적으로 줄어들기 때문에 비트코인의 저량 대 유량의 비율은 치솟는다.
2020년에는 금을 따라잡고, 그 4년 후인 2024년에는 다시 두 배 만큼 금을 능가하게 된다.
이러한 반감기는 총 32번 발생한다.
지수의 힘 덕분에 연간 비트코인 채굴량은 50년 후에는 100비트코인 미만으로, 75년 후에는 1비트코인 미만으로 떨어지게 된다.
블록 보상은 2140년 쯤 고갈되어 비트코인 생산은 사실상 중단될 것이다. 
이것은 매우 긴 게임이며, 당신이 이 글을 읽고 있는 현재는 아직 초기 단계에 해당한다.

\begin{figure}
	\includegraphics{assets/images/soundness-over-time.png}
	\caption{금과 비교한 비트코인의 저량 대 유량의 비율}
	\label{fig:soundness-over-time}
\end{figure}

\paragraph{}
\begin{comment}	
	As bitcoin approaches infinite stock to flow ratio it will be the
	soundest money in existence. Infinite soundness is hard to beat.
\end{comment}
비트코인의 건전성 즉 저량 대 유량의 비율은 무한대에 수렴하기 때문에
비트코인은 현존하는 가장 건전한 화폐가 될 것이다.
무한한 건전성을 누구도 이길 수 없다.

\paragraph{}
\begin{comment}	
	Viewed through the lens of economics, Bitcoin's \textit{difficulty adjustment}
	is probably its most important component. How hard it is to mine bitcoin depends
	on how quickly new bitcoins are mined.\footnote{It actually depends on how
		quickly valid blocks are found, but for our purposes, this is the same thing as
		\enquote{mining bitcoins} and will be so for the next 120 years.} It is the dynamic
	adjustment of the network's mining difficulty which enables us to predict its
	future supply.
\end{comment}
경제학 관점에서 볼 때, 비트코인의 난이도 조정이 아마 가장 중요한 요소일 것이다.
비트코인 채굴이 얼마나 어려운지는 신규 비트코인을 얼마나 빨리 채굴할 수 있느냐에 달려있다.\footnote{실제로는 얼마나 유효한 블록을 빨리 찾느냐에 달려있다. 
	하지만 행위의 목적을 표현하기 위해 \enquote{비트코인을 채굴한다}로 표현한다. 
	이 채굴은 향후 120년 동안 계속된다.}
네트워크의 채굴 난이도는 동적으로 결정되기 때문에 이를 통해 미래의 공급량을 예측할 수 있다.

\paragraph{}
\begin{comment}	
	The simplicity of the difficulty adjustment algorithm might distract
	from its profundity, but the difficulty adjustment truly is a revolution
	of Einsteinian proportions. It ensures that, no matter how much or how
	little effort is spent on mining, Bitcoin's controlled supply won't be
	disrupted. As opposed to every other resource, no matter how much
	energy someone will put into mining bitcoin, the total reward will not
	increase.
\end{comment}
난이도 조정 알고리즘의 단순함이 비트코인의 심오함을 흐트러뜨릴 수 있다. 
하지만, 비트코인 난이도 조정은 아인슈타인에 버금가는 혁명과도 유사하다.
이 방법은 채굴에 아무리 많은 노력을 들이더라도 비트코인의 통제된 공급이 중단되지 않도록 보장한다.
다른 모든 자원과는 달리 누군가 비트코인 채굴에 아무리 많은 에너지를 투입하여도 총 보상은 증가하지 않는다.

\paragraph{}
\begin{comment}	
	Just like $E=mc^2$ dictates the universal speed limit in our universe,
	Bitcoin's difficulty adjustment dictates the \textbf{universal money limit}
	in Bitcoin.
\end{comment}
$E=mc^2$로 우주의 보편적 속도의 한도가 결정되듯, 
비트코인의 난이도 조정으로 비트코인 전체 수량의 보편적 한도가 결정된다.

\paragraph{}
\begin{comment}	
	If it weren't for this difficulty adjustment, all bitcoins would have been mined
	already. If it weren't for this difficulty adjustment, Bitcoin probably wouldn't
	have survived in its infancy. It is what secures the network in its reward era.
	It is what ensures a steady and fair distribution\footnote{Dan Held,
		\textit{Bitcoin's Distribution was Fair}~\cite{distribution-was-fair}} of new
	bitcoin. It is the thermostat which regulates Bitcoin's monetary policy.
\end{comment}
난이도 조정이 없었다면 비트코인은 이미 고갈되었을 것이고, 초기 단계에서 살아남기 어려웠을 것이다.
난이도 조정 덕분에 채굴 보상이 존재하는 동안 네트워크가 안전하게 보호되며,
신규 발행된 비트코인이 안정적이고 공정하게 분배된다.\footnote{Dan Held,
	\textit{Bitcoin’s Distribution was Fair}~\cite{distribution-was-fair}}
난이도 조정은 비트코인 통화 정책의 규제 장치이다.

\begin{comment}	
	Einstein showed us something novel: no matter how hard you push an
	object, at a certain point you won't be able to get more speed out of
	it. Satoshi also showed us something novel: no matter how hard you dig
	for this digital gold, at a certain point you won't be able to get more
	bitcoin out of it. For the first time in human history, we have a
	monetary good which, no matter how hard you try, you won't be able to
	produce more of.
\end{comment}
아인슈타인은 물체를 아무리 세게 밀어도 특정 지점에 이르면 더 빠른 속도를 낼 수 없다는 새로운 사실을 보여주었다. 
사토시 또한 우리에게 새로운 것을 보여주었다. 
이 디지털 금을 아무리 열심히 채굴하여도 특정 시점이 되면 더는 새로운 비트코인을 얻을 수 없는 것이다. 
인류 역사상 처음으로 우리는 아무리 노력해도 더 이상 생산할 수 없는 금전적 재화를 갖게 되었다.

%\paragraph{Bitcoin taught me that sound money is essential.}
\paragraph{비트코인은 나에게 건전 화폐가 꼭 필요하다는 것을 가르쳐주었다.}

% ---
%
% #### Through the Looking-Glass
%
% - [Bitcoin's Energy Consumption: A Shift in Perspective][much energy]
%
% #### Down the Rabbit Hole
%
% - [The Ethics of Money Production][Jörg Guido Hülsmann] by Jörg Guido Hülsmann
% - [Mineral Commodity Summaries 2019][last few years] by the United States Geological Survey
% - [Bitcoin’s Distribution was Fair][fair distribution] by Dan Held
% - [Bitcoin's Controlled Supply][algorithmically controlled] on the Bitcoin Wiki
% - [Money Supply][how much money there is], [Speed of Light][universal speed limit] on Wikipedia
%
% <!-- Internal -->
% [much energy]: 
%
% [Fr. Bernard W. Dempsey, S.J.]: https://www.jstor.org/stable/29769582
% [Jörg Guido Hülsmann]: https://mises.org/sites/default/files/The%20Ethics%20of%20Money%20Production_2.pdf
% [stopped publishing]: https://www.federalreserve.gov/Releases/h6/discm3.htm
% [last few years]: https://minerals.usgs.gov/minerals/pubs/mcs/2018/mcs2018.pdf
% [my calculations]: https://www.wolframalpha.com/input/?i=volume+of+190000+metric+tons+gold+%2F+olympic+swimming+pool+volume
% [fair distribution]: https://blog.picks.co/bitcoins-distribution-was-fair-e2ef7bbbc892
%
% <!-- Bitcoin Wiki -->
% [algorithmically controlled]: https://en.bitcoin.it/wiki/Controlled_supply
%
% <!-- Wikipedia -->
% [how much money there is]: https://en.wikipedia.org/wiki/Money_supply
% [universal speed limit]: https://en.wikipedia.org/wiki/Speed_of_light#Upper_limit_on_speeds
% [alice]: https://en.wikipedia.org/wiki/Alice%27s_Adventures_in_Wonderland
% [carroll]: https://en.wikipedia.org/wiki/Lewis_Carroll
