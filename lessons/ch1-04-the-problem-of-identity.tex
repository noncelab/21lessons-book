\chapter{정체성의 문제}
\label{les:4}

\begin{chapquote}{루이스 캐롤, \textit{이상한 나라의 앨리스}}
	\enquote{넌 누구니?} 애벌레가 물었다.
\end{chapquote}

%Nic Carter, in an homage to Thomas Nagel's treatment of the same
%question in regards to a bat, wrote an excellent piece which discusses
%the following question: What is it like to be a bitcoin? He
%brilliantly shows that open, public blockchains in general, and Bitcoin
%in particular, suffer from the same conundrum as the ship of Theseus\footnote{In
	%the metaphysics of identity, the ship of Theseus is a thought experiment that
	%raises the question of whether an object that has had all of its components
	%replaced remains fundamentally the same object.~\cite{wiki:theseus}}: which
%Bitcoin is the real Bitcoin?
닉 카터는 그의 홈페이지에서 토마스 네이글
의 \enquote{박쥐가 된다는 것은 무엇인가?}를 오마주하여 
\enquote{비트코인이 된다는 것은 무엇인가?}라는 질문에 대한 글을 남겼다. 
그는 개방형 블록체인, 특히 비트코인이 테세우스의 배\footnote{테세우스의 배의 낡은 판자를 계속 교체하다 보면 어느 시점에는 원래 배 조각이 하나도 남지 않을 것이다. 이것을 테세우스의 배라고 할 수 있을까?~\cite{wiki:theseus}}
와 같은 난제에 해당된다고 서술한다.
과연 어떤 비트코인이 진짜 비트코인일까?

\begin{quotation}\begin{samepage}
		\enquote{비트코인의 컴포넌트가 얼마나 변화했는지 생각해보라. 비트코인의 전체 코드는
			원래의 버전과 거의 유사성이 없을 정도로 수정되었다.[...] 누가 무엇을 소유하는지에 대한
			기록, 즉 거래장부만이 유일하게 영구적이다.[...] 리더가 없다는 것의 의미는
			거래기록을 합법적으로 결정하는 쉬운 방법을 포기하는 것을 의미한다.}
		\begin{flushright} -- 닉 카터\footnote{Nic Carter, \textit{What is it like to be a bitcoin?} \cite{bitcoin-identity}}
\end{flushright}\end{samepage}\end{quotation}

%Consider just how little persistence Bitcoin's components have. The
%entire codebase has been reworked, altered, and expanded such that it
%barely resembles its original version. [...] The registry of who
%owns what, the ledger itself, is virtually the only persistent trait
%of the network [...]
%To be considered truly leaderless, you must surrender the easy
%solution of having an entity that can designate one chain as the
%legitimate one.}
%\begin{flushright} -- Nic Carter\footnote{Nic Carter, \textit{What is it like to be a bitcoin?} \cite{bitcoin-identity}}
%\end{flushright}\end{samepage}\end{quotation}

%It seems like the advancement of technology keeps forcing us to take
%these philosophical questions seriously. Sooner or later, self-driving
%cars will be faced with real-world versions of the trolley problem,
%forcing them to make ethical decisions about whose lives do matter and
%whose do not.
이러한 철학적 질문은 기술적 발전으로 인해 더 진지해진다. 
조만간 자율주행차는 누구의 생명이 더 중요하고 누구의 생명이 덜 중요한지에 대한
윤리적 결정을 해야 할 것이다.

%Cryptocurrencies, especially since the first contentious hard-fork,
%force us to think about and agree upon the metaphysics of identity.
%Interestingly, the two biggest examples we have so far have lead to two
%different answers. On August 1, 2017, Bitcoin split into two camps. The
%market decided that the unaltered chain is the original Bitcoin. One
%year earlier, on October 25, 2016, Ethereum split into two camps. The
%market decided that the \textit{altered} chain is the original Ethereum.
암호화폐에서 하드포크에 대한 논란은 정체성에 대한 형이상학적인 동의를 강요하고 있다. 
흥미롭게도 비트코인과 이더리움은 각각 다른 대답을 내놨다. 2017년 8월 1일 비트코인은 두 진영으로 
나뉘었다. 시장은 하드포크되지 않은 체인을 원래의 비트코인으로 받아들였다. 
2016년 10월 25일 이더리움도 두 진영으로 나뉘었다.
시장은 하드포크된 체인을 원래의 이더리움으로 받아들였다.

%If properly decentralized, the questions posed by the \textit{Ship of Theseus}
%will have to be answered in perpetuity for as long as these networks of
%value-transfer exist.
만약 정당하게 탈중앙화 되어있다면, 네트워크의 가치가 지속되는 한 테세우스의 배에 관한 질문의 
답변은 한결같아야 한다.

\paragraph{비트코인은 탈중앙화는 정체성과 모순된다는 것을 가르쳐주었다.}

% ---
%
% #### Down the Rabbit Hole
%
% - [What Is It Like to be a Bat?][in regards to a bat] by Thomas Nagel
% - [What is it like to be a bitcoin?] by Nic Carter
% - [Ship of Theseus], [trolley problem] on Wikipedia
%
% [in regards to a bat]: https://en.wikipedia.org/wiki/What_Is_it_Like_to_Be_a_Bat%3F
% [What is it like to be a bitcoin?]: https://medium.com/s/story/what-is-it-like-to-be-a-bitcoin-56109f3e6753
% [Ship of Theseus]: https://en.wikipedia.org/wiki/Ship_of_Theseus
% [trolley problem]: https://en.wikipedia.org/wiki/Trolley_problem
%
% <!-- Wikipedia -->
% [alice]: https://en.wikipedia.org/wiki/Alice%27s_Adventures_in_Wonderland
% [carroll]: https://en.wikipedia.org/wiki/Lewis_Carroll
