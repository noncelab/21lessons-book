\chapter{정체성의 문제}
\label{les:4}

\begin{chapquote}{루이스 캐롤, \textit{이상한 나라의 앨리스}}
	\enquote{넌 누구니?} 애벌레가 물었다.
\end{chapquote}

\paragraph{}
%Nic Carter, in an homage to Thomas Nagel's treatment of the same
%question in regards to a bat, wrote an excellent piece which discusses
%the following question: What is it like to be a bitcoin? He
%brilliantly shows that open, public blockchains in general, and Bitcoin
%in particular, suffer from the same conundrum as the ship of Theseus\footnote{In
	%the metaphysics of identity, the ship of Theseus is a thought experiment that
	%raises the question of whether an object that has had all of its components
	%replaced remains fundamentally the same object.~\cite{wiki:theseus}}: which
%Bitcoin is the real Bitcoin?
닉 카터는 토마스 네이글의 \enquote{박쥐가 된다는 것은 무엇인가?(What is it like to be bat?)}를 오마주하여 
\enquote{비트코인이 된다는 것은 무엇인가?(What is it like to be a bitcoin?)}라는 질문에 대한 훌륭한 글을 남겼다. 
그는 일반적으로 개방형 블록체인, 특히 비트코인은, 
테세우스의 배\footnote{정체성의 형이상학에서, 테세우스의 배의 낡은 판자를 계속 교체하다 보면 어느 시점에는 원래 배 조각이 하나도 남지 않을 것인데 이것을 테세우스의 배라고 할 수 있을까?에 대한
의문을 제기하는 사고 실험이다.~\cite{wiki:theseus}}와 같은 난제에 해당함을 탁월하게 서술했다.
과연 어떤 비트코인이 진짜 비트코인일까?

\begin{quotation}\begin{samepage}
		\enquote{비트코인 컴포넌트가 얼마나 변했는지 생각해보라. 
			비트코인의 전체 코드베이스는 재작업, 변경, 확장되어
			첫 버전과 유사성이 거의 없을 정도로 수정되었다.[\ldots]누가 무엇을 소유하는지에 대한
			기록, 즉 원장 자체만이 사실상 이 네트워크에서 유일하게 유지되고 있다.[\ldots]
			진정한 의미에서 리더가 없는 것으로 간주되려면 
			특정 체인을 적법한 체인으로 지정할 수 있는 주체가 있다는 쉬운 해결책을 포기해야만 한다.}
		\begin{flushright} -- 닉 카터\footnote{Nic Carter, \textit{What is it like to be a bitcoin?} \cite{bitcoin-identity}}
\end{flushright}\end{samepage}\end{quotation}

%Consider just how little persistence Bitcoin's components have. The
%entire codebase has been reworked, altered, and expanded such that it
%barely resembles its original version. [...] The registry of who
%owns what, the ledger itself, is virtually the only persistent trait
%of the network [...]
%To be considered truly leaderless, you must surrender the easy
%solution of having an entity that can designate one chain as the
%legitimate one.}
%\begin{flushright} -- Nic Carter\footnote{Nic Carter, \textit{What is it like to be a bitcoin?} \cite{bitcoin-identity}}
%\end{flushright}\end{samepage}\end{quotation}


\paragraph{}
%It seems like the advancement of technology keeps forcing us to take
%these philosophical questions seriously. Sooner or later, self-driving
%cars will be faced with real-world versions of the trolley problem,
%forcing them to make ethical decisions about whose lives do matter and
%whose do not.
이러한 철학적 질문은 기술의 발전으로 인해 더 진지하게 받아들여지게 된 것 같다.
조만간 자율주행차는 트롤리 문제에 직면하게 될 것이며, 누구의 생명이 더 중요하고 덜 중요한지 윤리적 결정을 내려야 할 것이다.
\footnote{역주: 트롤리 문제 혹은 트롤리 딜레마는 우리에게 '갈림길을 향해 달리는 제동장치가 고장난 수레 실험'으로 유명하다. 
이 실험은 다수를 위해 소수를 희생하는 것이(혹은 그 반대가) 과연 윤리적으로 올바른 선택인가 질문한다.}

\paragraph{}
%Cryptocurrencies, especially since the first contentious hard-fork,
%force us to think about and agree upon the metaphysics of identity.
%Interestingly, the two biggest examples we have so far have lead to two
%different answers. On August 1, 2017, Bitcoin split into two camps. The
%market decided that the unaltered chain is the original Bitcoin. One
%year earlier, on October 25, 2016, Ethereum split into two camps. The
%market decided that the \textit{altered} chain is the original Ethereum.
특히나 논쟁의 여지가 많은 첫 번째 하드포크 이후, 
암호화폐는 우리에게 정체성의 형이상학에 대해 생각해보고 동의하도록 강요하고 있다. 
흥미롭게도 비트코인과 이더리움은 각각 다른 대답을 내놨다. 
2017년 8월 1일, 비트코인은 두 진영으로 나뉘었다. 
시장은 하드포크되지 않은 체인을 원래의 비트코인으로 받아들였다. 
그보다 1년 정도 전인 2016년 10월 25일, 이더리움도 두 진영으로 나뉘었다.
시장은 하드포크된 체인을 원래의 이더리움이라고 결정했다.

\paragraph{}
%If properly decentralized, the questions posed by the \textit{Ship of Theseus}
%will have to be answered in perpetuity for as long as these networks of
%value-transfer exist.
제대로 탈중앙화되어 있다면, 네트워크의 가치가 지속되는 한 \textit{테세우스의 배} 질문에 대한 답변은 한결같아야 한다.

\paragraph{비트코인은 탈중앙화가 정체성과 모순된다는 것을 가르쳐주었다.}

% ---
%
% #### Down the Rabbit Hole
%
% - [What Is It Like to be a Bat?][in regards to a bat] by Thomas Nagel
% - [What is it like to be a bitcoin?] by Nic Carter
% - [Ship of Theseus], [trolley problem] on Wikipedia
%
% [in regards to a bat]: https://en.wikipedia.org/wiki/What_Is_it_Like_to_Be_a_Bat%3F
% [What is it like to be a bitcoin?]: https://medium.com/s/story/what-is-it-like-to-be-a-bitcoin-56109f3e6753
% [Ship of Theseus]: https://en.wikipedia.org/wiki/Ship_of_Theseus
% [trolley problem]: https://en.wikipedia.org/wiki/Trolley_problem
%
% <!-- Wikipedia -->
% [alice]: https://en.wikipedia.org/wiki/Alice%27s_Adventures_in_Wonderland
% [carroll]: https://en.wikipedia.org/wiki/Lewis_Carroll
