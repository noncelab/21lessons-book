\chapter{프라이버시는 죽지 않았다.}
\label{les:19}

\begin{chapquote}
	%{Lewis Carroll, \textit{Alice in Wonderland}}
	%The players all played at once without waiting for turns, and quarrelled all
	%the while at the tops of their voices, and in a very few minutes the Queen was
	%in a furious passion, and went stamping about and shouting \enquote{off with his
		%head!} of \enquote{off with her head!} about once in a minute.
	{루이스 캐롤, \textit{이상한 나라의 앨리스}}
	선수들은 차례를 기다리지 않고 한꺼번에 플레이했고, 내내 큰 목소리로 다투었다.
	여왕은 발을 구르며 1분에 한 번씩 맹렬하게 말했다.
	\enquote{그의 목을 쳐라!}
	\enquote{그녀의 목을 쳐라!}
\end{chapquote}

\begin{comment}
	If pundits are to believed, privacy has been dead since the
	80ies\footnote{\url{https://bit.ly/privacy-is-dead}}. The pseudonymous invention
	of Bitcoin and other events in recent history show that this is not the case.
	Privacy is alive, even though it is by no means easy to escape the surveillance
	state.
\end{comment}
전문가들의 말에 의하면 80년대 이후로 프라이버시는 죽었다.\footnote{\url{https://bit.ly/privacy-is-dead}}
하지만, 비트코인의 발명과 최근의 여러 사건들은 이것이 사실이 아님을 보여준다.
감시를 벗어나는 것이 결코 쉽지 않지만, 프라이버시는 살아있다.

\begin{comment}
	Satoshi went through great lengths to cover up his tracks and conceal
	his identity. Ten years later, it is still unknown if Satoshi Nakamoto
	was a single person, a group of people, male, female, or a
	time-traveling AI which bootstrapped itself to take over the world.
	Conspiracy theories aside, Satoshi chose to identify himself to be a
	Japanese male, which is why I don't assume but respect his chosen gender
	and refer to him as \textit{he}.
\end{comment}
사토시는 자신의 흔적을 지우고 신분을 감추기 위해 부단히 노력했다.
십년이 지난 지금도 사토시 나카모토가 한 사람인지 집단인지, 
남성인지 여성인지, 아니면 세상을 정복하기 위해 미래에서 온 인공지능인지 알 수 없다.
음모론은 차치하고, 사토시는 자신을 일본 남성으로 밝히길 선택했기 때문에
그의 선택을 존중하여 나는 그를 'he'라고 지칭한다.
\begin{figure}
	\includegraphics{assets/images/nope.png}
	%\caption{I am not Dorian Nakamoto.}
	\caption{나는 도리안 나카모토가 아닙니다.}
	\label{fig:nope}
\end{figure}

\begin{comment}
	Whatever his real identity might be, Satoshi was successful in hiding
	it. He set an encouraging example for everyone who wishes to remain
	anonymous: it is possible to have privacy online.
\end{comment}
그의 진짜 정체가 무엇이든 간에 사토시는 성공적으로 정체를 숨겼다.
그는 익명을 추구하는 모든 이들에게 힘을 북돋아 줄 만한 모범을 보였다.
온라인에서도 프라이버시 보호가 가능함을 말이다.

\begin{quotation}\begin{samepage}
		%\enquote{Encryption works. Properly implemented strong crypto systems are one
			%of the few things that you can rely on.}
		\enquote{암호학은 잘 작동합니다. 제대로 구현된 강력한 암호 시스템은 우리가 신뢰할 수 있는 몇 안 되는 것 중 하나입니다.}
		\begin{flushright} -- 에드워드 스노든\footnote{Edward Snowden, answers to reader questions\cite{snowden}}
\end{flushright}\end{samepage}\end{quotation}

\begin{comment}
	Satoshi wasn't the first pseudonymous or anonymous inventor, and he won't be the
	last. Some have directly imitated this pseudonymous publication style, like Tom
	Elvis Yedusor of MimbleWimble~\cite{mimblewimble-origin} fame, while others have
	published advanced mathematical proofs while remaining completely
	anonymous~\cite{4chan-math}.
\end{comment}
사토시는 최초의 익명 발명가도 아니고, 마지막 익명 발명가도 아닐 것이다.
밈블윔블(MimbleWimble)~\cite{mimblewimble-origin}로 유명한 톰 엘비스 예두서(Tom Elvis Yedusor)처럼 사토시의 익명 출판 스타일을 모방한 사람도 있고,
완전히 익명으로 고급 수학 증명을 출판한 사람도 있다.~\cite{4chan-math}


\begin{comment}
	It is a strange new world we are living in. A world where identity is
	optional, contributions are accepted based on merit, and people can
	collaborate and transact freely. It will take some adjustment to get
	comfortable with these new paradigms, but I strongly believe that all of
	this has the potential to change the world for the better.
\end{comment}
우리는 낯선 신세계를 살고 있다. 
이 세계에서 신원은 선택 사항이고, 능력에 따라 기여가 인정되며, 사람들은 자유롭게 협력하고 거래할 수 있다.
새로운 패러다임에 익숙해지려면 약간의 적응이 필요하겠지만, 이 모든 것이 세상을 더 나은 방향으로 변화시킬 잠재력이 있다고 굳게 믿고 있다.

\begin{comment}
	We should all remember that privacy is a fundamental human right\footnote{Universal Declaration of Human Rights, \textit{Article 12}.~\cite{article12}}. And as long
	as people exercise and defend these rights the battle for privacy is far from
	over.
\end{comment}
우리 모두는 프라이버시가 기본적인 인권이라는 사실을 기억해야 한다.\cite{article12}
그리고 사람들이 이러한 권리를 행사하고 보호하는 한 프라이버시를 지키기 위한 투쟁은 끝나지 않을 것이다.

%\paragraph{Bitcoin taught me that privacy is not dead.}
\paragraph{비트코인은 프라이버시가 죽지 않았다는 것을 가르쳐주었다.}

% ---
%
% #### Down the Rabbit Hole
%
% - [Universal Declaration of Human Rights][fundamental human right] by the United Nations
% - [A lower bound on the length of the shortest superpattern][anonymous] by Anonymous 4chan Poster, Robin Houston, Jay Pantone, and Vince Vatter
%
% [since the 80ies]: https://books.google.com/ngrams/graph?content=privacy+is+dead&year_start=1970&year_end=2019&corpus=15&smoothing=3&share=&direct_url=t1%3B%2Cprivacy%20is%20dead%3B%2Cc0
% [time-traveling AI]: https://blockchain24-7.com/is-crypto-creator-a-time-travelling-ai/
% ["I am not Dorian Nakamoto."]: http://p2pfoundation.ning.com/forum/topics/bitcoin-open-source?commentId=2003008%3AComment%3A52186
% [MimbleWimble]: https://github.com/mimblewimble/docs/wiki/MimbleWimble-Origin
% [anonymous]: https://oeis.org/A180632/a180632.pdf
% [fundamental human right]: http://www.un.org/en/universal-declaration-human-rights/
%
% <!-- Wikipedia -->
% [alice]: https://en.wikipedia.org/wiki/Alice%27s_Adventures_in_Wonderland
% [carroll]: https://en.wikipedia.org/wiki/Lewis_Carroll
