\chapter{ \enquote{신뢰하지 말고 검증하라}에 대한 고찰}
\label{les:16}

\begin{comment}
	\begin{chapquote}{Lewis Carroll, \textit{Alice in Wonderland}}
		\enquote{Now for the evidence,} said the King, \enquote{and then the sentence.}
	\end{chapquote}
\end{comment}
\begin{chapquote}{루이스 캐롤, \textit{이상한 나라의 앨리스}}
	\enquote{이제 입증되었으니,} 왕은 말했다, \enquote{선고를 내리겠다.}
\end{chapquote}

\begin{comment}
	Bitcoin aims to replace, or at least provide an alternative to,
	conventional currency. Conventional currency is bound to a centralized
	authority, no matter if we are talking about legal tender like the US
	dollar or modern monopoly money like Fortnite's V-Bucks. In both
	examples, you are bound to trust the central authority to issue, manage
	and circulate your money. Bitcoin unties this bound, and the main issue
	Bitcoin solves is the issue of \textit{trust}.
\end{comment}
비트코인은 기존 통화를 대체하거나 최소한 대안을 제공하는 것을 목표로 한다.
미국 달러와 같은 법정 화폐나 포트나이트의 V-Bucks와 같은 현대 독점 화폐는 중앙 집중형 권한에 매여있다.
우리가 이런 화폐를 쓰려면 발행, 관리, 유통을 하는 중앙 기관을 신뢰해야 한다. 
비트코인은 이러한 한계를 해결해주며, 비트코인이 해결하려는 주요 문제는 신뢰의 문제이다. 

%\begin{quotation}\begin{samepage}
%\enquote{The root problem with conventional currency is all the trust that's
	%required to make it work. [...] What is needed is an electronic
	%payment system based on cryptographic proof instead of trust}
%\begin{flushright} -- Satoshi Nakamoto\footnote{Satoshi Nakamoto, official Bitcoin announcement~\cite{bitcoin-announcement} and whitepaper~\cite{whitepaper}}
%\end{flushright}\end{samepage}\end{quotation}

\begin{quotation}\begin{samepage}
		\enquote{기존 화폐의 문제는 화폐가 동작하는 데 필요한 모든 종류의 신뢰이다. [...] 
			암호학 기반의 전자 결제 시스템에서 필요한 것은 신뢰가 아닌 증명이다.}
		\begin{flushright} -- 사토시 나카모토\footnote{Satoshi Nakamoto, official Bitcoin announcement~\cite{bitcoin-announcement} and whitepaper~\cite{whitepaper}}
\end{flushright}\end{samepage}\end{quotation}

\begin{comment}
	Bitcoin solves the problem of trust by being completely decentralized,
	with no central server or trusted parties. Not even trusted \textit{third}
	parties, but trusted parties, period. When there is no central
	authority, there simply \textit{is} no-one to trust. Complete decentralization
	is the innovation. It is the root of Bitcoin's resilience, the reason
	why it is still alive. Decentralization is also why we have mining,
	nodes, hardware wallets, and yes, the blockchain. The only thing you
	have to \enquote{trust} is that our understanding of mathematics and physics
	isn't totally off and that the majority of miners act honestly (which
	they are incentivized to do).
\end{comment}
비트코인은 중앙 서버나 신뢰 당사자 없이 완전히 탈중앙화되어 신뢰 문제를 해결한다.
신뢰할 수 있는 제3자는 물론 신뢰 당사자도 필요 없다.
중앙 권한이 없다면, 믿어야 할 사람도 없다.
완전한 탈중앙화는 혁신이다. 
그것이 비트코인 회복탄력성(resilience)의 핵심이며, 지금까지 비트코인이 생존한 이유이다.
이러한 탈중앙화 덕분에 우리는 채굴이 가능하고, 노드, 하드웨어 지갑, 블록체인을 직접 보유할 수 있다. 
우리가 유일하게 \enquote{신뢰}해야 할 것은 수학, 물리학의 법칙에 대한 우리의 이해가 잘못된 것이 아니며
대다수의 채굴자들이 인센티브를 얻기 위해 정직하게 행동한다는 것이다.(그들은 정직하게 행동하도록 장려된다.)

\begin{comment}
	While the regular world operates under the assumption of \textit{\enquote{trust,
			but verify,}} Bitcoin operates under the assumption of \textit{\enquote{don't
			trust, verify.}} Satoshi made the importance of removing trust very clear in
	both the introduction as well as the conclusion of the Bitcoin whitepaper.
\end{comment}
일반적으로 \enquote{신뢰하되 검증하라. (trust, but verify)}를 가정하지만,
비트코인은 \enquote{신뢰하지 말고 검증하라. (don't trust, verify)}를 가정한다.
사토시는 비트코인 백서의 서론과 결론에서 신뢰 제거의 중요성을 분명하게 밝혔다.

\begin{quotation}\begin{samepage}
		\enquote{결론: 신뢰에 의존하지 않고 전자 거래를 할 수 있는 시스템을 제안한다.}
		\begin{flushright} -- 사토시 나카모토\footnote{Satoshi Nakamoto, the Bitcoin whitepaper~\cite{whitepaper}}
\end{flushright}\end{samepage}\end{quotation}

\begin{comment}
	Note that \textit{without relying on trust} is used in a very specific context
	here. We are talking about trusted third parties, i.e. other entities
	which you trust to produce, hold, and process your money. It is assumed,
	for example, that you can trust your computer.
\end{comment}
여기서 신뢰에 의존하지 않는다는 것이 매우 구체적인 맥락에서 사용된다는 점에 유의해야 한다.
우리는 신뢰할 수 있는 제3자, 즉 돈을 발행, 보유 및 처리하기 위해 신뢰해야하는 주체에 관해 이야기하고 있다.
예를 들어, 당신이 컴퓨터를 신뢰할 수 있다고 가정해 보자.

\begin{comment}
	As Ken Thompson showed in his Turing Award lecture, trust is an
	extremely tricky thing in the computational world. When running a
	program, you have to trust all kinds of software (and hardware) which,
	in theory, could alter the program you are trying to run in a malicious
	way. As Thompson summarized in his \textit{Reflections on Trusting Trust}:
	\enquote{The moral is obvious. You can't trust code that you did not totally
		create yourself.}~\cite{trusting-trust}
\end{comment}
켄 톰슨(Ken Thompson)이 튜링 어워드(Turing Award) 강의에서 보여주었듯이 컴퓨터 세계에서 신뢰는 매우 까다로운 문제이다.
프로그램을 실행할 때 이론적으로 실행하려는 프로그램을 악의적인 방식으로 변경할 수 있는 모든 종류의 소프트웨어와 하드웨어를 신뢰해야 한다.
신뢰성에 대한 고찰(Reflections on Trusting Trust)에서 톰슨은
\enquote{교훈은 명확하다. 스스로 직접 작성한 코드가 아니면 신뢰할 수 없다.}\cite{trusting-trust}
라고 일축했다.

\begin{figure}
	\includegraphics{assets/images/ken-thompson-hack.png}
	\caption{켄 톰슨의 논문 '신뢰성에 대한 고찰'에서 발췌}
	\label{fig:ken-thompson-hack}
\end{figure}

\begin{comment}
	Thompson demonstrated that even if you have access to the source code,
	your compiler --- or any other program-handling program or
	hardware --- could be compromised and detecting this backdoor would be
	very difficult. Thus, in practice, a truly \textit{trustless} system does not
	exist. You would have to create all your software \textit{and} all your
	hardware (assemblers, compilers, linkers, etc.) from scratch, without
	the aid of any external software or software-aided machinery.
\end{comment}
톰슨은 코드에 대한 접근 권한이 있더라도 컴파일러(또는 기타 처리 프로그램, 하드웨어)
가 손상될 수 있으며 이 백도어를 감지하기가 매우 어렵다고 말한다. 
따라서 신뢰가 필요 없는 시스템은 존재하지 않는다.
당신은 외부 소프트웨어나 지원 도구 없이 모든 소프트웨어와 하드웨어(어셈블러, 컴파일러, 링커 등)를 만들어야만 한다.

\begin{quotation}\begin{samepage}
		\enquote{온전히 나의 힘으로 사과파이를 만들고 싶다면 먼저 우주를 발명해야 한다.}
		\begin{flushright} -- 칼 세이건\footnote{Carl Sagan, \textit{Cosmos} \cite{cosmos}}
\end{flushright}\end{samepage}\end{quotation}

\begin{comment}
	The Ken Thompson Hack is a particularly ingenious and hard-to-detect backdoor,
	so let's take a quick look at a hard-to-detect backdoor which works without
	modifying any software. Researchers found a way to compromise security-critical
	hardware by altering the polarity of silicon
	impurities.~\cite{becker2013stealthy} Just by changing the physical properties
	of the stuff that computer chips are made of they were able to compromise a
	cryptographically secure random number generator. Since this change can't be
	seen, the backdoor can't be detected by optical inspection, which is one of the
	most important tamper-detection mechanism for chips like these.
\end{comment}
켄 톰슨 핵(Ken Thompson Hack)은 매우 독창적이고 탐지하기 어려운 소프트웨어 수정 없이도 가능한 백도어 공격인데 이에 대해 간단히 살펴보자.
연구원들은 실리콘 불순물의 극성을 변경하여 보안이 중요한 하드웨어를 손상시키는 방법을 찾아냈다.\cite{becker2013stealthy}
컴퓨터 칩을 구성하는 재료의 물리적 속성을 변경하는 것만으로도 암호학적으로 안전한 난수 생성기를 손상시킬 수 있었다.
이 공격은 눈으로 볼 수 없기 때문에 가장 강력한 변조 감지 메커니즘 중 하나인 광학 검사로도 백도어를 감지해 낼 수 없다.

\begin{figure}
	\includegraphics{assets/images/stealthy-hardware-trojan.png}
	%\caption{Stealthy Dopant-Level Hardware Trojans by Becker, Regazzoni, Paar, Burleson}
	\caption{은밀한 도판트 레벨의 하드웨어 트로이 목마 바이러스}
	\label{fig:stealthy-hardware-trojan}
\end{figure}

\begin{comment}
	Sounds scary? Well, even if you would be able to build everything from
	scratch, you would still have to trust the underlying mathematics. You
	would have to trust that \textit{secp256k1} is an elliptic curve without
	backdoors. Yes, malicious backdoors can be inserted in the mathematical
	foundations of cryptographic functions and arguably this has already
	happened at least once.~\cite{wiki:Dual_EC_DRBG} There are good reasons to be paranoid, and the
	fact that everything from your hardware, to your software, to the
	elliptic curves used can have backdoors~\cite{wiki:backdoors} are some of them.
\end{comment}
두려운가? 처음부터 모든 것을 만들 수 있다고 해도, 수학을 신뢰하는 문제는 남아있다.
secp256k1이 백도어가 없는 타원 곡선이라는 것을 믿어야 하는 것이다.
그렇다. 악성 백도어는 암호화 기능의 기초가 되는 수학에 삽입될 수 있으며 아마 적어도 한번은 그랬을 것이다.\cite{wiki:Dual_EC_DRBG}.
하드웨어부터 소프트웨어, 타원곡선에 이르기까지, 모든 것이 백도어\cite{wiki:backdoors}를 포함할 수 있다는 사실은 편집증을 유발한다.


\begin{quotation}\begin{samepage}
		\enquote{신뢰하지 말고 검증하라(Don’t trust. Verify).}
		\begin{flushright} -- 도처의 비트코이너들
\end{flushright}\end{samepage}\end{quotation}

\begin{comment}
	The above exampes should illustrate that \textit{trustless} computing is
	utopic. Bitcoin is probably the one system which comes closest to this
	utopia, but still, it is \textit{trust-minimized} --- aiming to remove trust
	wherever possible. Arguably, the chain-of-trust is neverending, since
	you will also have to trust that computation requires energy, that P
	does not equal NP, and that you are actually in base reality and not
	imprisoned in a simulation by malicious actors.
\end{comment}
위의 예시는 신뢰가 필요 없는(trustless) 컴퓨팅이 유토피아라는 것을 나타낸다.
비트코인이 아마도 이 유토피아에 가장 가까운 시스템일 것이다.
아직 완벽하다고 말할 수 없으나 적어도 비트코인은 신뢰가 최소화되어 있고, 가능한한 신뢰를 제거하는 것을 목표로 하고 있다.
분명한 것은 신뢰의 사슬은 끝이 없다는 점이다.
계산에는 에너지가 필요하고, P는 NP와 같지 않으며, 실제로는 악의적인 참여자에게 휘둘리지 않고 있다는 점도 믿어야 한다.

\begin{comment}
	Developers are working on tools and procedures to minimize any remaining trust
	even further. For example, Bitcoin developers created
	Gitian\footnote{\url{https://gitian.org/}}, which is a software distribution
	method to create deterministic builds. The idea is that if multiple developers
	are able to reproduce identical binaries, the chance of malicious tampering is
	reduced. Fancy backdoors aren't the only attack vector. Simple blackmail or
	extortion are real threats as well. As in the main protocol, decentralization is
	used to minimize trust.
\end{comment}
개발자들은 여전히 남아있는 신뢰를 최소화하기 위한 도구와 절차를 개발하고 있다.
예를 들어, 비트코인 개발자들은 결정론적 빌드를 생성하기 위한 소프트웨어 배포 방법인 Gitian\footnote{\url{https://gitian.org/}}을 만들었다.
이 기법의 핵심 아이디어는 여러 개발자가 동일한 바이너리를 재현할 수 있으면 악의적 변조의 가능성이 줄어든다는 것이다.
화려한 백도어만이 공격이 아니다. 단순한 협박이나 갈취도 실제하는 위협이다. 
기본 프로토콜처럼 탈중앙화는 신뢰를 최소화하는 도구로 사용된다. 

\begin{comment}
	Various efforts are being made to improve upon the chicken-and-egg problem of
	bootstrapping which Ken Thompson's hack so brilliantly pointed
	out~\cite{web:bootstrapping}. One such effort is
	Guix\footnote{\url{https://guix.gnu.org}} (pronounced \textit{geeks}), which
	uses functionally declared package management leading to bit-for-bit
	reproducible builds by design. The result is that you don't have to trust any
	software-providing servers anymore since you can verify that the served binary
	was not tampered with by rebuilding it from scratch. Recently, a
	pull-request was merged to integrate Guix into the Bitcoin build process.\footnote{See PR 15277 of \texttt{bitcoin-core}: \\ \url{https://github.com/bitcoin/bitcoin/pull/15277}}
\end{comment}
켄 톰슨의 해킹이 지적한 '닭이 먼저냐, 달걀이 먼저냐'에 대한 문제를 개선하는 노력도 다양하게 이루어지고 있다.\cite{web:bootstrapping}
그 노력 중 하나는 기능적으로 선언된 패키지를 관리하여 설계에 따라 비트 단위로 재현 가능한 빌드를 제공하는 Guix(geeks로 발음)\footnote{\url{https://guix.gnu.org}}이다.
결과적으로 바이너리를 다시 빌드하여 변조되지 않았는지 확인할 수 있기 때문에 소프트웨어 제공 서버를 더 이상 신뢰할 필요가 없어진다.
최근 비트코인 빌드 프로세스로 통합하기 위한 Guix 풀 리퀘스트가 병합되었다.\footnote{PR 15277 of \texttt{bitcoin-core}: \\ \url{https://github.com/bitcoin/bitcoin/pull/15277}}

\begin{figure}
	\includegraphics{assets/images/guix-bootstrap-dependencies.png}
	%\caption{Which came first, the chicken or the egg?}
	\caption{닭이 먼저냐? 달걀이 먼저냐?}
	\label{fig:guix-bootstrap-dependencies}
\end{figure}

\begin{comment}
	Luckily, Bitcoin doesn't rely on a single algorithm or piece of
	hardware. One effect of Bitcoin's radical decentralization is a
	distributed security model. Although the backdoors described above are
	not to be taken lightly, it is unlikely that every software wallet,
	every hardware wallet, every cryptographic library, every node
	implementation, and every compiler of every language is compromised.
	Possible, but highly unlikely.
\end{comment}
다행히도 비트코인은 단일 알고리즘이나 하드웨어에 의존적이지 않다.
비트코인의 급진적인 탈중앙화 효과 중 하나는 분산형 보안 모델이라는 점이다. 
앞서 설명한 백도어를 가볍게 치부할 순 없지만 모든 소프트웨어 지갑, 하드웨어 지갑, 암호화 라이브러리, 노드 구현체 및 각종 언어의 모든 컴파일러가 손상될 가능성은 거의 없다.
가능한 일이기는 하나, 실제 일어날 일이 거의 없다는 것이다.

\begin{comment}
	Note that you can generate a private key without relying on any computational
	hardware or software. You can flip a coin~\cite{antonopoulos2014mastering} a
	couple of times, although depending on your coin and tossing style this source
	of randomness might not be sufficiently random. There is a reason why storage
	protocols like Glacier\footnote{\url{https://glacierprotocol.org/}} advise to
	use casino-grade dice as one of two sources of entropy.
\end{comment}
특정 하드웨어나 소프트웨어에 의존하지 않고도 개인키를 생성할 수 있다는 점에 주목하자.
동전을 몇 번 던져서 개인키를 생성할 수는 있지만\cite{antonopoulos2014mastering}, 동전을 던지는 방법에 따라 무작위성이 충분하지 않을 수 있다.
글래시어(Glacier)\footnote{\url{https://glacierprotocol.org/}} 같은 스토리지 프로토콜에서
두 가지 엔트로피 소스 중 하나로 카지노 급 주사위를 사용하도록 권장하는 이유가 있다.

\begin{comment}
	Bitcoin forced me to reflect on what trusting nobody actually entails.
	It raised my awareness of the bootstrapping problem, and the implicit
	chain-of-trust in developing and running software. It also raised my
	awareness of the many ways in which software and hardware can be
	compromised
\end{comment}
비트코인은 나로 하여금 아무도 신뢰하지 않는다는 것이 실제로 무엇을 의미하는지 생각하게 만들었다. 
이를 통해 부트스트래핑 문제와 소프트웨어 개발 및 실행에 있어 암묵적인 신뢰를 요하는 구조에 대한 인식이 높아졌다.
또한 소프트웨어와 하드웨어가 손상될 수 있는 다양한 방법에 대한 인식도 높아졌다.

\paragraph{비트코인은 나에게 신뢰하는 법이 아닌 검증하는 법을 가르쳐 주었다.}

% ---
%
% #### Down the Rabbit Hole
%
% - [The Bitcoin whitepaper][Nakamoto] by Satoshi Nakamoto
% - [Reflections on Trusting Trust][\textit{Reflections on Trusting Trust}] by Ken Thompson
% - [51% Attack][majority] on the Bitcoin Developer Guide
% - [Bootstrapping][bootstrapping], Guix Manual
% - [Secp256k1][secp256k1] on the Bitcoin Wiki
% - [ECC Backdoors][backdoors], [Dual EC DRBG][has already happened] on Wikipedia
%
% [Emmanuel Boutet]: https://commons.wikimedia.org/wiki/User:Emmanuel.boutet
% [\textit{Reflections on Trusting Trust}]: https://www.archive.ece.cmu.edu/~ganger/712.fall02/papers/p761-thompson.pdf
% [found a way]: https://scholar.google.com/scholar?hl=en&as_sdt=0%2C5&q=Stealthy+Dopant-Level+Hardware+Trojans&btnG=
% [Gitian]: https://gitian.org/
% [bootstrapping]: https://www.gnu.org/software/guix/manual/en/html_node/Bootstrapping.html
% [Guix]: https://www.gnu.org/software/guix/
% [pull-request]: https://github.com/bitcoin/bitcoin/pull/15277
% [flip a coin]: https://github.com/bitcoinbook/bitcoinbook/blob/develop/ch04.asciidoc#private-keys
% [Glacier]: https://glacierprotocol.org/
% [secp256k1]: https://en.bitcoin.it/wiki/Secp256k1
% [majority]: https://bitcoin.org/en/developer-guide#term-51-attack
%
% <!-- Wikipedia -->
% [backdoors]: https://en.wikipedia.org/wiki/Elliptic-curve_cryptography#Backdoors
% [has already happened]: https://en.wikipedia.org/wiki/Dual_EC_DRBG
% [Carl Sagan]: https://en.wikipedia.org/wiki/Cosmos_%28Carl_Sagan_book%29
% [alice]: https://en.wikipedia.org/wiki/Alice%27s_Adventures_in_Wonderland
% [carroll]: https://en.wikipedia.org/wiki/Lewis_Carroll
