\chapter{시간을 알려주는 데는 노력이 필요하다.}
\label{les:17}

%\begin{chapquote}{Lewis Carroll, \textit{Alice in Wonderland}}
%\enquote{Dear, dear! I shall be too late!}
\begin{chapquote}{루이스 캐롤, \textit{이상한 나라의 앨리스}}
	\enquote{이런, 이런! 너무 늦겠어!}
\end{chapquote}

\begin{comment}
	It is often said that bitcoins are mined because thousands of computers
	work on solving \textit{very complex} mathematical problems. Certain problems
	are to be solved, and if you compute the right answer, you \enquote{produce} a
	bitcoin. While this simplified view of bitcoin mining might be easier to
	convey, it does miss the point somewhat. Bitcoins aren't produced or
	created, and the whole ordeal is not really about solving particular
	math problems. Also, the math isn't particularly complex. What is
	complex is \textit{telling the time} in a decentralized system.
\end{comment}
흔히들 수천 대의 컴퓨터가 매우 복잡한 수학 문제를 풀면서 비트코인이 채굴된다고 말한다.
특정 문제를 풀어야하고, 정답을 계산해 내면 비트코인을 얻는다.
비트코인 채굴이라는 이 단순화된 관점은 전달하기 쉽지만 핵심을 놓칠 수 있는 표현이다.
비트코인은 생산되거나 생성되는 것이 아니며, 이 전체 과정은 특정 수학 문제를 푸는 것과는 관련이 없다.
또한 채굴에 활용되는 수학은 그다지 복잡하지 않다. 
진짜 복잡한 것은 탈중앙화된 시스템에서 시간을 알려주는 것이다.

\begin{comment}
	As outlined in the whitepaper, the proof-of-work system (aka mining) is
	a way to implement a distributed timestamp server.
\end{comment}
비트코인 백서에 설명된 대로 작업증명(proof-of-work, 일명 마이닝)은 탈중앙화된 타임스탬프\footnote{역주: 특정 시각을 기록하는 문자열} 서버를 구현하는 방법이다.

\begin{figure}
	\includegraphics{assets/images/bitcoin-whitepaper-timestamp-wide.png}
	%  \caption{Excerpts from the whitepaper. Did someone say timechain?}
	\caption{백서에서 발췌. 누가 타임체인이라고 했습니까?}
	\label{fig:bitcoin-whitepaper-timestamp-wide}
\end{figure}

\begin{comment}
	When I first learned how Bitcoin works I also thought that proof-of-work
	is inefficient and wasteful. After a while, I started to shift my
	perspective on Bitcoin's energy consumption~\cite{gigi:energy}. It seems that
	proof-of-work is still widely misunderstood today, in the year 10 AB
	(after Bitcoin).
\end{comment}
나는 비트코인을 처음 접했을 때, 작업증명을 비효율적이고 낭비라고 생각했다.
하지만 시간이 지나면서 비트코인의 에너지 소비\cite{gigi:energy}에 대한 관점이 바뀌기 시작했다.
비트코인이 세상에 나온 후 10년이 지난 지금도 여전히 작업증명은 오해받고 있는 것 같다.

\begin{comment}
	Since the problems to be solved in proof-of-work are made up, many
	people seem to believe that it is \textit{useless} work. If the focus is purely
	on the computation, this is an understandable conclusion. But Bitcoin
	isn't about computation. It is about \textit{independently agreeing on the
		order of things.}
\end{comment}
풀어야 할 문제가 인위적으로 만들어졌다는 이유로 작업증명을 쓸데없는 것이라고 생각하는 사람들이 많은 것 같다.
단순히 계산에만 초점을 맞춘다면 그렇게 판단할 수도 있다.
하지만 비트코인은 무언가를 계산하기 위해 작업증명을 사용한 것이 아니다.
작업증명은 독립적인 주체들이 어떤 일의 순서를 정하기 위해 합의에 이르는 과정이다.

\begin{comment}
	Proof-of-work is a system in which everyone can validate what happened
	and in what order it happened. This independent validation is what leads
	to consensus, an individual agreement by multiple parties about who owns
	what.
\end{comment}
작업증명은 모든 참여자가 발생한 사건과 사건의 순서를 검증하는 시스템이다.
이러한 독립적 검증을 통해 누가 무엇을 소유하는지 여러 당사자로부터 개별적 합의를 이끌어낸다. 

\begin{comment}
	In a radically decentralized environment, we don't have the luxury of absolute
	time. Any clock would introduce a trusted third party, a central point in the
	system which had to be relied upon and could be attacked. \enquote{Timing is the root
		problem,} as Grisha Trubetskoy points out~\cite{pow-clock}. And Satoshi
	brilliantly solved this problem by implementing a decentralized clock via a
	proof-of-work blockchain. Everyone agrees beforehand that the chain with the
	most cumulative work is the source of truth. It is per definition what actually
	happened. This agreement is what is now known as Nakamoto consensus.
\end{comment}
근원적으로 탈중앙화된 환경에서 절대적 시간이라는 것은 사치다.
모든 시계는 신뢰할 수 있는 제3자, 즉 중앙 시스템을 도입해야 하고 이는 언제든 해킹될 수 있다.
그리샤 트루베츠코이(Grisha Trubetskoy)가 지적한 것처럼 \enquote{시간이 근본적 문제}이다.\cite{pow-clock}
사토시는 작업증명 블록체인을 통해 탈중앙형 시계를 구현함으로써 이 문제를 훌륭히 해결했다.
가장 많은 작업이 누적된 체인이 '진실의 원천'이라는 것은 누구나 사전에 동의하는 사실이다.
실제로 비트코인이 구동되고 있는 것은 이 정의에 따른 것이다.
이 합의를 나카모토 합의라 한다.

\begin{quotation}\begin{samepage}
		%\enquote{The network timestamps transactions by hashing them into an ongoing
			%chain which serves as proof of the sequence of events witnessed}
		%\begin{flushright} -- Satoshi Nakamoto\footnote{Satoshi Nakamoto, the Bitcoin whitepaper~\cite{whitepaper}}
		\enquote{네트워크는 제출된 거래의 증거를 해싱함으로써 동작 중인 체인에 트랜잭션 타임스탬프를 기록한다.}
		\begin{flushright} -- Satoshi Nakamoto\footnote{사토시 나카모토, 비트코인 백서~\cite{whitepaper}}
\end{flushright}\end{samepage}\end{quotation}

\begin{comment}
	Without a consistent way to tell the time, there is no consistent way to
	tell before from after. Reliable ordering is impossible. As mentioned
	above, Nakamoto consensus is Bitcoin's way to consistently tell the
	time. The system's incentive structure produces a probabilistic,
	decentralized clock, by utilizing both greed and self-interest of
	competing participants. The fact that this clock is imprecise is
	irrelevant because the order of events is eventually unambiguous and can
	be verified by anyone.
\end{comment}
시간을 알 수 있는 일관된 방법이 없으면 사건의 이전과 이후를 구분할 방법이 없다.
신뢰할 수 있는 순서를 만드는 것이 불가능한 것이다.
위에서 언급했듯이, 나카모토 합의는 시간을 일관되게 알려주는 비트코인의 방식이다.
시스템의 인센티브 구조는 경쟁 참여자의 탐욕과 이기심을 모두 활용하여 확률적이고 탈중앙화된 시계를 만들어 낸다.
이 시계는 정확하지 않다. 하지만 사건의 순서가 모호하지 않고 누구나 선후관계를 확인할 수 있기 때문에 시각의 정확성은 중요하지 않다.

\begin{comment}
	Thanks to proof-of-work, both the work \textit{and} the validation of the work
	are radically decentralized. Everyone can join and leave at will, and
	everyone can validate everything at all times. Not only that, but
	everyone can validate the state of the system \textit{individually}, without
	having to rely on anyone else for validation.
\end{comment}
작업증명 덕분에 작업과 작업의 유효성 검증이 모두 근본적으로 탈중앙화된다.
누구나 마음대로 참여하고 탈퇴할 수 있으며, 모든 참여자가 항상 모든 것을 검증할 수 있다.
뿐만 아니라 다른 사람에게 의존하지 않고 시스템 상태를 스스로 검증할 수 있다.


\begin{comment}
	Understanding proof-of-work takes time. It is often counter-intuitive,
	and while the rules are simple, they lead to quite complex phenomena.
	For me, shifting my perspective on mining helped. Useful, not useless.
	Validation, not computation. Time, not blocks.
\end{comment}
작업증명을 이해하는 데는 시간이 걸린다.
작업증명의 어떤 부분은 직관에 어긋나고, 규칙은 단순하지만 상당히 복잡한 현상을 초래한다.
채굴에 대한 관점을 바꾸니 작업증명이 이해되었다.
쓸모없는 것이 아니고 유용한 것이다. 계산을 하는 것이 아니라 검증하는 것이다.
블록이 아니라 시간을 말이다.

%\paragraph{Bitcoin taught me that telling the time is tricky, especially if you are decentralized.}
\paragraph{비트코인은 나에게 시간을 알려주는 것이 탈중앙화된 경우엔 특히나 까다롭다는 것을 가르쳐주었다.}

% ---
%
% #### Through the Looking-Glass
%
% - [Bitcoin's Energy Consumption: A shift in perspective][energy]
%
% #### Down the Rabbit Hole
%
% - [Blockchain Proof-of-Work Is a Decentralized Clock][points out] by Gregory Trubetskoy
% - [The Anatomy of Proof-of-Work][pow-anatomy] by Hugo Nguyen
% - [PoW is efficient][pow-efficient] by Dan Held
% - [Mining][bw-mining], [Controlled supply][bw-supply] on the Bitcoin Wiki
%
% [points out]: https://grisha.org/blog/2018/01/23/explaining-proof-of-work/
% [energy]: 
% [whitepaper]: https://bitcoin.org/bitcoin.pdf
%
% [pow-efficient]: https://blog.picks.co/pow-is-efficient-aa3d442754d3
% [pow-anatomy]: https://bitcointechtalk.com/the-anatomy-of-proof-of-work-98c85b6f6667
% [bw-mining]: https://en.bitcoin.it/wiki/Mining
% [bw-supply]: https://en.bitcoin.it/wiki/Controlled_supply
%
% <!-- Wikipedia -->
% [alice]: https://en.wikipedia.org/wiki/Alice%27s_Adventures_in_Wonderland
% [carroll]: https://en.wikipedia.org/wiki/Lewis_Carroll
