\chapter{비트코인의 미래에 대한 비유}
\label{les:21}

\begin{chapquote}{루이스 캐롤, \textit{이상한 나라의 앨리스}}
	%\enquote{I know something interesting is sure to happen\ldots}
	\enquote{나는 흥미로운 일이 일어나리라 확신하고 있었다\ldots}
\end{chapquote}

\begin{comment}
	In the last couple of decades, it became apparent that technological
	innovation does not follow a linear trend. Whether you believe in the
	technological singularity or not, it is undeniable that progress is
	exponential in many fields. Not only that, but the rate at which
	technologies are being adopted is accelerating, and before you know it
	the bush in the local schoolyard is gone and your kids are using
	Snapchat instead. Exponential curves have the tendency to slap you in
	the face way before you see them coming.
\end{comment}
지난 수십 년 동안 기술 혁신이 선형적 추세를 따르지 않는다는 것이 명백해졌다.
기술 특이점을 믿든 믿지 않든, 많은 분야에서 기하급수적으로 발전이 이루어지고 있다는 것은 부인할 수 없는 사실이다.
뿐만 아니라, 기술이 채택되는 속도도 빨라지고 있다. 
어느새 동네 학교 운동장에 덤불은 사라졌고 아이들이 스냅챗을 사용하고 있다.
기하급수적인 곡선은 우리가 방심하는 사이 갑자기 존재감을 드러내곤 한다.

\begin{comment}
	Bitcoin is an exponential technology built upon exponential technologies.
	\textit{Our World in Data}\footnote{\url{https://ourworldindata.org/}}
	beautifully shows the rising speed of technological adoption, starting in 1903
	with the introduction of landlines (see Figure~\ref{fig:tech-adoption}).
	Landlines, electricity, computers, the internet, smartphones; all follow
	exponential trends in price-performance and adoption. Bitcoin does
	too~\cite{tech-adoption}.
\end{comment}
비트코인은 기하급수적 기술을 바탕으로 구축된 기하급수적 기술이다.
데이터로 보는 세상(Our World in Data)\footnote{\url{https://ourworldindata.org/}}에서는
1903년 유선 전화 도입을 시작으로 기술 채택의 속도가 증가하는 과정을 아름답게 보여준다.(그림 ~\ref{fig:tech-adoption})
유선전화, 전기, 컴퓨터, 인터넷, 스마트폰 등 모두 가격 대비 성능과 채택률에서 기하급수적 성장 추세를 따른다. 
비트코인도 마찬가지이다.~\cite{tech-adoption}

\begin{figure}
	\includegraphics{assets/images/tech-adoption.png}
	%\caption{Bitcoin is literally off the charts.}
	\caption{비트코인은 말그대로 차트를 벗어났다.}
	\label{fig:tech-adoption}
\end{figure}

\begin{comment}
	Bitcoin has not one but multiple network effects\footnote{Trace Mayer,
		\textit{The Seven Network Effects of Bitcoin}~\cite{7-network-effects}}, all of
	which resulting in exponential growth patterns in their respective area: price,
	users, security, developers, market share, and adoption as global money.
\end{comment}
비트코인은 여러 네트워크 효과\footnote{Trace Mayer,
	\textit{The Seven Network Effects of Bitcoin}~\cite{7-network-effects}}를 가지고 있다.
그 결과 가격, 사용자, 보안, 개발자, 시장 점유율, 글로벌 화폐로서의 채택 등 각 영역에서 기하급수적 성장 패턴을 보인다.

\begin{comment}
	Having survived its infancy, Bitcoin is continuing to grow every day in
	more aspects than one. Granted, the technology has not reached maturity
	yet. It might be in its adolescence. But if the technology is
	exponential, the path from obscurity to ubiquity is short.
\end{comment}
비트코인은 초기 단계를 넘어서 여러 측면에서 매일 성장을 거듭하고 있다.
물론 이 기술은 아직 성숙 단계에 도달하지 않았다. 아직 청소년기일 수도 있다.
하지만 기술이 기하급수적으로 발전한다면, 모호함에서 보편성으로의 전이는 순식간일 것이다.

\begin{figure}
	\includegraphics{assets/images/mobile-phone.png}
	%\caption{Mobile phone, ca 1965 vs 2019.}
	\caption{1965년과 2019년의 모바일 휴대폰}
	\label{fig:mobile-phone}
\end{figure}

\begin{comment}
	In his 2003 TED talk, Jeff Bezos chose to use electricity as a metaphor for the
	web's future.\footnote{\url{http://bit.ly/bezos-web}} All three phenomena ---
	electricity, the internet, Bitcoin --- are \textit{enabling} technologies,
	networks which enable other things. They are infrastructure to be built upon,
	foundational in nature.
\end{comment}
2003년 테드 강연에서 제프 베조스는 웹의 미래를 전기에 비유했다.\footnote{\url{http://bit.ly/bezos-web}}
전기, 인터넷, 비트코인이라는 세 가지 현상은 다른 모든 것을 가능하게하는 네트워크 기술을 실현한다.
이 세 가지 현상은 본질적으로 인프라적 성격을 갖는다.

\begin{comment}
	Electricity has been around for a while now. We take it for granted. The
	internet is quite a bit younger, but most people already take it for
	granted as well. Bitcoin is ten years old and has entered public
	consciousness during the last hype cycle. Only the earliest of adopters
	take it for granted. As more time passes, more and more people will
	recognize Bitcoin as something which simply is.\footnote{This is known as the
		\textit{Lindy Effect}. The Lindy effect is a theory that the future life expectancy
		of some non-perishable things like a technology or an idea is proportional to
		their current age, so that every additional period of survival implies a longer
		remaining life expectancy.~\cite{wiki:lindy}}
\end{comment}
전기는 오래전부터 존재해 왔다. 그래서 우리는 이를 당연하게 여긴다.
인터넷은 이보다 비교적 최근에 등장했지만 대부분의 사람들은 이미 인터넷을 당연하게 여긴다.
비트코인은 출시된 지 이제 10년이며 지난 과대광고 주기\footnote{역주: 기대감이 지나치게 높아진 시점} 동안 대중이 인식하기 시작했다.
아직은 초기 수용자들만이 비트코인을 당연하게 받아들인다.
시간이 지날수록 점점 더 많은 사람이 비트코인이 존재하는 것을 당연하게 여기게 될 것이다.\footnote{이러한 것을
	린디 효과라 한다. 린디 효과는 기술이나 아이디어처럼 썩지 않는 것의 미래 기대 수명이 현재 나이에 비례한다는 이론으로,
	생존 기간이 늘어날 때마다 남은 기대 수명도 길어진다.~\cite{wiki:lindy}}.

\begin{comment}
	In 1994, the internet was still confusing and unintuitive. Watching this old
	recording of the \textit{Today
		Show}\footnote{\url{https://youtu.be/UlJku_CSyNg}} makes it obvious that what
	feels natural and intuitive now actually wasn't back then. Bitcoin is still
	confusing and alien to most, but just like the internet is second nature for
	digital natives, spending and stacking
	sats\footnote{\url{https://twitter.com/hashtag/stackingsats}} will be second
	nature to the bitcoin natives of the future.
\end{comment}
1994년 당시, 인터넷은 여전히 혼란스럽고 직관적이지 않았다. 
투데이 쇼(Today show) 녹화 영상\footnote{\url{https://youtu.be/UlJku_CSyNg}}을 보면 
현재에는 자연스럽고 직관적이라고 느끼는 것들이 당시에는 그렇지 않았다는 것을 알 수 있다.
비트코인은 여전히 많은 사람들에게 혼란스럽고 낯설지만, 
디지털 세대에게 인터넷이 제2의 고향인 것처럼 
미래의 비트코인 세대들에게 사토시를 쌓는 것\footnote{\url{https://twitter.com/hashtag/stackingsats}}이 제2의 고향이 될 것이다.

\begin{quotation}\begin{samepage}
		\enquote{미래가 여기에 있습니다. 단지 대중화되지 않았을 뿐입니다.}
		\begin{flushright} -- 윌리엄 깁슨\footnote{William Gibson, \textit{The Science in Science Fiction(공상 과학 소설 속 과학)} \cite{william-gibson}}
\end{flushright}\end{samepage}\end{quotation}

\begin{comment}
	In 1995, about $15\%$ of American adults used the internet. Historical
	data from the Pew Research Center~\cite{pew-research} shows how the internet has woven
	itself into all our lives. According to a consumer survey by Kaspersky
	Lab~\cite{web:kaspersky}, 13\% of respondents have used Bitcoin and its clones to pay for
	goods in 2018. While payments aren't the only use-case of bitcoin, it is
	some indication of where we are in Internet time: in the early- to
	mid-90s.
\end{comment}
1995년에는 미국 인구의 약 $15\%$가 인터넷을 사용했다.
퓨 리서치 센터\cite{pew-research}의 과거 데이터는 인터넷이 우리 삶에 어떻게 스며들었는지를 보여준다.
카스퍼스키 랩\cite{web:kaspersky}의 소비자 설문 조사에 따르면 
응답자의 13\%가 2018년에 비트코인과 비트코인 유사품을 사용해 상품을 결제한 경험이 있다고 답했다.
결제만이 비트코인의 유일한 사용사례는 아니지만, 
이 지표는 인터넷 발전 단계로 보았을 때 비트코인이 90년대 초중반 어느 시점에 와 있는지를 보여준다.

\begin{comment}
	In 1997, Jeff Bezos stated in a letter to shareholders~\cite{bezos-letter} that
	\enquote{this is Day 1 for the Internet,} recognizing the great untapped
	potential for the internet and, by extension, his company. Whatever day this is
	for Bitcoin, the vast amounts of untapped potential are clear to all but the
	most casual observer.
\end{comment}
1997년 제프 베조스는 주주 서한\cite{bezos-letter}에 \enquote{오늘은 인터넷의 첫 날(Day 1) 입니다.}라고 적었다.
이 서한에 인터넷과 더 나아가 자신의 회사에 대한 엄청난 잠재력을 언급한 것이었다.
비트코인의 날이 언제가 되었든, 아직 개척되지 않은 엄청난 잠재력이 있다는 것은
비트코인을 무심코 바라보는 몇몇을 제외한 모든 사람에게 분명히 인식될 것이다.


\begin{figure}
	\includegraphics{assets/images/internet-evolution-white-dates.png}
	%  \caption{The internet, 1982 vs 2005. Source: cc-by Merit Network, Inc. and Barrett Lyon, Opte Project}
	\caption{1982년의 인터넷과 2005년의 인터넷 (출처: Merit Network)}
	\label{fig:internet-evolution-white-dates}
\end{figure}

\begin{comment}
	Bitcoin's first node went online in 2009 after Satoshi mined the \textit{genesis
		block}\footnote{The genesis block is the first block of the Bitcoin block chain.
		Modern versions of Bitcoin number it as block $0$, though very early versions
		counted it as block $1$. The genesis block is usually hardcoded into the
		software of the applications that utilize the Bitcoin block chain. It is a
		special case in that it does not reference a previous block and produces an
		unspendable subsidy. The \textit{coinbase} parameter contains, along with the
		normal data, the following text: \textit{\enquote{The Times 03/Jan/2009 Chancellor on
				brink of second bailout for banks}} \cite{btcwiki:genesis-block}} and released
	the software into the wild. His node wasn't alone for long. Hal Finney was one
	of the first people to pick up on the idea and join the network. Ten years
	later, as of this writing, more than
	$75.000$\footnote{\url{https://bit.ly/luke-nodecount}} nodes are running
	bitcoin.
\end{comment}
비트코인의 첫 번째 노드는 2009년 사토시가 소프트웨어를 공개한 후 제네시스 블록\footnote{
	제네시스 블록은 비트코인의 첫 번째 블록이다. 최신 버전의 비트코인은
	블록을 $0$으로 시작했지만, 초기 버전에서는 블록을 $1$로 계산했다.
	제네시스 블록은 일반적으로 비트코인 블록체인을 구동하는 응용프로그램의 소프트웨어에
	하드코딩 된다. 이전 블록이 없이 보상을 만들어 내는 데서 다른 블록과는 차별점이 있다.
	제네시스 블록의 코인 베이스 매개변수에는 다음 문구가 포함되어 있다.
	\enquote{The Times 03/Jan/2009 Chancellor on brink of second bailout for banks}\cite{btcwiki:genesis-block}}
을 채굴하며 온라인 상태가 되었다.
사토시 노드는 머지않아 혼자가 아니게 되었다.
할 피니는 사토시의 아이디어를 인정하고 비트코인 네트워크에 참여한 최초의 사람 중 한 명이었다.
10년 후, 이 글을 쓰는 시점에는 $75,000$개 이상\footnote{\url{https://bit.ly/luke-nodecount}}의 노드가 비트코인을 구동하고 있다.

\begin{figure}
	\centering
	\includegraphics[width=8cm]{assets/images/running-bitcoin.png}
	\caption{할피니가 2009년 1월 비트코인을 언급하는 첫 번째 트윗}
	\label{fig:running-bitcoin}
\end{figure}

\begin{comment}
	The protocol's base layer isn't the only thing growing exponentially.
	The lightning network, a second layer technology, is growing at an even
	faster rate.
\end{comment}
비트코인 프로토콜의 기본 레이어만 기하급수적으로 성장하고 있는 것이 아니다.
두 번째 레이어인 라이트닝 네트워크는 훨씬 더 빠른 속도로 성장하고 있다.

\begin{comment}
	In January 2018, the lightning network had $40$ nodes and $60$
	channels~\cite{web:lightning-nodes}. In April 2019, the network grew to more
	than $4000$ nodes and around $40.000$ channels. Keep in mind that this is still
	experimental technology where loss of funds can and does occur. Yet the trend is
	clear: thousands of people are reckless and eager to use it.
\end{comment}
2018년 1월, 라이트닝 네트워크에는 $40$개의 노드와 $60$개의 채널이 있었다.\cite{web:lightning-nodes}
2019년 4월, 네트워크는 $4,000$개 이상의 노드와  $40,000$개의 채널로 성장했다.
라이트닝 네트워크는 자금 손실이 발생할 수 있고, 실제로 발생하기도 하는 실험적인 기술임을 명심해야 한다.
하지만 수천 명의 사람들이 무모하고 열성적으로 라이트닝 네트워크를 사용하려는 추세만은 분명하다. 

\begin{figure}
	\includegraphics{assets/images/lnd-growth-lopp-white.png}
	%\caption{Lightning Network, January 2018 vs December 2018. Source: Jameson Lopp}
	\caption{2018년 1월과 2018년 12월의 라이트닝 네트워크 (출처: Jameson Lopp)}
	\label{fig:lnd-growth-lopp-white.png}
\end{figure}

\begin{comment}
	To me, having lived through the meteoric rise of the web, the parallels
	between the internet and Bitcoin are obvious. Both are networks, both
	are exponential technologies, and both enable new possibilities, new
	industries, new ways of life. Just like electricity was the best
	metaphor to understand where the internet is heading, the internet might
	be the best metaphor to understand where bitcoin is heading. Or, in the
	words of Andreas Antonopoulos, Bitcoin is \textit{The Internet of Money}.
	These metaphors are a great reminder that while history doesn't repeat
	itself, it often rhymes.
\end{comment}
웹의 급격한 성장을 경험한 나에게 인터넷과 비트코인의 유사점은 분명하다.
둘 다 네트워크이고 기하급수적인 기술이며 새로운 가능성, 새로운 산업, 새로운 삶의 방식을 가능하게 한다.
인터넷이 어디로 향하고 있는지 이해하기 위해 전기가 가장 좋은 비유였던 것처럼,
비트코인이 어디로 향하고 있는지 이해하기 위해 인터넷이 가장 좋은 비유가 될 수 있다.
안드레아스 안토노풀로스(Andreas Antonopoulos)의 말을 빌리자면, 비트코인은 돈을 위한 인터넷(The Internet of Money)이다.
역사가 반복되지 않는다 하더라도 종종 이러한 라임(rhymes)이 맞아 떨어질 수도 있다.

\begin{comment}
	Exponential technologies are hard to grasp and often underestimated.
	Even though I have a great interest in such technologies, I am
	constantly surprised by the pace of progress and innovation. Watching
	the Bitcoin ecosystem grow is like watching the rise of the internet in
	fast-forward. It is exhilarating.
\end{comment}
기하급수적으로 발전하는 기술은 파악하기도 어렵고 종종 과소평가되기도 한다.
이런 기술에 큰 관심을 가진 나로써도 그 발전과 혁신의 속도에 끊임없이 놀라고 있다.
비트코인 생태계가 성장하는 것을 지켜보는 것은 마치 인터넷의 부상을 빨리 감기로 보고 있는 것과 같다.
정말 짜릿하다.

\begin{comment}
	My quest of trying to make sense of Bitcoin has led me down the pathways
	of history in more ways than one. Understanding ancient societal
	structures, past monies, and how communication networks evolved were all
	part of the journey. From the handaxe to the smartphone, technology has
	undoubtedly changed our world many times over. Networked technologies
	are especially transformational: writing, roads, electricity, the
	internet. All of them changed the world. Bitcoin has changed mine and
	will continue to change the minds and hearts of those who dare to use
	it.
\end{comment}
비트코인을 이해하고자 하는 나의 탐구는 여러 방면으로 역사를 돌아보는 여정이었다.
고대 사회 구조, 과거의 화폐, 통신 네트워크가 어떻게 진화했는지 이해하는 것도 이 여정의 일부였다.
손도끼에서 스마트폰에 이르기까지 기술은 의심할 여지 없이 우리 세상을 여러 차례 변화시켰다.
특히 네트워크 기술은 문자, 도로, 전기, 인터넷과 같은 혁신을 일으켰다.
이 모든 것이 세상을 변화시켰다. 
비트코인은 나를 변화시켰고, 비트코인을 사용하는 사람들의 생각과 마음을 계속 변화시킬 것이다.

%\paragraph{Bitcoin taught me that understanding the past is essential to
	%understanding its future. A future which is just beginning\ldots}
\paragraph{비트코인은 과거를 이해하는 것이 미래를 이해하는 데 필수적이라는 것을 알려주었다. 미래는 이제 막 시작되었다.\ldots}

% ---
%
% #### Down the Rabbit Hole
%
% - [The Rising Speed of Technological Adoption][the rising speed of technological adoption] by Jeff Desjardins
% - [The 7 Network Effects of Bitcoin][multiple network effects] by Trace Mayer
% - [The Electricity Metaphor for the Web's Future][TED talk] by Jeff Bezos
% - [How the internet has woven itself into American life][data from the Pew Research Center] by Susannah Fox and Lee Rainie
% - [Genesis Block][genesis block] on the Bitcoin Wiki
% - [Lindy Effect][more time] on Wikipedia
%
% [Our World in Data]: https://ourworldindata.org/
% [the rising speed of technological adoption]: https://www.visualcapitalist.com/rising-speed-technological-adoption/
% [multiple network effects]: https://www.thrivenotes.com/the-7-network-effects-of-bitcoin/
% [TED talk]: https://www.ted.com/talks/jeff_bezos_on_the_next_web_innovation
% [recording of the Today Show]: https://www.youtube.com/watch?v=UlJku_CSyNg
% [William Gibson]: https://www.npr.org/2018/10/22/1067220/the-science-in-science-fiction
% [data from the Pew Research Center]: https://www.pewinternet.org/2014/02/27/part-1-how-the-internet-has-woven-itself-into-american-life/
% [consumer survey]: https://www.kaspersky.com/blog/money-report-2018/
% [letter to shareholders]: http://media.corporate-ir.net/media_files/irol/97/97664/reports/Shareholderletter97.pdf
% [running bitcoin]: https://twitter.com/halfin/status/1110302988?lang=en
% [40 nodes]: https://bitcoinist.com/bitcoin-lightning-network-mainnet-nodes/
% [reckless]: https://twitter.com/hashtag/reckless
% [Jameson Lopp]: https://twitter.com/lopp/status/1077200836072296449
% [\textit{The Internet of Money}]: https://theinternetofmoney.info/
% [stacking]: https://twitter.com/hashtag/stackingsats
%
% <!-- Bitcoin Wiki -->
% [genesis block]: https://en.bitcoin.it/wiki/Genesis_block
%
% <!-- Wikipedia -->
% [more time]: https://en.wikipedia.org/wiki/Lindy_effect
% [alice]: https://en.wikipedia.org/wiki/Alice%27s_Adventures_in_Wonderland
% [carroll]: https://en.wikipedia.org/wiki/Lewis_Carroll
