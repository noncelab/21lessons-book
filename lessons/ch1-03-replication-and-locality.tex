\chapter{복제와 국소성}
\label{les:3}

\begin{chapquote}{루이스 캐롤, \textit{이상한 나라의 앨리스}}
	다음으로 토끼의 화난 목소리가 들렸다. \enquote{팻! 팻! 어디야?}
\end{chapquote}

\paragraph{}
%Quantum mechanics aside, locality is a non-issue in the physical world.
%The question \textit{\enquote{Where is X?}} can be answered in a meaningful way, no
%matter if X is a person or an object. In the digital world, the question
%of \textit{where} is already a tricky one, but not impossible to answer. Where
%are your emails, really? A bad answer would be \enquote{the cloud}, which is
%just someone else's computer. Still, if you wanted to track down every
%storage device which has your emails on it you could, in theory, locate
%them.
양자역학까지 갈 것도 없이 물리 세계에서는 국소성(Locality)이 문제되지 않는다.
\footnote{역주: 고전역학에서 국소성이란 공간적으로 떨어져있는 두 물체가 서로 직접적으로 영향을 줄 수 없는 성질을 말한다. 
반대로 양자역학에서는 두 물체가 떨어져 있더라도 양자 차원의 힘에 의해 상호 영향을 미치기 때문에 두 물체 간에는 비국소성을 갖는다고 말한다.}
\enquote{X는 어디에 있는가?}라는 질문에 X가 사람이든 사물이든 대답할 수 있다. 
디지털 세계에서도 '어디에 있는가'라는 질문은 까다롭긴 해도 대답하기 불가능한 질문이 아니다. 
이메일이 진짜로 어디에 있는가? 이 질문을 듣고 \enquote{(내 컴퓨터가 아닌) 클라우드에 있어.}라고 성의 없이 대답할 수 있다.
하지만 이렇게 대답해도 모든 저장 장치를 추적하면 이론상 이메일이 있는 곳을 찾을 수 있다.

\paragraph{}
%With bitcoin, the question of \enquote{where} is \textit{really} tricky. Where,
%exactly, are your bitcoins?
그러나 비트코인의 경우, \enquote{어디에 있는가}라는 질문에 대답하기가 \textit{정말} 까다롭다. 비트코인은 정확히 어디에 있는 걸까?

\begin{quotation}\begin{samepage}
		\enquote{나는 수술 후 눈을 뜨고 주위를 둘러보며 한탄스러울 정도로 진부하지만 피할 수 없는 그 질문을 했다.`여기가 어디지?`}
		\begin{flushright} -- 다니엘 데넷\footnote{Daniel Dennett, \textit{Where Am I?}~\cite{where-am-i}}
		\end{flushright}
\end{samepage}\end{quotation}

\paragraph{}
%The problem is twofold: First, the distributed ledger is distributed by
%full replication, meaning the ledger is everywhere. Second, there are no
%bitcoins. Not only physically, but \textit{technically}.
이 문제는 이중적이다. 하나는 분산 원장은 모든 기록을 복제하여 어디에나 존재한다는 점이고, 다른 하나는 비트코인이 존재하지 않는다는 점이다. 
물리적으로도 \textit{기술적으로도}.

\paragraph{}
%Bitcoin keeps track of a set of unspent transaction outputs, without
%ever having to refer to an entity which represents a bitcoin. The
%existence of a bitcoin is inferred by looking at the set of unspent
%transaction outputs and calling every entry with 100 million base
%units a bitcoin.
비트코인은 어떤 수량의 비트코인을 나타내는 특정한 것을 참조하는 것이 아니라 미사용 트랜잭션 출력(UTXO)의 집합을 추적한다.
특정 수량의 비트코인이 존재한다는 것은 UTXO들 중 1억 기본 단위\footnote{역주: 1BTC = $10^{8}$sats}로 기록되는 모든 데이터를 호출할 수 있음을 의미한다. 

\begin{quotation}\begin{samepage}
		\enquote{비트코인은 지금 이 순간 어디로 이동 중일까요?[\ldots] 일단, 비트코인은
			없어요. 그냥 없어요. 존재하지 않아요. 공유된 원장이 존재할 뿐이죠. [\ldots] 
			물리적으로 존재하지 않아요. 원장이 물리적 위치에 존재하는 거예요.
			여기서 지리적 의미는 담지 맙시다. 당신 상식 선에서 이걸 이해하려고 해도 별 도움이 되지 않을 거예요.}
		\begin{flushright} -- 피터 반 발켄버그\footnote{Peter Van Valkenburgh on the What Bitcoin Did podcast, episode 49 \cite{wbd049}}
\end{flushright}\end{samepage}\end{quotation}

\paragraph{}
%So, what do you actually own when you say \textit{\enquote{I have a bitcoin}} if
%there are no bitcoins? Well, remember all these strange words which you were
%forced to write down by the wallet you used? Turns out these magic words are
%what you own: a magic spell\footnote{The Magic Dust of Cryptography: How digital
	%information is changing our society \cite{gigi:magic-spell}} which can be used
%to add some entries to the public ledger --- the keys to \enquote{move} some bitcoins.
%This is why, for all intents and purposes, your private keys \textit{are} your
%bitcoins. If you think I'm making all of this up feel free to send me your
%private keys.
비트코인이 어디에도 존재하지 않는다면, \textit{\enquote{비트코인을 갖고 있다.}}라는 문장에서 실제로 소유하고 있는 것은 무엇인가? 
혹시 사용하던 지갑 때문에 억지로 옮겨 적어야만 했던 이상한 단어들을 기억하는가?
바로 이 단어들이 비트코인 공개 원장에 장부를 추가할 수 있는 마법의 주문\footnote{The Magic Dust of Cryptography: How digital
	information is changing our society \cite{gigi:magic-spell}}, 즉 비트코인을 전송할 때 필요한 열쇠인 것이다. 
그렇기 때문에 이 모든 의도와 목적에 따라 개인키가 곧 비트코인인 것이다. 
이 모든 것이 내가 지어낸 것이라 생각한다면, 언제든 나에게 당신의 개인키를 보내주길 바란다.

\paragraph{비트코인은 나에게 국소성이 얼마나 까다로운 것인지 알려주었다.}

% ---
%
% #### Through the Looking-Glass
%
% - [The Magic Dust of Cryptography: How digital information is changing our society][a magic spell]
%
% #### Down the Rabbit Hole
%
% - [Where Am I?][Daniel Dennett] by Daniel Dennett
% - 🎧 [Peter Van Valkenburg on Preserving the Freedom to Innovate with Public Blockchains][wbd049] WBD #49 hosted by Peter McCormack
%
% <!-- Through the Looking-Glass -->
% [a magic spell]: 
%
% <!-- Down the Rabbit Hole -->
% [Daniel Dennett]: https://www.lehigh.edu/~mhb0/Dennett-WhereAmI.pdf
% [1st Amendment]: https://en.wikipedia.org/wiki/First_Amendment_to_the_United_States_Constitution
% [wbd049]: https://www.whatbitcoindid.com/podcast/coin-centers-peter-van-valkenburg-on-preserving-the-freedom-to-innovate-with-public-blockchains
%
% <!-- Wikipedia -->
% [alice]: https://en.wikipedia.org/wiki/Alice%27s_Adventures_in_Wonderland
% [carroll]: https://en.wikipedia.org/wiki/Lewis_Carroll
