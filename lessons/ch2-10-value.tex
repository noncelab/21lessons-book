\chapter{가치}
\label{les:10}

\begin{chapquote}{루이스 캐롤, \textit{이상한 나라의 앨리스}}
	\enquote{하얀 토끼가 다시 천천히 돌아오면서 무언가를 잃어버린 것처럼 걱정스럽게 주위를 둘러보았다\ldots}
\end{chapquote}

% \begin{comment}
% 	Value is somewhat paradoxical, and there are multiple theories\footnote{See
% 		\textit{Theory of value (economics)} on Wikipedia \cite{wiki:theory-of-value}}
% 	which try to explain why we value certain things over other things. People have
% 	been aware of this paradox for thousands of years. As Plato wrote in his
% 	dialogue with Euthydemus, we value some things because they are rare, and not
% 	merely based on their necessity for our survival.
% \end{comment}
\paragraph{}
가치는 다소 역설적인 개념이다. 
어떤 무언가가 다른 무언가에 비해 가치 있다고 판단하는 이유를 설명하는 여러 이론이 있다.
\footnote{\textit{가치 이론 (Theory of value (economics))} 위키피디아 \cite{wiki:theory-of-value}}
사람들은 수천 년 간 이 역설을 알고 있었다. 
플라톤이 에우티데모스와의 대화에서 말했듯이, 우리가 어떤 것을 소중히 여기는 이유는 단순히 생존을 위한 필요성 때문이 아니라 희소성 때문이기도 하다.

\begin{quotation}\begin{samepage}
		\enquote{그런데 두 분이 제정신이라면, 제자들에게도 똑같은 충고를 하실 겁니다.
			당신과 자기 자신들 말고는 어느 사람과도 절대 대화를 나누지 말라고 말이죠.
			에우티데모스! 핀다로스의 말처럼 귀한 것은 값지고 물은 아무리 훌륭해도 가장 싼 것이기 때문입니다.}
		\begin{flushright} -- 플라톤\footnote{Plato, \textit{Euthydemus} \cite{euthydemus}}
\end{flushright}\end{samepage}\end{quotation}

% \begin{comment}
% 	This paradox of value\footnote{See \textit{Paradox of value} on Wikipedia
% 		\cite{wiki:paradox-of-value}} shows something interesting about us humans: we
% 	seem to value things on a subjective\footnote{See \textit{Subjective theory of
% 			value} on Wikipedia \cite{wiki:subjective-theory-of-value}} basis, but do so
% 	with certain non-arbitrary criteria. Something might be \textit{precious} to us
% 	for a variety of reasons, but things we value do share certain characteristics.
% 	If we can copy something very easily, or if it is naturally abundant, we do not
% 	value it.
% \end{comment}

\paragraph{}
이 가치의 역설\footnote{\textit{가치의 역설(Paradox of value)} 위키피디아 \cite{wiki:paradox-of-value}}은 
인간의 매우 흥미로운 일면을 보여준다. 
우리는 주관적인 기준\footnote{\textit{주관적 가치 이론(Subjective theory of value)} 위키피디아 \cite{wiki:subjective-theory-of-value}}
을 가지고 가치를 측정하는 것 같지만, 사실은 임의의 기준이 아닌 특정한 기준에 따라 측정한다. 
어떠한 것이 여러 가지 이유로 나에게 가치가 있을 수 있지만, 우리가 가치 있다고 판단하는 것들은 대개 공통적 특성을 갖는다.
무언가를 아주 쉽게 모방할 수 있거나 자연에서 쉽게 얻을 수 있다면, 우리는 그것을 가치 있다고 말하지 않는다. 


% \begin{comment}
% 	It seems that we value something because it is scarce (gold, diamonds,
% 	time), difficult or labor-intensive to produce, can't be replaced (an
% 	old photograph of a loved one), is useful in a way in which it enables
% 	us to do things which we otherwise couldn't, or a combination of those,
% 	such as great works of art.
% 
\paragraph{}
우리가 무언가를 가치있게 여기는 이유는 
희소하고(금, 다이아몬드, 시간),
생산하기 어렵거나 노동집약적이며, 대체할 수 없고(사랑하는 사람의 오래된 사진),
우리가 할 수 없던 일을 가능하게 해줄 정도로 유용하기 때문이거나, 
또는 이러한 이유들이 결합되어 있기 때문인 것 같다.


% \begin{comment}
% 	Bitcoin is all of the above: it is extremely rare (21 million),
% 	increasingly hard to produce (reward halvening), can't be replaced (a
% 	lost private key is lost forever), and enables us to do some quite
% 	useful things. It is arguably the best tool for value transfer across
% 	borders, virtually resistant to censorship and confiscation in the
% 	process, plus, it is a self-sovereign store of value, allowing
% 	individuals to store their wealth independent of banks and governments,
% 	just to name two.
% \end{comment}
\paragraph{}
비트코인은 앞서 언급한 모든 것을 갖추고 있다. 극도로 희귀하고(2,100만 개),
생산하기 점점 어려워지고 있고(반감기), 대체할 수 없으며(개인키 분실 시 영원히 손실),
매우 유용한 작업을 수행할 수 있게 해준다. 비트코인은 틀림없이 국경을 넘어 가치를 이전하는 최고의 도구이며 
그 과정에서 발생할 수 있는 검열과 몰수에 저항한다. 게다가 비트코인은 자기주권적 가치 저장소로서
개인이 은행과 정부로부터 독립적으로 온전히 자신의 부를 저장할 수 있게 해준다.

\paragraph{비트코인은 가치는 주관적이지만, 임의적이지 않다는 점을 가르쳐주었다.}

% ---
%
% #### Down the Rabbit Hole
%
% - [Euthydemus] by Plato
% - [Theory of Value][multiple theories], [Paradox of Value][paradox of value], [Subjective Theory of Value][subjective] on Wikipedia
%
% [Euthydemus]: http://www.perseus.tufts.edu/hopper/text?doc=Perseus:text:1999.01.0178:text=Euthyd.
% [Plato]: http://www.perseus.tufts.edu/hopper/text?doc=plat.+euthyd.+304b
%
% <!-- Wikipedia -->
% [multiple theories]: https://en.wikipedia.org/wiki/Theory_of_value_%28economics%29
% [paradox of value]: https://en.wikipedia.org/wiki/Paradox_of_value
% [subjective]: https://en.wikipedia.org/wiki/Subjective_theory_of_value
% [alice]: https://en.wikipedia.org/wiki/Alice%27s_Adventures_in_Wonderland
% [carroll]: https://en.wikipedia.org/wiki/Lewis_Carroll
