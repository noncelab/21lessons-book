\chapter{금융 무지}
\label{les:8}

\begin{chapquote}{루이스 캐롤, \textit{이상한 나라의 앨리스}}
	\enquote{그렇게 물어보면 나를 얼마나 무지한 아이로 생각하겠어. 난 물어보지 않을 거야. 이 어딘가 쓰여 있는 걸 찾아봐야겠어.}
\end{chapquote}

%One of the most surprising things, to me, was the amount of finance,
%economics, and psychology required to get a grasp of what at first
%glance seems to be a purely \textit{technical} system --- a computer network.
%To paraphrase a little guy with hairy feet: \enquote{It's a dangerous business,
	%Frodo, stepping into Bitcoin. You read the whitepaper, and if you don't
	%keep your feet, there's no knowing where you might be swept off to.}
가장 놀라운 사실은 언뜻 보기엔 순수한 컴퓨터 네트워크 시스템인 것처럼 보이는데,
이것을 이해하려면 금융, 경제학, 심리학이 필요하다는 것이다.
호빗의 말을 인용하자면 이렇다고 볼 수 있다. 
\enquote{프로도, 비트코인에 발을 
	들이는 것은 위험한 일이야, 백서를 읽고 발을 들이지 않는다면 어디로 휩쓸릴지 몰라.}

%To understand a new monetary system, you have to get acquainted with the
%old one. I began to realize very soon that the amount of financial
%education I enjoyed in the educational system was essentially \textit{zero}.
새로운 화폐 시스템을 이해하기 위해서는 기존의 시스템을 알아야 한다. 
나는 줄곧 이제까지 배웠던 금융 교육이 아무 소용이 없다는 것을 깨달아야 했다.

\paragraph{}
%Like a five-year-old, I began to ask myself a lot of questions: How does the
%banking system work? How does the stock market work? What is fiat money? What is
%\textit{regular} money? Why is there so much
%debt?\footnote{\url{https://www.usdebtclock.org/}} How much money is actually
%printed, and who decides that?
다섯 살의 아이처럼 질문이 참 많아졌다. 
은행 시스템은 어떻게 작동하지? 주식 시장은 어떻게 작동하지? 
명목화폐는 무엇이지? 평범한 화폐는 무엇이지? 왜 그렇게 부채가 많지?\footnote{\url{https://www.usdebtclock.org/}} 
얼마나 많은 돈이 인쇄되고 있고, 그 결정은 누가 하는 거지?

\newpage

%After a mild panic about the sheer scope of my ignorance, I found
%reassurance in realizing that I was in good company.
내가 몰랐던 사실에 대해 매우 당황스러웠지만,
좋은 대안이 있다는 사실에 안심할 수 있었다.

\begin{quotation}\begin{samepage}
		\enquote{비트코인이 내가 금융기관에서 일한 모든 세월보다 돈에 대해 더 많은 것을 가르쳐주었다는 사실이 
			아이러니 하지 않는가? \ldots 중앙은행에서의 모든 경력을 포함하여}
		\begin{flushright} -- 애런\footnote{Aaron (\texttt{@aarontaycc}, \texttt{@fiatminimalist}), tweet from Dec.
				12, 2018~\cite{aarontaycc-tweet}}
\end{flushright}\end{samepage}\end{quotation}

\begin{quotation}\begin{samepage}
		\enquote{나는 암호화폐를 통해서 금융, 경제, 기술, 암호학, 인간 심리학, 정치, 게임이론 그리고 나 자신에 대해서
			3년 반 동안 대학에서 배운 것보다 더 많은 것을 배웠다.}
		\begin{flushright} -- 더니\footnote{Dunny (\texttt{@BitcoinDunny}), tweet from Nov. 28,
				2017~\cite{bitcoindunny-tweet}}
\end{flushright}\end{samepage}\end{quotation}

%These are just two of the many confessions all over twitter.\footnote{See
	%\url{http://bit.ly/btc-learned} for more confessions on twitter.} Bitcoin, as
%was explored in Lesson \ref{les:1}, is a living thing. Mises argued that
%economics also is a living thing. And as we all know from personal experience,
%living things are inherently difficult to understand.
이것은 많은 사람이 트위터에서 밝힌 고백 중 극히 일부에 불과하다.\footnote{See
	\url{http://bit.ly/btc-learned}} 
지난 교훈에서 살펴본 것처럼 비트코인은 살아있다. 
미제스는 경제학은 살아있는 생물이라 주장했다. 
개개인의 경험을 통해서는 이 살아있는 생물을 이해하기는 어렵다.

\begin{quotation}\begin{samepage}
		\enquote{과학은 끝없이 진행되는 지식 탐구의 과정에 불과하다. 
			이 시스템은 모든 인간의 무지로부터 영향을 받는다. 
			그러나 이 무지를 인정한다고 해서 오늘날의 경제학이 후진적이라는 의미는 아니다.
			경제학은 생명체라는 것을 의미하고, 
			생명체는 불완전하다는 것과 변화한다는 것을 의미한다.}
		\begin{flushright} -- 루드비히 폰 미제스\footnote{Ludwig von Mises, \textit{Human Action}
				\cite{human-action}}
\end{flushright}\end{samepage}\end{quotation}

\newpage

%We all read about various financial crises in the news, wonder about how
%these big bailouts work and are puzzled over the fact that no one ever
%seems to be held accountable for damages which are in the trillions. I
%am still puzzled, but at least I am starting to get a glimpse of what is
%going on in the world of finance.
우리는 금융 위기를 뉴스에서 접할 때 왜 대규모 구제 금융이 필요한지 궁금해해야 하고, 아무도 수조 달러에 해당하는
피해에 책임을 지지 않는다는 사실을 의아해 해야 한다. 
나는 혼란에도 불구하고 적어도 금융의 세계에서 무슨 일이 일어나고 있는지 엿볼 수 있었다.

%Some people even go as far as to attribute the general ignorance on
%these topics to systemic, willful ignorance. While history, physics,
%biology, math, and languages are all part of our education, the world of
%money and finance surprisingly is only explored superficially, if at
%all. I wonder if people would still be willing to accrue as much debt as
%they currently do if everyone would be educated in personal finance and
%the workings of money and debt. Then I wonder how many layers of
%aluminum make an effective tinfoil hat. Probably three.
어떤 사람들은 이러한 사실에 대해 고의로 알려 하지 않는다. 
역사, 물리학, 생물학, 수학, 언어는 교육으로 다루어지지만, 
놀랍게도 돈과 금융에 대한 교육은 표면적으로만 이루어진다. 
모든 사람이 개인 금융과 돈, 부채의 작용에 대해 알게 된다면 사람들은 현재와 같은 부채를 기꺼이 용납할 수 있을지 궁금하다. 
그리고 몇 겹의 알루미늄이 은박지 모자\footnote{역자: 은박지 모자를 쓰면 정부의 감시나 외계인의 정신 통제를 피할 수 있다는 믿음이 있다.}
를 만드는 데에 필요한지 궁금하다.
아마 3겹일 것이다.

\begin{quotation}\begin{samepage}
		\enquote{이런 문제, 구제 금융은 사고가 아니다. 그리고 학교에서 금융 교육을 하지 않는 것도 우연이 아니다. [...] 그것은 계획된 것이다.
			남북 전쟁 이전에 노예를 교육하는 것이 불법이었던 것처럼 우리는 학교에서 돈에 대해 배울 수 없다.}
		\begin{flushright} -- 로버트 기요사키\footnote{Robert Kiyosaki, \textit{Why the Rich
					are Getting Richer}\cite{robert-kiyosaki}}
\end{flushright}\end{samepage}\end{quotation}


%Like in The Wizard of Oz, we are told to pay no attention to the man behind the
%curtain. Unlike in The Wizard of Oz, we now have real
%wizardry\footnote{\url{http://bit.ly/btc-wizardry}}: a censorship-resistant,
%open, borderless network of value-transfer. There is no curtain, and the magic
%is visible to anyone.\footnote{\url{https://github.com/bitcoin/bitcoin}}
오즈의 마법사에서 말하는 것처럼 우리는 커튼 뒤의 남자에게 관심을 두지 말라고 한다. 
하지만 오즈의 마법사와 달리
우리는 검열에 저항할 수 있고 개방적이며 국경이 없는 가치 전송 네트워크라는 마법\footnote{\url{http://bit.ly/btc-wizardry}}을 가지고 있다.
그 마법은 커튼이 없어서 누구나 볼 수 있습니다.\footnote{\url{https://github.com/bitcoin/bitcoin}}

\paragraph{비트코인은 커튼 뒤를 들여다봐야 한다는 것과 나의 금융 무지의 받아들여야 하는 것을 가르쳐주었다.}

% ---
%
% #### Down the Rabbit Hole
%
% - [Human Action][Ludwig von Mises] by Ludwig von Mises
% - [Why the Rich are Getting Richer][Robert Kiyosaki] by Robert Kiyosaki
%
% [real wizardry]: https://external-preview.redd.it/8d03MWWOf2HIyKrT8ThBGO4WFv-u25JaYqhbEO9b1Sk.jpg?width=683&auto=webp&s=dc5922d84717c6a94527bafc0189fd4ca02a24bb
% [visible to anyone]: https://github.com/bitcoin/bitcoin
%
% <!-- Wikipedia -->
% [alice]: https://en.wikipedia.org/wiki/Alice%27s_Adventures_in_Wonderland
% [carroll]: https://en.wikipedia.org/wiki/Lewis_Carroll
