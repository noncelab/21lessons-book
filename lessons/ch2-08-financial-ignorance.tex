\chapter{금융적 무지}
\label{les:8}

\begin{chapquote}{루이스 캐롤, \textit{이상한 나라의 앨리스}}
	\enquote{그렇게 물어보면 나를 얼마나 무지한 아이로 보겠어. 안돼, 그런 건 질문하지 말아야지. 어딘가 쓰여있는 걸 봐야겠어.}
\end{chapquote}

\paragraph{}
%One of the most surprising things, to me, was the amount of finance,
%economics, and psychology required to get a grasp of what at first
%glance seems to be a purely \textit{technical} system --- a computer network.
%To paraphrase a little guy with hairy feet: \enquote{It's a dangerous business,
	%Frodo, stepping into Bitcoin. You read the whitepaper, and if you don't
	%keep your feet, there's no knowing where you might be swept off to.}
가장 놀라웠던 점 중 하나는 언뜻 보기엔 순수하게 \textit{기술적} 시스템인 컴퓨터 네트워크를 이해하기 위해, 
방대한 금융, 경제학, 심리학을 이해해야만 한다는 것을 알게 된 것이다.
한 호빗의 말을 빌리자면 이렇다. 
\enquote{비트코인에 발을 들이는 것은 위험한 일이야, 프로도. 백서를 읽고 발을 떼지 않으면 어디로 휩쓸릴지 몰라.}

\paragraph{}
%To understand a new monetary system, you have to get acquainted with the
%old one. I began to realize very soon that the amount of financial
%education I enjoyed in the educational system was essentially \textit{zero}.
새로운 화폐 시스템을 이해하려면 기존 시스템을 알아야 한다.  
나는 이내 교육 시스템에서 내가 누리던 금융 교육이 본질적으로 0이라는 것을 깨닫기 시작했다. 


%Like a five-year-old, I began to ask myself a lot of questions: How does the
%banking system work? How does the stock market work? What is fiat money? What is
%\textit{regular} money? Why is there so much
%debt?\footnote{\url{https://www.usdebtclock.org/}} How much money is actually
%printed, and who decides that?
그리고는 다섯 살짜리 아이처럼 스스로에게 질문을 하기 시작했다. 
은행 시스템은 어떻게 작동하나? 주식 시장은 어떻게 작동하나? 
법정화폐란 무엇인가? 일반적인 화폐는 무엇인가? 빚이 왜 이렇게 많은가?\footnote{\url{https://www.usdebtclock.org/}} 
실제로 인쇄되는 돈의 양은 얼마나되며, 이를 결정하는 사람은 누구인가?

%\newpage 

%After a mild panic about the sheer scope of my ignorance, I found
%reassurance in realizing that I was in good company.
\paragraph{}
내 무지의 범위에 잠깐 당황했지만, 동료들이 있다는 사실에 안도했다.

\begin{quotation}\begin{samepage}
		\enquote{내가 금융기관에서 일한 지난 몇 년보다 비트코인이 더 많은 것을 가르쳐주었다는 사실이 
			아이러니하지 않습니까? \ldots 중앙은행에서 커리어를 시작한 걸 포함해서}
		\begin{flushright} -- 애런\footnote{Aaron (\texttt{@aarontaycc}, \texttt{@fiatminimalist}), tweet from Dec.
				12, 2018~\cite{aarontaycc-tweet}}
\end{flushright}\end{samepage}\end{quotation}

\begin{quotation}\begin{samepage}
		\enquote{나는 지난 3년 반의 대학 생활 동안 보다 암호화폐 분야에서의 최근 3개월 동안 
		금융, 경제, 기술, 암호학, 인간 심리학, 정치, 게임 이론, 입법 그리고 나 자신에 대해서 더 많이 배웠습니다.}
		\begin{flushright} -- 더니\footnote{Dunny (\texttt{@BitcoinDunny}), tweet from Nov. 28,
				2017~\cite{bitcoindunny-tweet}}
\end{flushright}\end{samepage}\end{quotation}

%These are just two of the many confessions all over twitter.\footnote{See
	%\url{http://bit.ly/btc-learned} for more confessions on twitter.} Bitcoin, as
%was explored in Lesson \ref{les:1}, is a living thing. Mises argued that
%economics also is a living thing. And as we all know from personal experience,
%living things are inherently difficult to understand.
\paragraph{}
이것은 트위터 전체에 퍼져있는 수 많은 고백 중 단 두개에 불과하다. \footnote{\url{http://bit.ly/btc-learned}} 
지난 교훈에서 살펴본 것처럼 비트코인은 살아있다.\ref{les:1}
미제스는 경제학도 살아있는 생물이라 주장했다. 
그리고 우리 모두 개인적 경험을 통해 알고 있듯이 살아있는 생물을 이해하기란 어렵다.

\begin{quotation}\begin{samepage}
		\enquote{과학 시스템은 끝없이 진보하는 지식 탐색의 한 지점에 불과하다. 
			그것은 필연적으로 모든 인간 노력에 내재된 부족함으로 인해 영향을 받는다.
			그러나 이러한 사실을 인정한다고 해서 오늘날의 경제학이 후진적이라는 의미는 아니다.
			이는 단지 경제학이 살아있는 것이라는 의미일 뿐이다. 
			그리고 산다는 것은 불완전하다는 것과 변화한다는 것을 동시에 의미한다.}
		\begin{flushright} -- 루드비히 폰 미제스\footnote{Ludwig von Mises, \textit{Human Action}
				\cite{human-action}}
\end{flushright}\end{samepage}\end{quotation}

%\newpage

%We all read about various financial crises in the news, wonder about how
%these big bailouts work and are puzzled over the fact that no one ever
%seems to be held accountable for damages which are in the trillions. I
%am still puzzled, but at least I am starting to get a glimpse of what is
%going on in the world of finance.
\paragraph{}
우리 모두 뉴스에서 다양한 금융 위기 소식을 접하고, 대규모 구제 금융이 어떻게 작동하는지 궁금해하며, 
수조 달러에 달하는 손해를 아무도 책임지지 않는다는 사실에 당황한다.
여전히 의아하지만 적어도 나는 금융의 세계에서 무슨 일이 일어나고 있는지 엿볼 수 있게 되었다.

%Some people even go as far as to attribute the general ignorance on
%these topics to systemic, willful ignorance. While history, physics,
%biology, math, and languages are all part of our education, the world of
%money and finance surprisingly is only explored superficially, if at
%all. I wonder if people would still be willing to accrue as much debt as
%they currently do if everyone would be educated in personal finance and
%the workings of money and debt. Then I wonder how many layers of
%aluminum make an effective tinfoil hat. Probably three.
\paragraph{}
혹자는 심지어 이러한 경제에 대한 무지가 체계적이고 고의적이라 말한다.
역사, 물리학, 생물학, 수학, 언어가 우리 교육의 일부인 반면, 
놀랍게도 돈과 금융의 세계는 피상적으로만 다루어진다. 
모든 사람이 개인 금융과 돈과 부채의 작동 원리에 대해 교육을 받고도 지금처럼 많은 빚을 지게 될지 궁금하다.
그렇다면 알루미늄을 몇 겹이나 겹쳐야 효과적인 은박 모자\footnote{역자: 은박지 모자를 쓰면 정부의 감시나 외계인의 정신 통제를 피할 수 있다는 믿음이 있다.}
를 만들 수 있을까? 아마 세 겹일 것이다.

\begin{quotation}\begin{samepage}
		\enquote{이러한 붕괴와 구제 금융은 우연이 아니다. 그리고 학교에서 금융 교육을 하지 않는 것도 우연이 아니다. [...] 계획된 것이다.
			남북전쟁 이전에 노예를 교육하는 것이 불법이었던 것처럼, 학교에서 돈에 대해 배우는 것은 허용되지 않는다.}
		\begin{flushright} -- 로버트 기요사키\footnote{Robert Kiyosaki, \textit{Why the Rich
					are Getting Richer}\cite{robert-kiyosaki}}
\end{flushright}\end{samepage}\end{quotation}

%Like in The Wizard of Oz, we are told to pay no attention to the man behind the
%curtain. Unlike in The Wizard of Oz, we now have real
%wizardry\footnote{\url{http://bit.ly/btc-wizardry}}: a censorship-resistant,
%open, borderless network of value-transfer. There is no curtain, and the magic
%is visible to anyone.\footnote{\url{https://github.com/bitcoin/bitcoin}}
\paragraph{}
오즈의 마법사에서처럼 세상은 우리에게 장막 뒤에 있는 사람에게 관심을 두지 말라고 한다.
하지만 오즈의 마법사와는 달리, 이제 우리는 검열에 저항하는 개방적이며 국경 없는 가치 전송 네트워크인 진짜 마법사\footnote{\url{http://bit.ly/btc-wizardry}}를 만나게 되었다.
커튼은 없고, 누구나 마법을 볼 수 있다.\footnote{\url{https://github.com/bitcoin/bitcoin}}

\paragraph{비트코인은 장막 뒤에서 나의 금융적 무지와 직면하게 해주었다.}

% ---
%
% #### Down the Rabbit Hole
%
% - [Human Action][Ludwig von Mises] by Ludwig von Mises
% - [Why the Rich are Getting Richer][Robert Kiyosaki] by Robert Kiyosaki
%
% [real wizardry]: https://external-preview.redd.it/8d03MWWOf2HIyKrT8ThBGO4WFv-u25JaYqhbEO9b1Sk.jpg?width=683&auto=webp&s=dc5922d84717c6a94527bafc0189fd4ca02a24bb
% [visible to anyone]: https://github.com/bitcoin/bitcoin
%
% <!-- Wikipedia -->
% [alice]: https://en.wikipedia.org/wiki/Alice%27s_Adventures_in_Wonderland
% [carroll]: https://en.wikipedia.org/wiki/Lewis_Carroll
