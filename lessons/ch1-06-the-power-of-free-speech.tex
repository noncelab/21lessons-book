\chapter{언론 자유의 힘}
\label{les:6}

\begin{chapquote}{루이스 캐롤, \textit{이상한 나라의 앨리스}}
	\enquote{다시 한번 말씀해 주시겠어요?} 생쥐가 얼굴을 찡그리며, 하지만 아주 공손하게 물었다. \enquote{뭐라고 하셨죠?}
\end{chapquote}

\paragraph{}
%Bitcoin is an idea. An idea which, in its current form, is the
%manifestation of a machinery purely powered by text. Every aspect of
%Bitcoin is text: The whitepaper is text. The software which is run by
%its nodes is text. The ledger is text. Transactions are text. Public and
%private keys are text. Every aspect of Bitcoin is text, and thus
%equivalent to speech.
비트코인은 아이디어다. 현재 형태로는 순수하게 문자로만 구동되는 기계식 표현이다.
비트코인의 모든 것이 문자로 이루어져 있다. 백서가 문자로 되어있다. 노드가 실행하는 소프트웨어도 문자이다. 원장도 문자, 트랜잭션도 문자이다. 
공개키와 개인키도 물론 문자이다. 비트코인의 모든 것이 문자로 구성되어 있어 언어와 동일하다.

\begin{quotation}\begin{samepage}
		\enquote{
			의회는 종교의 설립을 존중하거나 자유로운 종교 행사를 금지하거나 
			언론의 자유, 출판의 자유, 또는 평화롭게 집회할 수 있는 국민의 권리를 저해하거나,  
			고충의 구제를 위해 정부에 청원할 권리를 제한하는 법률을 제정할 수 없다.}
		\begin{flushright} -- 미국 수정헌법 제1조
\end{flushright}\end{samepage}\end{quotation}

\paragraph{}
%Although the final battle of the Crypto Wars\footnote{The \textit{Crypto Wars}
	%is an unofficial name for the U.S. and allied governments' attempts to undermine
	%encryption.~\cite{eff-cryptowars}~\cite{wiki:cryptowars}} 약ment tries to
%outlaw text or speech, we slip down a path of absurdity which inevitably leads
%to abominations like illegal numbers\footnote{An illegal number is a number that
	%represents information which is illegal to possess, utter, propagate, or
	%otherwise transmit in some legal jurisdiction.\cite{wiki:illegal-number}} and
%illegal primes\footnote{An illegal prime is a prime number that represents
	%information whose possession or distribution is forbidden in some legal
	%jurisdictions. One of the first illegal primes was found in 2001. When
	%interpreted in a particular way, it describes a computer program that bypasses
	%the digital rights management scheme used on DVDs. Distribution of such a
	%program in the United States is illegal under the Digital Millennium Copyright
	%Act. An illegal prime is a kind of illegal number.\cite{wiki:illegal-prime}}.
크립토 전쟁(the Crypto Wars)\footnote{\textit{크립토 전쟁}은 미국과 연합 정부의 암호화를 약화하려는 시도에 대한 비공식적인 이름이다.~\cite{eff-cryptowars}~\cite{wiki:cryptowars}} 
은 아직 끝나지 않았지만, 문자 메시지 교환을 기반으로 한 시도와 아이디어를 범죄로 규정하기란 매우 어려울 것이다.
정부가 문자나 말을 불법화하려고 할 때마다 우리는 부조리한 길로 빠지게 되고, 필연적으로 
불법 숫자\footnote{불법 숫자는 일부 법적 관할에서 소유, 발언, 전파 또는 전송하는 것이 금지된 숫자이다. 모든 디지털 정보는 숫자이다.	결과적으로 특정 정보 집합을 전송하는 것은 불법일 수 있다.\cite{wiki:illegal-number}}
나 불법 소수\footnote{불법 소수(prime number)는 불법 숫자의 범주에 해당된다. 최초의 불법 소수는 2001년 필 카모디가 DVD의 복제 방지를 우회하는 컴퓨터 프로그램이다. 이러한 프로그램을 미국에서 배포하는 것은 불법이다.\cite{wiki:illegal-prime}}
 같은 혐오스러운 결론으로 귀결되게 된다.

\paragraph{}
%As long as there is a part of the world where speech is free as in
%\textit{freedom}, Bitcoin is unstoppable.
언론이 자유로운 세상이 존재하는 한, 비트코인을 멈출 수 없다.

\begin{quotation}\begin{samepage}
		\enquote{비트코인 거래에서 비트코인이 문자가 아닌 경우는 없다. 항상 문자이다. [...]
			비트코인은 문자다. 고로 비트코인은 언어이다.
			미국처럼 양도할 수 없는 권리가 보장되고 수정헌법 1조에 따라 
			명시적으로 출판 행위를 정부 감시에서 제외하는 자유 국가에서는 이를 규제할 수 없다.}
		\begin{flushright} -- 뷰티온\footnote{Beautyon, \textit{미국이 비트코인을 규제할 수 없는 이유(Why America can't regulate Bitcoin)} \cite{america-regulate-bitcoin}}
\end{flushright}\end{samepage}\end{quotation}

\paragraph{비트코인은 자유 사회에서 언론의 자유와 자유 소프트웨어를 막을 수 없다는 것을 가르쳐주었다.}

% ---
%
% #### Through the Looking-Glass
%
% - [The Magic Dust of Cryptography: How digital information is changing our society][a magic spell]
%
% #### Down the Rabbit Hole
%
% - [Why America can't regulate Bitcoin][Beautyon] by Beautyon
% - [First Amendment to the United States Constitution][1st Amendment], [Crypto Wars], [illegal numbers], [illegal primes] on Wikipedia
%
% <!-- Through the Looking-Glass -->
% [a magic spell]: 
%
% <!-- Down the Rabbit Hole -->
% [1st Amendment]: https://en.wikipedia.org/wiki/First_Amendment_to_the_United_States_Constitution
% [Crypto Wars]: https://en.wikipedia.org/wiki/Crypto_Wars
% [illegal numbers]: https://en.wikipedia.org/wiki/Illegal_number
% [illegal primes]: https://en.wikipedia.org/wiki/Illegal_prime
% [Beautyon]: https://hackernoon.com/why-america-cant-regulate-bitcoin-8c77cee8d794
%
% <!-- Wikipedia -->
% [alice]: https://en.wikipedia.org/wiki/Alice%27s_Adventures_in_Wonderland
% [carroll]: https://en.wikipedia.org/wiki/Lewis_Carroll
