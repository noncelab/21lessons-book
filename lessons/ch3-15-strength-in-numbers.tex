\chapter{숫자의 강력함}
\label{les:15}

\begin{chapquote}{루이스 캐롤, \textit{이상한 나라의 앨리스}}
	\enquote{어디 보자, 
		4 곱하기 5는 20, 
		4 곱하기 6은 13, 
		4 곱하기 7은 14, 
		어쩌지! 이런 속도로는 20까지 가지도 못하겠어!}
\end{chapquote}

\begin{comment}
	Numbers are an essential part of our everyday life. Large numbers,
	however, aren't something most of us are too familiar with. The largest
	numbers we might encounter in everyday life are in the range of
	millions, billions, or trillions. We might read about millions of people
	in poverty, billions of dollars spent on bank bailouts, and trillions of
	national debt. Even though it's hard to make sense of these headlines,
	we are somewhat comfortable with the size of those numbers.
\end{comment}
숫자는 일상 생활에서 꼭 필요한 일부이다. 
그러나 우리 대다수는 큰 숫자에 익숙하지 않다. 
우리가 일상생활에서 접하는 큰 숫자는 수백만(millions), 수십억(billions) 또는 수조(trillions) 정도이다.
수백만 명의 빈곤층, 구제 금융을 위해 지출된 수십억 달러, 수조 달러에 달하는 국가 부채 등이 우리가 접할 수 있는 큰 수이다.
비록 이런 헤드라인을 정확히 이해하기는 어렵지만, 이런 숫자는 어느 정도 편안하게 받아들일 수 있다.

\begin{comment}
	Although we might seem comfortable with billions and trillions, our
	intuition already starts to fail with numbers of this magnitude. Do you
	have an intuition how long you would have to wait for a
	million/billion/trillion seconds to pass? If you are anything like me,
	you are lost without actually crunching the numbers.
\end{comment}
하지만 수십억, 수조에 달하는 수를 읽는 것엔 익숙할지 몰라도, 이 숫자가 어느 정도의 규모인지 직관적으로 이해가 안되기 시작한다.
백만/십억/조 초가 지날 때까지 얼마나 기다려야 하는지 파악이 되는가?
나처럼 당신도 일일이 계산해 보지 않고는 감이 안 올 것이다. 

\begin{comment}
	Let's take a closer look at this example: the difference between each is an
	increase by three orders of magnitude: $10^6$, $10^9$, $10^{12}$. Thinking about
	seconds is not very useful, so let's translate this into something we can wrap
	our head around:
\end{comment}
이 예를 좀 더 자세히 살펴보자. $10^6$, $10^9$, $10^{12}$ 이 세 숫자는 각각 세 자릿수 씩 증가한다.  
초 단위로 생각하는 것엔 그닥 익숙하지 않으니, 머리를 싸매고서라도 이해할 수 있는 수준으로 바꿔보겠다.

\begin{itemize}
	\item $10^6$: 100만 초 전은 $1 \frac{1}{2}$주 전에 해당한다.
	\item $10^9$: 10억 초 전은 32년 전에 해당한다.
	\item $10^{12}$: 1조 초 전에는 맨하탄이 빙하 속에 묻혀있었다.\footnote{1조($10^{12}$)초는 $31,710$ 년이다. 마지막 최대 빙하기(Last  Glacial Maximum)는 $33,000$ 년 전이다.~\cite{wiki:LGM}}
\end{itemize}

\begin{figure}
	\includegraphics{assets/images/xkcd-1225.png}
	\caption{약 1조 초 전 각 도시의 빙하 두께 (출처: xkcd 1225)}
	\label{fig:xkcd-1225}
\end{figure}

\begin{comment}
	As soon as we enter the beyond-astronomical realm of modern
	cryptography, our intuition fails catastrophically. Bitcoin is built
	around large numbers and the virtual impossibility of guessing them.
	These numbers are way, way larger than anything we might encounter in
	day-to-day life. Many orders of magnitude larger. Understanding how
	large these numbers truly are is essential to understanding Bitcoin as a
	whole.
\end{comment}
현대 암호학에서 사용하는 천문학적 수에 도달하면 수를 파악하는 우리의 직관은 거의 불능 상태가 된다.
비트코인은 천문학적으로 큰 숫자와 이를 추측하는 것이 불가능하다는 점을 활용한다.
비트코인이 사용하는 숫자는 우리가 일상생활에서 접할 수 있는 그 어떤 숫자보다 훨씬 더 크다. 자릿수가 훨씬 크단 뜻이다.
비트코인을 전체적으로 이해하기 위해서 이 숫자가 실제로 얼마나 큰지 이해하는 것이 중요하다.

\begin{comment}
	Let's take SHA-256\footnote{SHA-256 is part of the SHA-2 family of cryptographic
		hash functions developed by the NSA.~\cite{wiki:sha2}}, one of the hash
	functions\footnote{Bitcoin uses SHA-256 in its block hashing
		algorithm.~\cite{btcwiki:block-hashing}} used in Bitcoin, as a concrete example.
	It is only natural to think about 256 bits as \enquote{two hundred fifty-six,} which
	isn't a large number at all. However, the number in SHA-256 is talking about
	orders of magnitude --- something our brains are not well-equipped to deal with.
\end{comment}
비트코인에서 사용하는 해시 함수인 SHA-256\footnote{SHA-256은 NSA에서 개발한. SHA-2 계열의 암호학 해시함수이다.~\cite{wiki:sha2}}
을 예로 들어보자.\footnote{SHA-256은 비트코인에서 블록 해시 알고리즘에 사용된다.~\cite{btcwiki:block-hashing}} 
여기서 사용되는 숫자 256비트를 단순히 "이백오십육"이라고 생각한다면, 이는 결코 큰 숫자가 아니다.
하지만, SHA-256에서 256은 자릿수를 의미한다. 이 숫자는 우리의 두뇌에서 처리할 수 없는 규모이다.

\begin{comment}
	While bit length is a convenient metric, the true meaning of 256-bit
	security is lost in translation. Similar to the millions ($10^6$) and
	billions ($10^9$) above, the number in SHA-256 is about orders of magnitude
	($2^{256}$).
\end{comment}
비트(bit)의 길이는 편리한 측정 기준이지만, 숫자를 자릿수로 변환하는 과정에서 256비트 보안성의 진정한 의미가 손상된다.
수백만($10^6$) 및 수십억($10^9$)과 유사하게 SHA-256에서 사용하는 숫자의 자리수는 엄청나게 크다.($2^{256}$).
\begin{comment}
	So, how strong is SHA-256, exactly?
\end{comment}
그렇다면, SHA-256은 정확히 얼마나 강력할까?

\begin{comment}
	\begin{quotation}\begin{samepage}
			\enquote{SHA-256 is very strong. It's not like the incremental step from MD5
				to SHA1. It can last several decades unless there's some massive
				breakthrough attack.}
			\begin{flushright} -- Satoshi Nakamoto\footnote{Satoshi Nakamoto, in a reply to questions about SHA-256 collisions. \cite{satoshi-sha256}}
	\end{flushright}\end{samepage}\end{quotation}
\end{comment}
\begin{quotation}\begin{samepage}
		\enquote{SHA-256은 매우 강력합니다. MD5에서 SHA1로의 점진적 증가와는 차원이 다릅니다.
			대규모의 획기적인 해킹 공격이 발생하지 않는 한 수십 년 동안 끄떡없을 것입니다.}
		\begin{flushright} --사토시 나카모토\footnote{Satoshi Nakamoto, in a reply to questions about SHA-256 collisions. \cite{satoshi-sha256}}
\end{flushright}\end{samepage}\end{quotation}


%Let's spell things out. $2^{256}$ equals the following number:
$2^{256}$이 얼마나 큰 수인지 적어보자.
\footnote{
	저자: 원문에는 다음과 같이 표현되어 있다. \\
	115 quattuorvigintillion 792 trevigintillion 89 duovigintillion 237
	unvigintillion 316 vigintillion 195 novemdecillion 423 octodecillion 570
	septendecillion 985 sexdecillion 8 quindecillion 687 quattuordecillion 907
	tredecillion 853 duodecillion 269 undecillion 984 decillion 665 nonillion
	640 octillion 564 septillion 39 sextillion 457 quintillion 584 quadrillion 7
	trillion 913 billion 129 million 639 thousand 936.}

\begin{comment}
	\begin{quotation}\begin{samepage}
			115 quattuorvigintillion 792 trevigintillion 89 duovigintillion 237
			unvigintillion 316 vigintillion 195 novemdecillion 423 octodecillion 570
			septendecillion 985 sexdecillion 8 quindecillion 687 quattuordecillion 907
			tredecillion 853 duodecillion 269 undecillion 984 decillion 665 nonillion
			640 octillion 564 septillion 39 sextillion 457 quintillion 584 quadrillion 7
			trillion 913 billion 129 million 639 thousand 936.
	\end{samepage}\end{quotation}
\end{comment}
\begin{quotation}
	\begin{samepage}
		115,792,089,237,316,195,423,570,985,008,687,907,853,
		269,984,665,640,564,039,457,584,007,913,129,639,936
	\end{samepage}
\end{quotation}

\begin{comment}
	That's a lot of nonillions! Wrapping your head around this number is
	pretty much impossible. There is nothing in the physical universe to
	compare it to. It is far larger than the number of atoms in the
	observable universe. The human brain simply isn't made to make sense of
	it.
\end{comment}
엄청나게 큰 숫자이다. 이 숫자를 직관적으로 이해하는 것은 불가능하다. 
우리가 사는 우주의 어떤 숫자와도 비교할 수 없다.
관측할 수 있는 우주의 원자 수보다 훨씬 많다. 
인간의 두뇌로 인지하는 것이 불가능하다.

\newpage

\begin{comment}
	One of the best visualizations of the true strength of SHA-256 is a video by
	Grant Sanderson. Aptly named \textit{\enquote{How secure is 256 bit
			security?}}\footnote{Watch the video at \url{https://youtu.be/S9JGmA5_unY}} it
	beautifully shows how large a 256-bit space is. Do yourself a favor and take the
	five minutes to watch it. As all other \textit{3Blue1Brown} videos it is not
	only fascinating but also exceptionally well made. Warning: You might fall down
	a math rabbit hole.
\end{comment}
그랜트 샌더슨의 영상에서 SHA-256의 진정한 강력함이 훌륭하게 시각화되었다.
\enquote{256비트 보안은 얼마나 안전한가?(How secure is 256 bit
	security?)}\footnote{\url{https://youtu.be/S9JGmA5_unY}}에서는
256비트의 공간이 얼마나 큰지 아름답게 보여준다. 잠시 시간을 내어 5분 간 감상해보자.
3Blue1Brown에서 만든 다른 영상도 그렇지만 이 영상도 매우 흥미롭고 잘 만들어졌다.
단, 수학 토끼굴에 빠질 수 있으니 주의하라.

\begin{comment}
	\begin{figure}
		\includegraphics{assets/images/youtube-vid-inverted.png}
		\caption{Illustration of SHA-256 security. Original graphic by Grant Sanderson aka 3Blue1Brown.}
		\label{fig:youtube-vid-inverted}
	\end{figure}
\end{comment}
\begin{figure}
	\includegraphics{assets/images/youtube-vid-inverted.png}
	\caption{그랜트 샌더슨 영상의 SHA-256을 설명하기 위한 일러스트}
	\label{fig:youtube-vid-inverted}
\end{figure}

\begin{comment}
	Bruce Schneier~\cite{web:schneier} used the physical limits of computation to put this
	number into perspective: even if we could build an optimal computer,
	which would use any provided energy to flip bits perfectly~\cite{wiki:landauer}, build a
	Dyson sphere\footnote{A Dyson sphere is a hypothetical megastructure that completely encompasses a star and captures a large percentage of its power output.~\cite{wiki:dyson}} around our sun, and let it run for 100 billion billion
	years, we would still only have a $25\%$ chance to find a needle in a
	256-bit haystack.
\end{comment}
브루스 슈나이어(Bruce Schneier)~\cite{web:schneier}는 물리적으로 계산하는데에 한계가 있음을 활용하여 이 숫자를 입체적으로 표현했다.
태양 주위에 다이슨 구체\footnote{다이슨 구체는 행성을 완전히 덮어서 엄청난 전력을 생산하는 가상의 거대 구조물이다.~\cite{wiki:dyson}}
를 구축한 에너지원을 사용해 1000억 년 동안 작동시킬 수 있는 '비트를 완벽하게 뒤집는 최적의 컴퓨터'를 만들었다 해도 256비트 건초 더미에서 바늘을 찾을 확률은 $25\%$에 불과하다.

\begin{comment}
	\begin{quotation}\begin{samepage}
			\enquote{These numbers have nothing to do with the technology of the devices;
				they are the maximums that thermodynamics will allow. And they
				strongly imply that brute-force attacks against 256-bit keys will be
				infeasible until computers are built from something other than matter
				and occupy something other than space.}
			\begin{flushright} -- Bruce Schneier\footnote{Bruce Schneier, \textit{Applied Cryptography} \cite{bruce-schneier}}
	\end{flushright}\end{samepage}\end{quotation}
\end{comment}
\begin{quotation}\begin{samepage}
		\enquote{이 숫자는 장치들의 기술과는 아무런 관련이 없다. 
			이는 열역학이 허용하는 최대값이다.
			그리고 컴퓨터가 완전히 다른 물질로 만들어지고, 완전히 다른 차원의 공간을 차지하기 전까지...
			256비트 키에 대한 무차별 대입 공격은 불가능할 것임을 강력하게 암시하는 것이다.}
		\begin{flushright} -- 브루스 슈나이어\footnote{Bruce Schneier, \textit{응용 암호학(Applied Cryptography)} \cite{bruce-schneier}}
\end{flushright}\end{samepage}\end{quotation}

\begin{comment}
	It is hard to overstate the profoundness of this. Strong cryptography
	inverts the power-balance of the physical world we are so used to.
	Unbreakable things do not exist in the real world. Apply enough force,
	and you will be able to open any door, box, or treasure chest.
\end{comment}
이 심오함은 아무리 강조해도 지나치지 않다.  
강력한 암호학은 우리가 상상할 수 있는 물리 세계의 힘의 균형을 뒤집는다. 
현실 세계에서 깨지지 않는 것들은 존재하지 않는다. 
충분한 힘을 가하면, 그것이 문, 박스, 보물 상자인지를 불문하고 어떤 것이든 열 수 있다. 

\begin{comment}
	Bitcoin's treasure chest is very different. It is secured by strong
	cryptography, which does not give way to brute force. And as long as the
	underlying mathematical assumptions hold, brute force is all we have.
	Granted, there is also the option of a global \$5 wrench attack (Figure~\ref{fig:xkcd-538})
	But torture won't work for all bitcoin addresses, and the cryptographic
	walls of bitcoin will defeat brute force attacks. Even if you come at it
	with the force of a thousand suns. Literally.
\end{comment}
하지만 비트코인 보물 상자는 매우 다르다. 
이 상자는 어떠한 무차별 대입 공격에도 굴복하지 않는 강력한 암호학으로 보호된다. 
우리의 수학적 상식 안에서는 무차별 대입 공격 외엔 딱히 다른 공격 방법도 존재하지 않는다. 
물론 \$5 렌치 공격(Figure~\ref{fig:xkcd-538})도 방법일 순 있다. 
그러나 이러한 사람을 고문하는 종류의 공격은 비트코인 주소 공격과는 다르며 비트코인 암호의 벽은 무차별 공격을 물리칠 것이다. 
문자 그대로 천 개의 태양의 힘으로 공격하더라도 말이다. 

\begin{figure}
	\centering
	\includegraphics[width=8cm]{assets/images/xkcd-538.png}
	\caption{\$5 렌치공격 (출처: xkcd 538)}
	\label{fig:xkcd-538}
\end{figure}

\begin{comment}
	This fact and its implications were poignantly summarized in the call
	to cryptographic arms: \textit{\enquote{No amount of coercive force will ever solve
	a math problem.}
\end{comment}
이 사실과 그 의미는 암호화 무기의 소명(A Call to Cryptographic Arms)에 통렬하게 요약되어 있다.
\enquote{어떠한 물리력도 수학 문제를 풀 수 없다.}

\begin{comment}
	\begin{quotation}\begin{samepage}
			\enquote{It isn't obvious that the world had to work this way. But somehow the
				universe smiles on encryption.}
			\begin{flushright} -- Julian Assange\footnote{Julian Assange, \textit{A Call to Cryptographic Arms} \cite{call-to-cryptographic-arms}}
	\end{flushright}\end{samepage}\end{quotation}
\end{comment}
\begin{quotation}\begin{samepage}
		\enquote{세상이 이런 식으로 돌아가야 했는지는 잘 모르겠다. 하지만 어쨌든 우주는 암호화를 향해 미소짓는다.}
		\begin{flushright} -- 줄리안 어산지\footnote{Julian Assange, \textit{암호화 무기의 소명(A Call to Cryptographic Arms)} \cite{call-to-cryptographic-arms}}
\end{flushright}\end{samepage}\end{quotation}

\begin{comment}
	Nobody yet knows for sure if the universe's smile is genuine or not. It
	is possible that our assumption of mathematical asymmetries is wrong and
	we find that P actually equals NP \cite{wiki:pnp}, or we find surprisingly quick
	solutions to specific problems \cite{wiki:discrete-log} which we currently assume to be hard.
	If that should be the case, cryptography as we know it will cease to
	exist, and the implications would most likely change the world beyond
	recognition.
\end{comment}
우주의 미소가 진짜인지 아닌지 아직 누구도 확신할 수 없다.  
수학적 비대칭성에 대한 우리의 가정이 잘못되어 P가 실제로 NP\cite{wiki:pnp}라는 것을 발견하거나, 
이산 로그 문제\cite{wiki:discrete-log}를 놀라울 정도로 빠르게 계산할 수 있는 방법을 찾아낼 수도 있다. 
만약 그런 일이 발생한다면 우리가 알고 있는 암호학은 더 이상 사용할 수 없게 될 것이며, 
우리가 상상할 수 없을 정도로 세상이 변화될 가능성이 크다. 

\begin{quotation}\begin{samepage}
		\enquote{Vires in Numeris} = \enquote{Strength in Numbers}\footnote{\textit{Vires in Numeris} 는 bitcointalk 사용자인 \textit{epii}~\cite{epii}에 의해 
			비트코인 모토로 처음 제안되었다.}
\end{samepage}\end{quotation}

\begin{comment}
	\textit{Vires in numeris} is not only a catchy motto used by bitcoiners. The
	realization that there is an unfathomable strength to be found in
	numbers is a profound one. Understanding this, and the inversion of
	existing power balances which it enables changed my view of the world
	and the future which lies ahead of us.
\end{comment}
숫자의 힘(Vires in numeris)은 비단 비트코이너 만을 위한 모토가 아니다.
숫자에 헤아릴 수 없는 강력한 힘이 있음을 깨닫는 것은 심오한 것이다.
이것을 이해하고 이를 통해 기득권의 균형을 뒤집을 수 있음이 우리 앞에 놓여있다는 사실은
이 세계와 우리 앞에 놓인 미래에 대한 나의 시각을 바꾸어 놓았다.

\begin{comment}
	One direct result of this is the fact that you don't have to ask anyone for permission to participate in Bitcoin. 
	There is no page to sign up, no company in charge, no government agency to send application forms to.
	Simply generate a large number and you are pretty much good to go. 
	The central authority of account creation is mathematics. And God only knows who is in charge of that.
\end{comment}
이러한 숫자의 힘으로 비트코인에 참여하기 위해 누구에게도 허가받을 필요가 없다.
가입 페이지도 없고, 담당 회사도 없고, 신청서를 보낼 정부 기관도 없다.
간단히 큰 숫자를 하나 생성하면 거의 모든 작업이 완료된다. 
수학이 비트코인 계정 생성의 권한을 가진다. 그 책임자가 누구인지는 신만이 알고 있다.

\begin{figure}
	\includegraphics{assets/images/elliptic-curve-examples.png}
	\caption{타원곡선의 예시 (출처: Emmanuel Boutet)}
	\label{fig:elliptic-curve-examples}
\end{figure}

\begin{comment}
	Bitcoin is built upon our best understanding of reality. While there are
	still many open problems in physics, computer science, and mathematics,
	we are pretty sure about some things. That there is an asymmetry between
	finding solutions and validating the correctness of these solutions is
	one such thing. That computation needs energy is another one. In other
	words: finding a needle in a haystack is harder than checking if the
	pointy thing in your hand is indeed a needle or not. And finding the
	needle takes work.
\end{comment}
비트코인은 현실 세계에 대한 최선의 이해를 토대로 구축되었다.
물리학, 컴퓨터 과학, 수학에는 아직 해결되지 않은 문제가 많지만, 그럼에도 몇 가지 확실한 것이 있다.
그 중 하나는 어떤 문제의 해결책을 찾는 것과 그 해결책의 정확성을 검증하는 것이 비대칭적이라는 것이다.
계산에 에너지가 소요된다는 것도 확실하다.
즉, 건초 더미에서 바늘을 찾는 것은 내가 찾은 뾰족한 것이 실제 바늘인지 아닌지 확인하는 것보다 더 어려운 문제라는 것이다. 
그리고 바늘을 찾는 노력이 필요하다는 것도 말이다.

\begin{comment}
	The vastness of Bitcoin's address space is truly mind-boggling. The
	number of private keys even more so. It is fascinating how much of our
	modern world boils down to the improbability of finding a needle in an
	unfathomably large haystack. I am now more aware of this fact than ever.
\end{comment}
비트코인 주소 공간은 정말 놀라울 정도로 방대하다. 개인 키의 수는 훨씬 더 크다.
헤아릴 수 없을 만큼 큰 건초 더미에서 바늘을 찾는 것이 불가능하다라는 사실에 많은 사람들이 매료되고 있다.
나는 이제 그 어느 때보다 이 사실을 명확하게 알고 있다.

%\paragraph{Bitcoin taught me that there is strength in numbers.}
\paragraph{비트코인은 나에게 숫자는 강력하다는 것을 가르쳐주었다.}


% ---
%
% #### Down the Rabbit Hole
%
% - [How secure is 256 bit security?]["How secure is 256 bit security?"] by 3Blue1Brown
% - [Block Hashing Algorithm][hash functions] on the Bitcoin Wiki
% - [Last Glacial Maximum][thick layer of ice], [SHA-2][SHA-256], [Dyson Sphere][Dyson sphere], [Landauer's Principle][flip bits perfectly] [P versus NP][P actually equals NP], [Discrete Logarithm][specific problems] on Wikipedia
%
% [thick layer of ice]: https://en.wikipedia.org/wiki/Last_Glacial_Maximum
% [xkcd \#1125]: https://xkcd.com/1225/
% [SHA-256]: https://en.wikipedia.org/wiki/SHA-2
% [hash functions]: https://en.bitcoin.it/wiki/Block_hashing_algorithm
% ["How secure is 256 bit security?"]: https://www.youtube.com/watch?v=S9JGmA5_unY
% [Bruce Schneier]: https://www.schneier.com/
% [flip bits perfectly]: https://en.wikipedia.org/wiki/Landauer%27s_principle#Equation
% [Dyson sphere]: https://en.wikipedia.org/wiki/Dyson_sphere
% [2]: https://books.google.com/books?id=Ok0nDwAAQBAJ&pg=PT316&dq=%22These+numbers+have+nothing+to+do+with+the+technology+of+the+devices;%22&hl=en&sa=X&ved=0ahUKEwjXttWl8YLhAhUphOAKHZZOCcsQ6AEIKjAA#v=onepage&q&f=false
% [wrench attack]: https://xkcd.com/538/
% [call to cryptographic arms]: https://cryptome.org/2012/12/assange-crypto-arms.htm
% [P actually equals NP]: https://en.wikipedia.org/wiki/P_versus_NP_problem#P_=_NP
% [specific problems]: https://en.wikipedia.org/wiki/Discrete_logarithm#Cryptography
% [3Blue1Brown]: https://twitter.com/3blue1brown
%
% <!-- Wikipedia -->
% [alice]: https://en.wikipedia.org/wiki/Alice%27s_Adventures_in_Wonderland
% [carroll]: https://en.wikipedia.org/wiki/Lewis_Carroll
