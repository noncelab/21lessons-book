\chapter{무결점의 개념}
\label{les:5}

\begin{chapquote}{루이스 캐롤, \textit{이상한 나라의 앨리스}}
	\enquote{그들의 머리가 사라졌습니다!} 병사들이 외쳤다\ldots
\end{chapquote}

\paragraph{}
%Everyone loves a good origin story. The origin story of Bitcoin is a
%fascinating one, and the details of it are more important than one might
%think at first. Who is Satoshi Nakamoto? Was he one person or a group of
%people? Was he a she? Time-traveling alien, or advanced AI? Outlandish
%theories aside, we will probably never know. And this is important.
누구나 기원이 훌륭한 서사를 좋아한다. 
비트코인의 서사는 흥미진진하며 그 세세한 내용들은 생각보다 매우 중요하다. 
사토시 나카모토는 누구일까? 한 사람일까, 아니면 여러 사람일까? 남자일까 여자일까? 시간 여행을 하는 외계인일까, 그것도 아니라면 고도의 인공지능일까? 
엉뚱한 상상력을 동원하더라도 우리는 아마 그의 정체를 절대 알 수 없을 것이다.
바로 이 사실이 매우 중요하다.

\paragraph{}
%Satoshi chose to be anonymous. He planted the seed of Bitcoin. He stuck
%around for long enough to make sure the network won't die in its
%infancy. And then he vanished.
사토시는 익명을 선택했다. 그는 비트코인의 씨앗을 심었다. 
그는 비트코인 초창기에 네트워크가 안정화될 때까지 충분히 오랜 기간 머물러있었다. 
그리고는 사라졌다.

\paragraph{}
%What might look like a weird anonymity stunt is actually crucial for a
%truly decentralized system. No centralized control. No centralized
%authority. No inventor. No-one to prosecute, torture, blackmail, or
%extort. An immaculate conception of technology.
사실 진정한 탈중앙화 시스템을 위해서는 익명성이라는 특이한 기능이 매우 중요하다.
중앙화된 통제권이 없다. 중앙화된 권한도 없다. 발명가도 없다.
기소하거나 고문하거나 협박하거나 강탈할 대상이 아무도 없다. 
이는 기술적으로 흠결없이 완벽한 개념이다.

\begin{quotation}\begin{samepage}
		\enquote{가장 멋진 점은 사토시가 사라졌다는 것입니다.}
		\begin{flushright} -- 지미 송\footnote{Jimmy Song, \textit{Why Bitcoin is Different} \cite{bitcoin-different}}
\end{flushright}\end{samepage}\end{quotation}

\newpage

\paragraph{}
%Since the birth of Bitcoin, thousands of other cryptocurrencies were
%created. None of these clones share its origin story. If you want to
%supersede Bitcoin, you will have to transcend its origin story. In a war
%of ideas, narratives dictate survival.
비트코인 이후 수천 개의 암호화폐가 만들어졌다. 
이러한 복제품 중 어느 것도 비트코인과 같은 서사를 가진 것은 없다.
비트코인을 대체하려면 비트코인의 탄생 스토리를 초월해야 한다. 
아이디어 전쟁에서는 내러티브가 생존을 좌우한다.

\begin{quotation}\begin{samepage}
		\enquote{금은 7,000년 전 처음으로 보석으로 만들어졌고 물물교환에 사용되었다. 매혹적인 금의 광채때문에 신이 내린 선물로 여겨졌다.}
		\begin{flushright} 오스트리안 민트\footnote{The Austrian Mint, \textit{Gold: The Extraordinary Metal} \cite{gold-gift-gods}}
\end{flushright}\end{samepage}\end{quotation}

\paragraph{}
%Like gold in ancient times, Bitcoin might be considered a gift from the
%gods. Unlike gold, Bitcoins origins are all too human. And this time, we
%know who the gods of development and maintenance are: people all over
%the world, anonymous or not.
고대의 금처럼 비트코인을 신이 내린 선물로 여길 수 있다. 
하지만 금과 달리 비트코인의 기원은 너무나도 인간적이다. 
그리고 비트코인을 누가 개발하고 유지하는지 알 수 있다.
바로 익명이든 아니든 이 세상 모든 사람들이다.

\paragraph{비트코인은 나에게 서사가 중요하다는 것을 가르쳐주었다.}

% ---
%
% #### Down the Rabbit Hole
%
% - [Why Bitcoin is different][Jimmy Song] by Jimmy Song
% - [Gold: The Extraordinary Metal] by the Austrian Mint
%
% <!-- Down the Rabbit Hole -->
% [Jimmy Song]: https://medium.com/@jimmysong/why-bitcoin-is-different-e17b813fd947
% [Gold: The Extraordinary Metal]: https://www.muenzeoesterreich.at/eng/discover/for-investors/gold-the-extraordinary-metal
%
% <!-- Wikipedia -->
% [alice]: https://en.wikipedia.org/wiki/Alice%27s_Adventures_in_Wonderland
% [carroll]: https://en.wikipedia.org/wiki/Lewis_Carroll
