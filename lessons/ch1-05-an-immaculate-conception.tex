\chapter{무결점의 개념}
\label{les:5}

\begin{chapquote}{루이스 캐롤, \textit{이상한 나라의 앨리스}}
	\enquote{머리가 사라졌습니다} 군인들이 대답했다\ldots
\end{chapquote}

%Everyone loves a good origin story. The origin story of Bitcoin is a
%fascinating one, and the details of it are more important than one might
%think at first. Who is Satoshi Nakamoto? Was he one person or a group of
%people? Was he a she? Time-traveling alien, or advanced AI? Outlandish
%theories aside, we will probably never know. And this is important.
사람들은 서사를 좋아한다. 
비트코인의 서사는 흥미롭고 빠지면 빠질수록 생각했던 것보다 심오하다. 
사토시 나카모토는 누구인가? 그는 한 사람인가? 아니면 단체인가? 여자인가? 
시간 여행을 하는 외계인이거나 인공지능인가? 
많은 상상력을 동원하더라도 우리는 아마 그의 정체를 절대 알 수 없을 것이다.
그리고 이 사실은 매우 중요하다.

%Satoshi chose to be anonymous. He planted the seed of Bitcoin. He stuck
%around for long enough to make sure the network won't die in its
%infancy. And then he vanished.
사토시 나카모토는 익명을 선택했다. 
그는 비트코인의 씨앗을 심었다. 
그는 네트워크가 안정화될 때까지 충분히 기다렸다.
그리고 그는 사라졌다.

%What might look like a weird anonymity stunt is actually crucial for a
%truly decentralized system. No centralized control. No centralized
%authority. No inventor. No-one to prosecute, torture, blackmail, or
%extort. An immaculate conception of technology.
익명성은 우리에게 익숙하지 않지만, 탈중앙화된 시스템에서는 매우 중요하다.
주도권이 없다.
발명가도 없다. 
누구도 기소하거나 고문하거나 협박하거나 강탈할 수 없다. 
이는 기술적으로 완벽한 개념이다.

\begin{quotation}\begin{samepage}
		\enquote{가장 멋진 점은 사토시가 사라졌다는 것이다.}
		\begin{flushright} -- 지미 송\footnote{Jimmy Song, \textit{Why Bitcoin is Different} \cite{bitcoin-different}}
\end{flushright}\end{samepage}\end{quotation}

\newpage

%Since the birth of Bitcoin, thousands of other cryptocurrencies were
%created. None of these clones share its origin story. If you want to
%supersede Bitcoin, you will have to transcend its origin story. In a war
%of ideas, narratives dictate survival.
비트코인 이후 수많은 암호화폐가 만들어졌다. 
어떤 암호화폐도 서사는 없다.
비트코인을 대체하려면 비트코인 서사를 초월해야 한다. 
이 치열한 아이디어 전쟁에서 서사는 생존을 좌우한다.

\begin{quotation}\begin{samepage}
		\enquote{금은 처음으로 보석으로 만들어진 이후 7,000년 동안 물물교환에 사용되었다. 금의 매혹적인 광채는 금을 신의 선물로 착각하게 했다.}
		\begin{flushright} 오스트리안 민트\footnote{The Austrian Mint, \textit{Gold: The Extraordinary Metal} \cite{gold-gift-gods}}
\end{flushright}\end{samepage}\end{quotation}

%Like gold in ancient times, Bitcoin might be considered a gift from the
%gods. Unlike gold, Bitcoins origins are all too human. And this time, we
%know who the gods of development and maintenance are: people all over
%the world, anonymous or not.
우리는 비트코인을 고대의 금처럼 신이 내린 선물로 착각할 수 있다. 
하지만 금과 달리 비트코인은 너무 인간적이다. 
그리고 비트코인을 발전시키고 유지하는 신이 누군지 우리는 다 안다. 
이 세상 모든 사람이다. 그 사람들이 익명이든 아니든.

\paragraph{비트코인은 서사가 중요하다는 것을 가르쳐주었다.}

% ---
%
% #### Down the Rabbit Hole
%
% - [Why Bitcoin is different][Jimmy Song] by Jimmy Song
% - [Gold: The Extraordinary Metal] by the Austrian Mint
%
% <!-- Down the Rabbit Hole -->
% [Jimmy Song]: https://medium.com/@jimmysong/why-bitcoin-is-different-e17b813fd947
% [Gold: The Extraordinary Metal]: https://www.muenzeoesterreich.at/eng/discover/for-investors/gold-the-extraordinary-metal
%
% <!-- Wikipedia -->
% [alice]: https://en.wikipedia.org/wiki/Alice%27s_Adventures_in_Wonderland
% [carroll]: https://en.wikipedia.org/wiki/Lewis_Carroll
