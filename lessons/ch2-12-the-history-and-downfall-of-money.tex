\chapter{화폐 몰락의 역사}
\label{les:12}

\begin{chapquote}{루이스 캐롤, \textit{이상한 나라의 앨리스}}
	\enquote{불 속에 뛰어들면 화상을 입는다, 칼로 손가락을 깊게 자르면 대개 피가 난다 등
		친구들이 알려준 간단한 규칙도 잊지 않았고, '독'이라고 적힌 병을 마시면
		조만간 죽을게 확실하단 사실도 잊지 않았다고 하네요.}
\end{chapquote}

\begin{comment}
	Many people think that money is backed by gold, which is locked away in
	big vaults, protected by thick
	walls. This ceased to be true many decades ago. I am not sure what I
	thought, since I was in much deeper trouble, having virtually no
	understanding of gold, paper money, or why it would need to be backed by
	something in the first place.
\end{comment}

\paragraph{}
많은 사람들은 돈이 두꺼운 벽으로 감싸진 큰 금고에 보관된 금으로 뒷받침된다고 생각한다. 
이것은 수십 년 전부터 더 이상 사실이 아니다.
사실상 나는 금이나 지폐, 혹은 애초에 왜 돈이 무언가에 의해 뒷받침되어야 하는지를 거의 이해하지 못한 채 더 깊은 혼돈에 빠진 이후로 무슨 생각을 했는지 조차 모르겠다.

\begin{comment}
	One part of learning about Bitcoin is learning about fiat money: what it
	means, how it came to be, and why it might not be the best idea we ever
	had. So, what exactly is fiat money? And how did we end up using it?
\end{comment}

\paragraph{}
비트코인 교훈의 일부는 명목화폐에 대해 배우는 것이다. 
명목화폐가 무엇을 의미하는지, 어떻게 생겨났는지, 
그것이 최선이 아닐 수 있는지에 대해서 말이다. 
그렇다면 명목화폐는 정확히 무엇일까? 
그리고 우리는 어떻게 명목화폐를 사용하게 되었을까?

\begin{comment}
	If something is imposed by \textit{fiat}, it simply means that it is imposed by
	formal authorization or proposition. Thus, fiat money is money simply
	because \textit{someone} says that it is money. Since all governments use fiat
	currency today, this someone is \textit{your} government. Unfortunately, you
	are not \textit{free} to disagree with this value proposition. You will quickly
	feel that this proposition is everything but non-violent. If you refuse
	to use this paper currency to do business and pay taxes the only people
	you will be able to discuss economics with will be your cellmates.
\end{comment}

\paragraph{}
어떤 것이 \textit{명목적}이라는 의미는, 공식적인 승인이나 발의에 의해 부여받은 것임을 의미한다. 
따라서 명목화폐는 단순히 \textit{누군가가} 그것이 돈이라고 말했기 때문에 돈이 된 것이다. 
오늘날 모든 정부가 명목화폐를 사용하므로 여기서 그 \textit{누군가}는 바로 정부이다. 
안타깝게도 우리에겐 이러한 가치 제안을 거부할 자유가 없다.
이것이 폭력적이지 않을 뿐 모든 것을 제약하는 것임을 금방 알아차렸을 것이다.
사업을 하고 세금을 내는 데에 이 종이 돈을 사용하지 않겠다고 하면,
당신이 경제에 대해 논의할 수 있는 사람은 감방 동료 밖에 없을 것이다.

\begin{comment}
	The value of fiat money does not stem from its inherent properties. How
	good a certain type of fiat money is, is only correlated to the
	political and fiscal (in)stability of those who dream it into existence.
	Its value is imposed by decree, arbitrarily.
\end{comment}

\paragraph{}
명목화폐의 가치는 그 자체가 갖는 고유한 속성에서 비롯되지 않는다.
명목화폐 간의 상대적 가치는 그 명목화폐의 존속을 꿈꾸는 사람들의 정치적, 재정적 (불)안정성에 따라 결정된다.
명목화폐의 가치는 법령이 임의로 부과한다.

\begin{figure}
	\centering
	\includegraphics[width=8cm]{assets/images/fiat-definition.png}
	\caption{명목화폐의 사전적 의미 --- `그대로 될지어다.(Let it be done)'}
	\label{fig:fiat-definition}
\end{figure}

\paragraph{}
%Until recently, two types of money were used: \textbf{commodity money}, made
%out of precious \textit{things}, and \textbf{representative money}, which simply
%\textit{represents} the precious thing, mostly in writing.
최근까지도 두 종류의 화폐가 쓰였다.
하나는 귀중품으로 만든 \textbf{상품화폐(commodity money)}이고, 
다른 하나는 귀중품의 가치를 뒷받침하는 \textbf{대리화폐(representative money)}이다.
대다수 대리화폐의 가치는 화폐에 쓰인 문구에서 비롯된다.

\paragraph{}
\begin{comment}
	We already touched on commodity money above. People used special bones,
	seashells, and precious metals as money. Later on, mainly coins made out of
	precious metals like gold and silver were used as money. The oldest coin found
	so far is made of a natural gold-and-silver mix and was made more than 2700
	years ago.\footnote{According to the Greek historian Herodotus, writing in the
		fifth century BC, the Lydians were the first people to have used gold and silver
		coinage. \cite{coinage-origins}} If something is new in Bitcoin, the concept of
	a coin is not it.
\end{comment}
우리는 이미 상품화폐에 대해 다루었다. 
사람들은 특별한 뼈와 조개껍질, 귀금속을 화폐로 사용했다. 
그 후에는 주로 금과 은으로 동전을 만들어 사용했다.
지금까지 발견된 가장 오래된 동전은 2700년도 더 된 것으로 순수 금과 은의 혼합물로 만들어졌다. \footnote{기원전 15년 전에 쓰인 헤로도토스의 그리스 역사에 의하면 리디아인들은 금, 은 주화를 사용한 최초의 사람들이었다고 쓰여있다.\cite{coinage-origins}}
"동전(coin)"이라는 개념을 비트코인이 처음 도입한 것이 아니라는 뜻이다.

\newpage

\begin{figure}
	\centering
	\includegraphics[width=5cm]{assets/images/lydian-coin-stater.png}
	\caption{리디아의 일렉트럼 동전. (출처: Classical Numismatic Group)}
	\label{fig:lydian-coin-stater}
\end{figure}

\paragraph{}
\begin{comment}
	Turns out that hoarding coins, or hodling, to use today's parlance, is
	almost as old as coins. The earliest coin hodler was someone who put
	almost a hundred of these coins in a pot and buried it in the
	foundations of a temple, only to be found 2500 years later. Pretty good
	cold storage if you ask me.
\end{comment}
전을 모으는 것(hoarding), 시쳇말로 호들링은 동전의 역사 만큼이나 오래된 것으로 밝혀졌다. 
최초의 동전 호들러는 100여개의 동전을 냄비에 넣고 사원 기둥에 묻은 사람이었다. 
이것은 2,500년이 지난 후에야 발견되었다. 냉동보관(cold storage)이 꽤 괜찮은 방법이라 할 수 있겠다.

\paragraph{}
\begin{comment}
	One of the downsides of using precious metal coins is that they can be
	clipped, effectively debasing the value of the coin. New coins can be
	minted from the clippings, inflating the money supply over time,
	devaluing every individual coin in the process. People were literally
	shaving off as much as they could get away with of their silver dollars.
	I wonder what kind of \textit{Dollar Shave Club} advertisements they had back
	in the day.
\end{comment}
금속으로된 동전의 단점 중 하나는 동전의 귀퉁이를 잘라서 가치를 떨어뜨릴 수 있다는 점이다. 
잘라낸 조각으로 새 동전을 주조하면 화폐의 양이 늘어나기 때문에 시간이 지날수록 동전의 가치가 떨어질 수 있다.
당시 사람들은 은화를 가능한 한 최대한 깎아내고(shaving off) 사용하였다.
그 시대의 \textit{달러 셰이브 클럽}(Dollar Shave Club)\footnote{역주: 미국 면도기 제조 기업} 광고는 어땠을지 궁금해진다.

\paragraph{}
\begin{comment}
	Since governments are only cool with inflation if they are the ones
	doing it, efforts were made to stop this guerrilla debasement. In
	classic cops-and-robbers fashion, coin clippers got ever more creative
	with their techniques, forcing the \enquote{masters of the mint} to get even
	more creative with their countermeasures. Isaac Newton, the
	world-renowned physicist of \textit{Principia Mathematica} fame, used to be one
	of these masters. He is attributed with adding the small stripes at the
	side of coins which are still present today. Gone were the days of easy
	coin shaving.
\end{comment}
정부는 정부 주도의 인플레이션에는 관대하지만, 
이러한 민간에서 행해지는 인플레이션인 동전 자르기 행위는 두고 볼 수 없었다. 
예전에 유행하던 경찰과 강도 놀이처럼, 
동전 자르기는 점점 창의적으로 자행되었기 때문에 조폐국은 이를 능가하는 창의성을 발휘해야만 했다.
당시 \textit{수학 원리(Principia Mathematica)}로 명성을 얻은 세계적인 물리학자 아이작 뉴턴은 이를 해결한 사람 중 한 명이다. 
동전 옆 테두리에 톱니바퀴 무늬를 추가하여 이를 해결하였는데, 이는 오늘날까지도 적용되고 있다.
이것으로 동전 자르기의 시대는 막을 내리게 된다.

\begin{figure}
	\includegraphics{assets/images/clipped-coins.png}
	\caption{심각하게 잘린 은화}
	\label{fig:clipped-coins}
\end{figure}

\paragraph{}
\begin{comment}
	Even with these methods of coin debasement\footnote{Besides clipping, sweating
		(shaking the coins in a bag and collecting the dust worn off) and plugging
		(punching a hole in the middle and hammering the coin flat to close the hole)
		were the most prominent methods of coin debasement. \cite{wiki:coin-debasement}}
	kept in check, coins still suffer from other issues. They are bulky and not very
	convenient to transport, especially when large transfers of value need to
	happen. Showing up with a huge bag of silver dollars every time you want to buy
	a Mercedes isn't very practical.
\end{comment}
이러한 가치 하락 방법\footnote{자르기 외에도 스웨팅(가방에 있는 동전을 흔들어 발생한 가루를 모으는 것),
	플러깅(가운데 구멍을 뚫고 동전을 납작하게 두드려 구멍을 막는 것) 등이 자주 발생하였다.
	\cite{wiki:coin-debasement}}
을 억제한다해도 동전은 여전히 다른 문제와 맞닥뜨리고 있다. 
특히 대량의 가치를 이전해야할 때, 동전을 사용하기엔 부피가 크고 운송이 그리 편하지 않다.
벤츠를 사고 싶을 때마다 엄청난 양의 은화가 든 가방을 들고 나타난다고 상상해보자.

\paragraph{}
\begin{comment}
	Speaking of German things: How the United States \textit{dollar} got its name is
	another interesting story. The word \enquote{dollar} is derived from the German word
	\textit{Thaler}, short for a \textit{Joachimsthaler}~\cite{wiki:thaler}. A
	Joachimsthaler was a coin minted in the town of \textit{Sankt Joachimsthal}.
	Thaler is simply a shorthand for someone (or something) coming from the valley,
	and because Joachimsthal was \textit{the} valley for silver coin production,
	people simply referred to these silver coins as \textit{Thaler.} Thaler (German)
	morphed into daalders (Dutch), and finally dollars (English).
\end{comment}
독일 얘기가 나온 참에, 독일인들이 말하는 미국 \textit{달러}가 어떻게 달러라는 이름을 갖게 됐는지에 대한 이야기가 매우 흥미롭다.
\enquote{달러(dollar)}라는 단어는 독일어 요하킴스탈러(Joachimsthaler)의 줄임말인 탈러(Thaler)에서 유래되었다고 한다\cite{wiki:thaler}. 
요하킴스탈러는 상트 요하킴스탈(Sankt Joachimsthal)\footnote{역주: 현재는 체코에 있으며 현지어로는 야히모프(Jáchymov)라 불린다.}이라는 마을에서 주조된 동전이었다.
탈러는 단순히 '요하킴스탈러 계곡 출신의 사람'의 약칭이었고 요하킴스탈러는 은화를 생산하는 계곡이었기 때문에 자연스럽게 은화를 탈러라고 부르게 되었다. 
요하킴스탈에서 온 누군가가 요하킴스탈러를 탈러라고 줄여 부르기 시작했고, 
독일어 탈러(Thaler)가 네덜란드어 달더스(daalders)로 변형되어 최종적으로 영어 달러(dollars)가 되었다고 한다.

\begin{figure}
	\centering
	\includegraphics[width=5cm]{assets/images/joachimsthaler.png}
	\caption{달러의 기원. 마법사의 모자와 로브를 입은 성 요하킴스가 그려져 있다. (출처: 위키피디아)}
	\label{fig:joachimsthaler}
\end{figure}

\paragraph{}
\begin{comment}
	The introduction of representative money heralded the downfall of hard
	money. Gold certificates were introduced in 1863, and about fifteen
	years later, the silver dollar was also slowly but surely being replaced
	by a paper proxy: the silver certificate. \cite{wiki:silver-certificate}
\end{comment}
대리화폐의 등장은 경화(hard money)의 몰락을 예고했다. 
금 증서는 1863년에 도입되었고 약 15년 후 은 달러도 느리지만 확실하게 종이 위임장 형태인 은 증서(silver certificate)로 대체되었다.\cite{wiki:silver-certificate}

\paragraph{}
\begin{comment}
	It took about 50 years from the introduction of the first silver
	certificates until these pieces of paper morphed into something that we
	would today recognize as one U.S. dollar.
\end{comment}
최초의 은 증서가 도입된 후 이 종이 조각이 
오늘날 우리가 미국 달러 1달러로 인식할 수 있는 형태로 변하기까지는 약 50년이 걸렸다.

\begin{figure}
	\centering
	\includegraphics{assets/images/us-silver-dollar-note-smaller.png}
	\caption{1928년의 미국의 은 1 달러화. `Payable to the bearer on demand.(요청 시 소지인에게 지급됨)'라고 적혀있다. (출처: 스미스소니언 재단 국립 화폐 컬렉션)}
	\label{fig:us-silver-dollar-note-smaller}
\end{figure}

\paragraph{}
\begin{comment}
	Note that the 1928 U.S. silver dollar in
	Figure~\ref{fig:us-silver-dollar-note-smaller} still goes by the name of
	\textit{silver certificate}, indicating that this is indeed simply a document
	stating that the bearer of this piece of paper is owed a piece of silver. It is
	interesting to see that the text which indicates this got smaller over time. The
	trace of \enquote{certificate} vanished completely after a while, being replaced
	by the reassuring statement that these are federal reserve notes.
\end{comment}
그림 \ref{fig:us-silver-dollar-note-smaller}의 1928년 미국 은화는 은 증서(silver certificate)라는
이름으로 사용되었으며 실제로 이 종이를 소지한 사람이 은의 소유자라는 것을 나타내었다. 
시간이 지남에 따라 이 문구가 점점 작아지는 것이 매우 흥미롭다. 
이 \enquote{증서(certificate)} 안 문구는 오래지않아 연방준비금이라는 문구로 대체되며 완전히 사라졌다.

\paragraph{}
\begin{comment}
	As mentioned above, the same thing happened to gold. Most of the world was on a
	bimetallic standard~\cite{wiki:bimetallism}, meaning coins were made
	primarily of gold and silver. Having certificates for gold, redeemable in gold
	coins, was arguably a technological improvement. Paper is more convenient,
	lighter, and since it can be divided arbitrarily by simply printing a smaller
	number on it, it is easier to break into smaller units.
\end{comment}
금에서도 같은 일이 일어났다. 
세계 대부분의 나라들은 주로 바이메탈 표준~\cite{wiki:bimetallism}에 따라 금화와 은화를 만들었다. 
금으로 교환할 수 있는 금 교환증이 있다는 것은 틀림없는 기술적 진보이다.
종이는 더 편하고 가볍다. 또 더 작은 숫자를 인쇄하면 가치를 임의로 나눌 수 있기 때문에 더 작은 단위 결제도 가능하다.

\paragraph{}
\begin{comment}
	To remind the bearers (users) that these certificates were
	representative for actual gold and silver, they were colored accordingly
	and stated this clearly on the certificate itself. You can fluently read
	the writing from top to bottom:
\end{comment}
소지자에게 이 증서가 실제 금과 은을 보유하고 있다는 것을 상기시키기 위해 지폐에는 이를 나타내는 색깔이 칠해졌고, 
증서 자체에 이에 대한 내용이 명기되었다. 여러분은 이 문구를 유창하게 읽을 수 있을 것이다.

\begin{comment}
	\begin{quotation}\begin{samepage}
			\enquote{This certifies that there have been deposited in the treasury of the
				United States of America one hundred dollars in gold coin payable to
				the bearer on demand.}
	\end{samepage}\end{quotation}
\end{comment}
\begin{quotation}\begin{samepage}
		\enquote{이 증서는 요구시 소지인에게 지불할 금화 100 달러가 미국 재무부에 예치되어 있음을 증명한다.}
\end{samepage}\end{quotation}

\begin{comment}
	\begin{figure}
		\centering
		\includegraphics{assets/images/us-gold-cert-100-smaller.png}
		\caption{A 1928 U.S. \$100 gold certificate. Picture cc-by-sa National Numismatic Collection, National Museum of American History.}
		\label{fig:us-gold-cert-100-smaller}
	\end{figure}
\end{comment}
\begin{figure}
	\centering
	\includegraphics{assets/images/us-gold-cert-100-smaller.png}
	\caption{1928년 미화 금 100 달러화 (출처: 미국 국립 박물관 국립 화폐 컬렉션)}
	\label{fig:us-gold-cert-100-smaller}
\end{figure}

\paragraph{}
\begin{comment}
	In 1963, the words \enquote{PAYABLE TO THE BEARER ON DEMAND} were removed from
	all newly issued notes. Five years later, the redemption of paper notes
	for gold and silver ended.
\end{comment}
1963년, 새로 발행된 모든 지폐에서 \enquote{PAYABLE TO THE BEARER ON DEMAND(요청 시 소지인에게 지급됨)}라는 문구가 제거되었다. 
그로부터 5년 후, 금과 은을 통한 달러의 속박은 막을 내리게 된다.

\paragraph{}
\begin{comment}
	The words hinting on the origins and the idea behind paper money were
	removed. The golden color disappeared. All that was left was the paper
	and with it the ability of the government to print as much of it as it
	wishes.
\end{comment}
지폐의 기원과 목적을 암시하는 단어는 삭제되었다. 황금색이 사라졌다. 
남은 것은 종이와 정부가 원하는 만큼 인쇄할 수 있는 능력뿐이었다.

\paragraph{}
\begin{comment}
	With the abolishment of the gold standard in 1971, this century-long
	sleight-of-hand was complete. Money became the illusion we all share to
	this day: fiat money. It is worth something because someone commanding
	an army and operating jails says it is wort능h something. As can be
	clearly read on every dollar note in circulation today, \enquote{THIS NOTE IS
		LEGAL TENDER}. In other words: It is valuable because the note says so.
\end{comment}
1971년 금본위제가 폐지되면서 한 세기에 걸친 속임수가 완성되었다. 
돈은 오늘날 우리 모두가 착각하는 환상, 즉 명목화폐가 되었다. 
군대를 지휘하고 감옥을 운영하는 자가 그것이 가치있다고 말함으로 인해 그 가치를 갖게 됐다. 
오늘날 유통되는 모든 달러에 적혀 있듯이 이 지폐는 법정 화폐이다. 
\enquote{THIS NOTE IS LEGAL TENDER(이 지폐는 법적인 화폐이다.)}
즉, 이 한 줄이 달러의 가치를 부여한다.

\begin{comment}
	\begin{figure}
		\centering
		\includegraphics{assets/images/us-dollar-2004.jpg}
		\caption{A 2004 series U.S. twenty dollar note used today. `THIS NOTE IS LEGAL TENDER'}
		\label{fig:us-dollar-2004}
	\end{figure}
\end{comment}
\begin{figure}
	\centering
	\includegraphics{assets/images/us-dollar-2004.jpg}
	\caption{2004년 현재의 미국 20달러. `THIS NOTE IS LEGAL TENDER'라고 적혀있다.}
	\label{fig:us-dollar-2004}
\end{figure}

\paragraph{}
\begin{comment}
	By the way, there is another interesting lesson on today's bank notes,
	hidden in plain sight. The second line reads that this is legal tender
	\enquote{FOR ALL DEBTS, PUBLIC AND PRIVATE}. What might be obvious to economists
	was surprising to me: All money is debt. My head is still hurting
	because of it, and I will leave the exploration of the relation of money
	and debt as an exercise to the reader.
\end{comment}
그건 그렇고, 눈에 잘 띄진 않지만, 오늘날의 지폐에는 흥미로운 교훈이 하나 더 있다. 
두 번째 줄에 적힌 \enquote{FOR ALL DEBTS, PUBLIC AND PRIVATE(공적 및 사적인 모든 부채를 위하여)}라는 문구가 그것이다. 
이 말인즉 모든 돈은 빚이라는 의미이다. 경제학자들은 알고 있었을지 모르나 나에게는 새로운 사실이었다. 
이것 때문에 아직도 머리가 아프다. 돈과 빚의 관계는 여러분에게 숙제로 남겨두겠다.

\paragraph{}
\begin{comment}
	As we have seen, gold and silver were used as money for millennia. Over
	time, coins made from gold and silver were replaced by paper. Paper
	slowly became accepted as payment. This acceptance created an
	illusion --- the illusion that the paper itself has value. The final
	move was to completely sever the link between the representation and the
	actual: abolishing the gold standard and convincing everyone that the
	paper in itself is precious.
\end{comment}
앞서 살펴본 것처럼 금과 은은 수천 년 동안 돈으로 사용되었다. 
시간이 지나면서 금과 은으로 만든 동전은 종이로 대체되었다. 
종이는 서서히 지불 수단으로 받아들여졌다. 
이러한 수용은 종이 자체에 가치가 있다는 환상을 만들어냈다. 
최종적으로 대리와 실체를 연결하는 고리는 완전히 끊어졌다.
즉 금본위제는 폐지되었고 종이 자체에 가치가 있다는 것을 모두가 확신하게 되었다.

\begin{comment}
	\paragraph{Bitcoin taught me about the history of money and the greatest sleight of
		hand in the history of economics: fiat currency.}
\end{comment}
\paragraph{비트코인은 나에게 돈의 역사와 경제학 역사상 가장 큰 속임수인 명목화폐에 대해 가르쳐주었다.}

% ---
%
% #### Down the Rabbit Hole
%
% - [Shelling Out: The Origins of Money] by Nick Szabo
% - [Methods of Coin Debasement][coin debasement], [Thaler], [U.S. Silver Certificate][silver certificates], [Bimetallism][bimetallic standard] on Wikipedia
%
% [oldest coin]: https://www.britishmuseum.org/explore/themes/money/the_origins_of_coinage.aspx
% [coin debasement]: https://en.wikipedia.org/wiki/Methods_of_coin_debasement
% [Thaler]: https://en.wikipedia.org/wiki/Thaler
% [Berlin-George]: https://en.wikipedia.org/wiki/File:Bohemia,_Joachimsthaler_1525_Electrotype_Copy._VF._Obverse..jpg
% [silver certificates]: https://en.wikipedia.org/wiki/Silver_certificate_%28United_States%29
% [bimetallic standard]: https://en.wikipedia.org/wiki/Bimetallism
% [Shelling Out: The Origins of Money]: https://nakamotoinstitute.org/shelling-out/
%
% <!-- Wikipedia -->
% [alice]: https://en.wikipedia.org/wiki/Alice%27s_Adventures_in_Wonderland
% [carroll]: https://en.wikipedia.org/wiki/Lewis_Carroll
