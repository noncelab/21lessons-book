\chapter{부분 지급준비금의 광기}
\label{les:13}

\begin{chapquote}{루이스 캐롤, \textit{이상한 나라의 앨리스}}
	아아! 너무 늦었어. 그녀는 계속 커지고 커져서 곧 바닥에 무릎을 꿇어야 했다.
	잠시 후에는 이마저도 비좁아졌고, 한쪽 팔꿈치를 문에 대고 누워서 다른 팔로 머리를 감쌌다.
	그러고도 계속 커져서 최후의 방법으로 한 팔은 창 밖으로, 한 발은 굴뚝 위로 내밀면서
	혼잣말을 했다. \enquote{이제 더 이상 할 수 있는게 없어. 난 어떻게 되는 거지?}
\end{chapquote}

\begin{comment}
	Value and money aren't trivial topics, especially in today's times. The
	process of money creation in our banking system is equally non-trivial,
	and I can't shake the feeling that this is deliberately so. What I have
	previously only encountered in academia and legal texts seems to be
	common practice in the financial world as well: nothing is explained in
	simple terms, not because it is truly complex, but because the truth is
	hidden behind layers and layers of jargon and \textit{apparent} complexity.
	\enquote{Expansionary monetary policy, quantitative easing, fiscal stimulus to
		the economy.} The audience nods along in agreement, hypnotized by the
	fancy words.
\end{comment}
가치와 돈은 특히 오늘날과 같은 시대에 사소한 주제가 아니다. 
은행 시스템에서 돈을 창출하는 과정 또한 사소하지 않은데 나는 이것이 지극히 의도적이었다는 의심을 지울 수 없다. 
학계와 법조문에서만 접했던 것이 금융계에서도 일반적 관행으로 쓰이는 것 같다.
진짜로 복잡해서가 아니라, 여러 겹의 전문 용어와 겉으로 드러나는 복잡성 뒤에 드러내고 싶지 않은 진실을 숨기고 있기 때문에 그 어떤 것도 쉽게 설명하지 않는다는 것이다.
\enquote{통화 확장 정책, 양적 완화, 재정 부양책} 
청중은 화려한 단어에 최면에 걸린 듯 고개를 끄덕이며 동의할 뿐이다.

\begin{comment}
	Fractional reserve banking and quantitative easing are two of those
	fancy words, obfuscating what is really happening by masking it as
	complex and difficult to understand. If you would explain them to a
	five-year-old, the insanity of both will become apparent quickly.
\end{comment}
부분 지급 준비 은행과 양적 완화라는 이 두 멋진 용어는 
복잡하고 이해하기 어려운 단어로 위장하여 실제 상황을 흐리게 만든다. 
5살짜리 아이에게 설명한다면 두 가지 모두 미친 짓이라는 것이 금세 드러날 것이다.

\begin{comment}
	Godfrey Bloom, addressing the European Parliament during a joint
	debate, said it way better than I ever could:
\end{comment}
고드프레이 블룸(Godfrey Bloom)은 유럽 의회의 토론에서 이를 훨씬 잘 설명했다.

\begin{comment}
	\begin{quotation}\begin{samepage}
			\enquote{[...] you do not really understand the concept of banking. All the
				banks are broke. Bank Santander, Deutsche Bank, Royal Bank of
				Scotland --- they're all broke! And why are they broke? It isn't an
				act of God. It isn't some sort of tsunami. They're broke because we
				have a system called `fractional reserve banking' which means that
				banks can lend money that they don't actually have! It's a criminal
				scandal and it's been going on for too long. [...]
				We have counterfeiting --- sometimes called quantitative
				easing --- but counterfeiting by any other name. The artificial
				printing of money which, if any ordinary person did, they'd go to
				prison for a very long time [...] and until we start sending
				bankers --- and I include central bankers and politicians --- to
				prison for this outrage it will continue.}
			\begin{flushright} -- Godfrey Bloom\footnote{Joint debate on the
					banking union~\cite{godfrey-bloom}}
	\end{flushright}\end{samepage}\end{quotation}
\end{comment}
\begin{quotation}\begin{samepage}
		\enquote{[...] 당신은 은행의 개념을 이해하지 못하고 있습니다. 
			모든 은행이 파산했습니다. 산탄데르 은행, 도이치 은행, 스코틀랜드 왕립은행 등 모두 파산했어요!
			왜 파산했을까요? 쓰나미 같은 천재지변이 아닙니다. 
			실제로 가지고 있지도 않은 돈을 빌려줄 수 있는 '부분 지급 준비금 은행'이라는 시스템 때문에 파산한 것입니다! 
			이것은 범죄이고 너무 오랫동안 계속되어 왔습니다.[...]	
			우리는 양적 완화라고도 불리는 위조를 하고있지만 다른 이름으로도 위조를 하고 있습니다.
			인위적으로 돈을 찍어내는 행위를 일반인이 저질렀다면 아주 오랫동안 감옥에 갇히게 될테죠.[...] 
			그리고 우리가 중앙은행가와 정치인들을 포함한 은행가들을 감옥에 보내기 전까지 이 분노는 계속될 것입니다.}
		\begin{flushright} -- 고드프레이 블룸\footnote{Joint debate on the
				banking union~\cite{godfrey-bloom}}
\end{flushright}\end{samepage}\end{quotation}

\begin{comment}
	Let me repeat the most important part: banks can lend money that they
	don't actually have.
\end{comment}
가장 중요한 부분이니 다시 한번 강조한다.
은행은 실제 가지고 있지도 않은 돈을 빌려줄 수 있다.

\begin{comment}
	Thanks to fractional reserve banking, a bank only has to keep a small
	\textit{fraction} of every dollar it gets. It's somewhere between $0$ and $10\%$,
	usually at the lower end, which makes things even worse.
\end{comment}
은행은 부분 준비금 제도 덕분에 덕분에 극히 \textit{일부}의 달러만 보유하면 된다. 
보통 $0$에서 $10\%$ 사이로, 낮기 때문에 상황은 더욱 악화된다.

\begin{comment}
	Let's use a concrete example to better understand this crazy idea: A
	fraction of $10\%$ will do the trick and we should be able to do all the
	calculations in our head. Win-win. So, if you take \$100 to a
	bank --- because you don't want to store it under your mattress --- they
	only have to keep the agreed upon \textit{fraction} of it. In our example that
	would be \$10, because 10\% of \$100 is \$10. Easy, right?
\end{comment}
이 정신나간 아이디어를 더 잘 이해하기 위해 구체적인 예를 들어보겠다. 
단순한 암산만 하면 이해할 수 있다. 준비율이 $10\%$라고 가정해 보자.
당신은 \$100를 가지고 있고, 이를 침대 밑에 보관하고 싶진 않다.
그래서 \$100를 은행에 맡긴다. 그러면 은행은 일부만 금고에 보관하면 된다.
준비율 10\%를 적용하면 \$10이기 때문에 \$10만 금고에 보관하면 되는 것이다.  
간단하다.

\begin{comment}
	So what do banks do with the rest of the money? What happens to your \$90? They
	do what banks do, they lend it to other people. The result is a money multiplier
	effect, which increases the money supply in the economy enormously
	(Figure~\ref{fig:money-multiplier}). Your initial deposit of \$100 will soon
	turn into \$190. By lending a 90\% fraction of the newly created \$90, there
	will soon be \$271 in the economy. And \$343.90 after that. The money supply is
	recursively increasing, since banks are literally lending money they don't
	have~\cite{wiki:money-multiplier}. Without a single Abracadabra, banks magically
	transform \$100 into one thousand dollars or more. Turns out 10x is easy. It
	only takes a couple of lending rounds.
\end{comment}
그렇다면 은행은 나머지 돈으로 무엇을 할까? 
당신의 \$90는 어떻게 되는 것인가? 
은행은 그 돈을 다른 사람에게 빌려준다. 
그 결과 통화 승수효과가 발생하여 통화 공급이 엄청나게 늘어난다(그림~\ref{fig:money-multiplier}). 
당신이 저축한 예금 \$100은 곧 \$190가 될 것이다. 
새롭게 탄생한 \$90의 90\%를 다시 대출함으로써 시장에는 곧 \$291가 존재하게 된다. 
그리고 또 \$343.90이 된다. 
은행이 말 그대로 가지고 있지 않은 돈을 빌려주기 때문에 통화 공급이 반복적으로 증가한다\cite{wiki:money-multiplier}. 
아브라카다브라 마법 없이도, 은행은 \$100을 \$1000로 바꿀 수 있다. 
몇 번의 대출만 해준다면 10배로 뻥튀기 하는 것 쯤은 식은 죽 먹기다.

\begin{comment}
	\begin{figure}
		\centering
		\includegraphics{assets/images/money-multiplier.png}
		\caption{The money multiplier effect}
		\label{fig:money-multiplier}
	\end{figure}
\end{comment}
\begin{figure}
	\centering
	\includegraphics{assets/images/money-multiplier.png}
	\caption{통화 승수 효과}
	\label{fig:money-multiplier}
\end{figure}

\paragraph{}
\begin{comment}
	Don't get me wrong: There is nothing wrong with lending. There is
	nothing wrong with interest. There isn't even anything wrong with good
	old regular banks to store your wealth somewhere more secure than in
	your sock drawer.
\end{comment}
오해말자. 대출은 잘못이 없다. 이자가 나쁜 것이 아니다. 
양말 서랍장보다 안전한 곳에 재산을 보관해주는 역사 깊은 일반 은행에는 아무런 문제가 없다. 

\begin{comment}
	Central banks, however, are a different beast. Abominations of financial
	regulation, half public half private, playing god with something which
	affects everyone who is part of our global civilization, without a
	conscience, only interested in the immediate future, and seemingly
	without any accountability or auditability (see Figure~\ref{fig:bsg}).
\end{comment}
중앙은행이 문제다. 
가증스러운 금융 규제, 공적도 사적도 아닌 애매한 위치, 
전 세계 사람들을 대상으로 절대자 행세를 하는, 양심은 저버린 채 근시안적 이익에만 관심있을 뿐인, 
어떠한 책임도 지지않고 견제도 받지 않는 조직이다.(그림~\ref{fig:bsg})

\begin{comment}
	\begin{figure}
		\centering
		\includegraphics{assets/images/bsg.jpg}
		\caption{Yellen is strongly opposed to audit the Fed, while Bitcoin Sign Guy is strongly in favor of buying bitcoin.}
		\label{fig:bsg}
	\end{figure}
\end{comment}
\begin{figure}
	\centering
	\includegraphics{assets/images/bsg.jpg}
	\caption{연준의 감사를 강력하게 반대하는 옐런. 비트코인 사세요(Buy Bitcoin) 팻말을 든 남자에게 찬성한다.}
	\label{fig:bsg}
\end{figure}

\begin{comment}
	While Bitcoin is still inflationary, it will cease to be so rather soon.
	The strictly limited supply of 21 million bitcoins will eventually do
	away with inflation completely. We now have two monetary worlds: an
	inflationary one where money is printed arbitrarily, and the world of
	Bitcoin, where final supply is fixed and easily auditable for everyone.
	One is forced upon us by violence, the other can be joined by anyone who
	wishes to do so. No barriers to entry, no one to ask for permission.
	Voluntary participation. That is the beauty of Bitcoin.
\end{comment}
비트코인은 여전히 발행되고 있지만 머지않아 발행을 멈추게 될 것이다. 
비트코인 총량은 2,100만 개로 엄격하게 제한되어 있기 때문에 결국 인플레이션은 완전히 없어질 것이다.
이제 우리에겐 두 가지 화폐 세계가 있다. 
하나는 돈이 임의로 인쇄되는 인플레이션 세계이고, 다른 하나는 최종 공급량이 고정되어 모든 사람이 쉽게 감사할 수 있는 비트코인 세계이다.
하나는 폭력으로 우리를 강요하고, 다른 하나는 원하면 누구든 참여할 수 있다.
진입 장벽도 없고 누군가에게 허가받을 필요도 없다. 
자발적인 참여. 그것이 비트코인의 아름다움이다.

\begin{comment}
	I would argue that the argument between Keynesian\footnote{Theories according to
		John Maynard Keynes and his deciples~\cite{wiki:keynesian}} and
	Austrian\footnote{School of economic thought based on methodological
		individualism~\cite{wiki:austrian}} economists is no longer purely academical.
	Satoshi managed to build a system for value transfer on steroids, creating the
	soundest money which ever existed in the process. One way or another, more and
	more people will learn about the scam which is fractional reserve banking. If
	they come to similar conclusions as most Austrians and Bitcoiners, they might
	join the ever-growing internet of money. Nobody can stop them if they choose to
	do so.
\end{comment}
나는 케인즈주의\footnote{존 메이너드 케인즈와 제자들의 이론~\cite{wiki:keynesian}}와 오스트리아학파\footnote{방법론적 개인주의에 입각한 경제학파~\cite{wiki:austrian}} 사이의 논쟁이 학문적 논쟁이라 생각하지 않는다. 
사토시 나카모토는 강력한 가치 전송 시스템을 구축하여 세상에서 가장 건전한 화폐를 만들었다. 
어떤 식으로든 점점 더 많은 사람이 지급준비금 은행이라는 사기에 대해 알게 될 것이다. 
이를 깨달은 사람들이 오스트리아 학파나 비트코이너와 유사한 결론에 도달한다면 
그들은 아마 지속적으로 성장하는 돈의 인터넷(the internet of money)에 참여하게 될 것이다. 
그 선택을 아무도 막을 수 없다.

%\paragraph{Bitcoin taught me that fractional reserve banking is pure insanity.}
\paragraph{비트코인은 부분 지급준비금 시스템이 진짜 광기임을 가르쳐주었다.}

% ---
%
% #### Down the Rabbit Hole
%
% - [The Creature From Jekyll Island] by G. Edward Griffin
% - [Money Multiplier][money multiplier], [Keynesian Economics][Keynesian], [Austrian School][Austrian] on Wikipedia
%
% [The Creature From Jekyll Island]: https://archive.org/details/pdfy--Pori1NL6fKm2SnY
%
% [joint debate]: https://www.youtube.com/watch?v=hYzX3YZoMrs
% [money multiplier]: https://en.wikipedia.org/wiki/Money_multiplier
% [auditability]: https://i.ytimg.com/vi/ThFGs347MW8/maxresdefault.jpg
% [Keynesian]: https://en.wikipedia.org/wiki/Keynesian_economics
% [Austrian]: https://en.wikipedia.org/wiki/Austrian_School
%
% <!-- Wikipedia -->
% [alice]: https://en.wikipedia.org/wiki/Alice%27s_Adventures_in_Wonderland
% [carroll]: https://en.wikipedia.org/wiki/Lewis_Carroll
