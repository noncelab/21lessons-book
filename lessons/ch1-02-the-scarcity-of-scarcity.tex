
\chapter{진정한 희소성}
\label{les:2}

\begin{chapquote}{앨리스}
	\enquote{충분해. 나는 더 크고 싶지 않아\ldots}
\end{chapquote}


%In general, the advance of technology seems to make things more abundant. More
%and more people are able to enjoy what previously have been luxurious goods.
%Soon, we will all live like kings. Most of us already do. As Peter Diamandis
%wrote in Abundance~\cite{abundance}: \enquote{Technology is a resource-liberating
	%mechanism. It can make the once scarce the now abundant.}
일반적으로 기술의 발전은 인류를 더 풍요롭게 한다. 기술로 인해 더 많은 사람이 더 많은 사치를 즐길 수 있게 되었다.
곧 우리는 왕처럼 살게 될 것이다. 그리고 우리 대부분은 이미 그렇게 살고 있다.
피터 디아만디스(Peter Diamandis)는 풍요(Abundance)라는 책에서 다음과 같이 말했다\cite{abundance}. \enquote{기술은 자원해방의 메커니즘이다.
	기술은 희소했던 것들을 풍부하게 만들어 준다.}

%Bitcoin, an advanced technology in itself, breaks this trend and creates
%a new commodity which is truly scarce. Some even argue that it is one of
%the scarcest things in the universe. The supply can't be inflated, no
%matter how much effort one chooses to expend towards creating more.
그 자체로 진보된 기술인 비트코인은 이러한 새로운 추세를 깨고 희소한 상품을 만들었다. 
어떤 사람들은 비트코인이 우주에서 가장 희귀한 것 이라고 주장한다.
비트코인은 공급을 더 늘릴 수 없다. 아무리 노력해도 더 만들어 낼 수 없다.

\begin{quotation}\begin{samepage}
		\enquote{비트코인과 시간만이 진정으로 희귀하다.}
		\begin{flushright} -- 사이페딘 아모스\footnote{Presentation on The Bitcoin Standard~\cite{bitcoinstandard-pres}}
\end{flushright}\end{samepage}\end{quotation}

%Paradoxically, it does so by a mechanism of copying. Transactions are
%broadcast, blocks are propagated, the distributed ledger is --- well,
%you guessed it --- distributed. All of these are just fancy words for
%copying. Heck, Bitcoin even copies itself onto as many computers as it
%can, by incentivizing individual people to run full nodes and mine new
%blocks.
역설적으로 이 희소성은 복제 메커니즘 때문에 가능하다. 
블록과 트랜잭션은 전파되고 장부는 분산되어 있다. 
이 모든 것은 복제를 설명하고 있다. 
비트코인은 개별 사람들이 노드를 실행하고 새 블록을 채굴하도록 장려하여 
가능한 한 많은 컴퓨터에 자신을 복사하게 만든다.

%All of this duplication wonderfully works together in a concerted effort
%to produce scarcity.
복제된 모든 것들이 희소성을 지키기 위해 공동의 노력으로 훌륭히 작동한다.

%\paragraph{In a time of abundance, Bitcoin taught me what real scarcity is.}
\paragraph{풍요의 시대에 비트코인은 진정한 희소성이 무엇인지 알려주었다.}

% ---
%
% #### Through the Looking-Glass
%
% - [Lesson 14: Sound money][lesson14]
%
% #### Down the Rabbit Hole
%
% - [The Bitcoin Standard: The Decentralized Alternative to Central Banking][bitcoin-standard]
% - [Abundance: The Future Is Better Than You Think][Abundance] by Peter Diamandis
% - [Presentation on The Bitcoin Standard][bitcoin-standard-presentation] by Saifedean Ammous
% - [Modeling Bitcoin's Value with Scarcity][planb-scarcity] by PlanB
% - 🎧 [Misir Mahmudov on the Scarcity of Time & Bitcoin][tftc60] TFTC #60 hosted by Marty Bent
% - 🎧 [PlanB – Modelling Bitcoin's digital scarcity through stock-to-flow techniques][slp67] SLP #67 hosted by Stephan Livera
%
% <!-- Through the Looking-Glass -->
% [lesson14]: {{ 'bitcoin/lessons/ch2-14-sound-money' | absolute_url }}
%
% <!-- Down the Rabbit Hole -->
% [Abundance]: https://www.diamandis.com/abundance
% [bitcoin-standard]: http://amzn.to/2L95bJW
% [bitcoin-standard-presentation]: https://www.bayernlb.de/internet/media/de/ir/downloads_1/bayernlb_research/sonderpublikationen_1/bitcoin_munich_may_28.pdf
% [planb-scarcity]: https://medium.com/@100trillionUSD/modeling-bitcoins-value-with-scarcity-91fa0fc03e25
% [tftc60]: https://anchor.fm/tales-from-the-crypt/episodes/Tales-from-the-Crypt-60-Misir-Mahmudov-e3aibh
% [slp67]: https://stephanlivera.com/episode/67
%
% <!-- Wikipedia -->
% [alice]: https://en.wikipedia.org/wiki/Alice%27s_Adventures_in_Wonderland
% [carroll]: https://en.wikipedia.org/wiki/Lewis_Carroll
