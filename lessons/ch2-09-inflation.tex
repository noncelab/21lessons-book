\chapter{인플레이션}
\label{les:9}

\begin{chapquote}{하트의 여왕\footnote{역자: 이상한 나라 앨리스의 등장인물}} 
	\enquote{내 사랑, 이 자리에 서 있기 위해서는 최대한 빨리 달려야 해. 그리고 만약 어디론가 가고 싶다면
		두 배로 빨리 달려야 해.}
\end{chapquote}

%Trying to understand monetary inflation, and how a non-inflationary
%system like Bitcoin might change how we do things, was the starting
%point of my venture into economics. I knew that inflation was the rate
%at which new money was created, but I didn't know too much beyond that.
인플레이션 현상을 이해하고 
비트코인처럼 인플레이션이 없는 시스템을 위해서
어떤 변화가 필요한지를 알아가는 것은 경제학 탐구의 출발점이었다. 
나는 인플레이션이 새로운 돈이 만들어지는 비율 정도라는 것은 알고 있었지만,
그 이면에는 무엇이 있는지는 알지 못했다.

%While some economists argue that inflation is a good thing, others argue
%that \enquote{hard} money which can't be inflated easily --- as we had in the
%days of the gold standard --- is essential for a healthy economy.
%Bitcoin, having a fixed supply of 21 million, agrees with the latter
%camp.
어떤 경제학자들은 인플레이션은 좋은 것이라 주장하는 반면, 
어떤 경제학자들은 금본위제 시대의 금처럼 
수량을 늘리기 어려운 경화(hard currency)가 건전한 경제에 필수적이라 주장한다. 
2,100만 개로 공급량이 고정된 비트코인은 후자의 의견을 반영한 것이다.

%Usually, the effects of inflation are not immediately obvious. Depending
%on the inflation rate (as well as other factors) the time between cause
%and effect can be several years. Not only that, but inflation affects
%different groups of people more than others. As Henry Hazlitt points out
%in \textit{Economics in One Lesson}: \enquote{The art of economics consists in looking
	%not merely at the immediate but at the longer effects of any act or
	%policy; it consists in tracing the consequences of that policy not
	%merely for one group but for all groups.}
일반적으로 인플레이션의 영향은 즉각적으로 나타나지 않는다. 
인플레이션율에 따라 이 시간은 몇 년이 걸릴 수도 있다. 
그뿐만 아니라 인플레이션은 무엇보다 광범위한 사람들에게 영향을 끼친다. 
헨리 해즐릿은 경제학의 교훈(Economics in One Lesson)에서 다음과 같이 말한다. 
\enquote{경제학에서 어떤 행동이나 정책은 즉각적인 효과보다는 긴 효과를 보고자 하는 데 있다.
	그것은 한 그룹이 아니라 모든 그룹에 대한 정책의 결과를 추적하는 것으로 구성된다.}


%One of my personal lightbulb moments was the realization that issuing
%new currency --- printing more money --- is a \textit{completely} different
%economic activity than all the other economic activities. While real
%goods and real services produce real value for real people, printing
%money effectively does the opposite: it takes away value from everyone
%who holds the currency which is being inflated.
개인적으로 가장 충격적인 사실은 
새로운 화폐를 발행하는 것 즉 돈을 인쇄하는 것은 일반적인 경제활동과는 완전히 다른 양상을 보인다는 것이다. 
실제 상품과 서비스를 생산하는 것은 사람들에게 가치를 제공하는 반면, 
돈을 인쇄하는 것은 인플레이션을 통해 사람들에게 가치를 빼앗아 간다.

\begin{quotation}\begin{samepage}
		%Mere inflation --- that is, the mere issuance of more money, with the
		%consequence of higher wages and prices --- may look like the creation
		%of more demand. But in terms of the actual production and exchange of
		%real things it is not.
		\enquote{인플레이션, 즉 더 많은 돈을 발행하는 것은 더 높은 임금과 물가로 인해 더 많은 수요를 창출하는 것처럼 보인다.
			하지만, 실제 생산과 거래가 생기는 것은 아니다.}
		\begin{flushright} -- 헨리 해즐릿\footnote{Henry Hazlitt, \textit{Economics in One Lesson} \cite{hazlitt}}
\end{flushright}\end{samepage}\end{quotation}

%The destructive force of inflation becomes obvious as soon as a little inflation
%turns into \textit{a lot}. If money hyperinflates things get ugly real
%quick.\footnote{\url{https://en.wikipedia.org/wiki/Hyperinflation}
	%\cite{wiki:hyperinflation}} As the inflating currency falls apart, it will fail
%to store value over time and people will rush to get their hands on any goods
%which might do.
인플레이션의 파괴력은 인플레이션의 규모가 커지는 순간 명백하게 나타난다. 
초인플레이션으로 화폐가 과도하게 팽창하면 상황은 빠르게 나빠진다.\footnote{\url{https://en.wikipedia.org/wiki/Hyperinflation}
	\cite{wiki:hyperinflation}} 
팽창하는 통화가 무너지면 사람들은 낮아지는 통화가치에 대응하기 위해 상품 구매를 서두른다.

\paragraph{}
%Another consequence of hyperinflation is that all the money which people
%have saved over the course of their life will effectively vanish. The
%paper money in your wallet will still be there, of course. But it will
%be exactly that: worthless paper.
초인플레이션의 또 다른 부작용은 사람들이 평생 저축한 돈이 사실상 사라지게 되는 것이다. 
지갑에는 돈이 그대로 있지만, 그 돈은 쓸모없는 종이와 같다.

\begin{figure}
	\includegraphics{assets/images/children-playing-with-money.png}
	\caption{바이마르 공화국의 초인플레이션 (1921-1923)}
	\label{fig:children-playing-with-money}
\end{figure}

\paragraph{}
%Money declines in value with so-called \enquote{mild} inflation as well. It
%just happens slowly enough that most people don't notice the diminishing
%of their purchasing power. And once the printing presses are running,
%currency can be easily inflated, and what used to be mild inflation
%might turn into a strong cup of inflation by the push of a button. As
%Friedrich Hayek pointed out in one of his essays, mild inflation usually
%leads to outright inflation.
인플레이션이 경미하다 하여도 돈의 가치는 결국 하락한다. 
대부분 사람이 알아차리지 못할 정도로 천천히 구매력이 감소한다. 
일단 인쇄기가 가동되면 통화는 쉽게 부풀려지고 경미한 인플레이션은 언제든 강력한 인플레이션으로 변질될 수 있다. 
프리드리히 하이에크가 그의 에세이에서 지적했듯이 
가벼운 인플레이션이 노골적인 인플레이션으로 변하는 일은 매우 흔하게 발생한다.

\begin{quotation}\begin{samepage}
		\enquote{`가벼운' 지속적인 인플레이션은 도움이 안 된다. 그것은 결국 노골적인 인플레이션으로 변질한다.}
		\begin{flushright} -- 프리드리히 하이에크\footnote{Friedrich Hayek, \textit{1980s
					Unemployment and the Unions} \cite{hayek-inflation}}
\end{flushright}\end{samepage}\end{quotation}

%Inflation is particularly devious since it favors those who are closer
%to the printing presses. It takes time for the newly created money to
%circulate and prices to adjust, so if you are able to get your hands on
%more money before everyone else's devaluates you are ahead of the
%inflationary curve. This is also why inflation can be seen as a hidden
%tax because in the end governments profit from it while everyone else
%ends up paying the price.
인플레이션은 인쇄기에 더 가까운 사람에게 유리하기 때문에 더 악랄하다. 
새로 만들어진 돈이 순환하고 가격이 조정되는 데에 시간이 걸리므로 
돈의 가치가 하락하기 전에 더 많은 돈을 손에 넣을 수 있다면
인플레이션이 나타나기 전에 돈을 쓸 수 있다. 
이러한 점으로 인해 인플레이션을 숨겨진 세금으로 간주하기도 한다.
왜냐하면 결국 인플레이션으로 정부는 이익을 얻지만, 다른 모든 사람은 그 대가를 지불하기 때문이다.

\begin{quotation}\begin{samepage}
		\enquote{광범위한 관점에서 우리의 역사는 정부의 이익을 위해 조작된 인플레이션의 역사이다. }
		\begin{flushright} -- 프리드리히 하이에크\footnote{Friedrich Hayek, \textit{Good Money} \cite{hayek-good-money}}
\end{flushright}\end{samepage}\end{quotation}

%So far, all government-controlled currencies have eventually been
%replaced or have collapsed completely. No matter how small the rate of
%inflation, \enquote{steady} growth is just another way of saying exponential
%growth. In nature as in economics, all systems which grow exponentially
%will eventually have to level off or suffer from catastrophic collapse.
역사적으로 모든 정부 통화는 결국 교체되거나 완전히 붕괴되었다. 
인플레이션 비율이 아무리 낮더라도 결국 기하급수적 팽창으로 귀결된다. 
자연에서와 마찬가지로 경제에서도 기하급수적으로 성장하는 
모든 시스템은 결국 평준화되거나 파국적인 붕괴를 피할 수 없다.

\paragraph{}
%\enquote{It can't happen in my country,} is what you're probably thinking. You don't
%think that if you are from Venezuela, which is currently suffering from
%hyperinflation. With an inflation rate of over 1 million percent, money is
%basically worthless. \cite{wiki:venezuela}
\enquote{우리나라는 괜찮아.}라고 생각할 수도 있다.
하지만 당신이 초인플레이션으로 고통받고 있는 베네수엘라 출신이라면 그렇게 생각하지 않을 것이다. 
인플레이션율이 100만 퍼센트가 넘는 상황에서 돈은 가치가 없다.\cite{wiki:venezuela}

\paragraph{}
%It might not happen in the next couple of years, or to the particular currency
%used in your country. But a glance at the list of historical
%currencies\footnote{See \textit{List of historical currencies} on Wikipedia.
	%\cite{wiki:historical-currencies}} shows that it will inevitably happen over a
%long enough period of time. I remember and used plenty of those listed: the
%Austrian schilling, the German mark, the Italian lira, the French franc, the
%Irish pound, the Croatian dinar, etc. My grandma even used the Austro-Hungarian
%Krone. As time moves on, the currencies currently in use\footnote{See
	%\textit{List of currencies} on Wikipedia \cite{wiki:list-of-currencies}} will
%slowly but surely move to their respective graveyards. They will hyperinflate or
%be replaced. They will soon be historical currencies. We will make them
%obsolete.
향후 몇 년 동안 당신의 국가에서 사용되는 특정 통화에서는 인플레이션이 발생하지 않을 수 있다. 
그러나 통화의 역사\footnote{See \textit{List of historical currencies} on Wikipedia.
	\cite{wiki:historical-currencies}}를
보면 충분히 오랜 기간에 걸쳐 불가피하게 인플레이션이 발생한다는 것을 알 수 있다.
나는 오스트리아 실링, 독일 마르크, 이탈리아 리라, 프랑스 프랑, 아일랜드 파운드, 크로아티아 디나르 등 여러
통화들을 기억하고 있거나 사용하고 있다. 
나의 할머니는 오스트리아-헝가리 크로네도 사용하셨다. 
시간이 지나면 현재 사용 중인 통화들도\footnote{See
	\textit{List of currencies} on Wikipedia \cite{wiki:list-of-currencies}}
느리지만 확실하게 각자의 무덤으로 이동하게 될 것이다.
이 통화들은 가치가 급락하거나 교체되어 역사의 뒤안길로 사라질 것이다.
우리가 그 화폐들을 폐물로 만들 것이다.

\begin{quotation}\begin{samepage}
		\enquote{정부가 화폐 공급을 부풀리려는 유혹에 굴복할 수밖에 없다는 사실은 역사가 잘 말해준다.}
		\begin{flushright} -- 사이페딘 아모스\footnote{Saifedean Ammous, \textit{The Bitcoin
					Standard} \cite{bitcoin-standard}}
\end{flushright}\end{samepage}\end{quotation}

\begin{comment}
	Why is Bitcoin different? In contrast to currencies mandated by the government,
	monetary goods which are not regulated by governments, but by the laws of
	physics\footnote{Gigi, \textit{Bitcoin's Energy Consumption - A shift in
			perspective} \cite{gigi:energy}}, tend to survive and even hold their respective
	value over time. The best example of this so far is gold, which, as the
	aptly-named \textit{Gold-to-Decent-Suit Ratio}\footnote{History shows that the
		price of an ounce of gold equals the price of a decent men's suit, according to Sionna
		investment managers \cite{web:gold-to-decent-suite-ratio}} shows, is holding its
	value over hundreds and even thousands of years. It might not be perfectly
	\enquote{stable} --- a questionable concept in the first place --- but the value it
	holds will at least be in the same order of magnitude.
\end{comment}
비트코인은 왜 다른가? 정부가 강제하는 통화와 달리 물리적 법칙에 의해 규제되는 화폐 상품\footnote{Gigi, \textit{Bitcoin's Energy Consumption - A shift in perspective}\cite{gigi:energy}}
은 시간이 지나도 가치를 유지하는 경향이 있다. 가장 좋은 예는 금이다. 
금 대비 적합한 양복 값의 비율(Gold-to-Decent-Suit Ratio)\footnote{시오나 투자 매니저에 따르면 역사적으로 금 1온스의 가격이 괜찮은 남성 정장 가격과 같다고 한다. \cite{web:gold-to-decent-suite-ratio}}
이라는 적절한 비유가 보여주듯이 금은 수백 년, 수천 년 동안 그 가치를 유지하고 있다. 
단기적으로는 금의 가치가 불안정해 보일 수도 있다.
하지만 분명한 것은 금의 희소성에 의한 가치는 유지될 것이라는 사실이다.

\begin{comment}
	If a monetary good or currency holds its value well over time and space,
	it is considered to be \textit{hard}. If it can't hold its value, because it
	easily deteriorates or inflates, it is considered a \textit{soft} currency. The
	concept of hardness is essential to understand Bitcoin and is worthy of
	a more thorough examination. We will return to it in the last economic
	lesson: sound money.
\end{comment}
금전적 상품이나 화폐가 시간과 공간을 초월하여 가치를 유지하기는 어렵다. 
쉽게 약화하거나 팽창하여 가치를 유지할 수 없는 경우 연화(soft currency)로 간주한다. 
돈의 강건성(hardness) 개념은 비트코인을 이해하는 데 필수적이며 더 철저히 검토되어야 한다.
우리는 마지막 교훈에서 화폐의 건전성(sound money)에 대해서 다시 다루게 될 것이다.

\paragraph{}
\begin{comment}
	As more and more countries suffer from
	hyperinflation more and more people will have to face the reality
	of hard and soft money. If we are lucky, maybe even some central bankers will be
	forced to re-evaluate their monetary policies. Whatever might happen, the
	insights I have gained thanks to Bitcoin will probably be invaluable, no matter
	the outcome.
\end{comment}
여러 국가가 초인플레이션에 시달릴수록 더 많은 사람이 연화와 경화의 차이를 알게 될 것이다. 
운이 좋다면 일부 중앙은행가들도 자국의 통화정책을 재검토하게 될지도 모른다. 
어떤 일이 일어나든, 어떤 결과가 나타나든
비트코인 덕분에 얻은 나의 통찰력은 매우 귀중하다.

\paragraph{비트코인은 인플레이션의 숨겨진 세금과 초인플레이션의 재앙을 가르쳐주었다.}

% ---
%
% #### Down the Rabbit Hole
%
% - [Economics in One Lesson][Henry Hazlitt] by Henry Hazlitt
% - [1980's Unemployment and the Unions][unions] by Friedrich Hayek
% - [Good Money, Part II][good-money]: Volume Six of the Collected Works of F.A. Hayek
% - [The Bitcoin Standard] by Saifedean Ammous
% - [Hyperinflation][hyperinflates], [economic crisis in Venezuela][wiki-venezuela], [list of historical currencies], [list of currencies][currently in use] on Wikipedia
%
% [unions]: https://books.google.com/books/about/1980s_unemployment_and_the_unions.html?id=xM9CAQAAIAAJ
% [good-money]: https://books.google.com/books?id=l_A1vVIaYBYC
%
% [Henry Hazlitt]: https://mises.org/library/economics-one-lesson
% [hyperinflates]: https://en.wikipedia.org/wiki/Hyperinflation
% [inflation cannot help]: https://books.google.com/books?id=zZu3AAAAIAAJ&dq=%22only+while+it+accelerates%22&focus=searchwithinvolume&q=%22steady+inflation+cannot+help%22
% [history of inflation]: https://books.google.com/books?id=l_A1vVIaYBYC&pg=PA142&dq=%22history+is+largely+a+history+of+inflation%22&hl=en&sa=X&ved=0ahUKEwi90NDLrdnfAhUprVkKHUx1CmIQ6AEIKjAA#v=onepage&q=%22history%20is%20largely%20a%20history%20of%20inflation%22&f=false
% [wiki-venezuela]: https://en.wikipedia.org/wiki/Crisis_in_Venezuela#Economic_crisis
% [by the laws of physics]: https://link.medium.com/9fzq2L0J3S
% [\textit{Gold-to-Decent-Suit Ratio}]: https://www.businesswire.com/news/home/20110819005774/en/History-Shows-Price-Ounce-Gold-Equals-Price
% [The Bitcoin Standard]: https://thesaifhouse.wordpress.com/book/
%
% <!-- Wikipedia -->
% [alice]: https://en.wikipedia.org/wiki/Alice%27s_Adventures_in_Wonderland
% [carroll]: https://en.wikipedia.org/wiki/Lewis_Carroll
