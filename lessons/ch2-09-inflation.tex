\chapter{인플레이션}
\label{les:9}

\begin{chapquote}{하트의 여왕} 
	\enquote{얘야, 제자리에 머물기 위해선 최대한 빨리 달려야 해. 그리고 만약 어디든 가고 싶다면, 그보다 적어도 두 배로 달려야 해.}
\end{chapquote}

%Trying to understand monetary inflation, and how a non-inflationary
%system like Bitcoin might change how we do things, was the starting
%point of my venture into economics. I knew that inflation was the rate
%at which new money was created, but I didn't know too much beyond that.
\paragraph{}
통화 인플레이션과 비트코인처럼 인플레이션 없는 시스템이 
우리가 일하는 방식을 어떻게 바꿀 수 있는지 이해하려는 노력이 나의 경제학 탐구의 출발점이었다. 
나는 인플레이션이 새로운 돈이 창출되는 비율이라는 것 정도는 알았지만 그 이상은 알지 못했다.

%While some economists argue that inflation is a good thing, others argue
%that \enquote{hard} money which can't be inflated easily --- as we had in the
%days of the gold standard --- is essential for a healthy economy.
%Bitcoin, having a fixed supply of 21 million, agrees with the latter
%camp.
\paragraph{}
일부 경제학자들이 인플레이션은 좋은 것이라 주장하는 반면, 어떤 경제학자들은 금본위제 시대의 금처럼 
쉽게 수량을 늘리기 어려운 경화(hard currency)가 건전한 경제에 필수라 주장한다. 
2,100만 개로 공급량이 고정된 비트코인은 후자의 의견을 반영한 것이다.

%Usually, the effects of inflation are not immediately obvious. Depending
%on the inflation rate (as well as other factors) the time between cause
%and effect can be several years. Not only that, but inflation affects
%different groups of people more than others. As Henry Hazlitt points out
%in \textit{Economics in One Lesson}: \enquote{The art of economics consists in looking
	%not merely at the immediate but at the longer effects of any act or
	%policy; it consists in tracing the consequences of that policy not
	%merely for one group but for all groups.}
\paragraph{}
일반적으로 인플레이션의 영향은 즉각적으로 나타나지 않는다. 
인플레이션율(및 기타 요인)에 따라 결과가 나타나기까지 몇 년이 걸릴 수도 있다. 
무엇보다 인플레이션은 광범위한 영향을 끼친다. 
헨리 해즐릿은 경제학의 교훈(Economics in One Lesson)에서 다음과 같이 말한다. 
\enquote{경제학은 어떤 행동이나 정책의 즉각적인 효과뿐만 아니라 장기적 효과를 살펴보고, 
	그 정책의 결과가 한 집단에 국한되지 않고 모든 집단에 미치는 영향을 추적하는 것으로 구성된다.}


%One of my personal lightbulb moments was the realization that issuing
%new currency --- printing more money --- is a \textit{completely} different
%economic activity than all the other economic activities. While real
%goods and real services produce real value for real people, printing
%money effectively does the opposite: it takes away value from everyone
%who holds the currency which is being inflated.
\paragraph{}
개인적으로 가장 충격적인 사실은 새로운 화폐를 발행하는 것, 즉 돈을 찍어내는 것이 
일반적인 경제 활동과는 \textit{완전히} 다른 행위라는 것이다. 
실제 상품과 서비스는 사람들을 위해 실질적인 가치를 창출하지만, 
돈을 찍어내는 것은 그 반대 역할을 한다. 통화를 부풀려 모든 사람들에게서 가치를 빼앗아가기 때문이다. 

\begin{quotation}\begin{samepage}
		%Mere inflation --- that is, the mere issuance of more money, with the
		%consequence of higher wages and prices --- may look like the creation
		%of more demand. But in terms of the actual production and exchange of
		%real things it is not.
		\enquote{단순히 인플레이션은, 즉 화폐를 더 많이 발행하고 그 결과 임금과 물가가 상승하는 것은 
		더 많은 수요를 창출하는 것처럼 보일 수 있다. 하지만, 실제 생산과 교환의 측면에서 실제 가치가 창출되는 것은 아니다.}
		\begin{flushright} -- 헨리 해즐릿\footnote{Henry Hazlitt, \textit{Economics in One Lesson} \cite{hazlitt}}
\end{flushright}\end{samepage}\end{quotation}

%The destructive force of inflation becomes obvious as soon as a little inflation
%turns into \textit{a lot}. If money hyperinflates things get ugly real
%quick.\footnote{\url{https://en.wikipedia.org/wiki/Hyperinflation}
	%\cite{wiki:hyperinflation}} As the inflating currency falls apart, it will fail
%to store value over time and people will rush to get their hands on any goods
%which might do.
\paragraph{}
인플레이션의 파괴력은 규모가 커지는 순간 명백하게 드러난다. 
화폐가 과도하게 팽창되면 상황은 순식간에 추악해진다.\footnote{\url{https://en.wikipedia.org/wiki/Hyperinflation} \cite{wiki:hyperinflation}} 
시간이 지남에 따라 팽창하던 화폐가 무너지면 가치를 저장하지 못하고 
사람들은 어떤 상품이든 손에 넣기 위해 서두르게 될 것이다.

%Another consequence of hyperinflation is that all the money which people
%have saved over the course of their life will effectively vanish. The
%paper money in your wallet will still be there, of course. But it will
%be exactly that: worthless paper.
\paragraph{}
초인플레이션의 또 다른 결과는 사람들이 평생 저축한 돈이 사실상 사라지게 된다는 것이다. 
물론 지갑에 돈이 그대로 있을테지만, 그 돈은 가치없는 종이와 같을 것이다.

\begin{figure}
	\includegraphics{assets/images/children-playing-with-money.png}
	\caption{바이마르 공화국의 초인플레이션 (1921-1923)}
	\label{fig:children-playing-with-money}
\end{figure}

\paragraph{}
%Money declines in value with so-called \enquote{mild} inflation as well. It
%just happens slowly enough that most people don't notice the diminishing
%of their purchasing power. And once the printing presses are running,
%currency can be easily inflated, and what used to be mild inflation
%might turn into a strong cup of inflation by the push of a button. As
%Friedrich Hayek pointed out in one of his essays, mild inflation usually
%leads to outright inflation.
소위 가벼운 인플레이션으로도 돈의 가치는 결국 하락한다. 
대부분의 사람들이 알아차리지 못할 정도로 천천히 구매력이 감소한다. 
그리고 일단 인쇄기가 가동되면 통화는 쉽게 부풀려질 수 있고, 예전에는 가벼운 인플레이션이었던 것이
버튼 하나만 누르면 강력한 인플레이션으로 바뀔 수 있다.
프리드리히 하이에크가 그의 에세이에서 지적했듯이, 가벼운 인플레이션이 명백한 인플레이션으로 변하는 일은 매우 흔하게 발생한다.

\begin{quotation}\begin{samepage}
		\enquote{가볍고 지속적인 인플레이션은 도움이 되지 않으며, 이는 결국 노골적인 인플레이션으로 이어진다.}
		\begin{flushright} -- 프리드리히 하이에크\footnote{Friedrich Hayek, \textit{1980s
					Unemployment and the Unions} \cite{hayek-inflation}}
\end{flushright}\end{samepage}\end{quotation}

\paragraph{}
%Inflation is particularly devious since it favors those who are closer
%to the printing presses. It takes time for the newly created money to
%circulate and prices to adjust, so if you are able to get your hands on
%more money before everyone else's devaluates you are ahead of the
%inflationary curve. This is also why inflation can be seen as a hidden
%tax because in the end governments profit from it while everyone else
%ends up paying the price.
인플레이션은 인쇄기에 더 가까운 사람에게 유리하기 때문에 특히 악랄하다. 
새로 만들어진 돈이 유통되고 가격이 조정되는 데에 시간이 걸리므로 
돈의 가치가 하락하기 전에 더 많은 돈을 손에 넣을 수 있다면 인플레이션이 나타나기 전에 돈을 쓸 수 있다. 
이것이 인플레이션을 숨겨진 세금으로 볼 수 있는 이유이기도 하다.
결국 정부는 인플레이션으로 이익을 얻고, 다른 모든 사람은 그 대가를 치르게 되기 때문이다.

\begin{quotation}\begin{samepage}
		\enquote{역사는 정부의 이익을 위해 조작한 인플레이션의 역사라 해도 과언이 아니다. }
		\begin{flushright} -- 프리드리히 하이에크\footnote{Friedrich Hayek, \textit{Good Money} \cite{hayek-good-money}}
\end{flushright}\end{samepage}\end{quotation}

%So far, all government-controlled currencies have eventually been
%replaced or have collapsed completely. No matter how small the rate of
%inflation, \enquote{steady} growth is just another way of saying exponential
%growth. In nature as in economics, all systems which grow exponentially
%will eventually have to level off or suffer from catastrophic collapse.
\paragraph{}
역사적으로 정부가 통제하는 모든 통화는 결국 교체되거나 완전히 붕괴되었다. 
인플레이션 비율이 아무리 낮더라도 꾸준한 증가는 결국 기하급수적 팽창으로 귀결될 뿐이다. 
자연에서와 마찬가지로 경제에서도 기하급수적으로 성장하는 
모든 시스템은 결국 평준화되거나 재앙적인 붕괴를 겪게 될 것이다.

\paragraph{}
%\enquote{It can't happen in my country,} is what you're probably thinking. You don't
%think that if you are from Venezuela, which is currently suffering from
%hyperinflation. With an inflation rate of over 1 million percent, money is
%basically worthless. \cite{wiki:venezuela}
아마 \enquote{우리나라는 괜찮아.}라고 생각할 수 있다.
하지만 당신이 현재 초인플레이션으로 고통받고 있는 베네수엘라 사람이라면 이야기가 다르다.  
인플레이션율이 100만 퍼센트가 넘는 상황에서 기본적으로 돈은 가치가 없다.\cite{wiki:venezuela}

\paragraph{}
%It might not happen in the next couple of years, or to the particular currency
%used in your country. But a glance at the list of historical
%currencies\footnote{See \textit{List of historical currencies} on Wikipedia.
	%\cite{wiki:historical-currencies}} shows that it will inevitably happen over a
%long enough period of time. I remember and used plenty of those listed: the
%Austrian schilling, the German mark, the Italian lira, the French franc, the
%Irish pound, the Croatian dinar, etc. My grandma even used the Austro-Hungarian
%Krone. As time moves on, the currencies currently in use\footnote{See
	%\textit{List of currencies} on Wikipedia \cite{wiki:list-of-currencies}} will
%slowly but surely move to their respective graveyards. They will hyperinflate or
%be replaced. They will soon be historical currencies. We will make them
%obsolete.
앞으로 몇 년 안에, 또는 당신의 국가에서 사용되는 특정 통화에서는 그러한 인플레이션이 발생하지 않을 수 있다. 
그러나 위키피디아에 있는 역사 속 통화 목록\footnote{\textit{List of historical currencies}\cite{wiki:historical-currencies}}
을 살펴보면 인플레이션은 오랜 기간에 걸쳐 반드시 일어날 것임을 알 수 있다.
오스트리아 실링, 독일 마르크, 이탈리아 리라, 프랑스 프랑, 아일랜드 파운드, 크로아티아 디나르 등.
우리 할머니는 오스트리아-헝가리 크로네도 사용하셨었다. 
시간이 지나면 현재 사용 중인 통화들도\footnote{\textit{List of currencies} 위키피디아의 \cite{wiki:list-of-currencies}}
느리지만 분명히 각자의 무덤으로 이동하게 될 것이다.
이 통화들은 과도하게 팽창하거나 교체되어 역사의 뒤안길로 사라질 것이다.
우리가 결국 그 화폐들을 쓸모 없게 만들 것이다.

\begin{quotation}\begin{samepage}
		\enquote{정부가 화폐 공급을 부풀리려는 유혹에 굴복할 수밖에 없다는 사실은 역사가 잘 말해준다.}
		\begin{flushright} -- 사이페딘 아모스\footnote{Saifedean Ammous, \textit{The Bitcoin
					Standard} \cite{bitcoin-standard}}
\end{flushright}\end{samepage}\end{quotation}

% \begin{comment}
% 	Why is Bitcoin different? In contrast to currencies mandated by the government,
% 	monetary goods which are not regulated by governments, but by the laws of
% 	physics\footnote{Gigi, \textit{Bitcoin's Energy Consumption - A shift in
% 			perspective} \cite{gigi:energy}}, tend to survive and even hold their respective
% 	value over time. The best example of this so far is gold, which, as the
% 	aptly-named \textit{Gold-to-Decent-Suit Ratio}\footnote{History shows that the
% 		price of an ounce of gold equals the price of a decent men's suit, according to Sionna
% 		investment managers \cite{web:gold-to-decent-suite-ratio}} shows, is holding its
% 	value over hundreds and even thousands of years. It might not be perfectly
% 	\enquote{stable} --- a questionable concept in the first place --- but the value it
% 	holds will at least be in the same order of magnitude.
% \end{comment}

\paragraph{}
그렇다면, 비트코인은 왜 다른가? 정부가 강제하는 통화와 달리, 
정부가 규제하지 않고 물리적 법칙에 의해 규제되는 화폐 상품\footnote{Gigi, 
	\textit{비트코인의 에너지 소비 - 관점의 전환(Bitcoin's Energy Consumption - A shift in perspective)}\cite{gigi:energy}}
은 시간이 지나도 생존하며 그 가치를 유지하는 경향을 보인다. 지금까지 가장 좋은 예는 금이다. 
금 대비 적합한 양복 값의 비율(Gold-to-Decent-Suit Ratio)\footnote{시오나(Siona) 
	투자 매니저에 따르면 역사적으로 금 1온스 가격은 괜찮은 남자 정장 가격과 동일하다. \cite{web:gold-to-decent-suite-ratio}}
이라는 이름에서 알 수 있듯이 금은 수백 년, 심지어 수천 년 동안 가치를 유지하고 있다. 
단기적으로 완벽하게 안정적이지는 않을 수도 있지만(애초에 완벽하게 안정적이라는 것이 의심스러운 개념이지만),
분명한 것은 적어도 금이 갖는 가치는 유지될 것이라는 사실이다.


% \begin{comment}
% 	If a monetary good or currency holds its value well over time and space,
% 	it is considered to be \textit{hard}. If it can't hold its value, because it
% 	easily deteriorates or inflates, it is considered a \textit{soft} currency. The
% 	concept of hardness is essential to understand Bitcoin and is worthy of
% 	a more thorough examination. We will return to it in the last economic
% 	lesson: sound money.
% \end{comment}

\paragraph{}
금전적 상품이나 화폐가 시간과 공간을 초월하여 가치를 유지한다면 견고하다고 여겨진다. 
반대로 쉽게 가치가 떨어지거나 부풀려져 가치를 유지할 수 없다면 연화(soft currency)로 간주된다. 
견고함(hardness)의 개념은 비트코인을 이해하는 데 필수적이며 좀 더 자세히 살펴볼 가치가 있다.
마지막 교훈에서 화폐의 건전성(sound money)에 대해서 다시 다루게 될 것이다.


% \begin{comment}
% 	As more and more countries suffer from
% 	hyperinflation more and more people will have to face the reality
% 	of hard and soft money. If we are lucky, maybe even some central bankers will be
% 	forced to re-evaluate their monetary policies. Whatever might happen, the
% 	insights I have gained thanks to Bitcoin will probably be invaluable, no matter
% 	the outcome.
% \end{comment}
\paragraph{}
점점 더 많은 국가에서 초인플레이션에 시달릴수록 점점 더 많은 사람들이 경화와 연화의 차이를 직시하게 될 것이다. 
운이 좋다면 일부 중앙은행가들도 자국의 통화정책을 재검토하게 될지 모른다. 
어떤 일이 일어나든 어떤 결과가 나타나든 비트코인 덕에 얻은 나의 통찰은 매우 귀중한 것이 될 것이다.

\paragraph{비트코인은 나에게 인플레이션이라는 숨겨진 세금과 초인플레이션의 재앙을 가르쳐주었다.}

% ---
%
% #### Down the Rabbit Hole
%
% - [Economics in One Lesson][Henry Hazlitt] by Henry Hazlitt
% - [1980's Unemployment and the Unions][unions] by Friedrich Hayek
% - [Good Money, Part II][good-money]: Volume Six of the Collected Works of F.A. Hayek
% - [The Bitcoin Standard] by Saifedean Ammous
% - [Hyperinflation][hyperinflates], [economic crisis in Venezuela][wiki-venezuela], [list of historical currencies], [list of currencies][currently in use] on Wikipedia
%
% [unions]: https://books.google.com/books/about/1980s_unemployment_and_the_unions.html?id=xM9CAQAAIAAJ
% [good-money]: https://books.google.com/books?id=l_A1vVIaYBYC
%
% [Henry Hazlitt]: https://mises.org/library/economics-one-lesson
% [hyperinflates]: https://en.wikipedia.org/wiki/Hyperinflation
% [inflation cannot help]: https://books.google.com/books?id=zZu3AAAAIAAJ&dq=%22only+while+it+accelerates%22&focus=searchwithinvolume&q=%22steady+inflation+cannot+help%22
% [history of inflation]: https://books.google.com/books?id=l_A1vVIaYBYC&pg=PA142&dq=%22history+is+largely+a+history+of+inflation%22&hl=en&sa=X&ved=0ahUKEwi90NDLrdnfAhUprVkKHUx1CmIQ6AEIKjAA#v=onepage&q=%22history%20is%20largely%20a%20history%20of%20inflation%22&f=false
% [wiki-venezuela]: https://en.wikipedia.org/wiki/Crisis_in_Venezuela#Economic_crisis
% [by the laws of physics]: https://link.medium.com/9fzq2L0J3S
% [\textit{Gold-to-Decent-Suit Ratio}]: https://www.businesswire.com/news/home/20110819005774/en/History-Shows-Price-Ounce-Gold-Equals-Price
% [The Bitcoin Standard]: https://thesaifhouse.wordpress.com/book/
%
% <!-- Wikipedia -->
% [alice]: https://en.wikipedia.org/wiki/Alice%27s_Adventures_in_Wonderland
% [carroll]: https://en.wikipedia.org/wiki/Lewis_Carroll
