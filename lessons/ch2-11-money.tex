\chapter{돈}
\label{les:11}

\begin{chapquote}{윌리엄 신부}
	\enquote{나는 젊었을 때부터, \ldots \\
		한 통에 5실링 하는 이 연고를 발라서, \\
		팔다리가 아주 유연해. \\
		너도 나한테 한두개 사서 발라보지 않을래?
	}\footnote{역자: 원작에는 1실링이라고 나온다.}
\end{chapquote}


% \begin{comment}
% 	What is money? We use it every day, yet this question is surprisingly
% 	difficult to answer. We are dependent on it in ways big and small, and
% 	if we have too little of it our lives become very difficult. Yet, we
% 	seldom think about the thing which supposedly makes the world go round.
% 	Bitcoin forced me to answer this question over and over again: What the
% 	hell is money?
% \end{comment}

\paragraph{}
돈이란 무엇인가? 
우리는 매일 돈을 사용하며 살아가지만 의외로 이 질문에 대답하기 어렵다. 
저마다 크고 작게 돈에 의존하고 있지만 돈이 너무 없으면 살기 힘들다.
그러나 우리는 돈이 세상을 어떻게 돌아가게 하는지 그 원동력에 대해선 거의 생각하지 않는다.
비트코인은 나에게 '도대체 돈이란 무엇인가?'라는 질문에 대한 대답을 강요했다.


% \begin{comment}
% 	In our \enquote{modern} world, most people will probably think of pieces of
% 	paper when they talk about money, even though most of our money is just
% 	a number in a bank account. We are already using zeros and ones as our
% 	money, so how is Bitcoin different? Bitcoin is different because at its
% 	core it is a very different \textit{type} of money than the money we currently
% 	use. To understand this, we will have to take a closer look at what
% 	money is, how it came to be, and why gold and silver was used for most
% 	of commercial history.
% \end{comment}

\paragraph{}
현대 사회를 사는 대부분의 사람들은 돈이라고 하면 종이 조각을 떠올리지만, 사실 대부분의 돈은 은행 계좌에 있는 숫자에 불과하다.
그렇다면 우리가 이미 0과 1을 돈으로 사용하고 있다는 것인데 비트코인은 어떻게 다른걸까? 
비트코인은 우리가 현재 사용하는 돈과 본질적으로 매우 다른 유형의 돈이다. 
이를 이해하려면 돈이 무엇인지, 화폐가 어떻게 생겨났는지, 
금과 은은 역사적으로 왜 가장 빈번하게 사용되었는지를 자세히 살펴보아야 한다.

% \paragraph{}
% \begin{comment}
% 	Seashells, gold, silver, paper, bitcoin. In the end, \textbf{money is whatever
% 		people use as money}, no matter its shape and form, or lack thereof.
% \end{comment}

\paragraph{}
조개껍데기, 금, 은, 종이, 비트코인. 결국 \textbf{사람들이 사용한다면 무엇이든 돈이 될 수 있다.}
돈의 모양, 형태, 또는 그것의 부족함과는 관계가 없다.


% \begin{comment}
% 	Money, as an invention, is ingenious. 
% 	A world without money is insanely
% 	complicated: How many fish will buy me new shoes? How many cows will buy
% 	me a house? What if I don't need anything right now but I need to get
% 	rid of my soon-to-be rotten apples? You don't need a lot of imagination
% 	to realize that a barter economy is maddeningly inefficient.
% \end{comment}
\paragraph{}
돈은 기발한 발명품이다.
돈이 없어지면 세상은 엄청나게 복잡해진다.
새 신발을 사려면 얼마나 많은 물고기가 필요할까? 
집을 사는데 몇 마리의 소가 필요할까? 
아무것도 살 필요가 없는 상황에서 곧 썩어버릴 사과를 없애야 한다면? 
물물교환 경제가 엄청나게 비효율임을 깨닫는 것은 그리 어렵지 않다.


% \begin{comment}
% 	The great thing about money is that it can be exchanged for \textit{anything
% 		else} --- that's quite the invention! As Nick
% 	Szabo\footnote{\url{http://unenumerated.blogspot.com/}} brilliantly summarizes
% 	in \textit{Shelling Out: The Origins of Money} \cite{shelling-out}, we humans
% 	have used all kinds of things as money: beads made of rare materials like ivory,
% 	shells, or special bones, various kinds of jewelry, and later on rare metals
% 	like silver and gold.
% \end{comment}

\begin{comment}
	돈의 가장 큰 장점은 무엇과도 교환할 수 있다는 것이다. 정말 대단한 발명품이다. 
	닉 재보\footnote{\url{http://unenumerated.blogspot.com/}}는 
	셸링 아웃: 화폐의 기원(Shelling out\footnote{역자: Shell out은 '지불하다.'라는 뜻으로 쓰이는데, 조개(shell)를 지불하는 데서 유래되었다고 한다.}: The Origins of Money)\cite{shelling-out}에서 
	이에 대해 훌륭하게 요약했다.
	\enquote{우리 인간은 상아, 조개, 특수 뼈와 같은 희귀한 재료로 만든 구슬, 다양한 종류의 장신구, 
		나중에는 은과 금과 같은 희귀 금속까지 모든 종류의 것을 돈으로 사용했다.}
\end{comment}


\begin{quotation}\begin{samepage}
		\enquote{이런 의미에서, 비트코인은 귀금속과 비슷하다. 
		공급량을 변경하여 가치를 유지하는 대신, 공급량은 미리 결정되어있고 그 가치가 변동된다.}
		\begin{flushright} -- 사토시 나카모토\footnote{Satoshi Nakamoto, in a reply to Sepp
				Hasslberger \cite{satoshi-precious-metal}}
\end{flushright}\end{samepage}\end{quotation}

% \begin{comment}
% 	Being the lazy creatures we are, we don't think too much about things
% 	which just work. Money, for most of us, works just fine. Like with our
% 	cars or our computers, most of us are only forced to think about the
% 	inner workings of these things if they break down. People who saw their
% 	life-savings vanish because of hyperinflation know the value of hard
% 	money, just like people who saw their friends and family vanish because
% 	of the atrocities of Nazi Germany or Soviet Russia know the value of
% 	privacy.
% \end{comment}

\begin{comment}
	인간은 게으른 동물이기 때문에 당장 잘 작동하는 것에 대해서는 별로 걱정하지 않는다. 
	우리 대부분에게 돈은 잘 작동한다.
	대부분의 사람들은 자동차나 컴퓨터처럼 고장이 난 경우에만 내부 작동 원리를 생각하게 된다.
	초인플레이션으로 인해 일생을 바쳐 모은 저축이 사라지는 것을 본 사람들은 경화(hard money)의 가치를 
	잘 알고 있다. 마치 나치 독일이나 소련 러시아의 잔혹 행위로 인해 친구와 가족의 죽음을 경험한 사람들이 프라이버시의 가치를 아는 것과 같다.
\end{comment}

% \begin{comment}
% 	The thing about money is that it is all-encompassing. Money is half of
% 	every transaction, which imbues the ones who are in charge with creating
% 	money with enormous power.
% \end{comment}
\begin{comment}
	문제는 돈이 우리의 모든 생활에 영향을 끼친다는 것이다.
	돈은 모든 거래의 절반을 차지하고 있고, 
	돈을 만드는 일을 담당하는 사람에게는 엄청난 힘이 있다.
\end{comment}

\begin{quotation}\begin{samepage}
		\enquote{돈이 모든 상거래의 절반을 차지하고 있고
			모든 문명의 흥망성쇠가 말 그대로 돈의 질에 따라 결정된다는 점을 감안할 때,
			우리도 알지 못하는 사이에 일어나고 있는 이 엄청난 힘에 대해 이야기하고 있는 것이다.
			그 힘이 지속되는 한 실제처럼 보이는 환상을 만들어낸다.
			이것이 연방준비제도가 가진 권력의 핵심이다.}
		\begin{flushright} -- 론 폴\footnote{Ron Paul, \textit{End the Fed} \cite{end-the-fed}}
\end{flushright}\end{samepage}\end{quotation}

%Bitcoin peacefully removes this power, since it does away with money
%creation and it does so without the use of force.
비트코인은 발권으로부터 나오는 힘을 평화적으로 무력화한다. 


% \begin{comment}
% 	Money went through multiple iterations. Most iterations were good. They
% 	improved our money in one way or another. Very recently, however, the
% 	inner workings of our money got corrupted. Today, almost all of our
% 	money is simply created \textit{out of thin air} by the powers that be. To
% 	understand how this came to be I had to learn about the history and
% 	subsequent downfall of money.
% \end{comment}

\begin{comment}
	돈은 여러 차례 반복을 거치고 있다. 
	대부분의 반복은 괜찮았다. 어떤 식으로든 개선되었다.
	그러나 최근에 화폐의 내부 작동이 손상되었다.
	오늘날 거의 모든 화폐는 권력에 의해 무에서 유를 창조하고 있다. 
	어떻게 이런 일이 벌어졌는지 이해하기 위해 나는 돈의 역사와 몰락에 대해 배워야 했다.
\end{comment}

% \begin{comment}
% 	If it will take a series of catastrophes or simply a monumental
% 	educational effort to correct this corruption remains to be seen. I pray
% 	to the gods of sound money that it will be the latter.
% \end{comment}
\begin{comment}
	이 부패를 바로잡기 위해서 일련의 재앙이 불어닥쳐야할지 아니면 단순히 훌륭한 교육이 필요할지는 아직 미지수이다. 
	하지만 건전 화폐의 신에게 후자이기를 기도해본다.
\end{comment}


\paragraph{비트코인은 나에게 돈이 무엇인지 알려주었다.}

% ---
%
% #### Down the Rabbit Hole
%
% - [End the Fed][Ron Paul] by Ron Paul
% - [Money, blockchains, and social scalability][social-scalability] by Nick Szabo
%
% [social-scalability]: https://unenumerated.blogspot.co.at/2017/02/money-blockchains-and-social-scalability.html
%
