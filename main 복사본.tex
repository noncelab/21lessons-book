% !TEX root = main.tex
% \documentclass{tufte-book}%[a4paper,twoside]
% See https://github.com/Tufte-LaTeX/tufte-latex/blob/master/sample-book.tex for details

% --- AMAZON BEGIN ---
% WITHOUT BLEED
% US Trade => 6x9
\documentclass[paper=6in:9in,pagesize=pdftex,
               headinclude=on,footinclude=on,12pt]{scrbook}
%
% Paper width
% W = 6in
% Paper height
% H = 9in
% Paper gutter
% BCOR = 0.5in
% Margin (0.5in imposed on lulu, recommended on createspace)
% m = 0.5in
% Text height
% h = H - 2m = 8in
% Text width
% w = W - 2m - BCOR = 4.5in
\areaset[0.50in]{4.5in}{8in}
% --- AMAZON END ---

% Copyright with title BEGIN
\usepackage{fancyhdr}
\def\secondpage{\clearpage\null\vfill
\pagestyle{empty}
\begin{minipage}[b]{0.9\textwidth}
\normalsize 21 Lessons \newline
\footnotesize What I've Learned From Falling Down the Bitcoin Rabbit Hole \par

Second edition. Version 0.3.12, git commit \texttt{8e70090}.

\footnotesize\raggedright
\setlength{\parskip}{0.5\baselineskip}
Copyright \copyright 2018--\the\year\ Gigi / \href{https://twitter.com/dergigi}{@dergigi} / \href{https://dergigi.com}{dergigi.com} \par

\includegraphics[width=2cm]{assets/images/cc-by-sa.pdf}

This book and its online version are distributed under the terms of the
Creative Commons Attribution-ShareAlike 4.0 license. A reference copy of this
license may be found at the official creative commons
page.\footnote{\url{https://creativecommons.org/licenses/by-sa/4.0}}

\end{minipage}
\vspace*{2\baselineskip}
\cleardoublepage
\rfoot{\thepage}}

\makeatletter
\g@addto@macro{\maketitle}{\secondpage}
\makeatother
% Copyright with title END

% Use serif font for chapters and parts
\setkomafont{disposition}{\bfseries}
\KOMAoptions{headings=small}

% Packages
\usepackage{setspace}
\usepackage{booktabs}
\usepackage{graphicx}
\setkeys{Gin}{width=\linewidth,totalheight=\textheight,keepaspectratio}
\graphicspath{{graphics/}}

%%
% For Quotes
\usepackage{csquotes}
\renewcommand\mkbegdispquote[2]{\makebox[0pt][r]{\textquotedblleft\,}}
\renewcommand\mkenddispquote[2]{\,\textquotedblright#2}

%%
% Just some sample text
\usepackage{lipsum}

%%
% For nicely typeset tabular material
\usepackage{booktabs}

%%
% Bibliography stuff: Biber, BibTex, BibLatex
%\usepackage[autostyle]{csquotes}
% \usepackage[
    % backend=biber,
    % style=authoryear-icomp,
    % sortlocale=de_DE,
    % natbib=true,
    % url=false,
    % doi=true,
    % eprint=false
% ]{biblatex}
% \usepackage[backend=biber]{biblatex}
\usepackage{url}
\usepackage{natbib}
\bibliographystyle{plain}

%%
% Hyperlinks
\usepackage[hidelinks]{hyperref}

%%
% For graphics / images
\usepackage{caption}
\usepackage{graphicx}
\setkeys{Gin}{width=\linewidth,totalheight=\textheight,keepaspectratio}
\graphicspath{{graphics/}}

% The fancyvrb package lets us customize the formatting of verbatim
% environments.  We use a slightly smaller font.
\usepackage{fancyvrb}
\fvset{fontsize=\normalsize}

%%
% Prints argument within hanging parentheses (i.e., parentheses that take
% up no horizontal space).  Useful in tabular environments.
\newcommand{\hangp}[1]{\makebox[0pt][r]{(}#1\makebox[0pt][l]{)}}

%%
% Prints an asterisk that takes up no horizontal space.
% Useful in tabular environments.
\newcommand{\hangstar}{\makebox[0pt][l]{*}}

%%
% Prints a trailing space in a smart way.
\usepackage{xspace}

% Prints the month name (e.g., January) and the year (e.g., 2008)
\newcommand{\monthyear}{%
  \ifcase\month\or January\or February\or March\or April\or May\or June\or
  July\or August\or September\or October\or November\or
  December\fi\space\number\year
}


% Prints an epigraph and speaker in sans serif, all-caps type.
\newcommand{\openepigraph}[2]{%
  %\sffamily\fontsize{14}{16}\selectfont
  \begin{fullwidth}
  \sffamily\large
  \begin{doublespace}
  \noindent\allcaps{#1}\\% epigraph
  \noindent\allcaps{#2}% author
  \end{doublespace}
  \end{fullwidth}
}

% Inserts a blank page
\newcommand{\blankpage}{\newpage\hbox{}\thispagestyle{empty}\newpage}

\usepackage{units}

% Typesets the font size, leading, and measure in the form of 10/12x26 pc.
\newcommand{\measure}[3]{#1/#2$\times$\unit[#3]{pc}}

% Macros for typesetting the documentation
\newcommand{\hlred}[1]{\textcolor{Maroon}{#1}}% prints in red
\newcommand{\hangleft}[1]{\makebox[0pt][r]{#1}}
\newcommand{\hairsp}{\hspace{1pt}}% hair space
\newcommand{\hquad}{\hskip0.5em\relax}% half quad space
\newcommand{\TODO}{\textcolor{red}{\bf TODO!}\xspace}
\newcommand{\na}{\quad--}% used in tables for N/A cells
\providecommand{\XeLaTeX}{X\lower.5ex\hbox{\kern-0.15em\reflectbox{E}}\kern-0.1em\LaTeX}
\newcommand{\tXeLaTeX}{\XeLaTeX\index{XeLaTeX@\protect\XeLaTeX}}
% \index{\texttt{\textbackslash xyz}@\hangleft{\texttt{\textbackslash}}\texttt{xyz}}
\newcommand{\tuftebs}{\symbol{'134}}% a backslash in tt type in OT1/T1
\newcommand{\doccmdnoindex}[2][]{\texttt{\tuftebs#2}}% command name -- adds backslash automatically (and doesn't add cmd to the index)
\newcommand{\doccmddef}[2][]{%
  \hlred{\texttt{\tuftebs#2}}\label{cmd:#2}%
  \ifthenelse{\isempty{#1}}%
    {% add the command to the index
      \index{#2 command@\protect\hangleft{\texttt{\tuftebs}}\texttt{#2}}% command name
    }%
    {% add the command and package to the index
      \index{#2 command@\protect\hangleft{\texttt{\tuftebs}}\texttt{#2} (\texttt{#1} package)}% command name
      \index{#1 package@\texttt{#1} package}\index{packages!#1@\texttt{#1}}% package name
    }%
}% command name -- adds backslash automatically
\newcommand{\doccmd}[2][]{%
  \texttt{\tuftebs#2}%
  \ifthenelse{\isempty{#1}}%
    {% add the command to the index
      \index{#2 command@\protect\hangleft{\texttt{\tuftebs}}\texttt{#2}}% command name
    }%
    {% add the command and package to the index
      \index{#2 command@\protect\hangleft{\texttt{\tuftebs}}\texttt{#2} (\texttt{#1} package)}% command name
      \index{#1 package@\texttt{#1} package}\index{packages!#1@\texttt{#1}}% package name
    }%
}% command name -- adds backslash automatically
\newcommand{\docopt}[1]{\ensuremath{\langle}\textrm{\textit{#1}}\ensuremath{\rangle}}% optional command argument
\newcommand{\docarg}[1]{\textrm{\textit{#1}}}% (required) command argument
\newenvironment{docspec}{\begin{quotation}\begin{samepage}\ttfamily\parskip0pt\parindent0pt\ignorespaces}{\end{flushright}\end{samepage}\end{quotation}}% command specification environment
\newcommand{\docenv}[1]{\texttt{#1}\index{#1 environment@\texttt{#1} environment}\index{environments!#1@\texttt{#1}}}% environment name
\newcommand{\docenvdef}[1]{\hlred{\texttt{#1}}\label{env:#1}\index{#1 environment@\texttt{#1} environment}\index{environments!#1@\texttt{#1}}}% environment name
\newcommand{\docpkg}[1]{\texttt{#1}\index{#1 package@\texttt{#1} package}\index{packages!#1@\texttt{#1}}}% package name
\newcommand{\doccls}[1]{\texttt{#1}}% document class name
\newcommand{\docclsopt}[1]{\texttt{#1}\index{#1 class option@\texttt{#1} class option}\index{class options!#1@\texttt{#1}}}% document class option name
\newcommand{\docclsoptdef}[1]{\hlred{\texttt{#1}}\label{clsopt:#1}\index{#1 class option@\texttt{#1} class option}\index{class options!#1@\texttt{#1}}}% document class option name defined
\newcommand{\docmsg}[2]{\bigskip\begin{fullwidth}\noindent\ttfamily#1\end{fullwidth}\medskip\par\noindent#2}
\newcommand{\docfilehook}[2]{\texttt{#1}\index{file hooks!#2}\index{#1@\texttt{#1}}}
\newcommand{\doccounter}[1]{\texttt{#1}\index{#1 counter@\texttt{#1} counter}}

% Generates the index
\usepackage{makeidx}
\makeindex

%%
% Chapter/Lesson Quotes
\makeatletter
\renewcommand{\@chapapp}{}% Not necessary...
\newenvironment{chapquote}[2][4em]
  {\setlength{\@tempdima}{#1}%
   \def\chapquote@author{#2}%
   \parshape 1 \@tempdima \dimexpr\textwidth-2\@tempdima\relax%
   \itshape}
  {\par\normalfont\hfill--\ \chapquote@author\hspace*{\@tempdima}\par\bigskip}
\makeatother

%%%%%%%%%%%%%%%%%%%%%%%%%%%%%%%%%%%%%%%%%%%%%%%%%%%%%%%%%%%%%%%%%%%%%%%%%%%%%%%%
%                                   DOCUMENT
%%%%%%%%%%%%%%%%%%%%%%%%%%%%%%%%%%%%%%%%%%%%%%%%%%%%%%%%%%%%%%%%%%%%%%%%%%%%%%%%

\begin{document}

\frontmatter

\title{21 Lessons}
\subtitle{What I've Learned from Falling Down the Bitcoin Rabbit Hole}
\author{Gigi}
\date{}

\maketitle

\cleardoublepage


\newpage \vspace*{8cm}
% Sets a PDF bookmark for the dedication
\pdfbookmark{Dedication}{dedication}
\thispagestyle{empty}
\begin{center}
	\Large \emph{
		%Dedicated to my wife, my child, and all the children of this world. May
		%bitcoin serve you well, and provide a vision for a future worth fighting for.
		나의 아내와 아이, 그리고 이 세상 모든 어린이에게 바칩니다.
		비트코인이 여러분에게 도움이 되고, 싸울 가치가 있는 미래에 대한 비전을 제공하길 바랍니다. 
	}
\end{center}

\chapter*{서문}
\pdfbookmark{Foreword}{foreword}

%Some call it a religious experience. Others call it Bitcoin.
\paragraph{}
누군가는 이를 두고 종교적 경험이라 부르고  
누군가는 비트코인이라 부른다.

%I first met Gigi in one of my spiritual homes -- Riga, Latvia -- the home of
%\textit{The Baltic Honeybadger} Conference, where the most fervent of the
%Bitcoin faithful make a yearly pilgrimage. After a deep lunchtime conversation,
%the bond Gigi and I forged was as set in stone as a Bitcoin transaction that was
%processed when we first shook hands a few hours prior.
\paragraph{}
나의 정신적 고향 중 하나인 라트비아 리가에서 지지(Gigi)를 처음 만났다. 리가는 발틱 벌꿀오소리 컨퍼런스(\textit{The Baltic Honeybadger Conference - 유럽에서 열리는 비트코인 컨퍼런스})의 본거지로 비트코인의 열렬한 지지자들이 매년 방문하는 곳이다. 
지지와 나는 점심시간 내내 깊은 대화를 나눈 후, 그와의 유대감은 몇 시간 전 우리가 처음 악수했을 당시 처리된 비트코인 거래 만큼이나 확고했다. 


%My other spiritual home, Christ Church, Oxford, where I had the privilege to
%study for my MBA, was where I had my \enquote{Rabbit Hole} moment. Like Gigi, I
%transcended the economic, technical and social realms, and was spiritually
%enveloped by Bitcoin. After \enquote{buying high} in the November 2013 bubble,
%there were several extremely hard-learned lessons to be had in the relentlessly
%crushing and seemingly never-ending 3-year bear market. These 21 Lessons would
%indeed have served me very well in that time. Many of these lessons are simply
%natural truths that, to the uninitiated, are obscured by an opaque, fragile
%film. By the end of this book however, the fa\c{c}ade will fragment fiercely.
\paragraph{}
나의 또 다른 정신적 고향인 옥스포드 크라이스트 처치(Christ Church)는 내가 MBA를 다녔던 특권을 누린 곳이자 \enquote{토끼굴}의 순간을 맞이했던 곳이다.
지지처럼 나는 경제적, 기술적, 사회적 영역을 초월했고 영적으로 비트코인에 둘러싸여 있었다. 
2013년 11월 버블 속에서 비트코인을 \enquote{고점 구매} 한 후, 끊임없이 파괴되고 결코 끝나지 않을 것처럼 보이는 3년의 약세장을 견디며 극도로 힘들게 얻은 몇가지 교훈이 있었다.
그 당시 내가 이 스물한 가지의 교훈을 알았더라면 큰 도움이 되었을 것이다. 
이 교훈들은 초심자에게 불투명하지만 쉽게 깨지는 필름으로 가려져있는 지극히 자연스러운 진실이다. 그러나 이 책이 끝날 무렵 그 필름은 산산조각 날 것이다.


%On a crystal-clear night in Oxford in late-August 2016, just a few weeks after
%the knife twisted in my heart again when the Bitfinex Exchange was hacked, I sat
%in quiet contemplation at Christ Church’s Master’s Garden. Times were tough, and
%I was at my mental and emotional breaking-point after what seemed to be a
%lifetime of torture; not because of financial loss, but of the crushing
%spiritual loss I felt being isolated in my world view. If only there were
%resources like this one at the time to see that I was not alone. The Master’s
%Garden is a very special place to me and many who came before me over the
%centuries. It was there where one Charles Dodgson, a Math Tutor at Christ
%Church, observed one of his young pupils, Alice Liddell, the daughter of the
%Dean of Christ Church. Dodgson, better known by his pen-name, Lewis Carroll,
%used Alice and The Garden as his inspiration, and in the magic of that hallowed
%turf, I stared deeply into the crypto-chasm, and it stared blazingly back,
%annihilating my arrogance, and slapping my self-pride square in the face. I was
%finally at peace.
\paragraph{}
2016년 8월 말 청명한 밤에, 비트파이넥스 해킹으로 또 한번의 심장을 후벼파는 아픔을 겪고 나서 몇 주 후, 나는 크라이스트 처치의 마스터스 가든에 조용히 앉아 생각에 잠겼다. 
힘든 시간을 보내며 평생 고문을 당한 것처럼 정신적, 감정적 한계에 다다라 있었다. 그것은 경제적 손실 때문이 아니라, 내 세계관에 고립된 느낌을 받아 정신적으로 참담했기 때문이었다. 
당시에 내가 혼자가 아니라는 것을 알 수 있는 이 스물한 가지 교훈 같은 자료들이 있었으면 참 좋았을 것이다. 
마스터스 가든은 수 세기에 걸쳐 나와 내 이전에 존재한 많은 이들에게 매우 특별한 장소이다.
그리스도 교회의 수학 교사인 찰스 도지슨(Charles Dodgson)이 그의 어린 학생 중 한 명인 그리스도 교회 학장의 딸 앨리스 리델을 관찰했던 의미 있는 곳이기도 하다.
그의 필명 '루이스 캐롤'로 더 유명한 도지슨은 앨리스와 정원에서 영감을 얻었다. 나는 그 신성한 정원의 마법으로 비밀의 틈을 깊숙이 바라보았고 
그것은 타오르는 듯한 눈빛으로 나의 오만함을 말살시키고 자만하던 나의 뺨을 때렸다. 그리고 나는 마침내 평화를 얻었다. 

%21 Lessons takes you on a true Bitcoin journey; not just a journey of
%philosophy, technology and economics, but of the soul.
\paragraph{}
스물 한가지 교훈은 철학, 기술, 경제학의 관점을 넘어 영혼에 이르는 진정한 비트코인 여행을 경험하게 해준다.

%As you dive deeper into the philosophy tersely laid out in 7 of the 21 Lessons,
%one can go as far as to understand the origin of all beings with enough time and
%contemplation. His 7 lessons on economics captures, in simple terms, how we are
%at the financial mercy of a small group of Mad Hatters, and how they have
%successfully managed to put blinders on our minds, hearts and souls. The 7
%lessons on technology lay out the beauty and technologically-Darwinian
%perfection of Bitcoin. Being a non-technical Bitcoiner, the lessons provide a
%salient review of the underlying technological nature of Bitcoin, and indeed,
%the nature of technology itself.
\paragraph{}
21개의 교훈 중 7개에 간결하게 제시된 철학에 더 깊이 파고들면 충분한 시간과 숙고를 통해 모든 존재의 기원을 이해하는 데까지 나아갈 수 있다. 
경제학에 대한 7개의 교훈은 간단한 용어로 미치광이 모자장수(Mad Hatters - 이상한 나라의 앨리스 등장 인물인 모자장수로 앨리스와 이상한 다과회를 갖는다.) 소수 집단이 어떻게 성공적으로 우리의 마음과 영혼의 눈을 멀게 할 수 있었는지 알려준다. 
기술에 관한 일곱 가지 교훈은 비트코인의 아름다움과 기술적으로 다윈적인(진화론적인) 완벽함을 설명한다. 
기술을 잘 모르는 비트코이너로서 이 교훈은 비트코인의 근본적인 기술적 특성과 기술 자체의 특성에 대한 핵심을 이해할 수 있게 해준다.

%In this transient experience we call life, we live, love and learn. But what is
%life but a timestamped order of events?
\paragraph{}
우리가 삶이라 부르는 이 일시적인 경험 속에서, 우리는 살아가고 사랑하며 배운다. 
하지만 어쩌면 삶이란 시간이 지정된(timestamped) 사건의 연속일 뿐 아닐까? 

%Conquering the Bitcoin mountain is not easy. False summits are rife, rocks are
%rough, and cracks and crevices are ubiquitously lying in wait to swallow you up.
%After reading this book, you will see that Gigi is the ultimate Bitcoin Sherpa,
%and I will appreciate him forever.
\paragraph{}
비트코인이라는 산을 정복하는 것은 쉽지 않다. 거짓된 정상은 만연하고 바위는 거칠며 당신을 집어삼키려는 균열과 틈새가 도사리고 있다.
이 책을 읽고 나면 지지가 궁극의 비트코인 셰르파(네팔에서 히말라야 산맥을 등산하는 사람을 가이드하는 현지인 또는 네팔 특정 민족을 칭한다.)임을 알게 될 것이다. 
그리고 나는 그를 영원히 존경할 것이다. 

\begin{flushright}
	%Hass McCook \\의
	%November 29, 2019
	2019년 11월 29일 \\
	하스 맥쿡(Hass McCook)
\end{flushright}


\newpage \vspace*{4cm}
\thispagestyle{empty}
\begin{quotation}
	\begin{center}
		\large
		%\enquote{Would you tell me, please, which way I ought to go from here?} \\~\\
		%\enquote{That depends a good deal on where you want to get to.} \\~\\
		%\enquote{I don't much care where --} \\~\\
		%\enquote{Then it doesn't matter which way you go.}
		\enquote{말해줄래, 제발, 난 어디로 가야 하지?} \\~\\
		\enquote{그건 네가 어디로 가고 싶은지에 달렸지.} \\~\\
		\enquote{어디든 상관없어.} \\~\\
		\enquote{그럼, 어디로 가든 상관없겠네.}
	\end{center}
	\begin{flushright} -- 루이스 캐롤, \textit{이상한 나라의 앨리스}\end{flushright}
\end{quotation}
No newline at end of file

\tableofcontents


\def\bitcoinB{\leavevmode
	{\setbox0=\hbox{\textsf{B}}%
		\dimen0\ht0 \advance\dimen0 0.2ex
		\ooalign{\hfil \box0\hfil\cr
			\hfil\vrule height \dimen0 depth.2ex\hfil\cr
		}%
	}%
}

\chapter*{이 책에 대하여 \\ (... 그리고 저자에 대하여)}
\pdfbookmark{About This Book (... and About the Author)}{about}

%This is a bit of an unusual book. But hey, Bitcoin is a bit of an unusual
%technology, so an unusual book about Bitcoin might be fitting. I'm not sure if
%I'm an unusual guy (I like to think of myself as a \textit{regular} guy) but the
%story of how this book came to be, and how I came to be an author, is worth
%telling.

\paragraph{}
이 책은 조금 특이하다. 
비트코인은 약간 특이한 기술이기 때문에 비트코인에 대한 책으론 특이한 것이 알맞을 수 있다. 
내가 특이한 사람인진 잘 모르겠지만(전 제가 평범한 사람이라고 생각하고 싶어요.) 
이 책이 어떻게 탄생하게 됐는지, 그리고 왜 내가 작가가 되었는지 이야기하는 것은 가치있다고 생각한다.

%First of all, I'm not an author. I'm an engineer. I didn't study writing. I
%studied code and coding. Second of all, I never intended to write a book, let
%alone a book about Bitcoin. Hell, I'm not even a native speaker.\footnote{The
	%reason why I'm writing these words in English is that my brain works in
	%mysterious ways. Whenever something technical comes up, it switches to English
	%mode.} I'm just a guy who caught the Bitcoin bug. Hard.

\paragraph{}
우선, 나는 작가가 아니다. 엔지니어다. 글쓰기 공부를 한 적이 없다. 나는 코드와 코딩을 전공했다. 
둘째, 비트코인에 대한 책은 말할 것도 없고 내가 책을 쓸거라 생각해 본 적이 없다. 심지어 원어민도 아니다. 
\footnote{제가 영어로 이 글을 쓰는 이유는 제 뇌가 신비한 방식으로 작동하기 때문입니다. 기술적인 내용이 떠오를 때마다 영어 모드로 전환됩니다.}
단지 비트코인 버그를 잡던 사람이다. 열심히.


%Who am \textit{I} to write a book about Bitcoin? That's a good question. The
%short answer is easy: I'm Gigi, and I'm a bitcoiner.

\paragraph{}
비트코인에 관련된 책을 쓰는 나는 누구인가? 좋은 질문이다. 
짧게 대답하면 쉽다. 내 이름은 Gigi이고, 비트코이너이다.

%The long answer is a bit more nuanced.

\paragraph{}
길게 대답하자면 약간 미묘해진다.

\paragraph{}
%My background is in computer science and software development. In a
%previous life, I was part of a research group that tried to make computers think
%and reason, among other things. In yet another previous life I wrote software
%for automated passport processing and related stuff which is even scarier. I
%know a thing or two about computers and our networked world, so I guess I have a
%bit of a head-start to understand the technical side of Bitcoin. However, as I
%try to outline in this book, the tech side of things is only a tiny sliver of
%the beast which is Bitcoin. And every single one of these slivers is important.
내 전공은 컴퓨터 과학과 소프트웨어 개발이다. 
이전에 나는 컴퓨터가 사고하고 추론할 수 있도록 연구하는 조직에 속해 있었다. 
또 자동화 여권을 개발하고 관련 작업을 하기도 했는데 그건 훨씬 무시무시한 일이었다. 
나는 컴퓨터와 네트워크 세계에 대해 한두 가지 정도는 알고 있기 때문에 비트코인의 기술적 측면을 이해하는 데 어느 정도 앞서 있다고 생각한다.
하지만 이 책에서 설명하고자 하는 것처럼 기술적인 측면은 비트코인이라는 거대한 야수를 이루는 단 하나의 작은 조각에 불과하다. 
그리고 이런 작은 조각 하나하나가 모두 중요하다. 

%This book came to be because of one simple question: \textit{\enquote{What have
		%you learned from Bitcoin?}} I tried to answer this question in a single tweet.
%Then the tweet turned into a tweetstorm. The tweetstorm turned into an article.
%The article turned into three articles. Three articles turned into 21 Lessons.
%And 21 Lessons turned into this book. So I guess I'm just really bad at
%condensing my thoughts into a single tweet.런런

\paragraph{}
이 책은 아주 간단한 질문으로 인해 탄생했다.  
\enquote{당신은 비트코인에서 무엇을 배웠습니까?}
나는 이 질문에 대한 답을 트윗 한 줄에 담으려고 노력했다.
그러자 그 트윗은 트윗 폭풍이 되어 돌아왔다. 트윗 폭풍은 기사로 바뀌었고, 이내 기사는 세 개가 되었다. 
세 개의 기사는 21개의 교훈으로 바뀌었다. 그리고 21개의 교훈이 이 책이 되었다. 내 생각을 트윗 한 줄로 압축하는 것에 정말 서툴렀던 것 같다.


%you might
%ask. Again, there is a short and a long answer. The short answer is that I
%simply had to. I was (and still am) \textit{possessed} by Bitcoin. I find it to
%be endlessly fascinating. I can't seem to stop thinking about it and the
%implications it will have on our global society. The long answer is that I
%believe that Bitcoin is the single most important invention of our time, and
%more people need to understand the nature of this invention. Bitcoin is
%still one of the most misunderstood phenomena of our modern world, and it took
%me years to fully realize the gravitas of this alien technology. Realizing what
%Bitcoin is and how it will transform our society is a profound experience. I%
%hope to plant the seeds which might lead to this realization in your head.
\paragraph{}
\enquote{왜 이 책을 썼는가?} 라고 물어볼 수 있겠다.
다시 말하지만 이 질문에는 짧은 대답과 긴 대답이 있다.  
짧은 대답은 단순히 그래야만 했기 때문이다. 나는 비트코인에 매료됐고 지금도 그렇다. 비트코인의 매력은 끝이 없다고 생각한다. 
비트코인과 비트코인이 글로벌 사회에 미칠 영향에 대한 생각을 멈출 수가 없다.
긴 대답을 하자면 비트코인은 우리 시대의 가장 중요한 발명품이며, 더 많은 사람들이 이 발명품의 본질을 이해해야 한다고 믿기 때문이다. 
비트코인은 여전히 현대 사회에서 가장 오해받는 현상 중 하나이고, 나 역시 이 낯선 기술의 중요함을 깨닫기까지 몇 년이 걸렸다. 
비트코인이 무엇인지, 그리고 비트코인이 우리 사회를 어떻게 변화시킬지 깨닫는 것은 심오한 경험이다. 
나는 여러분의 머릿속에 이러한 깨달음으로 이어질 수 있는 씨앗을 심고 싶다.


%While this section is titled \enquote{\textit{About This Book (... and About the
		%Author)}}, in the grand scheme of things, this book, who I am, and what I did
%doesn't really matter. I am just a node in the network, both literally
%\textit{and} figuratively. Plus, you shouldn't trust what I'm saying anyway. As
%we bitcoiners like to say: do your own research, and most importantly: don't
%trust, verify.
\paragraph{}
이 섹션의 제목이 \enquote{이 책에 대하여 (... 그리고 저자에 대하여)}이지만, 큰 틀에서 보면, 이 책, 내가 누군지, 내가 무엇을 했던 사람인지는 중요하지 않다. 
나는 말 그대로 그리고 비유적으로 네트워크의 한 노드에 불과하다. 
게다가 여러분은 내가 하는 말을 그대로 믿어선 안 된다. 
비트코이너들이 자주하는 말처럼, 스스로 조사하고 공부해야 한다. 가장 중요한 것은 믿지 말고 검증하라(Don't trust, verify.)는 것이다.

%I did my best to do my homework and provide plenty of sources for you, dear
%reader, to dive into. In addition to the footnotes and citations in this book, I
%try to keep an updated list of resources at
%\href{https://21lessons.com/rabbithole}{21lessons.com/rabbithole} and on
%\href{https://bitcoin-resources.com}{bitcoin-resources.com}, which also lists
%plenty of other curated resources, books, and podcasts that will help you to
%understand what Bitcoin is.
\paragraph{}
나는 독자 여러분이 공부하고 더 깊이 있게 살펴볼 수 있도록 다양한 자료를 제공하기 위해 최선을 다했다.
이 책의 각주 및 인용문 외에도 리소스 목록을 최신 상태로 유지하려고 노력하고 있다. 
\href{https://21lessons.com/rabbithole}{21lessons.com/rabbithole}과 \href{https://bitcoin-resources.com}{bitcoin-resources.com}
에는 비트코인을 이해하는 데 도움이 될 만한 엄선된 리소스, 서적, 팟캐스트도 많이 포함되어 있다.

\paragraph{}
%In short, this is simply a book about Bitcoin, written by a bitcoiner.
%Bitcoin doesn't need this book, and you probably don't need this book to
%understand Bitcoin. I believe that Bitcoin will be understood by you as soon as
%\textit{you} are ready, and I also believe that the first fractions of a bitcoin
%will find you as soon as you are ready to receive them. In essence, everyone
%will get \bitcoinB{}itcoin at exactly the right time. In the meanwhile, Bitcoin
%simply is, and that is enough.\footnote{Beautyon, \textit{Bitcoin is. And that
		%is enough.}~\cite{bitcoin-is}}
간단히 말해, 이 책은 비트코이너가 쓴 비트코인에 관한 책이다. 
비트코인은 꼭 이 책이 필요하지 않으며, 비트코인을 이해하기 위해 당신도 이 책이 필요없을지 모른다. 
당신이 준비가 되었다면 즉시 비트코인을 이해할 수 있을 것이고, 
비트코인을 이해했다면 비트코인을 받아들일 수 있을 것이다. 
본질적으로 모든 사람이 아주 적절한 시기에 \bitcoinB{}itcoin을 받아들이게 될 것이다. 
그 동안 비트코인은 그 자체로 충분하다.\footnote{뷰티온(Beautyon, 역자: 필명), \textit{비트코인이 있다. 그리고 그 자체로 충분하다. (Bitcoin is. And that is enough.)} ~\cite{bitcoin-is}}.
\chapter*{서문}

%Falling down the Bitcoin rabbit hole is a strange experience. Like many others,
%I feel like I have learned more in the last couple of years studying Bitcoin
%than I have during two decades of formal education.
비트코인 토끼 굴에 빠지는 것은 이상한 경험이다. 
다른 사람들과 마찬가지로 나는 지난 20년 동안의 받은 정식 교육보다 비트코인을 공부하며
더 많은 것을 배웠다는 사실을 깨달을 수 있었다.

%The following lessons are a distillation of what I’ve learned. First published
%as an article series titled \textit{“What I’ve Learned From Bitcoin,”} what follows
%can be seen as a third edition of the original series.
이 가르침들은 내가 배워온 것들을 무색하게 한다. 
이 글은 “내가 비트코인으로부터 배운 것들”에서 시작된 글의 세 번째 에디션이다.

%Like Bitcoin, these lessons aren't a static thing. I plan to work on them
%periodically, releasing updated versions and additional material in the future.
비트코인 그러하듯, 이 교훈들은 멈추지 않는다. 
나는 주기적으로 글을 업데이트하기로 했다.

%Unlike Bitcoin, future versions of this project do not have to be backward
%compatible. Some lessons might be extended, others might be reworked or
%replaced.
비트코인이 그렇지 않듯이, 이 글은 하위 호환성을 제공할 필요가 없다. 
어떤 가르침은 확장될 것이며, 어떤 것들은 대체될지도 모르기 때문이다.

%Bitcoin is an inexhaustible teacher, which is why I do not claim that these
%lessons are all-encompassing or complete. They are a reflection of my personal
%journey down the rabbit hole. There are many more lessons to be learned, and
%every person will learn something different from entering the world of Bitcoin.
비트코인은 포괄적이거나 완벽한 교훈을 주장하지 않기 때문에 지칠 줄 모르는 선생님과 같다. 
그저 토끼 굴을 따라 들어갈 뿐이다.
그곳에는 더 많은 가르침이 있고 모든 사람은 비트코인 세계에 들어가고 나서 각각 다른 것을 배운다.

%I hope that you will find these lessons useful and that the process of learning
%them by reading won’t be as arduous and painful as learning them firsthand.
이 교훈들이 당신에게 유용하기를 바라며, 그 과정이 힘들고 고통스럽지 않았으면 좋겠다.

% <!-- Internal -->
% [I]: 
%
% <!-- Twitter -->
% [dergigi]: https://twitter.com/dergigi
%
% <!-- Wikipedia -->
% [alice]: https://en.wikipedia.org/wiki/Alice%27s_Adventures_in_Wonderland
% [carroll]: https://en.wikipedia.org/wiki/Lewis_Carroll

%%
% Start the main matter (normal chapters)
\mainmatter

\part*{21 Lessons}

\newpage \vspace*{8cm}
\thispagestyle{empty}
\begin{quotation}
	\begin{center}
		\large
		\enquote{오 어리석은 앨리스!} 그녀는 스스로 대답했다. \enquote{
			여기서 어떻게 수업을 할 수 있겠어? 널 위한 공간도 거의 없고 수업 교재를 위한 공간도 전혀 없어!}
	\end{center}
	\begin{flushright} --루이스 캐롤, \textit{이상한 나라의 앨리스}\end{flushright}
\end{quotation}

\chapter*{들어가며}
\label{ch:introduction}

\begin{chapquote}{루이스 캐롤, \textit{이상한 나라의 앨리스}}
	\enquote{나는 이상한 세상에 들어가고 싶지 않아.}  앨리스가 말했다. 
	\enquote{너는 여기서 빠져나올 수 없어.}  고양이가 말했다. 
	\enquote{우리는 모두 미쳤어. 나도 미쳤어. 너도 미쳤어.} 
	\enquote{내가 미친것을 네가 어찌 알아?}  앨리스가 물었다. 
	\enquote{그래야만해.} 고양이가 대답했다, 
	\enquote{아니면 여기 오지 않았을 거야.}
\end{chapquote}

%In October 2018, Arjun Balaji asked the innocuous question,
%\textit{What have you learned from Bitcoin?} After trying to answer this
%question in a short tweet, and failing miserably, I realized that the things
%I've learned are far too numerous to answer quickly, if at all.
2018년 10월에 아준 발라지(Ajun Balaji)는 나에게 물었다.
\enquote{비트코인으로부터 배운 것이 뭐야?} 나는 트윗을 통해 이 질문에 대답하기 위해 노력했지만 할 수 없었다.
짧은 트윗으로 대답하기에는 너무 많은 것을 배웠다.

%The things I've learned are, obviously, about Bitcoin - or at least related to
%it. However, while some of the inner workings of Bitcoin are explained, the
%%following lessons are not an explanation of how Bitcoin works or what it is,
%they might, however, help to explore some of the things Bitcoin touches:
%philosophical questions, economic realities, and technological innovations.
내가 배운 것은 명백하게 비트코인에 관한 것이다. 하지만 비트코인의 작동 원리는 비트코인을 설명하기엔 부족하다.
이 책에서 다루는 내용은 비트코인의 작동 원리가 아니다. 우리는 비트코인의 작동 원리로부터 철학적
질문, 경제적 현실, 기술 혁신을 배울 수 있다.


\begin{center}
	\includegraphics[width=7cm]{assets/images/the-tweet.png}
\end{center}

%The \textit{21 Lessons} are structured in bundles of seven, resulting in three
%chapters. Each chapter looks at Bitcoin through a different lens, extracting
%what lessons can be learned by inspecting this strange network from a different
%angle.
스물 한가지 교훈은 일곱 개씩 묶어서 세 개의 챕터로 나누어진다. 각각 챕터에서는 비트코인을 다른 관점에서
바라본다. 이 이상한 네트워크를 다른 관점에서 바라보며 어떤 교훈이 있는지 살펴보자.

%\paragraph{\hyperref[ch:philosophy]{Chapter 1}}{explores the philosophical
	%teachings of Bitcoin. The interplay of immutability and change, the concept of
	%true scarcity, Bitcoin's immaculate conception, the problem of identity, the
	%contradiction of replication and locality, the power of free speech, and the
	%limits of knowledge.
	%}

\paragraph{\hyperref[ch:philosophy]{Chapter 1}}{에서는 비트코인의 철학적 가르침을 알아본다.
	\begin{itemize}
		\item 불변성과 변화
		\item 희소성의 개념
		\item 비트코인의 무결한 개념
		\item 비트코인의 정체성 문제
		\item 복제와 지역성의 모순
		\item 언론 자유의 힘
		\item 지식의 한계
\end{itemize}}

%explores the philosophical
%teachings of Bitcoin. The interplay of immutability and change, the concept of
%true scarcity, Bitcoin's immaculate conception, the problem of identity, the
%contradiction of replication and locality, the power of free speech, and the
%limits of knowledge.
%}

%\paragraph{\hyperref[ch:economics]{Chapter 2}}{explores the economic teachings
%of Bitcoin. Lessons about financial ignorance, inflation, value, money and the
%history of money, fractional reserve banking, and how Bitcoin is re-introducing
%sound money in a sly, roundabout way.}
\paragraph{\hyperref[ch:economics]{Chapter 2}}{에서는 비트코인의 경제적 가르침을 알아본다.
\begin{itemize}
	\item 금융 무지
	\item 인플레이션
	\item 가치
	\item 돈과 돈의 역사
	\item 지급 준비 제도
	\item 비트코인이 우회적인 방법으로 건전화폐 사회를 만드는 방법
	\end{itemize}}
	
	%\paragraph{\hyperref[ch:technology]{Chapter 3}}{explores some of the lessons
%learned by examining the technology of Bitcoin.  Why there is strength in
%numbers, reflections on trust, why telling time takes work, how moving slowly
%and not breaking things is a feature and not a bug, what Bitcoin's creation can
%tell us about privacy, why cypherpunks write code (and not laws), and what
%metaphors might be useful to explore Bitcoin's future.}
\paragraph{\hyperref[ch:technology]{Chapter 3}}{에서는 비트코인의 기술을 통한 가르침을 알아본다.
\begin{itemize}
	\item 왜 숫자들이 중요한지
	\item 신뢰에 대한 고찰
	\item 작업시간이 오래 걸리는 이유
	\item 느린 움직임과 깨지지 않는 것이 버그가 아닌 이유
	\item 비트코인이 프라이버시에 대해 말해주는 것
	\item 싸이퍼 펑크가 (법률이 아닌)코드를 작성하는 이유
	\item 미래를 탐험하는데에 유용할지 모르는 비트코인의 비유
	\end{itemize}}
	
	~
	
	%Each lesson contains several quotes and links throughout the text. If an idea is
	%worth exploring in more detail, you can follow the links to related works in the
	%footnotes or in the bibliography.
	각 교훈에는 인용문과 링크가 포함되어 있다. 만약 더 자세한 설명이 필요하다면 링크를 통해 자료를 확인해 보기
	바란다.
	
	%Even though some prior knowledge about Bitcoin is beneficial, I hope that these
	%lessons can be digested by any curious reader. While some relate to each other,
	%each lesson should be able to stand on its own and can be read independently. I
	%did my best to shy away from technical jargon, even though some domain-specific
	%vocabulary is unavoidable.
	비트코인에 대해 궁금한 독자가 이 책의 내용을 소화할 수 있기를 바란다. 각 교훈은 독립적으로 설명되기 때문에
	발췌독해도 괜찮다. 쉬운 이해를 위해 일부 피할 수 없는 경우를 제외하고는 최대한 기술적인 용어를 사용하지 않으려고 노력하였다.
	
	%I hope that my writing serves as inspiration for others to dig beneath the
	%surface and examine some of the deeper questions Bitcoin raises. My own
	%inspiration came from a multitude of authors and content creators to all of whom
	%I am eternally grateful.
	이 글을 통해 비트코인에 대해 더 탐구할 수 있는 계기가 되기를 바란다. 이 책의 영감은 다수 저자와
	콘텐츠 제작자들에게서 나왔다. 도움을 준 모든 이들에게 감사의 말을 전한다.
	
	%Last but not least: my goal in writing this is not to convince you of anything.
	%My goal is to make you think, and show you that there is way more to Bitcoin
	%than meets the eye. I can’t even tell you what Bitcoin is or what Bitcoin will
	%teach you. You will have to find that out for yourself.
	마지막으로, 이 글은 당신을 설득하기 위해 쓴 글이 아니다. 이 글의 목표는 비트코인이 눈에 보이는 것이
	전부가 아니라는 점을 알려주는 것이다. 나는 비트코인이 무엇이고 무엇을 가르치는지 말할 수 없다. 당신이 스스로
	찾아야 한다.
	
	\begin{quotation}\begin{samepage}
	\enquote{이 이후는 되돌릴 수 없다네. 당신이 파란 약을 먹는 순간 이야기는 끝나고 당신이 침대에서 일어나는 순간
		당신이 믿는 것을 볼 수 있네. 만약 빨간 약을 먹는다면\footnote{the \textit{orange} pill} 당신은 이상한 나라에 가게 될걸세, 
		그리고 이 토끼 굴이 얼마나 깊은지를 보여주지.}
	\begin{flushright} -- 모피어스
	\end{flushright}\end{samepage}\end{quotation}
	
	\begin{figure}
\includegraphics{assets/images/bitcoin-orange-pill.jpg}
\caption*{기억하라. 너에게 알려주고자 하는 것은 과장이 없는 진실이다.}
\label{fig:bitcoin-orange-pill}
\end{figure}

%
% [Morpheus]: https://en.wikipedia.org/wiki/Red_pill_and_blue_pill#The_Matrix_(1999)
% [this question]: https://twitter.com/arjunblj/status/1050073234719293440
%
% <!-- Internal -->
% [chapter1]: {{ 'bitcoin/lessons/ch1-00-philosophy' | absolute_url }}
% [chapter2]: {{ 'bitcoin/lessons/ch2-00-economics' | absolute_url }}
% [chapter3]: {{ 'bitcoin/lessons/ch3-00-technology' | absolute_url }}
%
% <!-- Wikipedia -->
% [alice]: https://en.wikipedia.org/wiki/Alice%27s_Adventures_in_Wonderland
% [carroll]: https://en.wikipedia.org/wiki/Lewis_Carroll

\part{철학}
\label{ch:philosophy}
\chapter*{철학}

\begin{chapquote}{루이스 캐롤, \textit{이상한 나라의 앨리스}}
	생쥐는 호기심 어린 눈으로 그녀를 바라보았고, 작은 눈으로 그녀에게 윙크하는 것 같았지만 아무 말도 하지 않았다.
\end{chapquote}

%Looking at Bitcoin superficially, one might conclude that it is slow, wasteful,
%unnecessarily redundant, and overly paranoid. Looking at Bitcoin inquisitively,
%one might find out that things are not as they seem at first glance.
비트코인을 겉핥기로 바라보는 것은 비트코인을 이해하는 데 도움이 되지 않는다. 조금만 더 호기심을 가지고 비트코인을 살펴보면, 보이는 것이 
전부가 아니라는 것을 알 수 있을 것이다.

%Bitcoin has a way of taking your assumptions and turning them on their heads.
%After a while, just when you were about to get comfortable again, Bitcoin will
%smash through the wall like a bull in a china shop and shatter your assumptions
%once more.
당신이 나름의 비트코인의 가정을 받아들이고 머릿속에서 오랫동안 고민하여 결론을 낼 수도 있다. 하지만,
당신이 이해했다고 생각할 때쯤에 비트코인은 당신의 가정을 깨트리고 다시 한번 머리를 복잡하게 만든다.

\begin{figure}
	\includegraphics{assets/images/blind-monks.jpg}
	\caption{비트코인 황소를 조사하는 맹인 승려}
	\label{fig:blind-monks}
\end{figure}

%Bitcoin is a child of many disciplines. Like blind monks examining an elephant,
%everyone who approaches this novel technology does so from a different angle.
%And everyone will come to different conclusions about the nature of the beast.
비트코인은 다양한 분야로부터 영향을 받아 탄생하였다. 
이 때문에 맹인 승려가 코끼리를 조사하는 것처럼 이 기술에 접근하는 모든 사람이 다른 각도에서 바라보게 된다. 
그리고 모두가 이 짐승의 본성에 대해 각각의 다른 결론을 낸다.

%The following lessons are about some of my assumptions which Bitcoin shattered,
%and the conclusions I arrived at. Philosophical questions of immutability,
%scarcity, locality, and identity are explored in the first four lessons.  Every
%part consists of seven lessons.
아래의 교훈들은 비트코인의 여러 가정과 내가 도달한 결론에 대한 내용이다. 첫 네 개의 교훈은
불변성, 희소성, 지역성, 정체성에 대한 철학적 질문을 다룰 예정이다.

~

\begin{samepage}
	Part~\ref{ch:philosophy} -- 철학:
	
	\begin{enumerate}
		\item 불변성과 변화
		\item 진정한 희소성
		\item 복제와 지역성
		\item 정체성의 문제
		\item 무결점의 개념
		\item 언론 자유의 힘
		\item 지식의 한계
	\end{enumerate}
\end{samepage}

%Lesson \ref{les:5} explores how Bitcoin's origin story is not only fascinating but
%absolutely essential for a leaderless system. The last two lessons of this
%chapter explore the power of free speech and the limits of our individual
%knowledge, reflected by the surprising depth of the Bitcoin rabbit hole.
다섯번째 교훈은 비트코인의 기원에 대한 이야기가 매력적인지, 그리고 리더의 부재가 왜 절대적으로 필요한지
알아본다. 마지막 두 개의 교훈은 깊은 토끼굴로 들어가 언론 자유에 대해 탐구하고 개인 지식의 한계에 관해 이야기하고자 한다. 

%I hope that you will find the world of Bitcoin as educational, fascinating and
%entertaining as I did and still do. I invite you to follow the white rabbit and
%explore the depths of this rabbit hole. Now hold on to your pocket watch, pop
%down, and enjoy the fall.
당신도 비트코인의 세계가 교육적이고 매력적이라는 사실을 깨닫길 바란다. 흰토끼를 따라 토끼 굴의 깊이를 탐험해보라.
지금부터 회중시계를 들고 토끼 굴로 떨어져보자.


\chapter{불변성과 변화}
\label{les:1}

%\begin{chapquote}{Alice}
%\enquote{I wonder if I've been changed in the night. Let me think. Was I the same when I
	%got up this morning? I almost think I can remember feeling a little different.
	%But if I'm not the same, the next question is `Who in the world am I?' Ah,
	%that's the great puzzle!}
%\end{chapquote}$
\begin{chapquote}{앨리스}
	\enquote{혹시 내가 변한 건가?
		왠지 다른 느낌이 들었거든. 만약에 내가 다른 모습이라면 다음 질문은 `도대체 나는 누굴까?' 야.
		아, 그것은 너무 큰 수수께끼야!}
\end{chapquote}



%Bitcoin is inherently hard to describe. It is a \textit{new thing}, and any
%attempt to draw a comparison to previous concepts -- be it by calling
%it digital gold or the internet of money -- is bound to fall short of
%the whole. Whatever your favorite analogy might be, two aspects of
%Bitcoin are absolutely essential: decentralization and immutability.
비트코인은 본질적으로 설명하기 어렵다. 
비트코인은 새로운 개념이고, 디지털 금, 인터넷 돈 등으로 설명하기엔 뭔가 부족해 보인다. 
당신이 가장 좋아하는 비유가 무엇이든 비트코인의 두 가지 측면은 절대적으로 필요하다. 
탈중앙성과 불변성이 그것이다.

\paragraph{}
%One way to think about Bitcoin is as an automated social contract\footnote{Hasu,
	%Unpacking Bitcoin's Social Contract~\cite{social-contract}}. The software is
%just one piece of the puzzle, and hoping to change Bitcoin by changing the
%software is an exercise in futility. One would have to convince the rest of the
%network to adopt the changes, which is more a psychological effort than a
%software engineering one.
비트코인을 이해하는 한 가지 방법은 자동화된 사회 계약으로 보는 것이다\footnote{Hasu, Unpacking Bitcoin's Social Contract~\cite{social-contract}}. 
비트코인 소프트웨어는 네트워크의 일부분일 뿐이다. 
소프트웨어를 변경한다고 해서 비트코인을 변경할 수 없기 때문이다. 
변경을 위해서는 네트워크의 나머지 참여자들을 설득 해야 하는데, 
이는 소프트웨어적 노력보다 심리적 노력이 더 필요하다.


\paragraph{}
%The following might sound absurd at first, like so many other things in
%this space, but I believe that it is profoundly true nonetheless: You
%won't change Bitcoin, but Bitcoin will change you.
당신은 비트코인을 변경하지 않지만, 비트코인은 당신을 변하게 할 수 있다.
조금 터무니없이 들리겠지만, 나는 이를 사실이라 믿는다.


\begin{quotation}\begin{samepage}
		\enquote{우리가 비트코인을 변하게 하는 것 보다 비트코인이 우리를 더 변하게 할 것이다.}
		\begin{flushright} -- 마티 벤트\footnote{Tales From the Crypt~\cite{tftc21}}
\end{flushright}\end{samepage}\end{quotation}

%It took me a long time to realize the profundity of this. Since Bitcoin
%is just software and all of it is open-source, you can simply change
%things at will, right? Wrong. \textit{Very} wrong. Unsurprisingly, Bitcoin's
%creator knew this all too well.
나는 이 심오함을 알아차리는 데 꽤 오랜 시간이 걸렸다.
비트코인은 단지 소프트웨어일 뿐이고 오픈 소스이기 때문에 
마음대로 변경할 수 있다고 생각했지만 매우 잘못된 생각이었다. 
비트코인 창시자는 이를 너무 잘 알고 있었다.

\begin{quotation}\begin{samepage}
		%\enquote{The nature of Bitcoin is such that once version 0.1 was released, the core
			%design was set in stone for the rest of its lifetime.}
		\enquote{비트코인의 본질은 비트코인 0.1 버전이 출시됨과 동시에 비트코인이 사라질 때까지 
			핵심 설계가 돌처럼 고정된다는 것이다.}
		\begin{flushright} -- 사토시 나카모토\footnote{BitcoinTalk forum post: `Re:
				Transactions and Scripts\ldots'~\cite{satoshi-set-in-stone}}
\end{flushright}\end{samepage}\end{quotation}

%Many people have attempted to change Bitcoin's nature. So far all of
%them have failed. While there is an endless sea of forks and altcoins,
%the Bitcoin network still does its thing, just as it did when the first
%node went online. The altcoins won't matter in the long run. The forks
%will eventually starve to death. Bitcoin is what matters. As long as our
%fundamental understanding of mathematics and/or physics doesn't change,
%the Bitcoin honeybadger will continue to not care.
많은 사람이 비트코인의 본질을 변경하고자 시도하였다. 그러나 모든 시도는 실패하였다. 알트코인이 수많은
포크를 진행하는 동안 비트코인 네트워크는 첫 번째 노드가 구동될 때와 같이 여전히 제 역할을 하고 있다.
알트코인은 장기적인 관점을 중요하지 않게 생각한다. 포크는 결국 굶어 죽을 것이다. 수학적, 물리학적 근본적인 
이해가 변하지 않는 한 비트코인 벌꿀오소리는 변하지 않을 것이다.

\begin{quotation}\begin{samepage}
		\enquote{비트코인은 삶의 새로운 형태를 보여준다. 비트코인은 인터넷에서 살아 숨 쉰다. 비트코인은
			사람들이 생존을 위해 구매를 지속하는 한 살아있을 것이다. [\ldots] 비트코인은 변하지 않는다. 논쟁의 여지가 없다.
			조작될 수도, 손상될 수도, 멈출 수도 없다. [\ldots] 핵전쟁으로 지구의 절반이 파괴되어도 
			부패하지 않고 살아남아 있을 것이다. }
		\begin{flushright} -- 랄프 머클\footnote{DAOs, Democracy and
				Governance,~\cite{merkle-dao}}
\end{flushright}\end{samepage}\end{quotation}


%The heartbeat of the Bitcoin network will outlast all of ours.
비트코인 네트워크의 심장박동은 모든 사람보다 오래 지속될 것이다.

~

%Realizing the above changed me way more than the past blocks of the Bitcoin
%blockchain ever will. It changed my time preference, my understanding of
%economics, my political views, and so much more. Hell, it is even changing
%people's diets\footnote{Inside the World of the Bitcoin
	%Carnivores,~\cite{carnivores}}. If all of this sounds crazy to you, you're in
%good company. All of this is crazy, and yet it is happening.

위의 내용을 깨닫는 것은 나를 변화시키는 것이었다. 
이 사실은 나의 시간선호도, 경제에 대한 이해, 정치적 견해 등을 바꾸었다. 
심지어 사람들의 식단도 바꾼다\footnote{Inside the World of the Bitcoin Carnivores,~\cite{carnivores}}. 
이 말들이 이상하게 들린다면 당신은 좋은 사회에 있는 것이다. 
모두가 미쳤고 미친 짓은 계속되고 있다.
~

\paragraph{비트코인은 비트코인이 변하지 않는다는 것을 알려주었다. 단지 내가 변할 뿐이다.}

% ---
%
% #### Through the Looking-Glass
%
% - [Bitcoin's Gravity: How idea-value feedback loops are pulling people in][gravity]
% - [Lesson 18: Move slowly and don't break things][lesson18]
%
% #### Down the Rabbit Hole
%
% - [Unpacking Bitcoin's Social Contract][automated social contract]: A framework for skeptics by Hasu
% - [DAOs, Democracy and Governance][Ralph Merkle] by Ralph C. Merkle
% - [Marty's Bent][bent]: A daily newsletter highlighting signal in Bitcoin by Marty Bent
% - [Technical Discussion on Bitcoin's Transactions and Scripts][Satoshi Nakamoto] by Satoshi Nakamoto, Gavin Andresen, and others
% - [Inside the World of the Bitcoin Carnivores][carnivores]: Why a small community of Bitcoin users is eating meat exclusively by Jordan Pearson
% - [Tales From the Crypt][tftc] hosted by Marty Bent
%
% <!-- Internal -->
% [gravity]: 
% [lesson18]: {{ 'bitcoin/lessons/ch3-18-move-slowly-and-dont-break-things' | absolute_url }}
%
% <!-- Further Reading -->
% [automated social contract]: https://medium.com/@hasufly/bitcoins-social-contract-1f8b05ee24a9
% [carnivores]: https://motherboard.vice.com/en_us/article/ne74nw/inside-the-world-of-the-bitcoin-carnivores
% [tftc]: https://tftc.io/tales-from-the-crypt/
% [bent]: https://tftc.io/martys-bent/
%
% <!-- Quotes -->
% [Ralph Merkle]: http://merkle.com/papers/DAOdemocracyDraft.pdf
% [Satoshi Nakamoto]: https://bitcointalk.org/index.php?topic=195.msg1611#msg1611
%
% <!-- Twitter People -->
% [Marty Bent]: https://twitter.com/martybent
%
% <!-- Wikipedia -->
% [alice]: https://en.wikipedia.org/wiki/Alice%27s_Adventures_in_Wonderland
% [carroll]: https://en.wikipedia.org/wiki/Lewis_Carroll


\chapter{진정한 희소성}
\label{les:2}

\begin{chapquote}{앨리스}
	\enquote{충분해. 나는 더 커지고 싶지 싶지 않아.\ldots}
\end{chapquote}


\paragraph{}
%In general, the advance of technology seems to make things more abundant. More
%and more people are able to enjoy what previously have been luxurious goods.
%Soon, we will all live like kings. Most of us already do. As Peter Diamandis
%wrote in Abundance~\cite{abundance}: \enquote{Technology is a resource-liberating
	%mechanism. It can make the once scarce the now abundant.}
일반적으로 기술의 발전은 모든 것을 더 풍요롭게 만드는 것 같다. 이전에는 사치품이었던 것을 더 많은 사람들이 누리게 되었다. 
이대로라면 머지않아 우리 모두 왕처럼 살게 될 것이다. 그리고 대부분 이미 그렇게 살고 있다.
피터 디아만디스(Peter Diamandis)가 풍요(Abundance)라는 책에 쓴 것처럼 말이다.\cite{abundance}. \enquote{기술은 자원을 해방하는 메커니즘이다.
	기술은 한때 희소했던 것들을 풍요롭게 만들 수 있다.}

\paragraph{}
%Bitcoin, an advanced technology in itself, breaks this trend and creates
%a new commodity which is truly scarce. Some even argue that it is one of
%the scarcest things in the universe. The supply can't be inflated, no
%matter how much effort one chooses to expend towards creating more.
그 자체로 첨단 기술인 비트코인은 이런 추세를 깨고 진짜 희소성을 갖는 새로운 상품을 만들어 냈다. 
어떤 사람들은 비트코인이 우주에서 가장 희귀한 것 중 하나라고 주장하기도 한다.
공급을 늘리려 아무리 노력해도 공급량을 늘릴 수가 없는 것이다. 

\begin{quotation}\begin{samepage}
		\enquote{비트코인과 시간. 오직 이 두 가지만이 진정으로 희귀하다.}
		\begin{flushright} -- 사이페딘 아모스\footnote{Presentation on The Bitcoin Standard~\cite{bitcoinstandard-pres}}
\end{flushright}\end{samepage}\end{quotation}

\paragraph{}
%Paradoxically, it does so by a mechanism of copying. Transactions are
%broadcast, blocks are propagated, the distributed ledger is --- well,
%you guessed it --- distributed. All of these are just fancy words for
%copying. Heck, Bitcoin even copies itself onto as many computers as it
%can, by incentivizing individual people to run full nodes and mine new
%blocks.
역설적으로 이 희소성은 복제 메커니즘 때문에 가능하다. 
트랜잭션이 브로드캐스팅되고, 블록이 전파되며, 분산 원장이... 뭐 짐작하겠지만, 어쨌든 분산된다.
이 모든 것은 '복사'를 있어보이게 설명하는 단어일 뿐이다.
심지어 비트코인은 개인이 전체 노드를 실행하고 새로운 블록을 채굴하도록 장려함으로써 
가능한 한 많은 컴퓨터에 스스로를 복사하게 만들기도 한다. 

%All of this duplication wonderfully works together in a concerted effort
%to produce scarcity.
복제된 모든 것들이 희소성을 지키기 위해 공동의 노력으로 훌륭히 작동한다.

%\paragraph{In a time of abundance, Bitcoin taught me what real scarcity is.}
\paragraph{풍요의 시대에 비트코인은 진정한 희소성이 무엇인지 가르쳐주었다.}

% ---
%
% #### Through the Looking-Glass
%
% - [Lesson 14: Sound money][lesson14]
%
% #### Down the Rabbit Hole
%
% - [The Bitcoin Standard: The Decentralized Alternative to Central Banking][bitcoin-standard]
% - [Abundance: The Future Is Better Than You Think][Abundance] by Peter Diamandis
% - [Presentation on The Bitcoin Standard][bitcoin-standard-presentation] by Saifedean Ammous
% - [Modeling Bitcoin's Value with Scarcity][planb-scarcity] by PlanB
% - 🎧 [Misir Mahmudov on the Scarcity of Time & Bitcoin][tftc60] TFTC #60 hosted by Marty Bent
% - 🎧 [PlanB – Modelling Bitcoin's digital scarcity through stock-to-flow techniques][slp67] SLP #67 hosted by Stephan Livera
%
% <!-- Through the Looking-Glass -->
% [lesson14]: {{ 'bitcoin/lessons/ch2-14-sound-money' | absolute_url }}
%
% <!-- Down the Rabbit Hole -->
% [Abundance]: https://www.diamandis.com/abundance
% [bitcoin-standard]: http://amzn.to/2L95bJW
% [bitcoin-standard-presentation]: https://www.bayernlb.de/internet/media/de/ir/downloads_1/bayernlb_research/sonderpublikationen_1/bitcoin_munich_may_28.pdf
% [planb-scarcity]: https://medium.com/@100trillionUSD/modeling-bitcoins-value-with-scarcity-91fa0fc03e25
% [tftc60]: https://anchor.fm/tales-from-the-crypt/episodes/Tales-from-the-Crypt-60-Misir-Mahmudov-e3aibh
% [slp67]: https://stephanlivera.com/episode/67
%
% <!-- Wikipedia -->
% [alice]: https://en.wikipedia.org/wiki/Alice%27s_Adventures_in_Wonderland
% [carroll]: https://en.wikipedia.org/wiki/Lewis_Carroll

\chapter{복제와 국소성}
\label{les:3}

\begin{chapquote}{루이스 캐롤, \textit{이상한 나라의 앨리스}}
	다음으로 토끼의 화난 목소리가 들렸다. \enquote{팻! 팻! 어디야?}
\end{chapquote}

\paragraph{}
%Quantum mechanics aside, locality is a non-issue in the physical world.
%The question \textit{\enquote{Where is X?}} can be answered in a meaningful way, no
%matter if X is a person or an object. In the digital world, the question
%of \textit{where} is already a tricky one, but not impossible to answer. Where
%are your emails, really? A bad answer would be \enquote{the cloud}, which is
%just someone else's computer. Still, if you wanted to track down every
%storage device which has your emails on it you could, in theory, locate
%them.
양자역학까지 갈 것도 없이 물리 세계에서는 국소성(Locality)이 문제가 되지 않는다.
\footnote{역주: 고전역학에서는 국소성이란 공간적으로 떨어져있는 두 물체가 서로 직접적으로 영향을 줄 수 없는 성질을 말한다. 
반대로 양자역학에서는 두 물체가 떨어져 있더라도 양자 차원의 힘에 의해 상호 영향을 미치기 때문에 두 물체 간에는 비국소성을 갖는다고 말한다.}
\enquote{X는 어디에 있는가?}라는 질문에 대해 X가 사람이든 사물이든 간에 의미있는 대답을 할 수 있다. 
디지털 세계에서 '어디에 있는가'라는 질문은 까다롭긴 해도 대답하기 불가능한 질문은 아니다. 
이메일이 진짜로 어디에 있는가?라는 질문에 대해 \enquote{(내 컴퓨터가 아닌) 클라우드에 있어.}라고 성의 없이 대답할 수 있다.
하지만 이렇게 대답해도 모든 저장 장치를 추적하면 이론상 이메일이 있는 곳을 찾을 수 있다.

\paragraph{}
%With bitcoin, the question of \enquote{where} is \textit{really} tricky. Where,
%exactly, are your bitcoins?
그러나 비트코인의 경우, \enquote{어디에 있는가}라는 질문에 대해 대답하기란 \textit{정말} 까다롭다. 비트코인은 정확히 어디에 있는 걸까?

\begin{quotation}\begin{samepage}
		\enquote{나는 수술 후 눈을 뜨고 주위를 둘러보며 한탄스러울 정도로 진부하지만 피할 수 없는 그 질문을 했다.`여기가 어디지?`}
		\begin{flushright} -- 다니엘 데넷\footnote{Daniel Dennett, \textit{Where Am I?}~\cite{where-am-i}}
\end{flushright}\end{samepage}\end{quotation}

\paragraph{}
%The problem is twofold: First, the distributed ledger is distributed by
%full replication, meaning the ledger is everywhere. Second, there are no
%bitcoins. Not only physically, but \textit{technically}.
이 문제는 이중적이다. 하나는 분산 원장은 모든 기록을 복제하여 어디에나 존재한다는 점이고, 다른 하나는 비트코인은 '없다'는 점이다. 
물리적으로도 \textit{기술적으로도}.

\paragraph{}
%Bitcoin keeps track of a set of unspent transaction outputs, without
%ever having to refer to an entity which represents a bitcoin. The
%existence of a bitcoin is inferred by looking at the set of unspent
%transaction outputs and calling every entry with 100 million base
%units a bitcoin.
비트코인은 어떤 수량의 비트코인을 나타내는 특정한 것을 참조하는 것이 아니라 미사용 트랜잭션 출력(UTXO)의 집합을 추적한다.
특정 수량의 비트코인이 존재한다는 것은 UTXO를 보고 1억 기본 단위\footnote{역주: 1BTC = 1sats}로 기록되는 모든 데이터를 호출할 수 있음을 의미한다. 

\begin{quotation}\begin{samepage}
		\enquote{비트코인은 지금 이 순간 어디로 이동 중일까?[...] 일단, 비트코인은
			없다. 그냥 없다. 존재하지 않는다. 공유된 원장에 존재할 뿐이다. [...] 
			물리적으로 존재하지 않는다. 원장이 물리적 위치에 존재하는 것이다.
			여기서 지리적 의미는 담지 말자. 이것을 당신 상식 선에서 이해하려고 해도 별 도움이 되지 않을 것이다.}
		\begin{flushright} -- 피터 반 발켄버그\footnote{Peter Van Valkenburgh on the What Bitcoin Did podcast, episode 49 \cite{wbd049}}
\end{flushright}\end{samepage}\end{quotation}

\paragraph{}
%So, what do you actually own when you say \textit{\enquote{I have a bitcoin}} if
%there are no bitcoins? Well, remember all these strange words which you were
%forced to write down by the wallet you used? Turns out these magic words are
%what you own: a magic spell\footnote{The Magic Dust of Cryptography: How digital
	%information is changing our society \cite{gigi:magic-spell}} which can be used
%to add some entries to the public ledger --- the keys to \enquote{move} some bitcoins.
%This is why, for all intents and purposes, your private keys \textit{are} your
%bitcoins. If you think I'm making all of this up feel free to send me your
%private keys.
비트코인이 어디에도 존재하지 않는다면, \textit{\enquote{비트코인을 갖고 있다.}}라고 할 때 당신이 실제로 소유하고 있는 것은 무엇일까? 
혹시 사용하던 지갑 때문에 억지로 옮겨 적어야만 했던 이상한 단어들을 기억하는가?
바로 이 단어들이 비트코인 공개 원장에 장부를 추가할 수 있는 마법의 주문\footnote{The Magic Dust of Cryptography: How digital
	information is changing our society \cite{gigi:magic-spell}}, 즉 비트코인을 전송할 때 필요한 열쇠인 것이다. 
그렇기 때문에 이 모든 의도와 목적에 따라 개인키가 곧 비트코인인 것이다. 
내가 이 모든 걸 지어냈다고 생각한다면, 언제든 나에게 당신의 개인키를 보내주길 바란다.

\paragraph{비트코인은 나에게 국소성이 얼마나 까다로운 것인지 알려주었다.}

% ---
%
% #### Through the Looking-Glass
%
% - [The Magic Dust of Cryptography: How digital information is changing our society][a magic spell]
%
% #### Down the Rabbit Hole
%
% - [Where Am I?][Daniel Dennett] by Daniel Dennett
% - 🎧 [Peter Van Valkenburg on Preserving the Freedom to Innovate with Public Blockchains][wbd049] WBD #49 hosted by Peter McCormack
%
% <!-- Through the Looking-Glass -->
% [a magic spell]: 
%
% <!-- Down the Rabbit Hole -->
% [Daniel Dennett]: https://www.lehigh.edu/~mhb0/Dennett-WhereAmI.pdf
% [1st Amendment]: https://en.wikipedia.org/wiki/First_Amendment_to_the_United_States_Constitution
% [wbd049]: https://www.whatbitcoindid.com/podcast/coin-centers-peter-van-valkenburg-on-preserving-the-freedom-to-innovate-with-public-blockchains
%
% <!-- Wikipedia -->
% [alice]: https://en.wikipedia.org/wiki/Alice%27s_Adventures_in_Wonderland
% [carroll]: https://en.wikipedia.org/wiki/Lewis_Carroll

\chapter{정체성의 문제}
\label{les:4}

\begin{chapquote}{루이스 캐롤, \textit{이상한 나라의 앨리스}}
	\enquote{넌 누구니?} 애벌레가 물었다.
\end{chapquote}

\paragraph{}
%Nic Carter, in an homage to Thomas Nagel's treatment of the same
%question in regards to a bat, wrote an excellent piece which discusses
%the following question: What is it like to be a bitcoin? He
%brilliantly shows that open, public blockchains in general, and Bitcoin
%in particular, suffer from the same conundrum as the ship of Theseus\footnote{In
	%the metaphysics of identity, the ship of Theseus is a thought experiment that
	%raises the question of whether an object that has had all of its components
	%replaced remains fundamentally the same object.~\cite{wiki:theseus}}: which
%Bitcoin is the real Bitcoin?
닉 카터는 토마스 네이글의 \enquote{박쥐가 된다는 것은 무엇인가?(What is it like to be bat?)}를 오마주하여 
\enquote{비트코인이 된다는 것은 무엇인가?(What is it like to be a bitcoin?)}라는 질문에 대한 훌륭한 글을 남겼다. 
그는 일반적으로 개방형 블록체인, 특히 비트코인은, 
테세우스의 배\footnote{정체성의 형이상학에서, 테세우스의 배의 낡은 판자를 계속 교체하다 보면 어느 시점에는 원래 배 조각이 하나도 남지 않을 것인데 이것을 테세우스의 배라고 할 수 있을까?에 대한
의문을 제기하는 사고 실험이다.~\cite{wiki:theseus}}와 같은 난제에 해당함을 탁월하게 서술했다.
과연 어떤 비트코인이 진짜 비트코인일까?

\begin{quotation}\begin{samepage}
		\enquote{비트코인 컴포넌트가 얼마나 변했는지 생각해보라. 
			비트코인의 전체 코드베이스는 재작업, 변경, 확장되어
			원래의 버전과 거의 유사성이 없을 정도로 수정되었다.[\ldots]누가 무엇을 소유하는지에 대한
			기록, 즉 원장 자체만이 사실상 이 네트워크에서 유일하게 유지되고 있다.[\ldots]
			진정한 의미에서 리더가 없는 것으로 간주되려면 
			특정 체인을 적법한 체인으로 지정할 수 있는 주체가 있다는 쉬운 해결책을 포기해야만 한다.}
		\begin{flushright} -- 닉 카터\footnote{Nic Carter, \textit{What is it like to be a bitcoin?} \cite{bitcoin-identity}}
\end{flushright}\end{samepage}\end{quotation}

%Consider just how little persistence Bitcoin's components have. The
%entire codebase has been reworked, altered, and expanded such that it
%barely resembles its original version. [...] The registry of who
%owns what, the ledger itself, is virtually the only persistent trait
%of the network [...]
%To be considered truly leaderless, you must surrender the easy
%solution of having an entity that can designate one chain as the
%legitimate one.}
%\begin{flushright} -- Nic Carter\footnote{Nic Carter, \textit{What is it like to be a bitcoin?} \cite{bitcoin-identity}}
%\end{flushright}\end{samepage}\end{quotation}


\paragraph{}
%It seems like the advancement of technology keeps forcing us to take
%these philosophical questions seriously. Sooner or later, self-driving
%cars will be faced with real-world versions of the trolley problem,
%forcing them to make ethical decisions about whose lives do matter and
%whose do not.
이러한 철학적 질문은 기술의 발전으로 인해 더 진지하게 받아들여지게 된 것 같다.
조만간 자율주행차는 트롤리 문제에 직면하게 될 것이며, 누구의 생명이 더 중요하고 덜 중요한지 윤리적 결정을 내려야 할 것이다.
\footnote{역주: 트롤리 문제 혹은 트롤리 딜레마는 우리에게 '갈림길을 향해 달리는 제동장치가 고장난 수레 실험'으로 유명하다. 
이 실험은 다수를 위해 소수를 희생하는 것이(혹은 그 반대가) 과연 윤리적으로 올바른 선택인가 질문한다.}

\paragraph{}
%Cryptocurrencies, especially since the first contentious hard-fork,
%force us to think about and agree upon the metaphysics of identity.
%Interestingly, the two biggest examples we have so far have lead to two
%different answers. On August 1, 2017, Bitcoin split into two camps. The
%market decided that the unaltered chain is the original Bitcoin. One
%year earlier, on October 25, 2016, Ethereum split into two camps. The
%market decided that the \textit{altered} chain is the original Ethereum.
특히나 논쟁의 여지가 많은 첫 번째 하드포크 이후, 
암호화폐는 우리에게 정체성의 형이상학에 대해 생각해보고 동의하도록 강요하고 있다. 
흥미롭게도 비트코인과 이더리움은 각각 다른 대답을 내놨다. 
2017년 8월 1일, 비트코인은 두 진영으로 나뉘었다. 
시장은 하드포크되지 않은 체인을 원래의 비트코인으로 받아들였다. 
그보다 1년 정도 전인 2016년 10월 25일, 이더리움도 두 진영으로 나뉘었다.
시장은 하드포크된 체인을 원래의 이더리움이라고 결정했다.

\paragraph{}
%If properly decentralized, the questions posed by the \textit{Ship of Theseus}
%will have to be answered in perpetuity for as long as these networks of
%value-transfer exist.
제대로 탈중앙화되어 있다면, 네트워크의 가치가 지속되는 한 \textit{테세우스의 배}에 관한 질문의 답변은 한결같아야 한다.

\paragraph{비트코인은 탈중앙화가 정체성과 모순된다는 것을 가르쳐주었다.}

% ---
%
% #### Down the Rabbit Hole
%
% - [What Is It Like to be a Bat?][in regards to a bat] by Thomas Nagel
% - [What is it like to be a bitcoin?] by Nic Carter
% - [Ship of Theseus], [trolley problem] on Wikipedia
%
% [in regards to a bat]: https://en.wikipedia.org/wiki/What_Is_it_Like_to_Be_a_Bat%3F
% [What is it like to be a bitcoin?]: https://medium.com/s/story/what-is-it-like-to-be-a-bitcoin-56109f3e6753
% [Ship of Theseus]: https://en.wikipedia.org/wiki/Ship_of_Theseus
% [trolley problem]: https://en.wikipedia.org/wiki/Trolley_problem
%
% <!-- Wikipedia -->
% [alice]: https://en.wikipedia.org/wiki/Alice%27s_Adventures_in_Wonderland
% [carroll]: https://en.wikipedia.org/wiki/Lewis_Carroll

\chapter{무결점의 개념}
\label{les:5}

\begin{chapquote}{루이스 캐롤, \textit{이상한 나라의 앨리스}}
	\enquote{그들의 머리가 사라졌습니다!} 병사들이 외쳤다\ldots
\end{chapquote}

\paragraph{}
%Everyone loves a good origin story. The origin story of Bitcoin is a
%fascinating one, and the details of it are more important than one might
%think at first. Who is Satoshi Nakamoto? Was he one person or a group of
%people? Was he a she? Time-traveling alien, or advanced AI? Outlandish
%theories aside, we will probably never know. And this is important.
누구나 기원이 훌륭한 서사를 좋아한다. 
비트코인의 서사는 흥미진진하며 그 세세한 내용들은 생각보다 매우 중요하다. 
사토시 나카모토는 누구일까? 한 사람일까, 아니면 여러 사람일까? 남자일까 여자일까? 시간 여행을 하는 외계인일까, 그것도 아니라면 고도의 인공지능일까? 
엉뚱한 상상력을 동원하더라도 우리는 아마 그의 정체를 절대 알 수 없을 것이다.
바로 이 사실이 매우 중요하다.

\paragraph{}
%Satoshi chose to be anonymous. He planted the seed of Bitcoin. He stuck
%around for long enough to make sure the network won't die in its
%infancy. And then he vanished.
사토시는 익명을 선택했다. 그는 비트코인의 씨앗을 심었다. 
그는 비트코인 초창기에 네트워크가 안정화될 때까지 충분히 오랜 기간 머물러있었다. 
그리고는 사라졌다.

\paragraph{}
%What might look like a weird anonymity stunt is actually crucial for a
%truly decentralized system. No centralized control. No centralized
%authority. No inventor. No-one to prosecute, torture, blackmail, or
%extort. An immaculate conception of technology.
사실 진정한 탈중앙화 시스템을 위해서는 익명성이라는 특이한 기능이 매우 중요하다.
중앙화된 통제권이 없다. 중앙화된 권한도 없다. 발명가도 없다.
기소하거나 고문하거나 협박하거나 강탈할 대상이 아무도 없다. 
이는 기술적으로 흠결없이 완벽한 개념이다.

\begin{quotation}\begin{samepage}
		\enquote{가장 멋진 점은 사토시가 사라졌다는 것입니다.}
		\begin{flushright} -- 지미 송\footnote{Jimmy Song, \textit{Why Bitcoin is Different} \cite{bitcoin-different}}
\end{flushright}\end{samepage}\end{quotation}

\newpage

\paragraph{}
%Since the birth of Bitcoin, thousands of other cryptocurrencies were
%created. None of these clones share its origin story. If you want to
%supersede Bitcoin, you will have to transcend its origin story. In a war
%of ideas, narratives dictate survival.
비트코인 이후 수천 개의 암호화폐가 만들어졌다. 
이러한 복제품 중 어느 것도 비트코인과 같은 서사를 가진 것은 없다.
비트코인을 대체하려면 비트코인의 탄생 스토리를 초월해야 한다. 
아이디어 전쟁에서는 내러티브가 생존을 좌우한다.

\begin{quotation}\begin{samepage}
		\enquote{금은 7,000년 전 처음으로 보석으로 만들어졌고 물물교환에 사용되었다. 매혹적인 금의 광채때문에 신이 내린 선물로 여겨졌다.}
		\begin{flushright} 오스트리안 민트\footnote{The Austrian Mint, \textit{Gold: The Extraordinary Metal} \cite{gold-gift-gods}}
\end{flushright}\end{samepage}\end{quotation}

\paragraph{}
%Like gold in ancient times, Bitcoin might be considered a gift from the
%gods. Unlike gold, Bitcoins origins are all too human. And this time, we
%know who the gods of development and maintenance are: people all over
%the world, anonymous or not.
고대의 금처럼 비트코인을 신이 내린 선물로 여길 수 있다. 
하지만 금과 달리 비트코인의 기원은 너무나도 인간적이다. 
그리고 비트코인을 누가 개발하고 유지하는지 알 수 있다.
익명이든 아니든, 바로 이 세상 모든 사람들이다.

\paragraph{비트코인은 나에게 서사가 중요하다는 것을 가르쳐주었다.}

% ---
%
% #### Down the Rabbit Hole
%
% - [Why Bitcoin is different][Jimmy Song] by Jimmy Song
% - [Gold: The Extraordinary Metal] by the Austrian Mint
%
% <!-- Down the Rabbit Hole -->
% [Jimmy Song]: https://medium.com/@jimmysong/why-bitcoin-is-different-e17b813fd947
% [Gold: The Extraordinary Metal]: https://www.muenzeoesterreich.at/eng/discover/for-investors/gold-the-extraordinary-metal
%
% <!-- Wikipedia -->
% [alice]: https://en.wikipedia.org/wiki/Alice%27s_Adventures_in_Wonderland
% [carroll]: https://en.wikipedia.org/wiki/Lewis_Carroll

\chapter{언론 자유의 힘}
\label{les:6}

\begin{chapquote}{루이스 캐롤, \textit{이상한 나라의 앨리스}}
	\enquote{다시 한번 말씀해 주시겠어요?} 생쥐가 얼굴을 찡그리며, 하지만 아주 공손하게 물었다. \enquote{뭐라고 하셨죠?}
\end{chapquote}

\paragraph{}
%Bitcoin is an idea. An idea which, in its current form, is the
%manifestation of a machinery purely powered by text. Every aspect of
%Bitcoin is text: The whitepaper is text. The software which is run by
%its nodes is text. The ledger is text. Transactions are text. Public and
%private keys are text. Every aspect of Bitcoin is text, and thus
%equivalent to speech.
비트코인은 아이디어다. 현재 형태로는 순수하게 문자로만 구동되는 기계식 표현이다.
비트코인의 모든 것이 문자로 이루어져 있다. 백서가 문자이다. 노드가 실행하는 소프트웨어도 문자이다. 원장도 문자, 트랜잭션도 문자이다. 
공개키와 개인키도 물론 문자이다. 비트코인의 모든 면이 문자로 이루어져있어 언어와 동일하다.

\begin{quotation}\begin{samepage}
		\enquote{
			의회는 종교의 설립을 존중하거나 자유로운 종교 행사를 금지하거나 
			언론의 자유, 출판의 자유, 또는 평화롭게 집회할 수 있는 국민의 권리를 저해하거나,  
			고충의 구제를 위해 정부에 청원할 권리를 제한하는 법률을 제정할 수 없다.}
		\begin{flushright} -- 미국 수정헌법 제1조
\end{flushright}\end{samepage}\end{quotation}

\paragraph{}
%Although the final battle of the Crypto Wars\footnote{The \textit{Crypto Wars}
	%is an unofficial name for the U.S. and allied governments' attempts to undermine
	%encryption.~\cite{eff-cryptowars}~\cite{wiki:cryptowars}} 약ment tries to
%outlaw text or speech, we slip down a path of absurdity which inevitably leads
%to abominations like illegal numbers\footnote{An illegal number is a number that
	%represents information which is illegal to possess, utter, propagate, or
	%otherwise transmit in some legal jurisdiction.\cite{wiki:illegal-number}} and
%illegal primes\footnote{An illegal prime is a prime number that represents
	%information whose possession or distribution is forbidden in some legal
	%jurisdictions. One of the first illegal primes was found in 2001. When
	%interpreted in a particular way, it describes a computer program that bypasses
	%the digital rights management scheme used on DVDs. Distribution of such a
	%program in the United States is illegal under the Digital Millennium Copyright
	%Act. An illegal prime is a kind of illegal number.\cite{wiki:illegal-prime}}.
아직 크립토 전쟁(the Crypto Wars)\footnote{\textit{크립토 전쟁}은 미국과 연합 정부의 암호화를 약화하려는 시도에 대한 비공식적인 이름이다.~\cite{eff-cryptowars}~\cite{wiki:cryptowars}} 
은 끝나지 않았지만, 문자 메시지의 교환을 기반으로 한 시도와 아이디어를 범죄로 규정하는 것은 매우 어려울 것이다.
정부가 문자나 말을 불법화하려고 할 때마다 우리는 부조리한 길로 빠지게 되고, 필연적으로 
불법 숫자\footnote{불법 숫자는 일부 법적 관할에서 소유, 발언, 전파 또는 전송하는 것이 금지된 숫자이다. 모든 디지털 정보는 숫자이다.	결과적으로 특정 정보 집합을 전송하는 것은 불법일 수 있다.\cite{wiki:illegal-number}}
나 불법 소수\footnote{불법 소수(prime number)는 불법 숫자의 범주에 해당된다. 최초의 불법 소수는 2001년 필 카모디가 DVD의 복제 방지를 우회하는 컴퓨터 프로그램이다. 이러한 프로그램을 미국에서 배포하는 것은 불법이다.\cite{wiki:illegal-prime}}
 같은 혐오스러운 결론이 발생하게 된다.

\paragraph{}
%As long as there is a part of the world where speech is free as in
%\textit{freedom}, Bitcoin is unstoppable.
언론이 자유로운 세상이 있는 한, 비트코인을 멈출 수 없다.

\begin{quotation}\begin{samepage}
		\enquote{비트코인 거래에서 비트코인이 문자가 아닌 경우는 없다. 항상 문자이다. [...]
			비트코인은 텍스트이다. 고로 비트코인은 언어이다.
			미국처럼 양도할 수 없는 권리가 보장되고 수정헌법 1조에 따라 
			출판 행위를 정부 감독에서 명시적으로 제외하는 자유 국가에서는 이를 규제할 수 없다.}
		\begin{flushright} -- 뷰티온\footnote{Beautyon, \textit{미국이 비트코인을 규제할 수 없는 이유(Why America can't regulate Bitcoin)} \cite{america-regulate-bitcoin}}
\end{flushright}\end{samepage}\end{quotation}

\paragraph{비트코인은 자유 사회에서 언론의 자유와 자유 소프트웨어를 막을 수 없다는 것을 가르쳐주었다.}

% ---
%
% #### Through the Looking-Glass
%
% - [The Magic Dust of Cryptography: How digital information is changing our society][a magic spell]
%
% #### Down the Rabbit Hole
%
% - [Why America can't regulate Bitcoin][Beautyon] by Beautyon
% - [First Amendment to the United States Constitution][1st Amendment], [Crypto Wars], [illegal numbers], [illegal primes] on Wikipedia
%
% <!-- Through the Looking-Glass -->
% [a magic spell]: 
%
% <!-- Down the Rabbit Hole -->
% [1st Amendment]: https://en.wikipedia.org/wiki/First_Amendment_to_the_United_States_Constitution
% [Crypto Wars]: https://en.wikipedia.org/wiki/Crypto_Wars
% [illegal numbers]: https://en.wikipedia.org/wiki/Illegal_number
% [illegal primes]: https://en.wikipedia.org/wiki/Illegal_prime
% [Beautyon]: https://hackernoon.com/why-america-cant-regulate-bitcoin-8c77cee8d794
%
% <!-- Wikipedia -->
% [alice]: https://en.wikipedia.org/wiki/Alice%27s_Adventures_in_Wonderland
% [carroll]: https://en.wikipedia.org/wiki/Lewis_Carroll

\chapter{지식의 한계}
\label{les:7}

\begin{chapquote}{루이스 캐롤, \textit{이상한 나라의 앨리스}}
	\enquote{아래로, 아래로, 아래로. 이 추락은 끝나지 않는 걸까요?}
\end{chapquote}

\paragraph{}
%Getting into Bitcoin is a humbling experience. I thought that I knew
%things. I thought that I was educated. I thought that I knew my computer
%science, at the very least. I studied it for years, so I have to know
%everything about digital signatures, hashes, encryption, operational
%security, and networks, right?
비트코인에 빠지는 것은 사람을 겸손하게 만드는 경험이다. 
나는 내가 많은 것을 안다고 생각했다. 
나는 교육받았고, 최소한 컴퓨터 과학은 알고 있다고 생각했다. 
수년간 공부했으니 디지털 서명, 해시, 암호학, 운영 보안, 네트워크에 대한 모든 것을 완벽히 알아야만 한다. 그렇지 않은가?

\paragraph{}
%Wrong.
틀렸다.

\paragraph{}
%Learning all the fundamentals which make Bitcoin work is hard.
%Understanding all of them deeply is borderline impossible.
비트코인을 동작하게 하는 모든 기초 지식을 배우기는 어렵다. 
이 모든 것을 깊이 이해하는 것은 거의 불가능에 가깝다.

\begin{quotation}\begin{samepage}
		\enquote{비트코인 토끼굴의 바닥을 본 사람은 아무도 없다.}
		\begin{flushright} -- 제임슨 롭\footnote{Jameson Lopp, tweet from Nov 11, 2018 \cite{lopp-tweet}}
\end{flushright}\end{samepage}\end{quotation}

\begin{figure}
	\centering
	\includegraphics[width=7cm]{assets/images/rabbit-hole-bottomless.png}
	\caption{비트코인 토끼굴은 무한하다.}
	\label{fig:rabbit-hole-bottomless}
\end{figure}

\paragraph{}
%My list of books to read keeps expanding way quicker than I could
%possibly read them. The list of papers and articles to read is virtually
%endless. There are more podcasts on all of these topics than I could
%ever listen to. It truly is humbling. Further, Bitcoin is evolving and
%it's almost impossible to stay up-to-date with the accelerating rate of
%innovation. The dust of the first layer hasn't even settled yet, and
%people have already built the second layer and are working on the third.
내가 읽어야할 책 목록은 내가 읽을 수 있는 것보다 훨씬 빠르게 늘어나고 있다. 
읽어야 할 논문과 기사는 사실상 끝이 없다.
이 모든 주제를 다루는 팟캐스트는 내가 들을 수 있는 것보다 더 많다. 
참으로 겸손해진다.
게다가 비트코인은 계속 진화하고 있으며, 가속화되는 혁신의 속도를 따라가는 것은 거의 불가능에 가깝다.
첫 번째 레이어의 먼지가 아직 가라앉지도 않았는데 사람들은 이미 두 번째 레이어를 만들었고, 이제는 세 번째 레이어를 위해 노력하고 있다.

\paragraph{비트코인은 내가 거의 아는게 없다는 걸 가르쳐주었다. 비트코인 토끼굴의 바닥을 알 수 없다.}
% ---
%
% #### Down the Rabbit Hole
%
% - [Bitcoin Literature] by the Satoshi Nakamoto Institute
% - [Bitcoin Information & Resources][lopp-resources] by Jameson Lopp
% - [Educational Resources][bitcoin-only] by Bitcoin Only
%
% <!-- Twitter -->
% [Jameson Lopp]: https://twitter.com/lopp/status/1061415918616698881
%
% <!-- Down the Rabbit Hole -->
% [lopp-resources]: https://www.lopp.net/bitcoin-information.html
% [bitcoin-only]: https://bitcoin-only.com/#learning
% [Bitcoin Literature]: https://nakamotoinstitute.org/literature/
%
% <!-- Wikipedia -->
% [alice]: https://en.wikipedia.org/wiki/Alice%27s_Adventures_in_Wonderland
% [carroll]: https://en.wikipedia.org/wiki/Lewis_Carroll

\part{경제학}
\label{ch:economics}
\chapter*{경제학}

\begin{chapquote}{루이스 캐롤, \textit{이상한 나라의 앨리스}}
	\enquote{정원 입구 근처에 커다란 장미나무가 서 있었는데, 나무에 달린 흰색 장미를 정원사 세 명이 바쁘게 빨간색으로 칠하고 있었다.
		앨리스는 참 신기한 일이라고 생각했다\ldots}
\end{chapquote}

%Money doesn’t grow on trees. To believe that it does is foolish, and our
%parents make sure that we know about that by repeating this saying like a
%mantra. We are encouraged to use money wisely, to not spend it frivolously,
%and to save it in good times to help us through the bad. Money, after all,
%does not grow on trees.
땅을 판다고 돈이 생기지 않는다. 그렇다고 믿는 것은 어리석은 일이며, 부모님은 이 진리를 되풀이하여 가르친다.
우리는 돈을 현명하게 사용하고, 경솔하게 써버려선 안되며, 좋은 시절에 저축하여 위기를 극복하라고 배운다. 
돈은 결코 땅을 판다고 저절로 생기지 않는다.

%Bitcoin taught me more about money than I ever thought I would need to know.
%Through it, I was forced to explore the history of money, banking, various
%schools of economic thought, and many other things. The quest to understand
%Bitcoin lead me down a plethora of paths, some of which I try to explore in
%this chapter.
비트코인은 내가 돈에 대해 알아야 한다고 생각했던 것보다 더 많은 것을 가르쳐주었다. 
나는 비트코인을 통해 돈의 역사, 은행, 다양한 경제학파 등 많은 것을 탐구할 수 밖에 없었다.
비트코인을 이해하기 위한 공부는 나를 수많은 길로 이끌었고, 그 중 일부를 이 장에서 살펴보고자 한다. 

%In the first seven lessons some of the philosophical questions Bitcoin touches
%on were discussed. The next seven lessons will take a closer look at money and
%economics.
앞서 살펴본 첫 일곱개의 교훈에서는 비트코인이 다루는 몇 가지 철학적 질문에 대해 논하였다.
이제 다음 일곱개의 교훈에서는 돈과 경제에 대해 자세히 살펴볼 것이다. 

~

\begin{samepage}
	Part~\ref{ch:economics} -- 경제학
	
	\begin{enumerate}
		\setcounter{enumi}{7}
		\item 금융적 무지
		\item 인플레이션
		\item 가치
		\item 돈
		\item 역사와 돈의 몰락
		\item 부분 지급준비금의 광기
		\item 건전 화폐
\end{enumerate}
\end{samepage}

%Again, I will only be able to scratch the surface. Bitcoin is not only
%ambitious, but also broad and deep in scope, making it impossible to cover all
%relevant topics in a single lesson, essay, article, or book. I doubt if it is
%even possible at all.
다시 강조하지만 나는 표면적인 내용만 다룰 수 있을 것이다.  
비트코인은 범위가 넓고 깊어서 하나의 강의, 에세이, 기사 또는 책에서 모든 주제를 다루는 것이 불가능하다.
과연 그렇게 하는게 가능할지 의문이다.  


%Bitcoin is a new form of money, which makes learning about
%economics paramount to understanding it. Dealing with the nature of human action
%and the interactions of economic agents, economics is probably one of the
%largest and fuzziest pieces of the Bitcoin puzzle.
비트코인은 새로운 형태의 돈으로 비트코인을 이해하기 위해서는 무엇보다 경제에 대해 배우는 것이 중요하다.
인간 행동의 본질과 경제 주체들의 상호 작용을 다루는 경제학은 아마도 비트코인 퍼즐에서 가장 크고 모호한 조각 중 하나일 것이다.

%Again, these lessons are an exploration of the various things I have learned
%from Bitcoin. They are a personal reflection of my journey down the rabbit hole.
%Having no background in economics, I am definitely out of my comfort zone and
%especially aware that any understanding I might have is incomplete. I will do my
%best to outline what I have learned, even at the risk of making a fool out of
%myself. After all, I am still trying to answer the question:
다시 한번 말하지만, 이 글은 내가 비트코인에서 배운 다양한 것들을 탐구한 것이다. 
비트코인 토끼굴로 내려가는 나의 여정을 개인적으로 회고한 것이기도 하다.  
경제학이 나의 전문 영역이 아니고, 나에게는 배경 지식도 없기 때문에 나의 이해가 불완전하다는 것을 알고 있다. 
하지만, 바보로 보일 부담을 감수하고서라도 내가 배운 것을 설명하기 위해 최선을 다할 것이다.
여전히 나는 같은 질문에 답하기 위해 노력 중이다.\enquote{비트코인으로부터 무엇을 배웠는가?}

%After seven lessons examined through the lens of philosophy, let’s use the lens
%of economics to look at seven more. Economy class is all I can offer this time.
%Final destination: \textit{sound money}.
철학적 관점에서 일곱 가지 교훈을 살펴보았으니, 이번에는 경제학적 관점에서 일곱 가지 교훈을 더 살펴보자.
이번에는 경제 수업이고 종착지는 건전화폐(sound money)이다.

% [the question]: https://twitter.com/arjunblj/status/1050073234719293440

\chapter{금융적 무지}
\label{les:8}

\begin{chapquote}{루이스 캐롤, \textit{이상한 나라의 앨리스}}
	\enquote{그렇게 물어보면 나를 얼마나 무지한 아이로 보겠어. 안돼, 그런 건 질문하지 말아야지. 어딘가 쓰여있는 걸 봐야겠어.}
\end{chapquote}

\paragraph{}
%One of the most surprising things, to me, was the amount of finance,
%economics, and psychology required to get a grasp of what at first
%glance seems to be a purely \textit{technical} system --- a computer network.
%To paraphrase a little guy with hairy feet: \enquote{It's a dangerous business,
	%Frodo, stepping into Bitcoin. You read the whitepaper, and if you don't
	%keep your feet, there's no knowing where you might be swept off to.}
가장 놀라웠던 점 중 하나는 언뜻 보기엔 순수하게 \textit{기술적} 시스템인 컴퓨터 네트워크를 이해하기 위해, 
방대한 금융, 경제학, 심리학을 이해해야만 한다는 것을 알게 된 것이다.
한 호빗의 말을 빌리자면 이렇다. 
\enquote{비트코인에 발을 들이는 것은 위험한 일이야, 프로도. 백서를 읽고 발을 떼지 않으면 어디로 휩쓸릴지 몰라.}

\paragraph{}
%To understand a new monetary system, you have to get acquainted with the
%old one. I began to realize very soon that the amount of financial
%education I enjoyed in the educational system was essentially \textit{zero}.
새로운 화폐 시스템을 이해하려면 기존 시스템을 알아야 한다.  
나는 이내 교육 시스템에서 내가 누리던 금융 교육이 본질적으로 0이라는 것을 깨닫기 시작했다. 


%Like a five-year-old, I began to ask myself a lot of questions: How does the
%banking system work? How does the stock market work? What is fiat money? What is
%\textit{regular} money? Why is there so much
%debt?\footnote{\url{https://www.usdebtclock.org/}} How much money is actually
%printed, and who decides that?
그리고는 다섯 살짜리 아이처럼 스스로에게 질문을 하기 시작했다. 
은행 시스템은 어떻게 작동하나? 주식 시장은 어떻게 작동하나? 
법정화폐란 무엇인가? 일반적인 화폐는 무엇인가? 빚이 왜 이렇게 많은가?\footnote{\url{https://www.usdebtclock.org/}} 
실제로 인쇄되는 돈의 양은 얼마나되며, 이를 결정하는 사람은 누구인가?

%\newpage 

%After a mild panic about the sheer scope of my ignorance, I found
%reassurance in realizing that I was in good company.
\paragraph{}
내 무지의 범위에 잠깐 당황했지만, 동료들이 있다는 사실에 안도했다.

\begin{quotation}\begin{samepage}
		\enquote{내가 금융기관에서 일한 지난 몇 년보다 비트코인이 더 많은 것을 가르쳐주었다는 사실이 
			아이러니하지 않습니까? \ldots 중앙은행에서 커리어를 시작한 걸 포함해서}
		\begin{flushright} -- 애런\footnote{Aaron (\texttt{@aarontaycc}, \texttt{@fiatminimalist}), tweet from Dec.
				12, 2018~\cite{aarontaycc-tweet}}
\end{flushright}\end{samepage}\end{quotation}

\begin{quotation}\begin{samepage}
		\enquote{나는 지난 3년 반의 대학 생활 동안 보다 암호화폐 분야에서의 최근 3개월 동안 
		금융, 경제, 기술, 암호학, 인간 심리학, 정치, 게임 이론, 입법 그리고 나 자신에 대해서 더 많이 배웠습니다.}
		\begin{flushright} -- 더니\footnote{Dunny (\texttt{@BitcoinDunny}), tweet from Nov. 28,
				2017~\cite{bitcoindunny-tweet}}
\end{flushright}\end{samepage}\end{quotation}

%These are just two of the many confessions all over twitter.\footnote{See
	%\url{http://bit.ly/btc-learned} for more confessions on twitter.} Bitcoin, as
%was explored in Lesson \ref{les:1}, is a living thing. Mises argued that
%economics also is a living thing. And as we all know from personal experience,
%living things are inherently difficult to understand.
\paragraph{}
이것은 트위터 전체에 퍼져있는 수 많은 고백 중 단 두개에 불과하다. \footnote{\url{http://bit.ly/btc-learned}} 
지난 교훈에서 살펴본 것처럼 비트코인은 살아있다.\ref{les:1}
미제스는 경제학도 살아있는 생물이라 주장했다. 
그리고 우리 모두 개인적 경험을 통해 알고 있듯이 살아있는 생물을 이해하기란 어렵다.

\begin{quotation}\begin{samepage}
		\enquote{과학 시스템은 끝없이 진보하는 지식 탐색의 한 지점에 불과하다. 
			그것은 필연적으로 모든 인간 노력에 내재된 부족함으로 인해 영향을 받는다.
			그러나 이러한 사실을 인정한다고 해서 오늘날의 경제학이 후진적이라는 의미는 아니다.
			이는 단지 경제학이 살아있는 것이라는 의미일 뿐이다. 
			그리고 산다는 것은 불완전하다는 것과 변화한다는 것을 동시에 의미한다.}
		\begin{flushright} -- 루드비히 폰 미제스\footnote{Ludwig von Mises, \textit{Human Action}
				\cite{human-action}}
\end{flushright}\end{samepage}\end{quotation}

%\newpage

%We all read about various financial crises in the news, wonder about how
%these big bailouts work and are puzzled over the fact that no one ever
%seems to be held accountable for damages which are in the trillions. I
%am still puzzled, but at least I am starting to get a glimpse of what is
%going on in the world of finance.
\paragraph{}
우리 모두 뉴스에서 다양한 금융 위기 소식을 접하고, 대규모 구제 금융이 어떻게 작동하는지 궁금해하며, 
수조 달러에 달하는 손해를 아무도 책임지지 않는다는 사실에 당황한다.
여전히 의아하지만 적어도 나는 금융의 세계에서 무슨 일이 일어나고 있는지 엿볼 수 있게 되었다.

%Some people even go as far as to attribute the general ignorance on
%these topics to systemic, willful ignorance. While history, physics,
%biology, math, and languages are all part of our education, the world of
%money and finance surprisingly is only explored superficially, if at
%all. I wonder if people would still be willing to accrue as much debt as
%they currently do if everyone would be educated in personal finance and
%the workings of money and debt. Then I wonder how many layers of
%aluminum make an effective tinfoil hat. Probably three.
\paragraph{}
혹자는 심지어 이러한 경제에 대한 무지가 체계적이고 고의적이라 말한다.
역사, 물리학, 생물학, 수학, 언어가 우리 교육의 일부인 반면, 
놀랍게도 돈과 금융의 세계는 피상적으로만 다루어진다. 
모든 사람이 개인 금융과 돈과 부채의 작동 원리에 대해 교육을 받고도 지금처럼 많은 빚을 지게 될지 궁금하다.
그렇다면 알루미늄을 몇 겹이나 겹쳐야 효과적인 은박 모자\footnote{역자: 은박지 모자를 쓰면 정부의 감시나 외계인의 정신 통제를 피할 수 있다는 믿음이 있다.}
를 만들 수 있을까? 아마 세 겹일 것이다.

\begin{quotation}\begin{samepage}
		\enquote{이러한 붕괴와 구제 금융은 우연이 아니다. 그리고 학교에서 금융 교육을 하지 않는 것도 우연이 아니다. [...] 계획된 것이다.
			남북전쟁 이전에 노예를 교육하는 것이 불법이었던 것처럼, 학교에서 돈에 대해 배우는 것은 허용되지 않는다.}
		\begin{flushright} -- 로버트 기요사키\footnote{Robert Kiyosaki, \textit{Why the Rich
					are Getting Richer}\cite{robert-kiyosaki}}
\end{flushright}\end{samepage}\end{quotation}

%Like in The Wizard of Oz, we are told to pay no attention to the man behind the
%curtain. Unlike in The Wizard of Oz, we now have real
%wizardry\footnote{\url{http://bit.ly/btc-wizardry}}: a censorship-resistant,
%open, borderless network of value-transfer. There is no curtain, and the magic
%is visible to anyone.\footnote{\url{https://github.com/bitcoin/bitcoin}}
\paragraph{}
오즈의 마법사에서처럼 세상은 우리에게 장막 뒤에 있는 사람에게 관심을 두지 말라고 한다.
하지만 오즈의 마법사와는 달리, 이제 우리는 검열에 저항하는 개방적이며 국경 없는 가치 전송 네트워크인 진짜 마법사\footnote{\url{http://bit.ly/btc-wizardry}}를 만나게 되었다.
커튼은 없고, 누구나 마법을 볼 수 있다.\footnote{\url{https://github.com/bitcoin/bitcoin}}

\paragraph{비트코인은 장막 뒤에서 나의 금융적 무지와 직면하게 해주었다.}

% ---
%
% #### Down the Rabbit Hole
%
% - [Human Action][Ludwig von Mises] by Ludwig von Mises
% - [Why the Rich are Getting Richer][Robert Kiyosaki] by Robert Kiyosaki
%
% [real wizardry]: https://external-preview.redd.it/8d03MWWOf2HIyKrT8ThBGO4WFv-u25JaYqhbEO9b1Sk.jpg?width=683&auto=webp&s=dc5922d84717c6a94527bafc0189fd4ca02a24bb
% [visible to anyone]: https://github.com/bitcoin/bitcoin
%
% <!-- Wikipedia -->
% [alice]: https://en.wikipedia.org/wiki/Alice%27s_Adventures_in_Wonderland
% [carroll]: https://en.wikipedia.org/wiki/Lewis_Carroll

\chapter{인플레이션}
\label{les:9}

\begin{chapquote}{하트의 여왕\footnote{역자: 이상한 나라 앨리스의 등장인물}} 
	\enquote{내 사랑, 이 자리에 서 있기 위해서는 최대한 빨리 달려야 해. 그리고 만약 어디론가 가고 싶다면
		두 배로 빨리 달려야 해.}
\end{chapquote}

%Trying to understand monetary inflation, and how a non-inflationary
%system like Bitcoin might change how we do things, was the starting
%point of my venture into economics. I knew that inflation was the rate
%at which new money was created, but I didn't know too much beyond that.
인플레이션 현상을 이해하고 
비트코인처럼 인플레이션이 없는 시스템을 위해서
어떤 변화가 필요한지를 알아가는 것은 경제학 탐구의 출발점이었다. 
나는 인플레이션이 새로운 돈이 만들어지는 비율 정도라는 것은 알고 있었지만,
그 이면에는 무엇이 있는지는 알지 못했다.

%While some economists argue that inflation is a good thing, others argue
%that \enquote{hard} money which can't be inflated easily --- as we had in the
%days of the gold standard --- is essential for a healthy economy.
%Bitcoin, having a fixed supply of 21 million, agrees with the latter
%camp.
어떤 경제학자들은 인플레이션은 좋은 것이라 주장하는 반면, 
어떤 경제학자들은 금본위제 시대의 금처럼 
수량을 늘리기 어려운 경화(hard currency)가 건전한 경제에 필수적이라 주장한다. 
2,100만 개로 공급량이 고정된 비트코인은 후자의 의견을 반영한 것이다.

%Usually, the effects of inflation are not immediately obvious. Depending
%on the inflation rate (as well as other factors) the time between cause
%and effect can be several years. Not only that, but inflation affects
%different groups of people more than others. As Henry Hazlitt points out
%in \textit{Economics in One Lesson}: \enquote{The art of economics consists in looking
	%not merely at the immediate but at the longer effects of any act or
	%policy; it consists in tracing the consequences of that policy not
	%merely for one group but for all groups.}
일반적으로 인플레이션의 영향은 즉각적으로 나타나지 않는다. 
인플레이션율에 따라 이 시간은 몇 년이 걸릴 수도 있다. 
그뿐만 아니라 인플레이션은 무엇보다 광범위한 사람들에게 영향을 끼친다. 
헨리 해즐릿은 경제학의 교훈(Economics in One Lesson)에서 다음과 같이 말한다. 
\enquote{경제학에서 어떤 행동이나 정책은 즉각적인 효과보다는 긴 효과를 보고자 하는 데 있다.
	그것은 한 그룹이 아니라 모든 그룹에 대한 정책의 결과를 추적하는 것으로 구성된다.}


%One of my personal lightbulb moments was the realization that issuing
%new currency --- printing more money --- is a \textit{completely} different
%economic activity than all the other economic activities. While real
%goods and real services produce real value for real people, printing
%money effectively does the opposite: it takes away value from everyone
%who holds the currency which is being inflated.
개인적으로 가장 충격적인 사실은 
새로운 화폐를 발행하는 것 즉 돈을 인쇄하는 것은 일반적인 경제활동과는 완전히 다른 양상을 보인다는 것이다. 
실제 상품과 서비스를 생산하는 것은 사람들에게 가치를 제공하는 반면, 
돈을 인쇄하는 것은 인플레이션을 통해 사람들에게 가치를 빼앗아 간다.

\begin{quotation}\begin{samepage}
		%Mere inflation --- that is, the mere issuance of more money, with the
		%consequence of higher wages and prices --- may look like the creation
		%of more demand. But in terms of the actual production and exchange of
		%real things it is not.
		\enquote{인플레이션, 즉 더 많은 돈을 발행하는 것은 더 높은 임금과 물가로 인해 더 많은 수요를 창출하는 것처럼 보인다.
			하지만, 실제 생산과 거래가 생기는 것은 아니다.}
		\begin{flushright} -- 헨리 해즐릿\footnote{Henry Hazlitt, \textit{Economics in One Lesson} \cite{hazlitt}}
\end{flushright}\end{samepage}\end{quotation}

%The destructive force of inflation becomes obvious as soon as a little inflation
%turns into \textit{a lot}. If money hyperinflates things get ugly real
%quick.\footnote{\url{https://en.wikipedia.org/wiki/Hyperinflation}
	%\cite{wiki:hyperinflation}} As the inflating currency falls apart, it will fail
%to store value over time and people will rush to get their hands on any goods
%which might do.
인플레이션의 파괴력은 인플레이션의 규모가 커지는 순간 명백하게 나타난다. 
초인플레이션으로 화폐가 과도하게 팽창하면 상황은 빠르게 나빠진다.\footnote{\url{https://en.wikipedia.org/wiki/Hyperinflation}
	\cite{wiki:hyperinflation}} 
팽창하는 통화가 무너지면 사람들은 낮아지는 통화가치에 대응하기 위해 상품 구매를 서두른다.

\paragraph{}
%Another consequence of hyperinflation is that all the money which people
%have saved over the course of their life will effectively vanish. The
%paper money in your wallet will still be there, of course. But it will
%be exactly that: worthless paper.
초인플레이션의 또 다른 부작용은 사람들이 평생 저축한 돈이 사실상 사라지게 되는 것이다. 
지갑에는 돈이 그대로 있지만, 그 돈은 쓸모없는 종이와 같다.

\begin{figure}
	\includegraphics{assets/images/children-playing-with-money.png}
	\caption{바이마르 공화국의 초인플레이션 (1921-1923)}
	\label{fig:children-playing-with-money}
\end{figure}

\paragraph{}
%Money declines in value with so-called \enquote{mild} inflation as well. It
%just happens slowly enough that most people don't notice the diminishing
%of their purchasing power. And once the printing presses are running,
%currency can be easily inflated, and what used to be mild inflation
%might turn into a strong cup of inflation by the push of a button. As
%Friedrich Hayek pointed out in one of his essays, mild inflation usually
%leads to outright inflation.
인플레이션이 경미하다 하여도 돈의 가치는 결국 하락한다. 
대부분 사람이 알아차리지 못할 정도로 천천히 구매력이 감소한다. 
일단 인쇄기가 가동되면 통화는 쉽게 부풀려지고 경미한 인플레이션은 언제든 강력한 인플레이션으로 변질될 수 있다. 
프리드리히 하이에크가 그의 에세이에서 지적했듯이 
가벼운 인플레이션이 노골적인 인플레이션으로 변하는 일은 매우 흔하게 발생한다.

\begin{quotation}\begin{samepage}
		\enquote{`가벼운' 지속적인 인플레이션은 도움이 안 된다. 그것은 결국 노골적인 인플레이션으로 변질한다.}
		\begin{flushright} -- 프리드리히 하이에크\footnote{Friedrich Hayek, \textit{1980s
					Unemployment and the Unions} \cite{hayek-inflation}}
\end{flushright}\end{samepage}\end{quotation}

%Inflation is particularly devious since it favors those who are closer
%to the printing presses. It takes time for the newly created money to
%circulate and prices to adjust, so if you are able to get your hands on
%more money before everyone else's devaluates you are ahead of the
%inflationary curve. This is also why inflation can be seen as a hidden
%tax because in the end governments profit from it while everyone else
%ends up paying the price.
인플레이션은 인쇄기에 더 가까운 사람에게 유리하기 때문에 더 악랄하다. 
새로 만들어진 돈이 순환하고 가격이 조정되는 데에 시간이 걸리므로 
돈의 가치가 하락하기 전에 더 많은 돈을 손에 넣을 수 있다면
인플레이션이 나타나기 전에 돈을 쓸 수 있다. 
이러한 점으로 인해 인플레이션을 숨겨진 세금으로 간주하기도 한다.
왜냐하면 결국 인플레이션으로 정부는 이익을 얻지만, 다른 모든 사람은 그 대가를 지불하기 때문이다.

\begin{quotation}\begin{samepage}
		\enquote{광범위한 관점에서 우리의 역사는 정부의 이익을 위해 조작된 인플레이션의 역사이다. }
		\begin{flushright} -- 프리드리히 하이에크\footnote{Friedrich Hayek, \textit{Good Money} \cite{hayek-good-money}}
\end{flushright}\end{samepage}\end{quotation}

%So far, all government-controlled currencies have eventually been
%replaced or have collapsed completely. No matter how small the rate of
%inflation, \enquote{steady} growth is just another way of saying exponential
%growth. In nature as in economics, all systems which grow exponentially
%will eventually have to level off or suffer from catastrophic collapse.
역사적으로 모든 정부 통화는 결국 교체되거나 완전히 붕괴되었다. 
인플레이션 비율이 아무리 낮더라도 결국 기하급수적 팽창으로 귀결된다. 
자연에서와 마찬가지로 경제에서도 기하급수적으로 성장하는 
모든 시스템은 결국 평준화되거나 파국적인 붕괴를 피할 수 없다.

\paragraph{}
%\enquote{It can't happen in my country,} is what you're probably thinking. You don't
%think that if you are from Venezuela, which is currently suffering from
%hyperinflation. With an inflation rate of over 1 million percent, money is
%basically worthless. \cite{wiki:venezuela}
\enquote{우리나라는 괜찮아.}라고 생각할 수도 있다.
하지만 당신이 초인플레이션으로 고통받고 있는 베네수엘라 출신이라면 그렇게 생각하지 않을 것이다. 
인플레이션율이 100만 퍼센트가 넘는 상황에서 돈은 가치가 없다.\cite{wiki:venezuela}

\paragraph{}
%It might not happen in the next couple of years, or to the particular currency
%used in your country. But a glance at the list of historical
%currencies\footnote{See \textit{List of historical currencies} on Wikipedia.
	%\cite{wiki:historical-currencies}} shows that it will inevitably happen over a
%long enough period of time. I remember and used plenty of those listed: the
%Austrian schilling, the German mark, the Italian lira, the French franc, the
%Irish pound, the Croatian dinar, etc. My grandma even used the Austro-Hungarian
%Krone. As time moves on, the currencies currently in use\footnote{See
	%\textit{List of currencies} on Wikipedia \cite{wiki:list-of-currencies}} will
%slowly but surely move to their respective graveyards. They will hyperinflate or
%be replaced. They will soon be historical currencies. We will make them
%obsolete.
향후 몇 년 동안 당신의 국가에서 사용되는 특정 통화에서는 인플레이션이 발생하지 않을 수 있다. 
그러나 통화의 역사\footnote{See \textit{List of historical currencies} on Wikipedia.
	\cite{wiki:historical-currencies}}를
보면 충분히 오랜 기간에 걸쳐 불가피하게 인플레이션이 발생한다는 것을 알 수 있다.
나는 오스트리아 실링, 독일 마르크, 이탈리아 리라, 프랑스 프랑, 아일랜드 파운드, 크로아티아 디나르 등 여러
통화들을 기억하고 있거나 사용하고 있다. 
나의 할머니는 오스트리아-헝가리 크로네도 사용하셨다. 
시간이 지나면 현재 사용 중인 통화들도\footnote{See
	\textit{List of currencies} on Wikipedia \cite{wiki:list-of-currencies}}
느리지만 확실하게 각자의 무덤으로 이동하게 될 것이다.
이 통화들은 가치가 급락하거나 교체되어 역사의 뒤안길로 사라질 것이다.
우리가 그 화폐들을 폐물로 만들 것이다.

\begin{quotation}\begin{samepage}
		\enquote{정부가 화폐 공급을 부풀리려는 유혹에 굴복할 수밖에 없다는 사실은 역사가 잘 말해준다.}
		\begin{flushright} -- 사이페딘 아모스\footnote{Saifedean Ammous, \textit{The Bitcoin
					Standard} \cite{bitcoin-standard}}
\end{flushright}\end{samepage}\end{quotation}

\begin{comment}
	Why is Bitcoin different? In contrast to currencies mandated by the government,
	monetary goods which are not regulated by governments, but by the laws of
	physics\footnote{Gigi, \textit{Bitcoin's Energy Consumption - A shift in
			perspective} \cite{gigi:energy}}, tend to survive and even hold their respective
	value over time. The best example of this so far is gold, which, as the
	aptly-named \textit{Gold-to-Decent-Suit Ratio}\footnote{History shows that the
		price of an ounce of gold equals the price of a decent men's suit, according to Sionna
		investment managers \cite{web:gold-to-decent-suite-ratio}} shows, is holding its
	value over hundreds and even thousands of years. It might not be perfectly
	\enquote{stable} --- a questionable concept in the first place --- but the value it
	holds will at least be in the same order of magnitude.
\end{comment}
비트코인은 왜 다른가? 정부가 강제하는 통화와 달리 물리적 법칙에 의해 규제되는 화폐 상품\footnote{Gigi, \textit{Bitcoin's Energy Consumption - A shift in perspective}\cite{gigi:energy}}
은 시간이 지나도 가치를 유지하는 경향이 있다. 가장 좋은 예는 금이다. 
금 대비 적합한 양복 값의 비율(Gold-to-Decent-Suit Ratio)\footnote{시오나 투자 매니저에 따르면 역사적으로 금 1온스의 가격이 괜찮은 남성 정장 가격과 같다고 한다. \cite{web:gold-to-decent-suite-ratio}}
이라는 적절한 비유가 보여주듯이 금은 수백 년, 수천 년 동안 그 가치를 유지하고 있다. 
단기적으로는 금의 가치가 불안정해 보일 수도 있다.
하지만 분명한 것은 금의 희소성에 의한 가치는 유지될 것이라는 사실이다.

\begin{comment}
	If a monetary good or currency holds its value well over time and space,
	it is considered to be \textit{hard}. If it can't hold its value, because it
	easily deteriorates or inflates, it is considered a \textit{soft} currency. The
	concept of hardness is essential to understand Bitcoin and is worthy of
	a more thorough examination. We will return to it in the last economic
	lesson: sound money.
\end{comment}
금전적 상품이나 화폐가 시간과 공간을 초월하여 가치를 유지하기는 어렵다. 
쉽게 약화하거나 팽창하여 가치를 유지할 수 없는 경우 연화(soft currency)로 간주한다. 
돈의 강건성(hardness) 개념은 비트코인을 이해하는 데 필수적이며 더 철저히 검토되어야 한다.
우리는 마지막 교훈에서 화폐의 건전성(sound money)에 대해서 다시 다루게 될 것이다.

\paragraph{}
\begin{comment}
	As more and more countries suffer from
	hyperinflation more and more people will have to face the reality
	of hard and soft money. If we are lucky, maybe even some central bankers will be
	forced to re-evaluate their monetary policies. Whatever might happen, the
	insights I have gained thanks to Bitcoin will probably be invaluable, no matter
	the outcome.
\end{comment}
여러 국가가 초인플레이션에 시달릴수록 더 많은 사람이 연화와 경화의 차이를 알게 될 것이다. 
운이 좋다면 일부 중앙은행가들도 자국의 통화정책을 재검토하게 될지도 모른다. 
어떤 일이 일어나든, 어떤 결과가 나타나든
비트코인 덕분에 얻은 나의 통찰력은 매우 귀중하다.

\paragraph{비트코인은 인플레이션의 숨겨진 세금과 초인플레이션의 재앙을 가르쳐주었다.}

% ---
%
% #### Down the Rabbit Hole
%
% - [Economics in One Lesson][Henry Hazlitt] by Henry Hazlitt
% - [1980's Unemployment and the Unions][unions] by Friedrich Hayek
% - [Good Money, Part II][good-money]: Volume Six of the Collected Works of F.A. Hayek
% - [The Bitcoin Standard] by Saifedean Ammous
% - [Hyperinflation][hyperinflates], [economic crisis in Venezuela][wiki-venezuela], [list of historical currencies], [list of currencies][currently in use] on Wikipedia
%
% [unions]: https://books.google.com/books/about/1980s_unemployment_and_the_unions.html?id=xM9CAQAAIAAJ
% [good-money]: https://books.google.com/books?id=l_A1vVIaYBYC
%
% [Henry Hazlitt]: https://mises.org/library/economics-one-lesson
% [hyperinflates]: https://en.wikipedia.org/wiki/Hyperinflation
% [inflation cannot help]: https://books.google.com/books?id=zZu3AAAAIAAJ&dq=%22only+while+it+accelerates%22&focus=searchwithinvolume&q=%22steady+inflation+cannot+help%22
% [history of inflation]: https://books.google.com/books?id=l_A1vVIaYBYC&pg=PA142&dq=%22history+is+largely+a+history+of+inflation%22&hl=en&sa=X&ved=0ahUKEwi90NDLrdnfAhUprVkKHUx1CmIQ6AEIKjAA#v=onepage&q=%22history%20is%20largely%20a%20history%20of%20inflation%22&f=false
% [wiki-venezuela]: https://en.wikipedia.org/wiki/Crisis_in_Venezuela#Economic_crisis
% [by the laws of physics]: https://link.medium.com/9fzq2L0J3S
% [\textit{Gold-to-Decent-Suit Ratio}]: https://www.businesswire.com/news/home/20110819005774/en/History-Shows-Price-Ounce-Gold-Equals-Price
% [The Bitcoin Standard]: https://thesaifhouse.wordpress.com/book/
%
% <!-- Wikipedia -->
% [alice]: https://en.wikipedia.org/wiki/Alice%27s_Adventures_in_Wonderland
% [carroll]: https://en.wikipedia.org/wiki/Lewis_Carroll

\chapter{가치}
\label{les:10}

\begin{chapquote}{루이스 캐롤, \textit{이상한 나라의 앨리스}}
	\enquote{하얀 토끼가 다시 천천히 돌아오면서 무언가를 잃어버린 것처럼 걱정스럽게 주위를 둘러보았다\ldots}
\end{chapquote}

% \begin{comment}
% 	Value is somewhat paradoxical, and there are multiple theories\footnote{See
% 		\textit{Theory of value (economics)} on Wikipedia \cite{wiki:theory-of-value}}
% 	which try to explain why we value certain things over other things. People have
% 	been aware of this paradox for thousands of years. As Plato wrote in his
% 	dialogue with Euthydemus, we value some things because they are rare, and not
% 	merely based on their necessity for our survival.
% \end{comment}
\paragraph{}
가치는 다소 역설적인 개념이다. 
어떤 무언가가 다른 무언가에 비해 가치 있다고 판단하는 이유를 설명하는 여러 이론이 있다.
\footnote{\textit{가치 이론 (Theory of value (economics))} 위키피디아 \cite{wiki:theory-of-value}}
사람들은 수천 년 간 이 역설을 알고 있었다. 
플라톤이 에우티데모스와의 대화에서 말했듯이, 우리가 어떤 것을 소중히 여기는 이유는 단순히 생존을 위한 필요성 때문이 아니라 희소성 때문이기도 하다.

\begin{quotation}\begin{samepage}
		\enquote{그런데 두 분이 제정신이라면, 제자들에게도 똑같은 충고를 하실 겁니다.
			당신과 자기 자신들 말고는 어느 사람과도 절대 대화를 나누지 말라고 말이죠.
			에우티데모스! 핀다로스의 말처럼 귀한 것은 값지고 물은 아무리 훌륭해도 가장 싼 것이기 때문입니다.}
		\begin{flushright} -- 플라톤\footnote{Plato, \textit{Euthydemus} \cite{euthydemus}}
\end{flushright}\end{samepage}\end{quotation}

% \begin{comment}
% 	This paradox of value\footnote{See \textit{Paradox of value} on Wikipedia
% 		\cite{wiki:paradox-of-value}} shows something interesting about us humans: we
% 	seem to value things on a subjective\footnote{See \textit{Subjective theory of
% 			value} on Wikipedia \cite{wiki:subjective-theory-of-value}} basis, but do so
% 	with certain non-arbitrary criteria. Something might be \textit{precious} to us
% 	for a variety of reasons, but things we value do share certain characteristics.
% 	If we can copy something very easily, or if it is naturally abundant, we do not
% 	value it.
% \end{comment}

\paragraph{}
이 가치의 역설\footnote{\textit{가치의 역설(Paradox of value)} 위키피디아 \cite{wiki:paradox-of-value}}은 
인간의 매우 흥미로운 일면을 보여준다. 
우리는 주관적인 기준\footnote{\textit{주관적 가치 이론(Subjective theory of value)} 위키피디아 \cite{wiki:subjective-theory-of-value}}
을 가지고 가치를 측정하는 것 같지만, 사실은 임의의 기준이 아닌 특정한 기준에 따라 측정한다. 
어떠한 것이 여러 가지 이유로 나에게 가치가 있을 수 있지만, 우리가 가치 있다고 판단하는 것들은 대개 공통적 특성을 갖는다.
무언가를 아주 쉽게 모방할 수 있거나 자연에서 쉽게 얻을 수 있다면, 우리는 그것을 가치 있다고 말하지 않는다. 


% \begin{comment}
% 	It seems that we value something because it is scarce (gold, diamonds,
% 	time), difficult or labor-intensive to produce, can't be replaced (an
% 	old photograph of a loved one), is useful in a way in which it enables
% 	us to do things which we otherwise couldn't, or a combination of those,
% 	such as great works of art.
% 
\paragraph{}
우리가 무언가를 가치있게 여기는 이유는 
희소하고(금, 다이아몬드, 시간),
생산하기 어렵거나 노동집약적이며, 대체할 수 없고(사랑하는 사람의 오래된 사진),
우리가 할 수 없던 일을 가능하게 해줄 정도로 유용하기 때문이거나, 
또는 이러한 이유들이 결합되어 있기 때문인 것 같다.


% \begin{comment}
% 	Bitcoin is all of the above: it is extremely rare (21 million),
% 	increasingly hard to produce (reward halvening), can't be replaced (a
% 	lost private key is lost forever), and enables us to do some quite
% 	useful things. It is arguably the best tool for value transfer across
% 	borders, virtually resistant to censorship and confiscation in the
% 	process, plus, it is a self-sovereign store of value, allowing
% 	individuals to store their wealth independent of banks and governments,
% 	just to name two.
% \end{comment}
\paragraph{}
비트코인은 앞서 언급한 모든 것을 갖추고 있다. 극도로 희귀하고(2,100만 개),
생산하기 점점 어려워지고 있고(반감기), 대체할 수 없으며(개인키 분실 시 영원히 손실),
매우 유용한 작업을 수행할 수 있게 해준다. 비트코인은 틀림없이 국경을 넘어 가치를 이전하는 최고의 도구이며 
그 과정에서 발생할 수 있는 검열과 몰수에 저항한다. 게다가 비트코인은 자기주권적 가치 저장소로서
개인이 은행과 정부로부터 독립적으로 온전히 자신의 부를 저장할 수 있게 해준다.

\paragraph{비트코인은 가치는 주관적이지만, 임의적이지 않다는 점을 가르쳐주었다.}

% ---
%
% #### Down the Rabbit Hole
%
% - [Euthydemus] by Plato
% - [Theory of Value][multiple theories], [Paradox of Value][paradox of value], [Subjective Theory of Value][subjective] on Wikipedia
%
% [Euthydemus]: http://www.perseus.tufts.edu/hopper/text?doc=Perseus:text:1999.01.0178:text=Euthyd.
% [Plato]: http://www.perseus.tufts.edu/hopper/text?doc=plat.+euthyd.+304b
%
% <!-- Wikipedia -->
% [multiple theories]: https://en.wikipedia.org/wiki/Theory_of_value_%28economics%29
% [paradox of value]: https://en.wikipedia.org/wiki/Paradox_of_value
% [subjective]: https://en.wikipedia.org/wiki/Subjective_theory_of_value
% [alice]: https://en.wikipedia.org/wiki/Alice%27s_Adventures_in_Wonderland
% [carroll]: https://en.wikipedia.org/wiki/Lewis_Carroll

\chapter{돈}
\label{les:11}

\begin{chapquote}{윌리엄 신부}
	\enquote{나는 젊었을 때부터, \ldots \\
		한 통에 5실링 하는 이 연고를 발라서, \\
		팔다리가 아주 유연해. \\
		너도 나한테 한두개 사서 발라보지 않을래?
	}\footnote{역자: 원작에는 1실링이라고 나온다.}
\end{chapquote}


% \begin{comment}
% 	What is money? We use it every day, yet this question is surprisingly
% 	difficult to answer. We are dependent on it in ways big and small, and
% 	if we have too little of it our lives become very difficult. Yet, we
% 	seldom think about the thing which supposedly makes the world go round.
% 	Bitcoin forced me to answer this question over and over again: What the
% 	hell is money?
% \end{comment}

\paragraph{}
돈이란 무엇인가? 
우리는 매일 돈을 사용하며 살아가지만 의외로 이 질문에 대답하기 어렵다. 
저마다 크고 작게 돈에 의존하고 있지만 돈이 너무 없으면 살기 힘들다.
그러나 우리는 돈이 세상을 어떻게 돌아가게 하는지 그 원동력에 대해선 거의 생각하지 않는다.
비트코인은 나에게 '도대체 돈이란 무엇인가?'라는 질문에 대한 대답을 강요했다.


% \begin{comment}
% 	In our \enquote{modern} world, most people will probably think of pieces of
% 	paper when they talk about money, even though most of our money is just
% 	a number in a bank account. We are already using zeros and ones as our
% 	money, so how is Bitcoin different? Bitcoin is different because at its
% 	core it is a very different \textit{type} of money than the money we currently
% 	use. To understand this, we will have to take a closer look at what
% 	money is, how it came to be, and why gold and silver was used for most
% 	of commercial history.
% \end{comment}

\paragraph{}
현대 사회를 사는 대부분의 사람들은 돈이라고 하면 종이 조각을 떠올리지만, 사실 대부분의 돈은 은행 계좌에 있는 숫자에 불과하다.
그렇다면 우리가 이미 0과 1을 돈으로 사용하고 있다는 것인데 비트코인은 어떻게 다른걸까? 
비트코인은 우리가 현재 사용하는 돈과 본질적으로 매우 다른 유형의 돈이다. 
이를 이해하려면 돈이 무엇인지, 화폐가 어떻게 생겨났는지, 
금과 은은 역사적으로 왜 가장 빈번하게 사용되었는지를 자세히 살펴보아야 한다.

% \paragraph{}
% \begin{comment}
% 	Seashells, gold, silver, paper, bitcoin. In the end, \textbf{money is whatever
% 		people use as money}, no matter its shape and form, or lack thereof.
% \end{comment}

\paragraph{}
조개껍데기, 금, 은, 종이, 비트코인. 결국 \textbf{사람들이 사용한다면 무엇이든 돈이 될 수 있다.}
돈의 모양, 형태, 또는 그것의 부족함과는 관계가 없다.


% \begin{comment}
% 	Money, as an invention, is ingenious. 
% 	A world without money is insanely
% 	complicated: How many fish will buy me new shoes? How many cows will buy
% 	me a house? What if I don't need anything right now but I need to get
% 	rid of my soon-to-be rotten apples? You don't need a lot of imagination
% 	to realize that a barter economy is maddeningly inefficient.
% \end{comment}
\paragraph{}
돈은 기발한 발명품이다.
돈이 없어지면 세상은 엄청나게 복잡해진다.
새 신발을 사려면 얼마나 많은 물고기가 필요할까? 
집을 사는데 몇 마리의 소가 필요할까? 
아무것도 살 필요가 없는 상황에서 곧 썩어버릴 사과를 없애야 한다면? 
물물교환 경제가 엄청나게 비효율임을 깨닫는 것은 그리 어렵지 않다.


% \begin{comment}
% 	The great thing about money is that it can be exchanged for \textit{anything
% 		else} --- that's quite the invention! As Nick
% 	Szabo\footnote{\url{http://unenumerated.blogspot.com/}} brilliantly summarizes
% 	in \textit{Shelling Out: The Origins of Money} \cite{shelling-out}, we humans
% 	have used all kinds of things as money: beads made of rare materials like ivory,
% 	shells, or special bones, various kinds of jewelry, and later on rare metals
% 	like silver and gold.
% \end{comment}

\begin{comment}
	돈의 가장 큰 장점은 무엇과도 교환할 수 있다는 것이다. 정말 대단한 발명품이다. 
	닉 재보\footnote{\url{http://unenumerated.blogspot.com/}}는 
	셸링 아웃: 화폐의 기원(Shelling out\footnote{역자: Shell out은 '지불하다.'라는 뜻으로 쓰이는데, 조개(shell)를 지불하는 데서 유래되었다고 한다.}: The Origins of Money)\cite{shelling-out}에서 
	이에 대해 훌륭하게 요약했다.
	\enquote{우리 인간은 상아, 조개, 특수 뼈와 같은 희귀한 재료로 만든 구슬, 다양한 종류의 장신구, 
		나중에는 은과 금과 같은 희귀 금속까지 모든 종류의 것을 돈으로 사용했다.}
\end{comment}


\begin{quotation}\begin{samepage}
		\enquote{이런 의미에서, 비트코인은 귀금속과 비슷하다. 
		공급량을 변경하여 가치를 유지하는 대신, 공급량은 미리 결정되어있고 그 가치가 변동된다.}
		\begin{flushright} -- 사토시 나카모토\footnote{Satoshi Nakamoto, in a reply to Sepp
				Hasslberger \cite{satoshi-precious-metal}}
\end{flushright}\end{samepage}\end{quotation}

% \begin{comment}
% 	Being the lazy creatures we are, we don't think too much about things
% 	which just work. Money, for most of us, works just fine. Like with our
% 	cars or our computers, most of us are only forced to think about the
% 	inner workings of these things if they break down. People who saw their
% 	life-savings vanish because of hyperinflation know the value of hard
% 	money, just like people who saw their friends and family vanish because
% 	of the atrocities of Nazi Germany or Soviet Russia know the value of
% 	privacy.
% \end{comment}

\begin{comment}
	인간은 게으른 동물이기 때문에 당장 잘 작동하는 것에 대해서는 별로 걱정하지 않는다. 
	우리 대부분에게 돈은 잘 작동한다.
	대부분의 사람들은 자동차나 컴퓨터처럼 고장이 난 경우에만 내부 작동 원리를 생각하게 된다.
	초인플레이션으로 인해 일생을 바쳐 모은 저축이 사라지는 것을 본 사람들은 경화(hard money)의 가치를 
	잘 알고 있다. 마치 나치 독일이나 소련 러시아의 잔혹 행위로 인해 친구와 가족의 죽음을 경험한 사람들이 프라이버시의 가치를 아는 것과 같다.
\end{comment}

% \begin{comment}
% 	The thing about money is that it is all-encompassing. Money is half of
% 	every transaction, which imbues the ones who are in charge with creating
% 	money with enormous power.
% \end{comment}
\begin{comment}
	문제는 돈이 우리의 모든 생활에 영향을 끼친다는 것이다.
	돈은 모든 거래의 절반을 차지하고 있고, 
	돈을 만드는 일을 담당하는 사람에게는 엄청난 힘이 있다.
\end{comment}

\begin{quotation}\begin{samepage}
		\enquote{돈이 모든 상거래의 절반을 차지하고 있고
			모든 문명의 흥망성쇠가 말 그대로 돈의 질에 따라 결정된다는 점을 감안할 때,
			우리도 알지 못하는 사이에 일어나고 있는 이 엄청난 힘에 대해 이야기하고 있는 것이다.
			그 힘이 지속되는 한 실제처럼 보이는 환상을 만들어낸다.
			이것이 연방준비제도가 가진 권력의 핵심이다.}
		\begin{flushright} -- 론 폴\footnote{Ron Paul, \textit{End the Fed} \cite{end-the-fed}}
\end{flushright}\end{samepage}\end{quotation}

%Bitcoin peacefully removes this power, since it does away with money
%creation and it does so without the use of force.
비트코인은 발권으로부터 나오는 힘을 평화적으로 무력화한다. 


% \begin{comment}
% 	Money went through multiple iterations. Most iterations were good. They
% 	improved our money in one way or another. Very recently, however, the
% 	inner workings of our money got corrupted. Today, almost all of our
% 	money is simply created \textit{out of thin air} by the powers that be. To
% 	understand how this came to be I had to learn about the history and
% 	subsequent downfall of money.
% \end{comment}

\begin{comment}
	돈은 여러 차례 반복을 거치고 있다. 
	대부분의 반복은 괜찮았다. 어떤 식으로든 개선되었다.
	그러나 최근에 화폐의 내부 작동이 손상되었다.
	오늘날 거의 모든 화폐는 권력에 의해 무에서 유를 창조하고 있다. 
	어떻게 이런 일이 벌어졌는지 이해하기 위해 나는 돈의 역사와 몰락에 대해 배워야 했다.
\end{comment}

% \begin{comment}
% 	If it will take a series of catastrophes or simply a monumental
% 	educational effort to correct this corruption remains to be seen. I pray
% 	to the gods of sound money that it will be the latter.
% \end{comment}
\begin{comment}
	이 부패를 바로잡기 위해서 일련의 재앙이 불어닥쳐야할지 아니면 단순히 훌륭한 교육이 필요할지는 아직 미지수이다. 
	하지만 건전 화폐의 신에게 후자이기를 기도해본다.
\end{comment}


\paragraph{비트코인은 나에게 돈이 무엇인지 알려주었다.}

% ---
%
% #### Down the Rabbit Hole
%
% - [End the Fed][Ron Paul] by Ron Paul
% - [Money, blockchains, and social scalability][social-scalability] by Nick Szabo
%
% [social-scalability]: https://unenumerated.blogspot.co.at/2017/02/money-blockchains-and-social-scalability.html
%

\chapter{화폐 몰락의 역사}
\label{les:12}

\begin{chapquote}{루이스 캐롤, \textit{이상한 나라의 앨리스}}
	\enquote{불 속에 뛰어들면 화상을 입는다, 칼로 손가락을 깊게 자르면 대개 피가 난다 등
		친구들이 알려준 간단한 규칙도 잊지 않았고, '독'이라고 적힌 병을 마시면
		조만간 죽을게 확실하단 사실도 잊지 않았다고 하네요.}
\end{chapquote}

\begin{comment}
	Many people think that money is backed by gold, which is locked away in
	big vaults, protected by thick
	walls. This ceased to be true many decades ago. I am not sure what I
	thought, since I was in much deeper trouble, having virtually no
	understanding of gold, paper money, or why it would need to be backed by
	something in the first place.
\end{comment}

\paragraph{}
많은 사람들은 돈이 두꺼운 벽으로 감싸진 큰 금고에 보관된 금으로 뒷받침된다고 생각한다. 
이것은 수십 년 전부터 더 이상 사실이 아니다.
사실상 나는 금이나 지폐, 혹은 애초에 왜 돈이 무언가에 의해 뒷받침되어야 하는지를 거의 이해하지 못한 채 더 깊은 혼돈에 빠진 이후로 무슨 생각을 했는지 조차 모르겠다.

\begin{comment}
	One part of learning about Bitcoin is learning about fiat money: what it
	means, how it came to be, and why it might not be the best idea we ever
	had. So, what exactly is fiat money? And how did we end up using it?
\end{comment}

\paragraph{}
비트코인 교훈의 일부는 명목화폐에 대해 배우는 것이다. 
명목화폐가 무엇을 의미하는지, 어떻게 생겨났는지, 
그것이 최선이 아닐 수 있는지에 대해서 말이다. 
그렇다면 명목화폐는 정확히 무엇일까? 
그리고 우리는 어떻게 명목화폐를 사용하게 되었을까?

\begin{comment}
	If something is imposed by \textit{fiat}, it simply means that it is imposed by
	formal authorization or proposition. Thus, fiat money is money simply
	because \textit{someone} says that it is money. Since all governments use fiat
	currency today, this someone is \textit{your} government. Unfortunately, you
	are not \textit{free} to disagree with this value proposition. You will quickly
	feel that this proposition is everything but non-violent. If you refuse
	to use this paper currency to do business and pay taxes the only people
	you will be able to discuss economics with will be your cellmates.
\end{comment}

\paragraph{}
어떤 것이 \textit{명목적}이라는 의미는, 공식적인 승인이나 발의에 의해 부여받은 것임을 의미한다. 
따라서 명목화폐는 단순히 \textit{누군가가} 그것이 돈이라고 말했기 때문에 돈이 된 것이다. 
오늘날 모든 정부가 명목화폐를 사용하므로 여기서 그 \textit{누군가}는 바로 정부이다. 
안타깝게도 우리에겐 이러한 가치 제안을 거부할 자유가 없다.
이것이 폭력적이지 않을 뿐 모든 것을 제약하는 것임을 금방 알아차렸을 것이다.
사업을 하고 세금을 내는 데에 이 종이 돈을 사용하지 않겠다고 하면,
당신이 경제에 대해 논의할 수 있는 사람은 감방 동료 밖에 없을 것이다.

\begin{comment}
	The value of fiat money does not stem from its inherent properties. How
	good a certain type of fiat money is, is only correlated to the
	political and fiscal (in)stability of those who dream it into existence.
	Its value is imposed by decree, arbitrarily.
\end{comment}

\paragraph{}
명목화폐의 가치는 그 자체가 갖는 고유한 속성에서 비롯되지 않는다.
명목화폐 간의 상대적 가치는 그 명목화폐의 존속을 꿈꾸는 사람들의 정치적, 재정적 (불)안정성에 따라 결정된다.
명목화폐의 가치는 법령이 임의로 부과한다.

\begin{figure}
	\centering
	\includegraphics[width=8cm]{assets/images/fiat-definition.png}
	\caption{명목화폐의 사전적 의미 --- `그대로 될지어다.(Let it be done)'}
	\label{fig:fiat-definition}
\end{figure}

\paragraph{}
%Until recently, two types of money were used: \textbf{commodity money}, made
%out of precious \textit{things}, and \textbf{representative money}, which simply
%\textit{represents} the precious thing, mostly in writing.
최근까지도 두 종류의 화폐가 쓰였다.
하나는 귀중품으로 만든 \textbf{상품화폐(commodity money)}이고, 
다른 하나는 귀중품의 가치를 뒷받침하는 \textbf{대리화폐(representative money)}이다.
대다수 대리화폐의 가치는 화폐에 쓰인 문구에서 비롯된다.

\paragraph{}
\begin{comment}
	We already touched on commodity money above. People used special bones,
	seashells, and precious metals as money. Later on, mainly coins made out of
	precious metals like gold and silver were used as money. The oldest coin found
	so far is made of a natural gold-and-silver mix and was made more than 2700
	years ago.\footnote{According to the Greek historian Herodotus, writing in the
		fifth century BC, the Lydians were the first people to have used gold and silver
		coinage. \cite{coinage-origins}} If something is new in Bitcoin, the concept of
	a coin is not it.
\end{comment}
우리는 이미 상품화폐에 대해 다루었다. 
사람들은 특별한 뼈와 조개껍질, 귀금속을 화폐로 사용했다. 
그 후에는 주로 금과 은으로 동전을 만들어 사용했다.
지금까지 발견된 가장 오래된 동전은 2700년도 더 된 것으로 순수 금과 은의 혼합물로 만들어졌다. \footnote{기원전 15년 전에 쓰인 헤로도토스의 그리스 역사에 의하면 리디아인들은 금, 은 주화를 사용한 최초의 사람들이었다고 쓰여있다.\cite{coinage-origins}}
"동전(coin)"이라는 개념을 비트코인이 처음 도입한 것이 아니라는 뜻이다.

\newpage

\begin{figure}
	\centering
	\includegraphics[width=5cm]{assets/images/lydian-coin-stater.png}
	\caption{리디아의 일렉트럼 동전. (출처: Classical Numismatic Group)}
	\label{fig:lydian-coin-stater}
\end{figure}

\paragraph{}
\begin{comment}
	Turns out that hoarding coins, or hodling, to use today's parlance, is
	almost as old as coins. The earliest coin hodler was someone who put
	almost a hundred of these coins in a pot and buried it in the
	foundations of a temple, only to be found 2500 years later. Pretty good
	cold storage if you ask me.
\end{comment}
전을 모으는 것(hoarding), 시쳇말로 호들링은 동전의 역사 만큼이나 오래된 것으로 밝혀졌다. 
최초의 동전 호들러는 100여개의 동전을 냄비에 넣고 사원 기둥에 묻은 사람이었다. 
이것은 2,500년이 지난 후에야 발견되었다. 냉동보관(cold storage)이 꽤 괜찮은 방법이라 할 수 있겠다.

\paragraph{}
\begin{comment}
	One of the downsides of using precious metal coins is that they can be
	clipped, effectively debasing the value of the coin. New coins can be
	minted from the clippings, inflating the money supply over time,
	devaluing every individual coin in the process. People were literally
	shaving off as much as they could get away with of their silver dollars.
	I wonder what kind of \textit{Dollar Shave Club} advertisements they had back
	in the day.
\end{comment}
금속으로된 동전의 단점 중 하나는 동전의 귀퉁이를 잘라서 가치를 떨어뜨릴 수 있다는 점이다. 
잘라낸 조각으로 새 동전을 주조하면 화폐의 양이 늘어나기 때문에 시간이 지날수록 동전의 가치가 떨어질 수 있다.
당시 사람들은 은화를 가능한 한 최대한 깎아내고(shaving off) 사용하였다.
그 시대의 \textit{달러 셰이브 클럽}(Dollar Shave Club)\footnote{역자: 미국 면도기 제조 기업} 광고는 어땠을지 궁금해진다.

\paragraph{}
\begin{comment}
	Since governments are only cool with inflation if they are the ones
	doing it, efforts were made to stop this guerrilla debasement. In
	classic cops-and-robbers fashion, coin clippers got ever more creative
	with their techniques, forcing the \enquote{masters of the mint} to get even
	more creative with their countermeasures. Isaac Newton, the
	world-renowned physicist of \textit{Principia Mathematica} fame, used to be one
	of these masters. He is attributed with adding the small stripes at the
	side of coins which are still present today. Gone were the days of easy
	coin shaving.
\end{comment}
정부는 정부 주도의 인플레이션에는 관대하지만, 
이러한 민간에서 행해지는 인플레이션인 동전 자르기 행위는 두고 볼 수 없었다. 
예전에 유행하던 경찰과 강도 놀이처럼, 
동전 자르기는 점점 창의적으로 자행되었기 때문에 조폐국은 이를 능가하는 창의성을 발휘해야만 했다.
당시 \textit{수학 원리(Principia Mathematica)}로 명성을 얻은 세계적인 물리학자 아이작 뉴턴은 이를 해결한 사람 중 한 명이다. 
동전 옆 테두리에 톱니바퀴 무늬를 추가하여 이를 해결하였는데, 이는 오늘날까지도 적용되고 있다.
이것으로 동전 자르기의 시대는 막을 내리게 된다.

\begin{figure}
	\includegraphics{assets/images/clipped-coins.png}
	\caption{심각하게 잘린 은화}
	\label{fig:clipped-coins}
\end{figure}

\paragraph{}
\begin{comment}
	Even with these methods of coin debasement\footnote{Besides clipping, sweating
		(shaking the coins in a bag and collecting the dust worn off) and plugging
		(punching a hole in the middle and hammering the coin flat to close the hole)
		were the most prominent methods of coin debasement. \cite{wiki:coin-debasement}}
	kept in check, coins still suffer from other issues. They are bulky and not very
	convenient to transport, especially when large transfers of value need to
	happen. Showing up with a huge bag of silver dollars every time you want to buy
	a Mercedes isn't very practical.
\end{comment}
이러한 가치 하락 방법\footnote{자르기 외에도 스웨팅(가방에 있는 동전을 흔들어 발생한 가루를 모으는 것),
	플러깅(가운데 구멍을 뚫고 동전을 납작하게 두드려 구멍을 막는 것) 등이 자주 발생하였다.
	\cite{wiki:coin-debasement}}
을 억제한다해도 동전은 여전히 다른 문제와 맞닥뜨리고 있다. 
특히 대량의 가치를 이전해야할 때, 동전을 사용하기엔 부피가 크고 운송이 그리 편하지 않다.
벤츠를 사고 싶을 때마다 엄청난 양의 은화가 든 가방을 들고 나타난다고 상상해보자.

\paragraph{}
\begin{comment}
	Speaking of German things: How the United States \textit{dollar} got its name is
	another interesting story. The word \enquote{dollar} is derived from the German word
	\textit{Thaler}, short for a \textit{Joachimsthaler}~\cite{wiki:thaler}. A
	Joachimsthaler was a coin minted in the town of \textit{Sankt Joachimsthal}.
	Thaler is simply a shorthand for someone (or something) coming from the valley,
	and because Joachimsthal was \textit{the} valley for silver coin production,
	people simply referred to these silver coins as \textit{Thaler.} Thaler (German)
	morphed into daalders (Dutch), and finally dollars (English).
\end{comment}
독일 얘기가 나온 참에, 독일인들이 말하는 미국 \textit{달러}가 어떻게 달러라는 이름을 갖게 됐는지에 대한 이야기가 매우 흥미롭다.
\enquote{달러(dollar)}라는 단어는 독일어 요하킴스탈러(Joachimsthaler)의 줄임말인 탈러(Thaler)에서 유래되었다고 한다\cite{wiki:thaler}. 
요하킴스탈러는 상트 요하킴스탈(Sankt Joachimsthal)\footnote{역자: 현재는 체코에 있으며 현지어로는 야히모프(Jáchymov)라 불린다.}이라는 마을에서 주조된 동전이었다.
탈러는 단순히 '요하킴스탈러 계곡 출신의 사람'의 약칭이었고 요하킴스탈러는 은화를 생산하는 계곡이었기 때문에 자연스럽게 은화를 탈러라고 부르게 되었다. 
요하킴스탈에서 온 누군가가 요하킴스탈러를 탈러라고 줄여 부르기 시작했고, 
독일어 탈러(Thaler)가 네덜란드어 달더스(daalders)로 변형되어 최종적으로 영어 달러(dollars)가 되었다고 한다.

\begin{figure}
	\centering
	\includegraphics[width=5cm]{assets/images/joachimsthaler.png}
	\caption{달러의 기원. 마법사의 모자와 로브를 입은 성 요하킴스가 그려져 있다. (출처: 위키피디아)}
	\label{fig:joachimsthaler}
\end{figure}

\paragraph{}
\begin{comment}
	The introduction of representative money heralded the downfall of hard
	money. Gold certificates were introduced in 1863, and about fifteen
	years later, the silver dollar was also slowly but surely being replaced
	by a paper proxy: the silver certificate. \cite{wiki:silver-certificate}
\end{comment}
대리화폐의 등장은 경화(hard money)의 몰락을 예고했다. 
금 증서는 1863년에 도입되었고 약 15년 후 은 달러도 느리지만 확실하게 종이 위임장 형태인 은 증서(silver certificate)로 대체되었다.\cite{wiki:silver-certificate}

\paragraph{}
\begin{comment}
	It took about 50 years from the introduction of the first silver
	certificates until these pieces of paper morphed into something that we
	would today recognize as one U.S. dollar.
\end{comment}
최초의 은 증서가 도입된 후 이 종이 조각이 
오늘날 우리가 미국 달러 1달러로 인식할 수 있는 형태로 변하기까지는 약 50년이 걸렸다.

\begin{figure}
	\centering
	\includegraphics{assets/images/us-silver-dollar-note-smaller.png}
	\caption{1928년의 미국의 은 1 달러화. `Payable to the bearer on demand.(요청 시 소지인에게 지급됨)'라고 적혀있다. (출처: 스미스소니언 재단 국립 화폐 컬렉션)}
	\label{fig:us-silver-dollar-note-smaller}
\end{figure}

\paragraph{}
\begin{comment}
	Note that the 1928 U.S. silver dollar in
	Figure~\ref{fig:us-silver-dollar-note-smaller} still goes by the name of
	\textit{silver certificate}, indicating that this is indeed simply a document
	stating that the bearer of this piece of paper is owed a piece of silver. It is
	interesting to see that the text which indicates this got smaller over time. The
	trace of \enquote{certificate} vanished completely after a while, being replaced
	by the reassuring statement that these are federal reserve notes.
\end{comment}
그림 \ref{fig:us-silver-dollar-note-smaller}의 1928년 미국 은화는 은 증서(silver certificate)라는
이름으로 사용되었으며 실제로 이 종이를 소지한 사람이 은의 소유자라는 것을 나타내었다. 
시간이 지남에 따라 이 문구가 점점 작아지는 것이 매우 흥미롭다. 
이 \enquote{증서(certificate)} 안 문구는 오래지않아 연방준비금이라는 문구로 대체되며 완전히 사라졌다.

\paragraph{}
\begin{comment}
	As mentioned above, the same thing happened to gold. Most of the world was on a
	bimetallic standard~\cite{wiki:bimetallism}, meaning coins were made
	primarily of gold and silver. Having certificates for gold, redeemable in gold
	coins, was arguably a technological improvement. Paper is more convenient,
	lighter, and since it can be divided arbitrarily by simply printing a smaller
	number on it, it is easier to break into smaller units.
\end{comment}
금에서도 같은 일이 일어났다. 
세계 대부분의 나라들은 주로 바이메탈 표준~\cite{wiki:bimetallism}에 따라 금화와 은화를 만들었다. 
금으로 교환할 수 있는 금 교환증이 있다는 것은 틀림없는 기술적 진보이다.
종이는 더 편하고 가볍다. 또 더 작은 숫자를 인쇄하면 가치를 임의로 나눌 수 있기 때문에 더 작은 단위 결제도 가능하다.

\paragraph{}
\begin{comment}
	To remind the bearers (users) that these certificates were
	representative for actual gold and silver, they were colored accordingly
	and stated this clearly on the certificate itself. You can fluently read
	the writing from top to bottom:
\end{comment}
소지자에게 이 증서가 실제 금과 은을 보유하고 있다는 것을 상기시키기 위해 지폐에는 이를 나타내는 색깔이 칠해졌고, 
증서 자체에 이에 대한 내용이 명기되었다. 여러분은 이 문구를 유창하게 읽을 수 있을 것이다.

\begin{comment}
	\begin{quotation}\begin{samepage}
			\enquote{This certifies that there have been deposited in the treasury of the
				United States of America one hundred dollars in gold coin payable to
				the bearer on demand.}
	\end{samepage}\end{quotation}
\end{comment}
\begin{quotation}\begin{samepage}
		\enquote{이 증서는 요구시 소지인에게 지불할 금화 100 달러가 미국 재무부에 예치되어 있음을 증명한다.}
\end{samepage}\end{quotation}

\begin{comment}
	\begin{figure}
		\centering
		\includegraphics{assets/images/us-gold-cert-100-smaller.png}
		\caption{A 1928 U.S. \$100 gold certificate. Picture cc-by-sa National Numismatic Collection, National Museum of American History.}
		\label{fig:us-gold-cert-100-smaller}
	\end{figure}
\end{comment}
\begin{figure}
	\centering
	\includegraphics{assets/images/us-gold-cert-100-smaller.png}
	\caption{1928년 미화 금 100 달러화 (출처: 미국 국립 박물관 국립 화폐 컬렉션)}
	\label{fig:us-gold-cert-100-smaller}
\end{figure}

\paragraph{}
\begin{comment}
	In 1963, the words \enquote{PAYABLE TO THE BEARER ON DEMAND} were removed from
	all newly issued notes. Five years later, the redemption of paper notes
	for gold and silver ended.
\end{comment}
1963년, 새로 발행된 모든 지폐에는 \enquote{PAYABLE TO THE BEARER ON DEMAND(요청 시 소지인에게 지급됨)}라는 문구가 제거되었다. 
그로부터 5년 후, 금과 은을 통한 달러의 속박은 막을 내리게 된다.

\paragraph{}
\begin{comment}
	The words hinting on the origins and the idea behind paper money were
	removed. The golden color disappeared. All that was left was the paper
	and with it the ability of the government to print as much of it as it
	wishes.
\end{comment}
지폐의 기원과 목적을 암시하는 단어는 삭제되었다. 황금색이 사라졌다. 
남은 것은 종이와 정부가 원하는 만큼 인쇄할 수 있는 능력뿐이었다.

\paragraph{}
\begin{comment}
	With the abolishment of the gold standard in 1971, this century-long
	sleight-of-hand was complete. Money became the illusion we all share to
	this day: fiat money. It is worth something because someone commanding
	an army and operating jails says it is wort능h something. As can be
	clearly read on every dollar note in circulation today, \enquote{THIS NOTE IS
		LEGAL TENDER}. In other words: It is valuable because the note says so.
\end{comment}
1971년 금본위제가 폐지되면서 한 세기에 걸친 속임수가 완성되었다. 
돈은 오늘날 우리 모두가 착각하는 환상, 즉 명목화폐가 되었다. 
군대를 지휘하고 감옥을 운영하는 자가 그것이 가치있다고 말함으로 인해 그 가치를 갖게 됐다. 
오늘날 유통되는 모든 달러에 적혀 있듯이 이 지폐는 법정 화폐이다. 
\enquote{THIS NOTE IS LEGAL TENDER(이 지폐는 법적인 화폐이다.)}
즉, 이 한 줄이 달러의 가치를 부여한다.

\begin{comment}
	\begin{figure}
		\centering
		\includegraphics{assets/images/us-dollar-2004.jpg}
		\caption{A 2004 series U.S. twenty dollar note used today. `THIS NOTE IS LEGAL TENDER'}
		\label{fig:us-dollar-2004}
	\end{figure}
\end{comment}
\begin{figure}
	\centering
	\includegraphics{assets/images/us-dollar-2004.jpg}
	\caption{2004년 현재의 미국 20달러. `THIS NOTE IS LEGAL TENDER'라고 적혀있다.}
	\label{fig:us-dollar-2004}
\end{figure}

\paragraph{}
\begin{comment}
	By the way, there is another interesting lesson on today's bank notes,
	hidden in plain sight. The second line reads that this is legal tender
	\enquote{FOR ALL DEBTS, PUBLIC AND PRIVATE}. What might be obvious to economists
	was surprising to me: All money is debt. My head is still hurting
	because of it, and I will leave the exploration of the relation of money
	and debt as an exercise to the reader.
\end{comment}
그건 그렇고, 눈에 잘 띄진 않지만, 오늘날의 지폐에는 흥미로운 교훈이 하나 더 있다. 
두 번째 줄에 적힌 \enquote{FOR ALL DEBTS, PUBLIC AND PRIVATE(공적 및 사적인 모든 부채를 위하여)}라는 문구가 그것이다. 
이 말인즉 모든 돈은 빚이라는 의미이다. 경제학자들은 알고 있었을지 모르나 나에게는 새로운 사실이었다. 
이것 때문에 아직도 머리가 아프다. 돈과 빚의 관계는 여러분에게 숙제로 남겨두겠다.

\paragraph{}
\begin{comment}
	As we have seen, gold and silver were used as money for millennia. Over
	time, coins made from gold and silver were replaced by paper. Paper
	slowly became accepted as payment. This acceptance created an
	illusion --- the illusion that the paper itself has value. The final
	move was to completely sever the link between the representation and the
	actual: abolishing the gold standard and convincing everyone that the
	paper in itself is precious.
\end{comment}
앞서 살펴본 것처럼 금과 은은 수천 년 동안 돈으로 사용되었다. 
시간이 지나면서 금과 은으로 만든 동전은 종이로 대체되었다. 
종이는 서서히 지불 수단으로 받아들여졌다. 
이러한 수용은 종이 자체에 가치가 있다는 환상을 만들어냈다. 
최종적으로 대리와 실체를 연결하는 고리는 완전히 끊어졌다.
즉 금본위제는 폐지되었고 종이 자체에 가치가 있다는 것을 모두가 확신하게 되었다.

\begin{comment}
	\paragraph{Bitcoin taught me about the history of money and the greatest sleight of
		hand in the history of economics: fiat currency.}
\end{comment}
\paragraph{비트코인은 나에게 돈의 역사와 경제학 역사상 가장 큰 속임수인 명목화폐에 대해 가르쳐주었다.}

% ---
%
% #### Down the Rabbit Hole
%
% - [Shelling Out: The Origins of Money] by Nick Szabo
% - [Methods of Coin Debasement][coin debasement], [Thaler], [U.S. Silver Certificate][silver certificates], [Bimetallism][bimetallic standard] on Wikipedia
%
% [oldest coin]: https://www.britishmuseum.org/explore/themes/money/the_origins_of_coinage.aspx
% [coin debasement]: https://en.wikipedia.org/wiki/Methods_of_coin_debasement
% [Thaler]: https://en.wikipedia.org/wiki/Thaler
% [Berlin-George]: https://en.wikipedia.org/wiki/File:Bohemia,_Joachimsthaler_1525_Electrotype_Copy._VF._Obverse..jpg
% [silver certificates]: https://en.wikipedia.org/wiki/Silver_certificate_%28United_States%29
% [bimetallic standard]: https://en.wikipedia.org/wiki/Bimetallism
% [Shelling Out: The Origins of Money]: https://nakamotoinstitute.org/shelling-out/
%
% <!-- Wikipedia -->
% [alice]: https://en.wikipedia.org/wiki/Alice%27s_Adventures_in_Wonderland
% [carroll]: https://en.wikipedia.org/wiki/Lewis_Carroll

\chapter{부분 지급준비금의 광기}
\label{les:13}

\begin{chapquote}{루이스 캐롤, \textit{이상한 나라의 앨리스}}
	아아! 너무 늦었어. 그녀는 계속 커지고 커져서 곧 바닥에 무릎을 꿇어야 했다.
	잠시 후에는 이마저도 할 공간이 없었고, 한쪽 팔꿈치를 문에 대고 누워서 다른 팔로 머리를 감쌌다.
	그래도 계속 커져서 마지막 방법으로 그녀는 한 팔을 창 밖으로, 한 발을 굴뚝 위로 내밀면서
	혼잣말을 했다. \enquote{이제 더 이상 할 수 있는게 없어. 난 어떻게 되는 거지?}
\end{chapquote}

\begin{comment}
	Value and money aren't trivial topics, especially in today's times. The
	process of money creation in our banking system is equally non-trivial,
	and I can't shake the feeling that this is deliberately so. What I have
	previously only encountered in academia and legal texts seems to be
	common practice in the financial world as well: nothing is explained in
	simple terms, not because it is truly complex, but because the truth is
	hidden behind layers and layers of jargon and \textit{apparent} complexity.
	\enquote{Expansionary monetary policy, quantitative easing, fiscal stimulus to
		the economy.} The audience nods along in agreement, hypnotized by the
	fancy words.
\end{comment}
가치와 돈은 특히 오늘날과 같은 시대에 사소한 주제가 아니다. 
은행 시스템에서 돈을 창출하는 과정 또한 사소하지 않은데 나는 이것이 지극히 의도적이었다는 의심을 지울 수 없다. 
학계와 법조문에서만 접했던 것이 금융계에서도 일반적 관행으로 쓰이는 것 같다.
진짜로 복잡해서가 아니라, 여러 겹의 전문 용어와 겉으로 드러나는 복장성 뒤에 진실이 숨겨져있기 때문에 어떤 것도 쉽게 설명하지 않는다는 것이다.
\enquote{통화 확장 정책, 양적 완화, 재정 부양책} 멋
청중은 화려한 단어에 최면에 걸린 듯 고개를 끄덕이며 동의할 뿐이다.

\begin{comment}
	Fractional reserve banking and quantitative easing are two of those
	fancy words, obfuscating what is really happening by masking it as
	complex and difficult to understand. If you would explain them to a
	five-year-old, the insanity of both will become apparent quickly.
\end{comment}
부분 지급 준비 은행과 양적 완화라는 이 두 멋진 용어는 
복잡하고 이해하기 어려운 단어로 위장하여 실제 상황을 흐리게 만든다. 
5살짜리 아이에게 설명한다면 두 가지 모두 미친 짓이라는 것이 금세 드러날 것이다.

\begin{comment}
	Godfrey Bloom, addressing the European Parliament during a joint
	debate, said it way better than I ever could:
\end{comment}
고드프레이 블룸(Godfrey Bloom)은 유럽 의회의 토론에서 이를 훨씬 잘 설명했다.

\begin{comment}
	\begin{quotation}\begin{samepage}
			\enquote{[...] you do not really understand the concept of banking. All the
				banks are broke. Bank Santander, Deutsche Bank, Royal Bank of
				Scotland --- they're all broke! And why are they broke? It isn't an
				act of God. It isn't some sort of tsunami. They're broke because we
				have a system called `fractional reserve banking' which means that
				banks can lend money that they don't actually have! It's a criminal
				scandal and it's been going on for too long. [...]
				We have counterfeiting --- sometimes called quantitative
				easing --- but counterfeiting by any other name. The artificial
				printing of money which, if any ordinary person did, they'd go to
				prison for a very long time [...] and until we start sending
				bankers --- and I include central bankers and politicians --- to
				prison for this outrage it will continue.}
			\begin{flushright} -- Godfrey Bloom\footnote{Joint debate on the
					banking union~\cite{godfrey-bloom}}
	\end{flushright}\end{samepage}\end{quotation}
\end{comment}
\begin{quotation}\begin{samepage}
		\enquote{[...] 당신은 은행의 개념을 이해하지 못하고 있습니다. 
			모든 은행이 파산했습니다. 산탄데르 은행, 도이치 은행, 스코틀랜드 왕립은행 등 모두 파산했어요!
			왜 파산했을까요? 천재지변도 아니고 쓰나미 같은 것도 아닙니다. 
			은행이 실제로 가지고 있지 않은 돈을 빌려줄 수 있는 '부분 지급 준비금 은행'이라는 시스템이 있기 때문에 파산한 것입니다! 
			이것은 범죄이고 너무 오랫동안 계속되어 왔습니다.[...]	
			우리는 양적 완화라고도 불리는 위조를 하고있지만 다른 이름으로도 위조를 하고 있습니다.
			인위적으로 돈을 찍어내는 행위를 일반인이 저질렀다면 아주 오랫동안 감옥에 갇히게 될 것입니다.[...] 
			그리고 우리가 중앙은행가와 정치인들을 포함한 은행가들을 감옥에 보내기 전까지 이 분노는 계속될 것입니다.}
		\begin{flushright} -- 고드프레이 블룸\footnote{Joint debate on the
				banking union~\cite{godfrey-bloom}}
\end{flushright}\end{samepage}\end{quotation}

\begin{comment}
	Let me repeat the most important part: banks can lend money that they
	don't actually have.
\end{comment}
가장 중요한 부분이니 다시 한번 강조한다.
은행은 실제로 가지고 있지 않은 돈을 빌려줄 수 있다.

\begin{comment}
	Thanks to fractional reserve banking, a bank only has to keep a small
	\textit{fraction} of every dollar it gets. It's somewhere between $0$ and $10\%$,
	usually at the lower end, which makes things even worse.
\end{comment}
부분 준비금 은행 덕분에 은행은 달러의 극히 \textit{일부만} 보유하면 된다. 
보톤 $0$에서 $10\%$ 사이로, 낮은 편에 속하기 때문에 상황은 더욱 악화된다.

\begin{comment}
	Let's use a concrete example to better understand this crazy idea: A
	fraction of $10\%$ will do the trick and we should be able to do all the
	calculations in our head. Win-win. So, if you take \$100 to a
	bank --- because you don't want to store it under your mattress --- they
	only have to keep the agreed upon \textit{fraction} of it. In our example that
	would be \$10, because 10\% of \$100 is \$10. Easy, right?
\end{comment}
이 정신나간 아이디어를 더 잘 이해하기 위해 구체적인 예를 들어보겠다. 
단순한 암산만 하면 이해할 수 있다. 준비율이 $10\%$라고 가정해 보자.
당신에게 \$100가 있는데 이를 침대 밑에 보관하고 싶지 않다.
그래서 은행에 \$100를 맡긴다면 은행은 일부만 금고에 보관하고 있으면 된다.
준비율 10\%를 적용하면 \$10이기 때문에 \$10만 금고에 보관하면 되는 것이다.  
간단하다.

\begin{comment}
	So what do banks do with the rest of the money? What happens to your \$90? They
	do what banks do, they lend it to other people. The result is a money multiplier
	effect, which increases the money supply in the economy enormously
	(Figure~\ref{fig:money-multiplier}). Your initial deposit of \$100 will soon
	turn into \$190. By lending a 90\% fraction of the newly created \$90, there
	will soon be \$271 in the economy. And \$343.90 after that. The money supply is
	recursively increasing, since banks are literally lending money they don't
	have~\cite{wiki:money-multiplier}. Without a single Abracadabra, banks magically
	transform \$100 into one thousand dollars or more. Turns out 10x is easy. It
	only takes a couple of lending rounds.
\end{comment}
그렇다면 은행은 나머지 돈으로 무엇을 할까? 
당신의 \$90는 어떻게 되는 것인가? 
은행은 그 돈을 다른 사람에게 이를 빌려준다. 
그 결과 통화 승수효과가 발생하여 통화 공급이 엄청나게 늘어난다(그림~\ref{fig:money-multiplier}). 
당신이 저축한 예금 \$100은 곧 \$190으로 바뀔 것이다. 
새로 생성된 \$90의 90\%를 다시 대출함으로써 시장에는 곧 \$291이 존재하게 된다. 
그리고 또 \$343.90이 된다. 
은행이 말 그대로 가지고 있지 않은 돈을 빌려주기 때문에 통화 공급이 반복적으로 증가한다\cite{wiki:money-multiplier}. 
아브라카다브라 마법이 없어도, 은행은 \$100을 \$1000로 바꿀 수 있다. 
몇 번의 대출만 있다면 10배로 뻥튀기 하는 것은 식은 죽 먹기다.

\begin{comment}
	\begin{figure}
		\centering
		\includegraphics{assets/images/money-multiplier.png}
		\caption{The money multiplier effect}
		\label{fig:money-multiplier}
	\end{figure}
\end{comment}
\begin{figure}
	\centering
	\includegraphics{assets/images/money-multiplier.png}
	\caption{통화 승수 효과}
	\label{fig:money-multiplier}
\end{figure}

\paragraph{}
\begin{comment}
	Don't get me wrong: There is nothing wrong with lending. There is
	nothing wrong with interest. There isn't even anything wrong with good
	old regular banks to store your wealth somewhere more secure than in
	your sock drawer.
\end{comment}
오해하지 말라. 대출은 잘못이 없다. 이자가 나쁜 것이 아니다. 
양말 서랍보다 더 안전한 곳에 재산을 보관하는 오래된 일반 은행에는 아무런 문제가 없다. 

\begin{comment}
	Central banks, however, are a different beast. Abominations of financial
	regulation, half public half private, playing god with something which
	affects everyone who is part of our global civilization, without a
	conscience, only interested in the immediate future, and seemingly
	without any accountability or auditability (see Figure~\ref{fig:bsg}).
\end{comment}
그러나 중앙은행은 또 다른 짐승이다. 
가증스러운 금융 규제, 공적도 사적도 아닌 애매한 위치, 
전 세계 사람들을 대상으로 절대자 행세를 하는, 양심도 없고 근시안적 이익에만 관심이 있는, 
어떤 책임이나 견제도 없는 것처럼 보이는 조직이다.(그림~\ref{fig:bsg})

\begin{comment}
	\begin{figure}
		\centering
		\includegraphics{assets/images/bsg.jpg}
		\caption{Yellen is strongly opposed to audit the Fed, while Bitcoin Sign Guy is strongly in favor of buying bitcoin.}
		\label{fig:bsg}
	\end{figure}
\end{comment}
\begin{figure}
	\centering
	\includegraphics{assets/images/bsg.jpg}
	\caption{연준의 감사를 강력하게 반대하는 옐런. 비트코인을 사라(Buy Bitcoin) 팻말을 든 남자에게 찬성한다.}
	\label{fig:bsg}
\end{figure}

\begin{comment}
	While Bitcoin is still inflationary, it will cease to be so rather soon.
	The strictly limited supply of 21 million bitcoins will eventually do
	away with inflation completely. We now have two monetary worlds: an
	inflationary one where money is printed arbitrarily, and the world of
	Bitcoin, where final supply is fixed and easily auditable for everyone.
	One is forced upon us by violence, the other can be joined by anyone who
	wishes to do so. No barriers to entry, no one to ask for permission.
	Voluntary participation. That is the beauty of Bitcoin.
\end{comment}
비트코인은 지금도 발행되고 있지만 머지않아 멈추게 될 것이다. 
비트코인 총량은 2,100만 개로 엄격하게 제한되어 있기 때문에 결국 인플레이션은 완전히 없어질 것이다.
이제 우리는 두 가지 화폐 세계를 갖게 되었다. 
하나는 돈이 임의로 인쇄되는 인플레이션 세계이고, 다른 하나는 최종 공급량이 고정되어 모든 사람이 쉽게 감사할 수 있는 비트코인 세계이다.
하나는 폭력으로 우리를 강요하고, 다른 하나는 원하는 누구든 참여할 수 있다.
진입 장벽도 없고 누군가에게 허가받을 필요도 없다. 
자발적인 참여. 그것이 비트코인의 아름다움이다.

\begin{comment}
	I would argue that the argument between Keynesian\footnote{Theories according to
		John Maynard Keynes and his deciples~\cite{wiki:keynesian}} and
	Austrian\footnote{School of economic thought based on methodological
		individualism~\cite{wiki:austrian}} economists is no longer purely academical.
	Satoshi managed to build a system for value transfer on steroids, creating the
	soundest money which ever existed in the process. One way or another, more and
	more people will learn about the scam which is fractional reserve banking. If
	they come to similar conclusions as most Austrians and Bitcoiners, they might
	join the ever-growing internet of money. Nobody can stop them if they choose to
	do so.
\end{comment}
나는 케인즈주의\footnote{존 메이너드 케인즈와 제자들의 이론~\cite{wiki:keynesian}}와 오스트리아학파\footnote{방법론적 개인주의에 입각한 경제학파~\cite{wiki:austrian}} 사이의 논쟁이 학문적인 논쟁이라 생각하지 않는다. 
사토시 나카모토는 강력한 가치 전송 시스템을 구축하여 세상에서 가장 건전한 화폐를 만들었다. 
어떤 식으로든 점점 더 많은 사람이 지급준비금 은행의 사기에 대해 알게 될 것이다. 
이를 깨달은 사람들이 오스트리아 학파나 비트코이너와 유사한 결론에 도달하면 
그들은 아마 지속적으로 성장하는 돈을 위한 인터넷(the internet of money)에 참여하게 될 것이다. 
그들의 선택을 아무도 막을 수 없다.

%\paragraph{Bitcoin taught me that fractional reserve banking is pure insanity.}
\paragraph{비트코인은 부분 지급준비금 시스템이 진짜 광기라는 것을 가르쳐주었다.}

% ---
%
% #### Down the Rabbit Hole
%
% - [The Creature From Jekyll Island] by G. Edward Griffin
% - [Money Multiplier][money multiplier], [Keynesian Economics][Keynesian], [Austrian School][Austrian] on Wikipedia
%
% [The Creature From Jekyll Island]: https://archive.org/details/pdfy--Pori1NL6fKm2SnY
%
% [joint debate]: https://www.youtube.com/watch?v=hYzX3YZoMrs
% [money multiplier]: https://en.wikipedia.org/wiki/Money_multiplier
% [auditability]: https://i.ytimg.com/vi/ThFGs347MW8/maxresdefault.jpg
% [Keynesian]: https://en.wikipedia.org/wiki/Keynesian_economics
% [Austrian]: https://en.wikipedia.org/wiki/Austrian_School
%
% <!-- Wikipedia -->
% [alice]: https://en.wikipedia.org/wiki/Alice%27s_Adventures_in_Wonderland
% [carroll]: https://en.wikipedia.org/wiki/Lewis_Carroll

\chapter{건전 화폐}
\label{les:14}

\begin{chapquote}{루이스 캐롤, \textit{이상한 나라의 앨리스}}
	\begin{comment}	
		\enquote{The first thing I've got to do,} said Alice to herself, as she wandered about
		in the wood, \enquote{is to grow to my right size, and the second thing is to find my
			way into that lovely garden. I think that will be the best plan.}
	\end{comment}
	\enquote{내가 첫 번째로 해야 할 일은,} 앨리스가 숲속을 배회하며 혼잣말을 했다. \enquote{원래 크기로 돌아가는 것이고, 
		두 번째 할 일은 아름다운 정원으로 가는 방법을 찾는거야. 이게 최선이야.}
\end{chapquote}

\paragraph{}
\begin{comment}	
	The most important lesson I have learned from Bitcoin is that in the
	long run, hard money is superior to soft money. Hard money, also
	referred to as \textit{sound money}, is any globally traded currency that
	serves as a reliable store of value.
\end{comment}
내가 비트코인에서 배운 가장 중요한 교훈은 장기적으로 경화(hard money)가 연화(soft money)보다 우월하다는 것이다.
건전 화폐(sound money)라고도 불리는 경화는 신뢰할 수 있는 가치 저장소 역할을 하는 전 세계적으로 거래되는 화폐를 의미한다.

\paragraph{}
\begin{comment}	
	Granted, Bitcoin is still young and volatile. Critics will say that it
	does not store value reliably. The volatility argument is missing the
	point. Volatility is to be expected. The market will take a while to
	figure out the just price of this new money. Also, as is often jokingly
	pointed out, it is grounded in an error of measurement. If you think in
	dollars you will fail to see that one bitcoin will always be worth one
	bitcoin.
\end{comment}
물론, 비트코인은 아직 초기 단계에 있고 변동성이 크다. 
이를 두고 비평가들은 가치를 안정적으로 저장하지 못한다고 주장할 것이다.
하지만, 변동성 논쟁은 요점이 아니다. 변동성은 얼마든지 나타날 수 있다. 
시장은 이 새로운 돈의 정당한 가격을 결정하는 데 시간이 걸릴 것이다. 
또한 흔히 농담조로 지적하듯이 이러한 가치 측정 방법에는 오류가 있다. 
비트코인의 가치를 달러로 측정하면 1비트코인이 항상 1비트코인의 가치가 있다는 사실을 놓치게 된다는 점을 알아야 한다.

\begin{quotation}\begin{samepage}
		\begin{comment}	
			\enquote{A fixed money supply, or a supply altered only in accord with
				objective and calculable criteria, is a necessary condition to a
				meaningful just price of money.}
		\end{comment}
		\enquote{고정된 공급, 즉 객관적이고 계산 가능한 기준대로 변경되는 공급이 정당한 화폐 가격을 위한 필수 조건이다.}
		\begin{flushright} -- 버나드 뎀프시\footnote{Fr. Bernard W. Dempsey, S.J., \textit{Interest and Usury}~\cite[p.~210]{dempsey_interest_1943}}
\end{flushright}\end{samepage}\end{quotation}

\newpage

\paragraph{}
\begin{comment}	
	As a quick stroll through the graveyard of forgotten currencies has
	shown, money which can be printed will be printed. So far, no human in
	history was able to resist this temptation.
\end{comment}
역사적으로 잊혀진 많은 화폐를 훑어보면 알 수 있듯이 돈은 계속 인쇄될 것이다. 
지금까지 역사상 어떤 인간도 이 유혹을 뿌리치지 못했다.

\paragraph{}
\begin{comment}	
	Bitcoin does away with the temptation to print money in an ingenious
	way. Satoshi was aware of our greed and fallibility --- this is why he
	chose something more reliable than human restraint: mathematics.
\end{comment}
비트코인은 기발한 방법으로 돈을 인쇄하려는 유혹을 물리친다. 
사토시는 우리의 탐욕 때문에 오류를 범할 것이라 예상했다.
이것이 그가 인간의 통제보다 신뢰할 수 있는 것, 즉 수학을 선택한 이유이다.

\begin{figure}
	\centering
	\begin{equation}
		\sum\limits_{i=0}^{32} \frac{210000 \lfloor \frac{50*10^8}{2^i} \rfloor}{10^8}
	\end{equation}
	\caption{비트코인 공급량 공식}
	\label{fig:supply-formula-white}
\end{figure}

\paragraph{}
\begin{comment}	
	While this formula is useful to describe Bitcoin's supply, it is actually
	nowhere to be found in the code. Issuance of new bitcoin is done in an
	algorithmically controlled fashion, by reducing the reward which is paid to
	miners every four years~\cite{btcwiki:supply}. The formula above is used to
	quickly sum up what is happening under the hood. What really happens can be best
	seen by looking at the change in block reward, the reward paid out to whoever
	finds a valid block, which roughly happens every 10 minutes.
\end{comment}다
이 공식은 비트코인의 공급을 설명하는 데 유용하지만 실제 코드에서는 찾을 수 없다. 
신규 비트코인 발행은 4년마다 채굴자에게 지급되는 보상을 줄이는 알고리즘을 통해 제어된다.\cite{btcwiki:supply}
위 공식은 단지 비트코인 내부의 작동 방식을 빠르게 이해하고 요약하는 데 사용되는 것이다. 
실제로는 대략 10분마다 발생하는 유효한 블록을 찾는 채굴자에게 지급되는 보상의 변화를 보면 잘 알 수 있다.

\begin{figure}
	\includegraphics{assets/images/you-are-here.png}
	\caption{비트코인의 공급량 제한}
	\label{fig:you-are-here.png}
\end{figure}

\paragraph{}
\begin{comment}	
	Formulas, logarithmic functions and exponentials are not exactly
	intuitive to understand. The concept of \textit{soundness} might be easier to
	understand if looked at in another way. Once we know how much there is
	of something, and once we know how hard this something is to produce or
	get our hands on, we immediately understand its value. What is true for
	Picasso's paintings, Elvis Presley's guitars, and Stradivarius violins
	is also true for fiat currency, gold, and bitcoins.
\end{comment}
공식, 대수 함수, 지수는 이해하기 직관적이지 않다. 
건전성이라는 개념은 다른 방식으로 살펴보면 더 쉽게 이해할 수 있다.
무언가가 얼마나 많은지 알게 되면, 그리고 그것이 생산하거나 획득하는 것이 얼마나 어려운지 알게 되면 
우리는 즉시 그 가치를 이해하게 된다. 
피카소의 그림, 엘비스 프레슬리의 기타, 스트라디바리우스의 바이올린이 왜 가치있는지 안다면,
법정화폐, 금, 비트코인이 얼마나 가치 있는지도 알 수 있을 것이다.

\paragraph{}
\begin{comment}	
	The hardness of fiat currency depends on who is in charge of the
	respective printing presses. Some governments might be more willing to
	print large amounts of currency than others, resulting in a weaker
	currency. Other governments might be more restrictive in their money
	printing, resulting in harder currency.
\end{comment}
법정화폐의 건전성은 인쇄기를 담당하는 사람에게 달려있다. 
어떤 정부는 다른 정부보다 더 많은 양의 돈을 인쇄하여 통화 가치의 약세를 초래할 수 있다.
한편 어떤 정부는 화폐 발행을 제한하여 통화 강세를 이끌 수도 있다.


\begin{samepage}\begin{quotation}
		\begin{comment}	
			\enquote{One important aspect of this new reality is that institutions like
				the Fed cannot go bankrupt. They can print any amount of money that
				they might need for themselves at virtually zero cost.}
		\end{comment}
		\enquote{이 새로운 현실의 한 가지 중요한 측면은 연준과 같은 기관이 파산할 수 없다는 것이다. 
		그들은 사실상 비용을 전혀 들이지 않고 자신이 필요한 만큼 돈을 인쇄할 수 있다.}
		\begin{flushright} -- 요르그 귀도 홀스만\footnote{Jörg Guido Hülsmann, \textit{The
					Ethics of Money Production}~\cite{hulsmann2008ethics}}
\end{flushright}\end{quotation}\end{samepage}

\paragraph{}
\begin{comment}	
	Before we had fiat currencies, the soundness of money was determined by
	the natural properties of the stuff which we used as money. The amount
	of gold on earth is limited by the laws of physics. Gold is rare because
	supernovae and neutron star collisions are rare. The \enquote{flow} of gold is
	limited because extracting it is quite an effort. Being a heavy element
	it is mostly buried deep underground.
\end{comment}
법정화폐 이전 시대에는 화폐로 사용되는 물질의 자연적 속성이 화폐 건전성을 결정했다. 
물리 법칙에 의해 지구에 매장된 금은 한정되어 있다.
금이 새로 생기려면 초신성과 중성자성이 충돌해야 하는데, 이러한 일은 거의 발생하지 않는다.
금을 추출하는 것은 상당히 어렵기 때문에 공급이 제한적이다.
무거운 원소들은 대체로 지하 깊은 곳에 묻혀 있기 때문이다.

\paragraph{}
\begin{comment}	
	The abolishment of the gold standard gave way to a new reality: adding new money
	requires just a drop of ink. In our modern world adding a couple of zeros to the
	balance of a bank account requires even less effort: flipping a few bits in a
	bank computer is enough.
\end{comment}
금본위제 폐지는 우리를 새로운 국면으로 이끌었다.
이제 화폐 발행을 위해 잉크 한 방울만 있으면 된다.
현대 사회에서는 훨씬 쉽다. 은행 계좌 잔액에 0 몇 개를 더 추가하기만 하면 된다.
은행 컴퓨터에서 몇 개의 비트만 뒤집으면 되는 것이다.

\paragraph{}
\begin{comment}	
	The principle outlined above can be expressed more generally as the
	ratio of \enquote{stock} to \enquote{flow}. Simply put, the \textit{stock} is how much of
	something is currently there. For our purposes, the stock is a measure
	of the current money supply. The \textit{flow} is how much there is produced
	over a period of time (e.g. per year). The key to understanding sound
	money is in understanding this stock-to-flow ratio.
\end{comment}
위에 설명한 원리는 \enquote{저량(stock)} 대 \enquote{유량(flow)}을 사용하여 좀 더 일반적으로 표현할 수 있다.
간단히 말해 \enquote{저량}은 현재 존재하는 양을 말한다. 여기에서는 현재 통화 공급량을 의미한다. 
\enquote{유량}은 일정 기간 동안의 생산량을 나타낸다.
건전 화폐 이해의 핵심은 이 저량 대 유량 비율을 이해하는 것이다.

\paragraph{}
\begin{comment}	
	Calculating the stock-to-flow ratio for fiat currency is difficult, because how
	much money there is depends on how you look at it.~\cite{wiki:money-supply} You
	could count only banknotes and coins (M0), add traveler checks and check
	deposits (M1), add saving accounts and mutual funds and some other things (M2),
	and even add certificates of deposit to all of that (M3). Further, how all of
	this is defined and measured varies from country to country and since the US
	Federal Reserve stopped publishing \cite{web:fed-m3} numbers for M3, we will
	have to make do with the M2 monetary supply. I would love to verify these
	numbers, but I guess we have to trust the fed for now.
\end{comment}
정부화폐의 저량 대 유량 비율을 계산하는 것은 어렵다. 
어떤 돈을 기준으로 하느냐에 따라 달라질 수 있기 때문이다\cite{wiki:money-supply}.
지폐와 동전(M0)만 계산할 수도, 여행자 수표와 예금(M1)을 추가할 수도, 
저축 계좌와 뮤추얼 펀드 등(M2)을 추가할 수도, 심지어 예금 증서(M3)를 추가할 수도 있다.
게다가 이것들의 정의와 측정하는 방법이 국가마다 다르다.
미국 연방준비 은행이 M3 수치 공개를 중단했기 때문에 우리는 M2로 확인할 수 밖에 없다.\cite{web:fed-m3}
이 수치를 직접 검증하고 싶지만 지금은 연준을 믿을 수 밖에 없다.

\paragraph{}
\begin{comment}	
	Gold, one of the rarest metals on earth, has the highest stock-to-flow
	ratio. According to the US Geological Survey, a little more than 190,000
	tons have been mined. In the last few years, around 3100 tons of gold
	have been mined per year.~\cite{mineral-commodity-summaries}
\end{comment}
지구상에서 가장 희귀한 금속인 금은 저량 대 유량의 비율이 가장 높다.
미국 지질조사국에 따르면 금은 현재까지 약 19만 톤 조금 넘게 채굴되었다.
그 중 지난 몇 년 동안 매년 약 3,100톤이 채굴되었다.~\cite{mineral-commodity-summaries}

\paragraph{}
\begin{comment}	
	Using these numbers, we can easily calculate the stock-to-flow ratio for
	gold (see Figure~\ref{fig:stock-to-flow-gold}).
\end{comment}
이 숫자를 사용하여 우리는 금의 저량 대 유량 비율을 쉽게 계산할 수 있다. (그림 ~\ref{fig:stock-to-flow-gold})

\begin{figure}
	\centering
	\begin{equation}
		\frac{190,000 t}{3,100 t} = ~ 61
	\end{equation}
	\caption{금의 저량 대 유량 비율}
	\label{fig:stock-to-flow-gold}
\end{figure}

\paragraph{}
\begin{comment}	
	Nothing has a higher stock-to-flow ratio than gold. This is why gold, up to now,
	was the hardest, soundest money in existence. It is often said that all the gold
	mined so far would fit in two olympic-sized swimming pools. According to my
	calculations\footnote{\url{https://bit.ly/gold-pools}}, we would need four. So
	maybe this needs updating, or Olympic-sized swimming pools got smaller.
\end{comment}
금보다 저량 대 유량 비율이 더 높은 것은 없다. 
이것이 금이 지금까지, 그리고 현존하는 가장 견고하고 건전한 화폐인 이유이다. 
지금까지 채굴된 금의 양이 올림픽 규격 수영장 두 개에 정도 될 것이라고들 말한다.
하지만 내 계산에 따르면 수영장 네 개가 필요하다.\footnote{\url{https://bit.ly/gold-pools}}
수치가 잘못되었거나 올림픽 규모의 수영장 크기가 작아지지 않았다면 말이다.

\paragraph{}
\begin{comment}	
	Enter Bitcoin. As you probably know, bitcoin mining was all the rage in
	the last couple of years. This is because we are still in the early
	phases of what is called the \textit{reward era}, where mining nodes are
	rewarded with \textit{a lot} of bitcoin for their computational effort. We are
	currently in reward era number 3, which began in 2016 and will end in
	early 2020, probably in May. While the bitcoin supply is predetermined,
	the inner workings of Bitcoin only allow for approximate dates.
	Nevertheless, we can predict with certainty how high Bitcoin's
	stock-to-flow ratio will be. Spoiler alert: it will be high.
\end{comment}
이제 비트코인의 저량 대 유량 비율을 보자. 
잘 알겠지만, 비트코인 채굴은 지난 몇 년간 대유행이었다.
아직 채굴 노드가 계산 노력에 비해 많은 보상을 받는 초기 보상 단계에 있기 때문이다.
우리는 현재 2016년에 시작되어 2020년 5월에 끝나는 세 번째 반감기를 지나고 있다.
반감기에 따른 비트코인의 공급량은 미리 정해져 있지만 날짜는 정확하지 않을 수 있다.
그럼에도 불구하고 비트코인의 저량 대 유량의 비율을 계산할 수 있다.
미리 스포일러를 하자면 높다.


\begin{comment}	
	How high? Well, it turns out that Bitcoin will get infinitely hard (see
	Figure~\ref{fig:stock-to-flow-white-cropped}).
\end{comment}
높지 않은가? 비트코인은 무한대가 될 때까지 높아질 것이다.(그림 ~\ref{fig:stock-to-flow-white-cropped})

\begin{figure}
	\includegraphics{assets/images/stock-to-flow-white-cropped.png}
	\caption{달러, 금, 비트코인의 저량과 유량}
	\label{fig:stock-to-flow-white-cropped}
\end{figure}

\paragraph{}
\begin{comment}	
	Due to an exponential decrease of the mining reward, the flow of new
	bitcoin will diminish resulting in a sky-rocketing stock-to-flow ratio.
	It will catch up to gold in 2020, only to surpass it four years later by
	doubling its soundness again. Such a doubling will occur 32 times in
	total. Thanks to the power of exponentials, the number of bitcoin mined
	per year will drop below 100 bitcoin in 50 years and below 1 bitcoin in
	75 years. The global faucet which is the block reward will dry up
	somewhere around the year 2140, effectively stopping the production of
	bitcoin. This is a long game. If you are reading this, you are still
	early.
\end{comment}
채굴 보상이 기하급수적으로 줄어들기 때문에 비트코인의 저량 대 유량의 비율은 치솟는다.
2020년에는 금을 따라잡고, 그 4년 후인 2024년에는 다시 두 배 만큼 금을 능가하게 된다.
이러한 반감기는 총 32번 발생한다.
지수의 힘 덕분에 연간 비트코인 채굴량은 50년 후에는 100비트코인 미만으로, 75년 후에는 1비트코인 미만으로 떨어지게 된다.
블록 보상은 2140년 쯤 고갈되어 비트코인 생산은 사실상 중단될 것이다. 
이것은 매우 긴 게임이며, 당신이 이 글을 읽고 있는 현재는 아직 초기 단계에 해당한다.

\begin{figure}
	\includegraphics{assets/images/soundness-over-time.png}
	\caption{금과 비교한 비트코인의 저량 대 유량의 비율}
	\label{fig:soundness-over-time}
\end{figure}

\paragraph{}
\begin{comment}	
	As bitcoin approaches infinite stock to flow ratio it will be the
	soundest money in existence. Infinite soundness is hard to beat.
\end{comment}
비트코인의 건전성 즉 저량 대 유량의 비율은 무한대에 수렴하기 때문에
비트코인은 현존하는 가장 건전한 화폐가 될 것이다.
무한한 건전성을 누구도 이길 수 없다.

\paragraph{}
\begin{comment}	
	Viewed through the lens of economics, Bitcoin's \textit{difficulty adjustment}
	is probably its most important component. How hard it is to mine bitcoin depends
	on how quickly new bitcoins are mined.\footnote{It actually depends on how
		quickly valid blocks are found, but for our purposes, this is the same thing as
		\enquote{mining bitcoins} and will be so for the next 120 years.} It is the dynamic
	adjustment of the network's mining difficulty which enables us to predict its
	future supply.
\end{comment}
경제학 관점에서 볼 때, 비트코인의 난이도 조정이 아마 가장 중요한 요소일 것이다.
비트코인 채굴이 얼마나 어려운지는 신규 비트코인을 얼마나 빨리 채굴하느냐에 달려있다.\footnote{실제로는 얼마나 유효한 블록을 빨리 찾느냐에 달려있다. 
	하지만 행위의 목적을 표현하기 위해 \enquote{비트코인을 채굴한다}로 표현한다. 
	이 채굴은 향후 120년 동안 계속된다.}
네트워크의 채굴 난이도는 동적으로 결정되기 때문에 이를 통해 미래의 공급량을 예측할 수 있다.

\paragraph{}
\begin{comment}	
	The simplicity of the difficulty adjustment algorithm might distract
	from its profundity, but the difficulty adjustment truly is a revolution
	of Einsteinian proportions. It ensures that, no matter how much or how
	little effort is spent on mining, Bitcoin's controlled supply won't be
	disrupted. As opposed to every other resource, no matter how much
	energy someone will put into mining bitcoin, the total reward will not
	increase.
\end{comment}
난이도 조정 알고리즘의 단순함이 비트코인의 심오함을 흐트러뜨릴 수 있다. 
하지만, 비트코인 난이도 조정은 아인슈타인에 버금가는 혁명과도 유사하다.
이 방법은 채굴에 아무리 많은 노력을 들이더라도 비트코인의 통제된 공급이 중단되지 않도록 보장한다.
다른 모든 자원과는 달리 누군가 비트코인 채굴에 아무리 많은 에너지를 투입하여도 총 보상은 증가하지 않는다.

\paragraph{}
\begin{comment}	
	Just like $E=mc^2$ dictates the universal speed limit in our universe,
	Bitcoin's difficulty adjustment dictates the \textbf{universal money limit}
	in Bitcoin.
\end{comment}
$E=mc^2$가 우주의 보편적 속도의 한도를 결정하듯, 
비트코인의 난이도 조정은 비트코인 전체 수량의 보편적 한도를 결정한다.

\paragraph{}
\begin{comment}	
	If it weren't for this difficulty adjustment, all bitcoins would have been mined
	already. If it weren't for this difficulty adjustment, Bitcoin probably wouldn't
	have survived in its infancy. It is what secures the network in its reward era.
	It is what ensures a steady and fair distribution\footnote{Dan Held,
		\textit{Bitcoin's Distribution was Fair}~\cite{distribution-was-fair}} of new
	bitcoin. It is the thermostat which regulates Bitcoin's monetary policy.
\end{comment}
난이도 조정이 없었다면 비트코인은 이미 고갈되었을 것이고, 초기 단계에서 살아남기 어려웠을 것이다.
난이도 조정 덕분에 채굴 보상이 존재하는 동안 네트워크가 안전하게 보호되며,
신규 발행된 비트코인이 안정적이고 공정하게 분배된다.\footnote{Dan Held,
	\textit{Bitcoin’s Distribution was Fair}~\cite{distribution-was-fair}}
난이도 조정은 비트코인 통화 정책을 규제하는 조절 장치이다.

\begin{comment}	
	Einstein showed us something novel: no matter how hard you push an
	object, at a certain point you won't be able to get more speed out of
	it. Satoshi also showed us something novel: no matter how hard you dig
	for this digital gold, at a certain point you won't be able to get more
	bitcoin out of it. For the first time in human history, we have a
	monetary good which, no matter how hard you try, you won't be able to
	produce more of.
\end{comment}
아인슈타인은 물체를 아무리 세게 밀어도 특정 지점에 이르면 더 빠른 속도를 낼 수 없다는 새로운 사실을 보여주었다. 
사토시 또한 우리에게 새로운 것을 보여주었다. 
이 디지털 금을 아무리 열심히 채굴하여도 특정 시점이 되면 더이상 새로운 비트코인을 얻을 수 없는 것이다. 
인류 역사상 처음으로 우리는 아무리 노력해도 더 이상 생산할 수 없는 금전적 재화를 갖게 되었다.

%\paragraph{Bitcoin taught me that sound money is essential.}
\paragraph{비트코인은 나에게 건전 화폐가 꼭 필요하다는 것을 가르쳐주었다.}

% ---
%
% #### Through the Looking-Glass
%
% - [Bitcoin's Energy Consumption: A Shift in Perspective][much energy]
%
% #### Down the Rabbit Hole
%
% - [The Ethics of Money Production][Jörg Guido Hülsmann] by Jörg Guido Hülsmann
% - [Mineral Commodity Summaries 2019][last few years] by the United States Geological Survey
% - [Bitcoin’s Distribution was Fair][fair distribution] by Dan Held
% - [Bitcoin's Controlled Supply][algorithmically controlled] on the Bitcoin Wiki
% - [Money Supply][how much money there is], [Speed of Light][universal speed limit] on Wikipedia
%
% <!-- Internal -->
% [much energy]: 
%
% [Fr. Bernard W. Dempsey, S.J.]: https://www.jstor.org/stable/29769582
% [Jörg Guido Hülsmann]: https://mises.org/sites/default/files/The%20Ethics%20of%20Money%20Production_2.pdf
% [stopped publishing]: https://www.federalreserve.gov/Releases/h6/discm3.htm
% [last few years]: https://minerals.usgs.gov/minerals/pubs/mcs/2018/mcs2018.pdf
% [my calculations]: https://www.wolframalpha.com/input/?i=volume+of+190000+metric+tons+gold+%2F+olympic+swimming+pool+volume
% [fair distribution]: https://blog.picks.co/bitcoins-distribution-was-fair-e2ef7bbbc892
%
% <!-- Bitcoin Wiki -->
% [algorithmically controlled]: https://en.bitcoin.it/wiki/Controlled_supply
%
% <!-- Wikipedia -->
% [how much money there is]: https://en.wikipedia.org/wiki/Money_supply
% [universal speed limit]: https://en.wikipedia.org/wiki/Speed_of_light#Upper_limit_on_speeds
% [alice]: https://en.wikipedia.org/wiki/Alice%27s_Adventures_in_Wonderland
% [carroll]: https://en.wikipedia.org/wiki/Lewis_Carroll

\part{기술}
\label{ch:technology}
\chapter*{기술}

\begin{comment}
	\begin{chapquote}{Lewis Carroll, \textit{Alice in Wonderland}}
		\enquote{Now, I'll manage better this time} she said to herself, and began by taking
		the little golden key, and unlocking the door that led into the garden
	\end{chapquote}
\end{comment}
\begin{chapquote}{루이스 캐롤, \textit{이상한 나라의 앨리스}}
	\enquote{이번에는 더 잘할 수 있어.} 앨리스는 혼자 중얼거리며 작은 황금 열쇠를 꺼내 정원으로 이어지는 문을 열었다.
\end{chapquote}

\begin{comment}
	Golden keys, clocks which only work by chance, races to solve
	strange riddles, and builders that don't have faces or names. What sounds like
	fairy tales from Wonderland is daily business in the world of Bitcoin.
\end{comment}
황금 열쇠, 어쩌다 우연히 작동되는 시계, 이상한 수수께끼를 풀기 위한 경쟁, 얼굴도 이름도 없는 건축가.
이상한 나라의 동화처럼 들리는 이것들은 비트코인 세계에서 일상적인 일이다.

\begin{comment}
	As we explored in Chapter~\ref{ch:economics}, large parts of the current financial system are systematically broken. 
	Like Alice, we can only hope to manage better this time. 
	But, thanks to a pseudonymous inventor, we have incredibly sophisticated technology to support us this time around: Bitcoin.
\end{comment}
경제학~\ref{ch:economics}에서 살펴보았듯, 우리는 현재 금융 시스템이 상당히 손상되었다는 것을 안다.
앨리스처럼 우리도 이번에는 더 잘 관리할 수 있길 바랄 뿐이다.
그러나 이번에 다른 점은 익명의 발명가 덕분에 우리를 도와줄 수 있는 놀랍도록 정교한 기술인 비트코인을 갖게 되었다는 것이다. 

\begin{comment}
	Solving problems in a radically decentralized and adversarial environment
	requires unique solutions. What would otherwise be trivial problems to solve
	are everything but in this strange world of nodes. Bitcoin relies on strong
	cryptography for most solutions, at least if looked at through the lens of
	technology. Just how strong this cryptography is will be explored in one of the
	following lessons.
\end{comment}
철저히 탈중앙화되고 적대적인 환경에서 발생하는 문제를 해결하려면 특별한 해결책이 필요하다.
이 노드들로 이루어진 이상한 나라에서 사소한 문제들을 해결하지 못한다면 모든 것이 문제가 되고 만다.
기술적 관점에서 볼 때 비트코인은 대부분의 해결책을 강력한 암호학에 의존하고 있다. 
암호학이 얼마나 강력한지는 차차 살펴볼 예정이다.

\begin{comment}
	Cryptography is what Bitcoin uses to remove trust in authorities.
	Instead of relying on centralized institutions, the system relies on the final
	authority of our universe: physics. Some grains of trust still remain, however.
	We will examine these grains in the second lesson of this chapter.
\end{comment}
비트코인은 기관에 대한 신뢰를 제거하기 위해 암호학 기술을 사용한다.
비트코인은 중앙 기관에 의존하는 대신 만물의 법칙인 물리학에 의존한다.
그럼에도 여전히 일부의 신뢰 문제가 남아있다.
이에 대해서는 이 장의 두 번째 교훈에서 살펴볼 것이다.

~

\begin{samepage}
	Part~\ref{ch:technology} -- Technology:
	
	\begin{enumerate}
		\setcounter{enumi}{14}
		\item 숫자의 강력함
		\item \enquote{신뢰하지 말고 검증하라}의 고찰
		\item 시간을 알려주는 데는 노력이 필요하다.
		\item 천천히 움직여라, 아무것도 깨뜨리지 않도록.
		\item 프라이버시는 죽지 않았다.
		\item 사이퍼펑크는 코드를 작성한다.
		\item 비트코인의 미래에 대한 비유
	\end{enumerate}
\end{samepage}

\begin{comment}
	The last couple of lessons explore the ethos of technological development in
	Bitcoin, which is arguably as important as the technology itself. Bitcoin is not
	the next shiny app on your phone. It is the foundation of a new economic
	reality, which is why Bitcoin should be treated as nuclear-grade financial
	software.
\end{comment}
마지막 두 개의 교훈에서는 기술 자체만큼이나 중요한 비트코인의 개발 정신을 살펴본다.
비트코인은 당신의 휴대전화에서 구동될 빛나는 차세대 앱이 아니다.
비트코인은 새로운 경제 현상의 토대이다. 
그렇기 때문에, 비트코인을 핵폭탄급 금융 소프트웨어로 취급해야 한다.

\begin{comment}
	Where are we in this financial, societal, and technological revolution? 
	
	Networks and technologies of the past may serve as metaphors for Bitcoin's future, which
	are explored in the last lesson of this chapter.
\end{comment}
우리는 이 금융 혁명 혹은 사회 혁명이자 기술 혁명의 어디쯤 와 있을까?
과거의 네트워크와 기술들은 비트코인의 미래를 예측할 수 있는 실마리가 될 수 있다. 이에 대해서는 마지막 교훈에서 살펴본다. 

\begin{comment}
	Once more, strap in and enjoy the ride. Like all exponential technologies, we
	are about to go parabolic.
\end{comment}
다시 한번 안전벨트를 매고 즐겨보자. 다른 모든 기하급수적으로 성장하는 기술들처럼, 우리도 곧 포물선을 그리게 될 것이다.

\chapter{숫자의 강력함}
\label{les:15}

\begin{chapquote}{루이스 캐롤, \textit{이상한 나라의 앨리스}}
	\enquote{어디 보자, 
		4 곱하기 5는 20, 
		4 곱하기 6은 13, 
		4 곱하기 7은 14, 
		어쩌지! 이런 속도로는 20까지 가지도 못하겠어!}
\end{chapquote}

\begin{comment}
	Numbers are an essential part of our everyday life. Large numbers,
	however, aren't something most of us are too familiar with. The largest
	numbers we might encounter in everyday life are in the range of
	millions, billions, or trillions. We might read about millions of people
	in poverty, billions of dollars spent on bank bailouts, and trillions of
	national debt. Even though it's hard to make sense of these headlines,
	we are somewhat comfortable with the size of those numbers.
\end{comment}
숫자는 일상 생활에서 꼭 필요한 일부이다. 
그러나 우리 대다수는 큰 숫자에 익숙하지 않다. 
우리가 일상생활에서 접하는 큰 숫자는 수백만(millions), 수십억(billions) 또는 수조(trillions) 정도이다.
수백만 명의 빈곤층, 구제 금융을 위해 지출된 수십억 달러, 수조 달러에 달하는 국가 부채 등이 우리가 접할 수 있는 큰 수이다.
비록 이런 헤드라인을 정확히 이해하기는 어렵지만, 이런 숫자는 어느 정도 편안하게 받아들일 수 있다.

\begin{comment}
	Although we might seem comfortable with billions and trillions, our
	intuition already starts to fail with numbers of this magnitude. Do you
	have an intuition how long you would have to wait for a
	million/billion/trillion seconds to pass? If you are anything like me,
	you are lost without actually crunching the numbers.
\end{comment}
하지만 수십억, 수조에 달하는 수를 읽는 것엔 익숙할지 몰라도, 이 숫자가 어느 정도의 규모인지 직관적으로 이해가 안되기 시작한다.
백만/십억/조 초가 지날 때까지 얼마나 기다려야 하는지 파악이 되는가?
나처럼 당신도 일일이 계산해 보지 않고는 감이 안 올 것이다. 

\begin{comment}
	Let's take a closer look at this example: the difference between each is an
	increase by three orders of magnitude: $10^6$, $10^9$, $10^{12}$. Thinking about
	seconds is not very useful, so let's translate this into something we can wrap
	our head around:
\end{comment}
이 예를 좀 더 자세히 살펴보자. $10^6$, $10^9$, $10^{12}$ 이 세 숫자는 각각 세 자릿수 씩 증가한다.  
초 단위로 생각하는 것엔 그닥 익숙하지 않으니, 머리를 싸매고서라도 이해할 수 있는 수준으로 바꿔보겠다.

\begin{itemize}
	\item $10^6$: 100만 초 전은 $1 \frac{1}{2}$주 전에 해당한다.
	\item $10^9$: 10억 초 전은 32년 전에 해당한다.
	\item $10^{12}$: 1조 초 전에는 맨하탄이 빙하 속에 묻혀있었다.\footnote{1조($10^{12}$)초는 $31,710$ 년이다. 마지막 최대 빙하기(Last  Glacial Maximum)는 $33,000$ 년 전이다.~\cite{wiki:LGM}}
\end{itemize}

\begin{figure}
	\includegraphics{assets/images/xkcd-1225.png}
	\caption{약 1조 초 전 각 도시의 빙하 두께 (출처: xkcd 1225)}
	\label{fig:xkcd-1225}
\end{figure}

\begin{comment}
	As soon as we enter the beyond-astronomical realm of modern
	cryptography, our intuition fails catastrophically. Bitcoin is built
	around large numbers and the virtual impossibility of guessing them.
	These numbers are way, way larger than anything we might encounter in
	day-to-day life. Many orders of magnitude larger. Understanding how
	large these numbers truly are is essential to understanding Bitcoin as a
	whole.
\end{comment}
현대 암호학에서 사용하는 천문학적 수에 도달하면 수를 파악하는 우리의 직관은 거의 불능 상태가 된다.
비트코인은 천문학적으로 큰 숫자와 이를 추측하는 것이 불가능하다는 점을 활용한다.
비트코인이 사용하는 숫자는 우리가 일상생활에서 접할 수 있는 그 어떤 숫자보다 훨씬 더 크다. 자릿수가 훨씬 크단 뜻이다.
비트코인을 전체적으로 이해하기 위해서 이 숫자가 실제로 얼마나 큰지 이해하는 것이 중요하다.

\begin{comment}
	Let's take SHA-256\footnote{SHA-256 is part of the SHA-2 family of cryptographic
		hash functions developed by the NSA.~\cite{wiki:sha2}}, one of the hash
	functions\footnote{Bitcoin uses SHA-256 in its block hashing
		algorithm.~\cite{btcwiki:block-hashing}} used in Bitcoin, as a concrete example.
	It is only natural to think about 256 bits as \enquote{two hundred fifty-six,} which
	isn't a large number at all. However, the number in SHA-256 is talking about
	orders of magnitude --- something our brains are not well-equipped to deal with.
\end{comment}
비트코인에서 사용하는 해시 함수인 SHA-256\footnote{SHA-256은 NSA에서 개발한. SHA-2 계열의 암호학 해시함수이다.~\cite{wiki:sha2}}
을 예로 들어보자.\footnote{SHA-256은 비트코인에서 블록 해시 알고리즘에 사용된다.~\cite{btcwiki:block-hashing}} 
여기서 사용되는 숫자 256비트를 단순히 "이백오십육"이라고 생각한다면, 이는 결코 큰 숫자가 아니다.
하지만, SHA-256에서 256은 자릿수를 의미한다. 이 숫자는 우리의 두뇌에서 처리할 수 없는 규모이다.

\begin{comment}
	While bit length is a convenient metric, the true meaning of 256-bit
	security is lost in translation. Similar to the millions ($10^6$) and
	billions ($10^9$) above, the number in SHA-256 is about orders of magnitude
	($2^{256}$).
\end{comment}
비트(bit)의 길이는 편리한 측정 기준이지만, 숫자를 자릿수로 변환하는 과정에서 256비트 보안성의 진정한 의미가 손상된다.
수백만($10^6$) 및 수십억($10^9$)과 유사하게 SHA-256에서 사용하는 숫자의 자리수는 엄청나게 크다.($2^{256}$).
\begin{comment}
	So, how strong is SHA-256, exactly?
\end{comment}
그렇다면, SHA-256은 정확히 얼마나 강력할까?

\begin{comment}
	\begin{quotation}\begin{samepage}
			\enquote{SHA-256 is very strong. It's not like the incremental step from MD5
				to SHA1. It can last several decades unless there's some massive
				breakthrough attack.}
			\begin{flushright} -- Satoshi Nakamoto\footnote{Satoshi Nakamoto, in a reply to questions about SHA-256 collisions. \cite{satoshi-sha256}}
	\end{flushright}\end{samepage}\end{quotation}
\end{comment}
\begin{quotation}\begin{samepage}
		\enquote{SHA-256은 매우 강력하다. MD5에서 SHA1로의 점진적 증가와는 차원이 다르다.
			대규모의 획기적인 해킹 공격이 발생하지 않는 한 수십 년 동안 끄떡없을 것이다.}
		\begin{flushright} --사토시 나카모토\footnote{Satoshi Nakamoto, in a reply to questions about SHA-256 collisions. \cite{satoshi-sha256}}
\end{flushright}\end{samepage}\end{quotation}


%Let's spell things out. $2^{256}$ equals the following number:
$2^{256}$이 얼마나 큰 수인지 적어보자.
\footnote{
	저자: 원문에는 다음과 같이 표현되어 있다. \\
	115 quattuorvigintillion 792 trevigintillion 89 duovigintillion 237
	unvigintillion 316 vigintillion 195 novemdecillion 423 octodecillion 570
	septendecillion 985 sexdecillion 8 quindecillion 687 quattuordecillion 907
	tredecillion 853 duodecillion 269 undecillion 984 decillion 665 nonillion
	640 octillion 564 septillion 39 sextillion 457 quintillion 584 quadrillion 7
	trillion 913 billion 129 million 639 thousand 936.}

\begin{comment}
	\begin{quotation}\begin{samepage}
			115 quattuorvigintillion 792 trevigintillion 89 duovigintillion 237
			unvigintillion 316 vigintillion 195 novemdecillion 423 octodecillion 570
			septendecillion 985 sexdecillion 8 quindecillion 687 quattuordecillion 907
			tredecillion 853 duodecillion 269 undecillion 984 decillion 665 nonillion
			640 octillion 564 septillion 39 sextillion 457 quintillion 584 quadrillion 7
			trillion 913 billion 129 million 639 thousand 936.
	\end{samepage}\end{quotation}
\end{comment}
\begin{quotation}
	\begin{samepage}
		115,792,089,237,316,195,423,570,985,008,687,907,853,
		269,984,665,640,564,039,457,584,007,913,129,639,936
	\end{samepage}
\end{quotation}

\begin{comment}
	That's a lot of nonillions! Wrapping your head around this number is
	pretty much impossible. There is nothing in the physical universe to
	compare it to. It is far larger than the number of atoms in the
	observable universe. The human brain simply isn't made to make sense of
	it.
\end{comment}
엄청나게 큰 숫자이다. 이 숫자를 직관적으로 이해하는 것은 불가능하다. 
우리가 사는 우주의 어떤 숫자와도 비교할 수 없다.
관측할 수 있는 우주의 원자 수보다 훨씬 많다. 
인간의 두뇌로 인지하는 것이 불가능하다.

\newpage

\begin{comment}
	One of the best visualizations of the true strength of SHA-256 is a video by
	Grant Sanderson. Aptly named \textit{\enquote{How secure is 256 bit
			security?}}\footnote{Watch the video at \url{https://youtu.be/S9JGmA5_unY}} it
	beautifully shows how large a 256-bit space is. Do yourself a favor and take the
	five minutes to watch it. As all other \textit{3Blue1Brown} videos it is not
	only fascinating but also exceptionally well made. Warning: You might fall down
	a math rabbit hole.
\end{comment}
그랜트 샌더슨의 영상에서 SHA-256의 진정한 강력함이 훌륭하게 시각화되었다.
\enquote{256비트 보안은 얼마나 안전한가?(How secure is 256 bit
	security?)}\footnote{\url{https://youtu.be/S9JGmA5_unY}}에서는
256비트의 공간이 얼마나 큰지 아름답게 보여준다. 잠시 시간을 내어 5분 간 감상해보자.
3Blue1Brown에서 만든 다른 영상도 그렇지만 이 영상도 매우 흥미롭고 잘 만들어졌다.
단, 수학 토끼굴에 빠질 수 있으니 주의하라.

\begin{comment}
	\begin{figure}
		\includegraphics{assets/images/youtube-vid-inverted.png}
		\caption{Illustration of SHA-256 security. Original graphic by Grant Sanderson aka 3Blue1Brown.}
		\label{fig:youtube-vid-inverted}
	\end{figure}
\end{comment}
\begin{figure}
	\includegraphics{assets/images/youtube-vid-inverted.png}
	\caption{그랜트 샌더슨 영상의 SHA-256을 설명하기 위한 일러스트}
	\label{fig:youtube-vid-inverted}
\end{figure}

\begin{comment}
	Bruce Schneier~\cite{web:schneier} used the physical limits of computation to put this
	number into perspective: even if we could build an optimal computer,
	which would use any provided energy to flip bits perfectly~\cite{wiki:landauer}, build a
	Dyson sphere\footnote{A Dyson sphere is a hypothetical megastructure that completely encompasses a star and captures a large percentage of its power output.~\cite{wiki:dyson}} around our sun, and let it run for 100 billion billion
	years, we would still only have a $25\%$ chance to find a needle in a
	256-bit haystack.
\end{comment}
브루스 슈나이어(Bruce Schneier)~\cite{web:schneier}는 물리적으로 계산하는데에 한계가 있음을 활용하여 이 숫자를 입체적으로 표현했다.
태양 주위에 다이슨 구체\footnote{다이슨 구체는 행성을 완전히 덮어서 엄청난 전력을 생산하는 가상의 거대 구조물이다.~\cite{wiki:dyson}}
를 구축한 에너지원을 사용해 1000억 년 동안 작동시킬 수 있는 '비트를 완벽하게 뒤집는 최적의 컴퓨터'를 만들었다 해도 256비트 건초 더미에서 바늘을 찾을 확률은 $25\%$에 불과하다.

\begin{comment}
	\begin{quotation}\begin{samepage}
			\enquote{These numbers have nothing to do with the technology of the devices;
				they are the maximums that thermodynamics will allow. And they
				strongly imply that brute-force attacks against 256-bit keys will be
				infeasible until computers are built from something other than matter
				and occupy something other than space.}
			\begin{flushright} -- Bruce Schneier\footnote{Bruce Schneier, \textit{Applied Cryptography} \cite{bruce-schneier}}
	\end{flushright}\end{samepage}\end{quotation}
\end{comment}
\begin{quotation}\begin{samepage}
		\enquote{이 숫자는 장치들의 기술과는 아무런 관련이 없다. 
			이는 열역학이 허용하는 최대값이다.
			그리고 컴퓨터가 완전히 다른 물질로 만들어지고, 완전히 다른 차원의 공간을 차지하기 전까지...
			256비트 키에 대한 무차별 대입 공격은 불가능할 것임을 강력하게 암시하는 것이다.}
		\begin{flushright} -- 브루스 슈나이어\footnote{Bruce Schneier, \textit{응용 암호학(Applied Cryptography)} \cite{bruce-schneier}}
\end{flushright}\end{samepage}\end{quotation}

\begin{comment}
	It is hard to overstate the profoundness of this. Strong cryptography
	inverts the power-balance of the physical world we are so used to.
	Unbreakable things do not exist in the real world. Apply enough force,
	and you will be able to open any door, box, or treasure chest.
\end{comment}
이 심오함은 아무리 강조해도 지나치지 않다.  
강력한 암호학은 우리가 상상할 수 있는 물리 세계의 힘의 균형을 뒤집는다. 
현실 세계에서 깨지지 않는 것들은 존재하지 않는다. 
충분한 힘을 가하면, 그것이 문, 박스, 보물 상자인지를 불문하고 어떤 것이든 열 수 있다. 

\begin{comment}
	Bitcoin's treasure chest is very different. It is secured by strong
	cryptography, which does not give way to brute force. And as long as the
	underlying mathematical assumptions hold, brute force is all we have.
	Granted, there is also the option of a global \$5 wrench attack (Figure~\ref{fig:xkcd-538})
	But torture won't work for all bitcoin addresses, and the cryptographic
	walls of bitcoin will defeat brute force attacks. Even if you come at it
	with the force of a thousand suns. Literally.
\end{comment}
하지만 비트코인 보물 상자는 매우 다르다. 
이 상자는 어떠한 무차별 대입 공격에도 굴복하지 않는 강력한 암호학으로 보호된다. 
우리의 수학적 상식 안에서는 무차별 대입 공격 외엔 딱히 다른 공격 방법도 존재하지 않는다. 
물론 \$5 렌치 공격(Figure~\ref{fig:xkcd-538})도 방법일 순 있다. 
그러나 이러한 사람을 고문하는 종류의 공격은 비트코인 주소 공격과는 다르며 비트코인 암호의 벽은 무차별 공격을 물리칠 것이다. 
문자 그대로 천 개의 태양의 힘으로 공격하더라도 말이다. 

\begin{figure}
	\centering
	\includegraphics[width=8cm]{assets/images/xkcd-538.png}
	\caption{\$5 렌치공격 (출처: xkcd 538)}
	\label{fig:xkcd-538}
\end{figure}

\begin{comment}
	This fact and its implications were poignantly summarized in the call
	to cryptographic arms: \textit{\enquote{No amount of coercive force will ever solve
	a math problem.}
\end{comment}
이 사실과 그 의미는 암호화 무기의 소명(A Call to Cryptographic Arms)에 통렬하게 요약되어 있다.
\enquote{어떠한 물리력도 수학 문제를 풀 수 없다.}

\begin{comment}
	\begin{quotation}\begin{samepage}
			\enquote{It isn't obvious that the world had to work this way. But somehow the
				universe smiles on encryption.}
			\begin{flushright} -- Julian Assange\footnote{Julian Assange, \textit{A Call to Cryptographic Arms} \cite{call-to-cryptographic-arms}}
	\end{flushright}\end{samepage}\end{quotation}
\end{comment}
\begin{quotation}\begin{samepage}
		\enquote{세상이 이런 식으로 돌아가야 했는지는 잘 모르겠다. 하지만 어쨌든 우주는 암호화를 향해 미소짓는다.}
		\begin{flushright} -- 줄리안 어산지\footnote{Julian Assange, \textit{암호화 무기의 소명(A Call to Cryptographic Arms)} \cite{call-to-cryptographic-arms}}
\end{flushright}\end{samepage}\end{quotation}

\begin{comment}
	Nobody yet knows for sure if the universe's smile is genuine or not. It
	is possible that our assumption of mathematical asymmetries is wrong and
	we find that P actually equals NP \cite{wiki:pnp}, or we find surprisingly quick
	solutions to specific problems \cite{wiki:discrete-log} which we currently assume to be hard.
	If that should be the case, cryptography as we know it will cease to
	exist, and the implications would most likely change the world beyond
	recognition.
\end{comment}
우주의 미소가 진짜인지 아닌지 아직 누구도 확신할 수 없다.  
수학적 비대칭성에 대한 우리의 가정이 잘못되어 P가 실제로 NP\cite{wiki:pnp}라는 것을 발견하거나, 
이산 로그 문제\cite{wiki:discrete-log}를 놀라울 정도로 빠르게 계산할 수 있는 방법을 찾아낼 수도 있다. 
만약 그런 일이 발생한다면 우리가 알고 있는 암호학은 더 이상 사용할 수 없게 될 것이며, 
우리가 상상할 수 없을 정도로 세상이 변화될 가능성이 크다. 

\begin{quotation}\begin{samepage}
		\enquote{Vires in Numeris} = \enquote{Strength in Numbers}\footnote{\textit{Vires in Numeris} 는 bitcointalk 사용자인 \textit{epii}~\cite{epii}에 의해 
			비트코인 모토로 처음 제안되었다.}
\end{samepage}\end{quotation}

\begin{comment}
	\textit{Vires in numeris} is not only a catchy motto used by bitcoiners. The
	realization that there is an unfathomable strength to be found in
	numbers is a profound one. Understanding this, and the inversion of
	existing power balances which it enables changed my view of the world
	and the future which lies ahead of us.
\end{comment}
숫자의 힘(Vires in numeris)은 비단 비트코이너 만을 위한 모토가 아니다.
숫자에 헤아릴 수 없는 강력한 힘이 있음을 깨닫는 것은 심오한 것이다.
이것을 이해하고 이를 통해 기득권의 균형을 뒤집을 수 있음이 우리 앞에 놓여있다는 사실은
이 세계와 우리 앞에 놓인 미래에 대한 나의 시각을 바꾸어 놓았다.

\begin{comment}
	One direct result of this is the fact that you don't have to ask anyone for permission to participate in Bitcoin. 
	There is no page to sign up, no company in charge, no government agency to send application forms to.
	Simply generate a large number and you are pretty much good to go. 
	The central authority of account creation is mathematics. And God only knows who is in charge of that.
\end{comment}
이러한 숫자의 힘으로 비트코인에 참여하기 위해 누구에게도 허가받을 필요가 없다.
가입 페이지도 없고, 담당 회사도 없고, 신청서를 보낼 정부 기관도 없다.
간단히 큰 숫자를 하나 생성하면 거의 모든 작업이 완료된다. 
수학이 비트코인 계정 생성의 권한을 가진다. 그 책임자가 누구인지는 신만이 알고 있다.

\begin{figure}
	\includegraphics{assets/images/elliptic-curve-examples.png}
	\caption{타원곡선의 예시 (출처: Emmanuel Boutet)}
	\label{fig:elliptic-curve-examples}
\end{figure}

\begin{comment}
	Bitcoin is built upon our best understanding of reality. While there are
	still many open problems in physics, computer science, and mathematics,
	we are pretty sure about some things. That there is an asymmetry between
	finding solutions and validating the correctness of these solutions is
	one such thing. That computation needs energy is another one. In other
	words: finding a needle in a haystack is harder than checking if the
	pointy thing in your hand is indeed a needle or not. And finding the
	needle takes work.
\end{comment}
비트코인은 현실 세계에 대한 최선의 이해를 토대로 구축되었다.
물리학, 컴퓨터 과학, 수학에는 아직 해결되지 않은 문제가 많지만, 그럼에도 몇 가지 확실한 것이 있다.
그 중 하나는 어떤 문제의 해결책을 찾는 것과 그 해결책의 정확성을 검증하는 것이 비대칭적이라는 것이다.
계산에 에너지가 소요된다는 것도 확실하다.
즉, 건초 더미에서 바늘을 찾는 것은 내가 찾은 뾰족한 것이 실제 바늘인지 아닌지 확인하는 것보다 더 어려운 문제라는 것이다. 
그리고 바늘을 찾는 노력이 필요하다는 것도 말이다.

\begin{comment}
	The vastness of Bitcoin's address space is truly mind-boggling. The
	number of private keys even more so. It is fascinating how much of our
	modern world boils down to the improbability of finding a needle in an
	unfathomably large haystack. I am now more aware of this fact than ever.
\end{comment}
비트코인 주소 공간은 정말 놀라울 정도로 방대하다. 개인 키의 수는 훨씬 더 크다.
헤아릴 수 없을 만큼 큰 건초 더미에서 바늘을 찾는 것이 불가능하다라는 사실에 많은 사람들이 매료되고 있다.
나는 이제 그 어느 때보다 이 사실을 명확하게 알고 있다.

%\paragraph{Bitcoin taught me that there is strength in numbers.}
\paragraph{비트코인은 나에게 숫자는 강력하다는 것을 가르쳐주었다.}


% ---
%
% #### Down the Rabbit Hole
%
% - [How secure is 256 bit security?]["How secure is 256 bit security?"] by 3Blue1Brown
% - [Block Hashing Algorithm][hash functions] on the Bitcoin Wiki
% - [Last Glacial Maximum][thick layer of ice], [SHA-2][SHA-256], [Dyson Sphere][Dyson sphere], [Landauer's Principle][flip bits perfectly] [P versus NP][P actually equals NP], [Discrete Logarithm][specific problems] on Wikipedia
%
% [thick layer of ice]: https://en.wikipedia.org/wiki/Last_Glacial_Maximum
% [xkcd \#1125]: https://xkcd.com/1225/
% [SHA-256]: https://en.wikipedia.org/wiki/SHA-2
% [hash functions]: https://en.bitcoin.it/wiki/Block_hashing_algorithm
% ["How secure is 256 bit security?"]: https://www.youtube.com/watch?v=S9JGmA5_unY
% [Bruce Schneier]: https://www.schneier.com/
% [flip bits perfectly]: https://en.wikipedia.org/wiki/Landauer%27s_principle#Equation
% [Dyson sphere]: https://en.wikipedia.org/wiki/Dyson_sphere
% [2]: https://books.google.com/books?id=Ok0nDwAAQBAJ&pg=PT316&dq=%22These+numbers+have+nothing+to+do+with+the+technology+of+the+devices;%22&hl=en&sa=X&ved=0ahUKEwjXttWl8YLhAhUphOAKHZZOCcsQ6AEIKjAA#v=onepage&q&f=false
% [wrench attack]: https://xkcd.com/538/
% [call to cryptographic arms]: https://cryptome.org/2012/12/assange-crypto-arms.htm
% [P actually equals NP]: https://en.wikipedia.org/wiki/P_versus_NP_problem#P_=_NP
% [specific problems]: https://en.wikipedia.org/wiki/Discrete_logarithm#Cryptography
% [3Blue1Brown]: https://twitter.com/3blue1brown
%
% <!-- Wikipedia -->
% [alice]: https://en.wikipedia.org/wiki/Alice%27s_Adventures_in_Wonderland
% [carroll]: https://en.wikipedia.org/wiki/Lewis_Carroll

\chapter{ \enquote{신뢰하지 말고 검증하라}의 고찰}
\label{les:16}

\begin{comment}
	\begin{chapquote}{Lewis Carroll, \textit{Alice in Wonderland}}
		\enquote{Now for the evidence,} said the King, \enquote{and then the sentence.}
	\end{chapquote}
\end{comment}
\begin{chapquote}{루이스 캐롤, \textit{이상한 나라의 앨리스}}
	\enquote{이제 증거가 있으니,} 왕은 말했다, \enquote{선고를 내리겠다.}
\end{chapquote}

\begin{comment}
	Bitcoin aims to replace, or at least provide an alternative to,
	conventional currency. Conventional currency is bound to a centralized
	authority, no matter if we are talking about legal tender like the US
	dollar or modern monopoly money like Fortnite's V-Bucks. In both
	examples, you are bound to trust the central authority to issue, manage
	and circulate your money. Bitcoin unties this bound, and the main issue
	Bitcoin solves is the issue of \textit{trust}.
\end{comment}
비트코인은 기존 통화를 대체하거나 대안을 제공하는 것을 목표로 한다.
미국의 달러나 포트나이트의 V-Bucks와 같은 기존의 현대 독점 화폐는 
중앙 집중형 권한에 의해 묶여있다.
두 화폐 모두 당신은 발행, 관리, 유통하는 중앙 기관들을 신뢰해야 한다.
비트코인은 이러한 한계를 극복하였다.
그리고 비트코인은 화폐에 있어서 가장 중요한 문제인 신뢰의 문제를 해결하였다.


%\begin{quotation}\begin{samepage}
%\enquote{The root problem with conventional currency is all the trust that's
	%required to make it work. [...] What is needed is an electronic
	%payment system based on cryptographic proof instead of trust}
%\begin{flushright} -- Satoshi Nakamoto\footnote{Satoshi Nakamoto, official Bitcoin announcement~\cite{bitcoin-announcement} and whitepaper~\cite{whitepaper}}
%\end{flushright}\end{samepage}\end{quotation}

\begin{quotation}\begin{samepage}
		\enquote{기존 화폐의 문제는 화폐가 동작하는 데 필요한 모든 종류의 신뢰이다. [...] 
			암호학 기반의 전자 결제 시스템에서 필요한 것은 신뢰가 아닌 증명이다.}
		\begin{flushright} -- 사토시 나카모토\footnote{Satoshi Nakamoto, official Bitcoin announcement~\cite{bitcoin-announcement} and whitepaper~\cite{whitepaper}}
\end{flushright}\end{samepage}\end{quotation}

\begin{comment}
	Bitcoin solves the problem of trust by being completely decentralized,
	with no central server or trusted parties. Not even trusted \textit{third}
	parties, but trusted parties, period. When there is no central
	authority, there simply \textit{is} no-one to trust. Complete decentralization
	is the innovation. It is the root of Bitcoin's resilience, the reason
	why it is still alive. Decentralization is also why we have mining,
	nodes, hardware wallets, and yes, the blockchain. The only thing you
	have to \enquote{trust} is that our understanding of mathematics and physics
	isn't totally off and that the majority of miners act honestly (which
	they are incentivized to do).
\end{comment}
비트코인은 중앙 서버나 신뢰 당사자들 없이 완전히 탈중앙화되어 신뢰 문제를 해결한다.
제삼자는 물론 신뢰 당사자도 필요 없다. 이상이다.
중앙 권한이 없다면, 믿어야 할 사람도 없다.
완전한 탈중앙화는 혁신이다. 
그것이 비트코인 유연성의 핵심이며, 비트코인이 여전히 살아 숨 쉬는 이유이다.
이러한 탈중앙화 덕분에 우리는 채굴이 가능하고, 노드, 하드웨어 지갑, 블록체인을 직접 가질 수 있다. 
우리가 \enquote{신뢰}해야 할 것은 수학, 물리학의 법칙과 
다수 채굴자가 인센티브를 얻기 위해 정직하게 행동한다는 것이다.

\begin{comment}
	While the regular world operates under the assumption of \textit{\enquote{trust,
			but verify,}} Bitcoin operates under the assumption of \textit{\enquote{don't
			trust, verify.}} Satoshi made the importance of removing trust very clear in
	both the introduction as well as the conclusion of the Bitcoin whitepaper.
\end{comment}
보통의 세상에서는 \enquote{신뢰하되 검증하라. (trust, but verify)}를 가정하지만,
비트코인은 \enquote{신뢰하지 말고 검증하라. (don't trust, verify)}를 가정한다.
사토시는 비트코인 백서의 서론과 결론 모두에서 신뢰 제거의 중요성을 분명하게 밝히고 있다.

\begin{quotation}\begin{samepage}
		\enquote{결론: 신뢰에 의존하지 않고 전자 거래를 할 수 있는 시스템을 제안한다.}
		\begin{flushright} -- 사토시 나카모토\footnote{Satoshi Nakamoto, the Bitcoin whitepaper~\cite{whitepaper}}
\end{flushright}\end{samepage}\end{quotation}

\begin{comment}
	Note that \textit{without relying on trust} is used in a very specific context
	here. We are talking about trusted third parties, i.e. other entities
	which you trust to produce, hold, and process your money. It is assumed,
	for example, that you can trust your computer.
\end{comment}
신뢰에 의존하지 않는 점은 특정한 맥락에서 사용되기 때문에 유의해야 한다.
우리는 신뢰할 수 있는 제삼자, 즉 돈을 발행, 보유, 처리를 위한 주체에 관해 이야기하고 있다.
예를 들어 당신의 컴퓨터를 신뢰할 수 있다고 가정해 보자.

\begin{comment}
	As Ken Thompson showed in his Turing Award lecture, trust is an
	extremely tricky thing in the computational world. When running a
	program, you have to trust all kinds of software (and hardware) which,
	in theory, could alter the program you are trying to run in a malicious
	way. As Thompson summarized in his \textit{Reflections on Trusting Trust}:
	\enquote{The moral is obvious. You can't trust code that you did not totally
		create yourself.}~\cite{trusting-trust}
\end{comment}
켄 톰슨(Ken Thompson)이 튜링 어워드(Turing Award) 강의에서 보여주었듯이 컴퓨터 세계에서 신뢰는 매우 까다로운 문제이다.
프로그램을 실행할 때는 악의적인 방식으로 변경된 모든 종류의 소프트웨어와 하드웨어를 신뢰해야 한다.
신뢰하는 신뢰의 고찰(Reflections on Trusting Trust)에서 톰슨은
\enquote{도덕은 명확하다. 스스로 직접 작성한 코드가 아니면 신뢰하기 어렵다.}\cite{trusting-trust}
라고 말하고 있다.

\begin{figure}
	\includegraphics{assets/images/ken-thompson-hack.png}
	\caption{켄 톰슨의 논문 `신뢰하는 신뢰의 고찰'의 발췌}
	\label{fig:ken-thompson-hack}
\end{figure}

\begin{comment}
	Thompson demonstrated that even if you have access to the source code,
	your compiler --- or any other program-handling program or
	hardware --- could be compromised and detecting this backdoor would be
	very difficult. Thus, in practice, a truly \textit{trustless} system does not
	exist. You would have to create all your software \textit{and} all your
	hardware (assemblers, compilers, linkers, etc.) from scratch, without
	the aid of any external software or software-aided machinery.
\end{comment}
톰슨은 코드에 대한 접근 권한이 있더라도 컴파일러(또는 기타 처리 프로그램, 하드웨어)
가 손상될 수 있으며 이 백도어를 감지하기가 매우 어렵다고 말한다. 
따라서 신뢰가 필요 없는 시스템은 존재하지 않는다.
당신은 외부 소프트웨어나 지원 도구 없이 모든 소프트웨어와 하드웨어(어셈블러, 컴파일러, 링커 등)를 만들어야 한다.

\begin{quotation}\begin{samepage}
		\enquote{온전히 나의 힘으로 사과파이를 만들고 싶다면 먼저 우주를 발명해야 한다.}
		\begin{flushright} -- 칼 세이건\footnote{Carl Sagan, \textit{Cosmos} \cite{cosmos}}
\end{flushright}\end{samepage}\end{quotation}

\begin{comment}
	The Ken Thompson Hack is a particularly ingenious and hard-to-detect backdoor,
	so let's take a quick look at a hard-to-detect backdoor which works without
	modifying any software. Researchers found a way to compromise security-critical
	hardware by altering the polarity of silicon
	impurities.~\cite{becker2013stealthy} Just by changing the physical properties
	of the stuff that computer chips are made of they were able to compromise a
	cryptographically secure random number generator. Since this change can't be
	seen, the backdoor can't be detected by optical inspection, which is one of the
	most important tamper-detection mechanism for chips like these.
\end{comment}
켄 톰슨은 소프트웨어를 수정하지 않고 백도어에서 은밀하게 동작하는 해킹이 가능하다고 말한다.
실리콘 불순물의 극성을 변경하여 보안에 중요한 하드웨어를 위조하는 방식이다\cite{becker2013stealthy}.
컴퓨터 반도체를 구성하는 물질의 물리적 속성을 변경하는 것만으로도 암호학적으로 안전한 난수 생성기를 손상시킬 수 있었다.
이 백도어 공격은 눈으로 볼 수 없기 때문에 가장 강력한 해킹 방지 메커니즘 중의 하나인 광학 검사로도 감지할 수 없다.

\begin{figure}
	\includegraphics{assets/images/stealthy-hardware-trojan.png}
	%\caption{Stealthy Dopant-Level Hardware Trojans by Becker, Regazzoni, Paar, Burleson}
	\caption{은밀한 도판트 레벨의 하드웨어 트로이 목마 바이러스}
	\label{fig:stealthy-hardware-trojan}
\end{figure}

\begin{comment}
	Sounds scary? Well, even if you would be able to build everything from
	scratch, you would still have to trust the underlying mathematics. You
	would have to trust that \textit{secp256k1} is an elliptic curve without
	backdoors. Yes, malicious backdoors can be inserted in the mathematical
	foundations of cryptographic functions and arguably this has already
	happened at least once.~\cite{wiki:Dual_EC_DRBG} There are good reasons to be paranoid, and the
	fact that everything from your hardware, to your software, to the
	elliptic curves used can have backdoors~\cite{wiki:backdoors} are some of them.
\end{comment}
무섭게 들리는가? 처음부터 모든 것을 구축하더라도 기본 수학은 여전히 신뢰해야 한다.
secp256k1은 백도어가 없는 타원 곡선임을 믿어야 한다.
그렇다. 수학적 기반에 악의적인 백도어를 심어 암호학 함수를 공격하는 것이 가능하다. 
이러한 공격은 적어도 한 번 이상 발생했을 것이다\cite{wiki:Dual_EC_DRBG}.
하드웨어에서 소프트웨어, 타원곡선에 이르기까지의 모든 것이 백도어\cite{wiki:backdoors}를 가질 수 있다는 사실은
편집증을 유발한다.


\begin{quotation}\begin{samepage}
		\enquote{신뢰하지 말고 검증하라(Don’t trust. Verify).}
		\begin{flushright} -- 모든 비트코이너들
\end{flushright}\end{samepage}\end{quotation}

\begin{comment}
	The above examp과과es should illustrate that \textit{trustless} computing is
	utopic. Bitcoin is probably the one system which comes closest to this
	utopia, but still, it is \textit{trust-minimized} --- aiming to remove trust
	wherever possible. Arguably, the chain-of-trust is neverending, since
	you will also have to trust that computation requires energy, that P
	does not equal NP, and that you are actually in base reality and not
	imprisoned in a simulation by malicious actors.
\end{comment}
위의 예시는 신뢰가 필요 없는 컴퓨터가 유토피아라는 것을 나타낸다.
비트코인은 아마도 이 유토피아에 가장 가까운 시스템일 것이다.
아직 완벽하다고 말할 수 없으나 적어도 비트코인은 신뢰를 최소화하거나 제거하는 것을 목표로 하고 있다.
분명한 것은 신뢰의 사슬은 끝이 없다는 점이다.
계산에는 에너지가 필요하다는 것과 P가 NP와 같지 않으며, 
악의적인 참여자에게 휘둘리지 않고 있다는 것, 이 모든 것을 믿어야 하기 때문이다.

\begin{comment}
	Developers are working on tools and procedures to minimize any remaining trust
	even further. For example, Bitcoin developers created
	Gitian\footnote{\url{https://gitian.org/}}, which is a software distribution
	method to create deterministic builds. The idea is that if multiple developers
	are able to reproduce identical binaries, the chance of malicious tampering is
	reduced. Fancy backdoors aren't the only attack vector. Simple blackmail or
	extortion are real threats as well. As in the main protocol, decentralization is
	used to minimize trust.
\end{comment}
개발자들은 여전히 남아있는 신뢰를 최소화하기 위한 도구 및 절차를 위한 작업을 하고 있다.
그 예로 비트코인 개발자들은 결정론적 빌드를 만드는 소프트웨어 배포 기법인 Gitian\footnote{\url{https://gitian.org/}}을 개발하였다.
이 기법의 핵심 아이디어는 여러 개발자가 동일한 바이너리를 재현할 수 있으면 악의적 변조의 가능성이 줄어든다는 것이다.
백도어만이 공격 방법은 아니다. 단순한 협박이나 갈취도 공격이 될 수 있다.
메인 프로토콜처럼 이 프로젝트에서도 탈중앙화를 통해 신뢰를 최소화한다.

\begin{comment}
	Various efforts are being made to improve upon the chicken-and-egg problem of
	bootstrapping which Ken Thompson's hack so brilliantly pointed
	out~\cite{web:bootstrapping}. One such effort is
	Guix\footnote{\url{https://guix.gnu.org}} (pronounced \textit{geeks}), which
	uses functionally declared package management leading to bit-for-bit
	reproducible builds by design. The result is that you don't have to trust any
	software-providing servers anymore since you can verify that the served binary
	was not tampered with by rebuilding it from scratch. Recently, a
	pull-request was merged to integrate Guix into the Bitcoin build process.\footnote{See PR 15277 of \texttt{bitcoin-core}: \\ \url{https://github.com/bitcoin/bitcoin/pull/15277}}
\end{comment}
켄 톰슨의 해킹이 지적한 닭이 먼저냐 달걀이 먼저냐에 대한 부트스트래핑 문제를 개선하는 노력도 이루어지고 있다\cite{web:bootstrapping}.
그 노력 중 하나로 기능적으로 선언된 패키지를 관리하여 설계에 따라 비트 단위로 재현이 가능한 빌드를 제공하는 Guix(geeks로 발음)\footnote{\url{https://guix.gnu.org}}이 그것이다.
바이너리가 처음부터 다시 빌드하여 변조되지 않았음을 확인할 수 있어서 더 이상 소프트웨어에서 제공하는 서버를 신뢰할 필요가 없다.
최근에 Guix는 비트코인 빌드 프로세스에 통합되기 위해 머지되었다\footnote{PR 15277 of \texttt{bitcoin-core}: \\ \url{https://github.com/bitcoin/bitcoin/pull/15277}}.

\begin{figure}
	\includegraphics{assets/images/guix-bootstrap-dependencies.png}
	%\caption{Which came first, the chicken or the egg?}
	\caption{닭이 먼저냐? 달걀이 먼저냐?}
	\label{fig:guix-bootstrap-dependencies}
\end{figure}

\begin{comment}
	Luckily, Bitcoin doesn't rely on a single algorithm or piece of
	hardware. One effect of Bitcoin's radical decentralization is a
	distributed security model. Although the backdoors described above are
	not to be taken lightly, it is unlikely that every software wallet,
	every hardware wallet, every cryptographic library, every node
	implementation, and every compiler of every language is compromised.
	Possible, but highly unlikely.
\end{comment}
다행히도 비트코인은 하나의 알고리즘과 하드웨어에 의존하지 않는다.
비트코인의 급진적인 탈중앙화 효과 중 하나는 분산 보안 모델이다. 
백도어를 무시할 수는 없지만 모든 비트코인 소프트웨어 지갑, 하드웨어 지갑, 
암호화 라이브러리, 노드의 구현과 컴파일러가 손상될 가능성은 없다.
가능하지만, 가능성이 거의 없다.

\begin{comment}
	Note that you can generate a private key without relying on any computational
	hardware or software. You can flip a coin~\cite{antonopoulos2014mastering} a
	couple of times, although depending on your coin and tossing style this source
	of randomness might not be sufficiently random. There is a reason why storage
	protocols like Glacier\footnote{\url{https://glacierprotocol.org/}} advise to
	use casino-grade dice as one of two sources of entropy.
\end{comment}
특정 컴퓨터 하드웨어나 소프트웨어에 의존하지 않고도 개인키를 생성할 수 있다.
동전을 몇 번 던져서 개인키를 생성할 수 있지만\cite{antonopoulos2014mastering},
동전을 던지는 스타일에 따라 무작위성이 충분하지 않을 수 있다.
글래시어(Glacier)\footnote{\url{https://glacierprotocol.org/}}와 같은 스토리지 프로토콜에서는 
개인키 생성 시 카지노 수준의 주사위를 사용하도록 조언한다.

\begin{comment}
	Bitcoin forced me to reflect on what trusting nobody actually entails.
	It raised my awareness of the bootstrapping problem, and the implicit
	chain-of-trust in developing and running software. It also raised my
	awareness of the many ways in which software and hardware can be
	compromised고고
\end{comment}
비트코인은 아무도 믿지 않는 것이 가능하다는 것을 보여주었다.
비트코인은 부트스트래핑 문제와 소프트웨어 개발 및 실행에 있어 암묵적인 신뢰에 대한 인식을 높였다.
그리고 소프트웨어나 하드웨어가 손상될 수 있는 여러 가지 방법에 대한 인식을 높였다.

\paragraph{비트코인은 나에게 신뢰하지 말고 검증하라고 가르쳤다.}

% ---
%
% #### Down the Rabbit Hole
%
% - [The Bitcoin whitepaper][Nakamoto] by Satoshi Nakamoto
% - [Reflections on Trusting Trust][\textit{Reflections on Trusting Trust}] by Ken Thompson
% - [51% Attack][majority] on the Bitcoin Developer Guide
% - [Bootstrapping][bootstrapping], Guix Manual
% - [Secp256k1][secp256k1] on the Bitcoin Wiki
% - [ECC Backdoors][backdoors], [Dual EC DRBG][has already happened] on Wikipedia
%
% [Emmanuel Boutet]: https://commons.wikimedia.org/wiki/User:Emmanuel.boutet
% [\textit{Reflections on Trusting Trust}]: https://www.archive.ece.cmu.edu/~ganger/712.fall02/papers/p761-thompson.pdf
% [found a way]: https://scholar.google.com/scholar?hl=en&as_sdt=0%2C5&q=Stealthy+Dopant-Level+Hardware+Trojans&btnG=
% [Gitian]: https://gitian.org/
% [bootstrapping]: https://www.gnu.org/software/guix/manual/en/html_node/Bootstrapping.html
% [Guix]: https://www.gnu.org/software/guix/
% [pull-request]: https://github.com/bitcoin/bitcoin/pull/15277
% [flip a coin]: https://github.com/bitcoinbook/bitcoinbook/blob/develop/ch04.asciidoc#private-keys
% [Glacier]: https://glacierprotocol.org/
% [secp256k1]: https://en.bitcoin.it/wiki/Secp256k1
% [majority]: https://bitcoin.org/en/developer-guide#term-51-attack
%
% <!-- Wikipedia -->
% [backdoors]: https://en.wikipedia.org/wiki/Elliptic-curve_cryptography#Backdoors
% [has already happened]: https://en.wikipedia.org/wiki/Dual_EC_DRBG
% [Carl Sagan]: https://en.wikipedia.org/wiki/Cosmos_%28Carl_Sagan_book%29
% [alice]: https://en.wikipedia.org/wiki/Alice%27s_Adventures_in_Wonderland
% [carroll]: https://en.wikipedia.org/wiki/Lewis_Carroll

\chapter{시간을 알려주는 데는 노력이 필요하다.}
\label{les:17}

%\begin{chapquote}{Lewis Carroll, \textit{Alice in Wonderland}}
%\enquote{Dear, dear! I shall be too late!}
\begin{chapquote}{루이스 캐롤, \textit{이상한 나라의 앨리스}}
	\enquote{이런, 이런! 너무 늦겠어!}
\end{chapquote}

\begin{comment}
	It is often said that bitcoins are mined because thousands of computers
	work on solving \textit{very complex} mathematical problems. Certain problems
	are to be solved, and if you compute the right answer, you \enquote{produce} a
	bitcoin. While this simplified view of bitcoin mining might be easier to
	convey, it does miss the point somewhat. Bitcoins aren't produced or
	created, and the whole ordeal is not really about solving particular
	math problems. Also, the math isn't particularly complex. What is
	complex is \textit{telling the time} in a decentralized system.
\end{comment}
흔히들 수천 대의 컴퓨터가 매우 복잡한 수학 문제를 풀면서 비트코인이 채굴된다고 말한다.
특정 문제를 풀어야하고, 정답을 계산해 내면 비트코인을 얻는다.
비트코인 채굴이라는 이 단순화된 관점은 전달하기 쉽지만 핵심을 놓칠 수 있는 표현이다.
비트코인은 생산되거나 생성되는 것이 아니며, 이 전체 과정은 특정 수학 문제를 푸는 것과는 관련이 없다.
또한 채굴에 활용되는 수학은 그다지 복잡하지 않다. 
진짜 복잡한 것은 탈중앙화된 시스템에서 시간을 알려주는 것이다.

\begin{comment}
	As outlined in the whitepaper, the proof-of-work system (aka mining) is
	a way to implement a distributed timestamp server.
\end{comment}
비트코인 백서에 설명된 대로 작업증명(proof-of-work, 일명 마이닝)은 탈중앙화된 타임스탬프\footnote{역주: 특정 시각을 기록하는 문자열} 서버를 구현하는 방법이다.

\begin{figure}
	\includegraphics{assets/images/bitcoin-whitepaper-timestamp-wide.png}
	%  \caption{Excerpts from the whitepaper. Did someone say timechain?}
	\caption{백서에서 발췌. 누가 타임체인이라고 했습니까?}
	\label{fig:bitcoin-whitepaper-timestamp-wide}
\end{figure}

\begin{comment}
	When I first learned how Bitcoin works I also thought that proof-of-work
	is inefficient and wasteful. After a while, I started to shift my
	perspective on Bitcoin's energy consumption~\cite{gigi:energy}. It seems that
	proof-of-work is still widely misunderstood today, in the year 10 AB
	(after Bitcoin).
\end{comment}
나는 비트코인을 처음 접했을 때, 작업증명을 비효율적이고 낭비라고 생각했다.
하지만 시간이 지나면서 비트코인의 에너지 소비\cite{gigi:energy}에 대한 관점이 바뀌기 시작했다.
비트코인이 세상에 나온 후 10년이 지난 지금도 여전히 작업증명은 오해받고 있는 것 같다.

\begin{comment}
	Since the problems to be solved in proof-of-work are made up, many
	people seem to believe that it is \textit{useless} work. If the focus is purely
	on the computation, this is an understandable conclusion. But Bitcoin
	isn't about computation. It is about \textit{independently agreeing on the
		order of things.}
\end{comment}
풀어야 할 문제가 인위적으로 만들어졌다는 이유로 작업증명을 쓸데없는 것이라고 생각하는 사람들이 많은 것 같다.
단순히 계산에만 초점을 맞춘다면 그렇게 판단할 수도 있다.
하지만 비트코인은 무언가를 계산하기 위해 작업증명을 사용한 것이 아니다.
작업증명은 독립적인 주체들이 어떤 일의 순서를 정하기 위해 합의에 이르는 과정이다.

\begin{comment}
	Proof-of-work is a system in which everyone can validate what happened
	and in what order it happened. This independent validation is what leads
	to consensus, an individual agreement by multiple parties about who owns
	what.
\end{comment}
작업증명은 모든 참여자가 발생한 사건과 사건의 순서를 검증하는 시스템이다.
이러한 독립적 검증을 통해 누가 무엇을 소유하는지 여러 당사자로부터 개별적 합의를 이끌어낸다. 

\begin{comment}
	In a radically decentralized environment, we don't have the luxury of absolute
	time. Any clock would introduce a trusted third party, a central point in the
	system which had to be relied upon and could be attacked. \enquote{Timing is the root
		problem,} as Grisha Trubetskoy points out~\cite{pow-clock}. And Satoshi
	brilliantly solved this problem by implementing a decentralized clock via a
	proof-of-work blockchain. Everyone agrees beforehand that the chain with the
	most cumulative work is the source of truth. It is per definition what actually
	happened. This agreement is what is now known as Nakamoto consensus.
\end{comment}
근원적으로 탈중앙화된 환경에서 절대적 시간이라는 것은 사치다.
모든 시계는 신뢰할 수 있는 제3자, 즉 중앙 시스템을 도입해야 하고 이는 언제든 해킹될 수 있다.
그리샤 트루베츠코이(Grisha Trubetskoy)가 지적한 것처럼 \enquote{시간이 근본적 문제}이다.\cite{pow-clock}
사토시는 작업증명 블록체인을 통해 탈중앙형 시계를 구현함으로써 이 문제를 훌륭히 해결했다.
가장 많은 작업이 누적된 체인이 '진실의 원천'이라는 것은 누구나 사전에 동의하는 사실이다.
실제로 비트코인이 구동되고 있는 것은 이 정의에 따른 것이다.
이 합의를 나카모토 합의라 한다.

\begin{quotation}\begin{samepage}
		%\enquote{The network timestamps transactions by hashing them into an ongoing
			%chain which serves as proof of the sequence of events witnessed}
		%\begin{flushright} -- Satoshi Nakamoto\footnote{Satoshi Nakamoto, the Bitcoin whitepaper~\cite{whitepaper}}
		\enquote{네트워크는 제출된 거래의 증거를 해싱함으로써 동작 중인 체인에 트랜잭션 타임스탬프를 기록한다.}
		\begin{flushright} -- Satoshi Nakamoto\footnote{사토시 나카모토, 비트코인 백서~\cite{whitepaper}}
\end{flushright}\end{samepage}\end{quotation}

\begin{comment}
	Without a consistent way to tell the time, there is no consistent way to
	tell before from after. Reliable ordering is impossible. As mentioned
	above, Nakamoto consensus is Bitcoin's way to consistently tell the
	time. The system's incentive structure produces a probabilistic,
	decentralized clock, by utilizing both greed and self-interest of
	competing participants. The fact that this clock is imprecise is
	irrelevant because the order of events is eventually unambiguous and can
	be verified by anyone.
\end{comment}
시간을 알 수 있는 일관된 방법이 없으면 사건의 선후를 구분할 방법이 없다.
신뢰할 수 있는 순서를 만드는 것이 불가능한 것이다.
위에서 언급했듯이, 나카모토 합의는 시간을 일관되게 알려주는 비트코인만의 방식이다.
시스템의 인센티브 구조는 경쟁 참여자의 탐욕과 이기심을 모두 활용하여 확률적이고 탈중앙화된 시계를 만들어 낸다.
이 시계는 정확하지 않다. 하지만 사건의 순서가 모호하지 않고 누구나 선후관계를 확인할 수 있기 때문에 시각의 정확성은 중요하지 않다.

\begin{comment}
	Thanks to proof-of-work, both the work \textit{and} the validation of the work
	are radically decentralized. Everyone can join and leave at will, and
	everyone can validate everything at all times. Not only that, but
	everyone can validate the state of the system \textit{individually}, without
	having to rely on anyone else for validation.
\end{comment}
작업증명 덕분에 작업과 작업의 유효성 검증이 근본적으로 탈중앙화된다. 
누구나 마음대로 참여를 결정할 수 있으며, 모든 참여자가 항상 모든 것을 검증할 수 있다. 
뿐만 아니라 다른 사람에게 의존하지 않고 시스템 상태를 스스로 검증할 수 있다.


\begin{comment}
	Understanding proof-of-work takes time. It is often counter-intuitive,
	and while the rules are simple, they lead to quite complex phenomena.
	For me, shifting my perspective on mining helped. Useful, not useless.
	Validation, not computation. Time, not blocks.
\end{comment}
작업증명을 이해하는 데는 시간이 걸린다.
작업증명의 어떤 부분은 직관에 어긋나고, 규칙은 단순하지만 상당히 복잡한 현상을 초래한다. 
채굴에 대한 관점을 바꾸니 작업증명이 이해되었다. 쓸모없는 것이 아니고 유용한 것이다. 
계산하는 것이 아니라 검증하는 것이다. 블록이 아닌 시간을 말이다.


%\paragraph{Bitcoin taught me that telling the time is tricky, especially if you are decentralized.}
\paragraph{비트코인은 나에게 시간을 알려주는 것이 탈중앙화된 경우엔 특히나 까다롭다는 것을 가르쳐주었다.}

% ---
%
% #### Through the Looking-Glass
%
% - [Bitcoin's Energy Consumption: A shift in perspective][energy]
%
% #### Down the Rabbit Hole
%
% - [Blockchain Proof-of-Work Is a Decentralized Clock][points out] by Gregory Trubetskoy
% - [The Anatomy of Proof-of-Work][pow-anatomy] by Hugo Nguyen
% - [PoW is efficient][pow-efficient] by Dan Held
% - [Mining][bw-mining], [Controlled supply][bw-supply] on the Bitcoin Wiki
%
% [points out]: https://grisha.org/blog/2018/01/23/explaining-proof-of-work/
% [energy]: 
% [whitepaper]: https://bitcoin.org/bitcoin.pdf
%
% [pow-efficient]: https://blog.picks.co/pow-is-efficient-aa3d442754d3
% [pow-anatomy]: https://bitcointechtalk.com/the-anatomy-of-proof-of-work-98c85b6f6667
% [bw-mining]: https://en.bitcoin.it/wiki/Mining
% [bw-supply]: https://en.bitcoin.it/wiki/Controlled_supply
%
% <!-- Wikipedia -->
% [alice]: https://en.wikipedia.org/wiki/Alice%27s_Adventures_in_Wonderland
% [carroll]: https://en.wikipedia.org/wiki/Lewis_Carroll

\chapter{천천히 움직여라 아무것도 깨뜨리지 않도록}
\label{les:18}

\begin{chapquote}{루이스 캐롤, \textit{이상한 나라의 앨리스}}
	%So the boat wound slowly along, beneath the bright summer-day, with its merry crew and its music of voices and laughter\ldots
	그리하여 배는 밝은 여름날 아래 천천히 나아갔다. 즐거운 선원들과 음악의 선율, 그리고 웃음 소리와 함께\ldots
\end{chapquote}

\begin{comment}
	It might be a dead mantra, but \enquote{move fast and break things} is still how
	much of the tech world operates. The idea that it doesn't matter if you
	get things right the first time is a basic pillar of the \textit{fail early,
		fail often} mentality. Success is measured in growth, so as long as you
	are growing everything is fine. If something doesn't work at first you
	simply pivot and iterate. In other words: throw enough shit against the
	wall and see what sticks.
\end{comment}
오래된 진리일지 모르겠지만, \enquote{빠르게 움직여라 무언가 깨뜨릴 정도로\footnote{역주: 페이스북의 모토}}는 기술 세계에서 여전히 통하는 방식이다.
처음부터 제대로 해내는 것이 중요하지 않다는 생각은 \textit{일찌감치 실패하고 자주 실패하라}는 식의 사고방식이다.
성공은 성장으로 측정되기 때문에 성장하고 있는 한 모든 것이 괜찮다. 
처음에 무언가가 잘 작동하지 않으면 방향을 전환하고 반복하면 된다. 
다른 말로, 똥인지 된장인지는 먹어봐야 안다는 뜻이다.

\begin{comment}
	Bitcoin is very different. It is different by design. It is different
	out of necessity. As Satoshi pointed out, e-currency has been tried
	many times before, and all previous attempts have failed because there
	was a head which could be cut off. The novelty of Bitcoin is that it is
	a beast without heads.
\end{comment}
비트코인은 사뭇 다르다. 설계가 다르다. 필요성부터 다르다. 
사토시가 말한 것처럼 전자 화폐는 이전에도 여러 번 시도되었으나, 잘려버릴 수도 있는 머리, 즉 리더가 있었기에 모두 실패하고 말았다. 
비트코인의 참신함은 이 머리가 없다는 점이다.

\begin{quotation}\begin{samepage}
	\begin{comment}
		\enquote{A lot of people automatically dismiss e-currency as a lost cause
			because of all the companies that failed since the 1990's. I hope it's
			obvious it was only the centrally controlled nature of those systems
			that doomed them.}
		\begin{flushright} -- Satoshi Nakamoto\footnote{Satoshi Nakamoto, in a reply to Sepp Hasslberger \cite{satoshi-centralized-nature}}
		\end{comment}
		\enquote{많은 사람들이 1990년대부터 시도된 전자 화폐의 실패 원인을 회사가 망했기 때문이라고 생각합니다.
			분명히 말하지만 전자 화폐 시스템이 망한 이유는 중앙에서 제어되는 특성을 가졌기 때문이었습니다.}
		\begin{flushright} -- 사토시 나카모토\footnote{Satoshi Nakamoto, in a reply to Sepp Hasslberger \cite{satoshi-centralized-nature}}
\end{flushright}\end{samepage}\end{quotation}
	
\begin{comment}
	One consequence of this radical decentralization is an inherent
	resistance to change. \enquote{Move fast and break things} does not and will
	never work on the Bitcoin base layer. Even if it would be desirable, it
	wouldn't be possible without convincing \textit{everyone} to change their ways.
	That's distributed consensus. That's the nature of Bitcoin.
\end{comment}
이러한 급진적인 탈중앙화의 결과 중 하나는 내재되어있는 변화에 대한 저항이다.
\enquote{빠르게 움직여라 무언가 깨뜨릴 정도로}의 방식은 비트코인에서는 통하지 않으며, 앞으로도 그럴 것이다.
설사 이것이 바람직하다 하더라도, 모든 사람들이 자신의 방식을 바꾸리라 설득되지 않는 한 실행 불가능할 것이다. 
이것이 바로 분산 합의이자 비트코인의 본질이다.


\begin{quotation}\begin{samepage}
	\begin{comment}
		\enquote{The nature of Bitcoin is such that once version 0.1 was released, the
			core design was set in stone for the rest of its lifetime.}
		\begin{flushright} -- Satoshi Nakamoto\footnote{Satoshi Nakamoto, in a reply to Gavin Andresen \cite{satoshi-centralized-nature}}
		\end{comment}
		\enquote{비트코인의 특성상 0.1 버전이 출시되고 나면 핵심 설계는 비트코인이 사라질 때까지 확정된 것이나 다름없습니다.}
		\begin{flushright} -- 사토시 나카모토\footnote{Satoshi Nakamoto, in a reply to Gavin Andresen \cite{satoshi-centralized-nature}}
\end{flushright}\end{samepage}\end{quotation}

\begin{comment}
	This is one of the many paradoxical properties of Bitcoin. We all came
	to believe that anything which is software can be changed easily. But
	the nature of the beast makes changing it bloody hard.
\end{comment}
이것은 비트코인의 많은 역설적 특징 중 하나이다.
우리 모두 소프트웨어는 쉽게 바꿀 수 있다고 믿게 되었다.
하지만 비트코인은 변경하기 매우 어렵다. 
	
\begin{comment}
	As Hasu beautifully shows in Unpacking Bitcoin's Social
	Contract~\cite{social-contract}, changing the rules of Bitcoin is only possible
	by \textit{proposing} a change, and consequently \textit{convincing} all users
	of Bitcoin to adopt this change. This makes Bitcoin very resilient to change,
	even though it is software.
\end{comment}
하수(Hasu)는 비트코인의 사회 계약 풀기(Unpacking Bitcoin's Social Contract)\cite{social-contract}에서
비트코인의 규칙을 변경하기 위해서는 제안을 통해서만 가능하며, 모든 사용자가 이를 채택하도록 설득해야 한다고 언급했다.
이로 인해 비트코인은 소프트웨어임에도 불구하고 변화에 매우 복원력이 강하다.

\begin{comment}
	This resilience is one of the most important properties of Bitcoin.
	Critical software systems have to be antifragile, which is what the
	interplay of Bitcoin's social layer and its technical layer guarantees.
	Monetary systems are adversarial by nature, and as we have known for
	thousands of years solid foundations are essential in an adversarial
	environment.
\end{comment}
이러한 복원력은 비트코인의 주요 특성 중 하나이다.
중요한 역할을 수행하는 소프트웨어 시스템은 사회적 계층과 기술적 계층의 상호작용이 보장하는 
안티프래질\footnote{역주: 충격을 받으면 더 강해지는 특성} 특성을 갖춰야 한다.
그동안의 화폐 시스템은 본질적으로 적대적이었다.
우리가 수천 년 동안 보았듯이 이러한 적대적 환경에서는 견고한 기반이 필수이다.
	
\begin{quotation}\begin{samepage}
	\begin{comment}
		\enquote{The rain came down, the floods came, and the winds blew, and beat on
			that house; and it didn't fall, for it was founded on the rock.}
		\begin{flushright} -- Matthew 7:24--27
		\end{comment}
		\enquote{비가 내리고 창수가 나고 바람이 불어 그 집에 부딪치되 무너지지 아니하나니 이는 주추를 반석 위에 놓은 까닭이요.}
		\begin{flushright} -- 마태복음 7:24--27
\end{flushright}\end{samepage}\end{quotation}
		
\begin{comment}
	Arguably, in this parable of the wise and the foolish builders Bitcoin
	isn't the house. It is the rock. Unchangeable, unmoving, providing the
	foundation for a new financial system.
\end{comment}
성경에 등장하는 '현명한 건축가와 어리석은 건축가' 우화에 비유한다면 비트코인은 집(house)이 아니다. 반석(rock)이다.
\footnote{역주: 집을 짓되 깊이 파고 주추를 반석 위에 놓은 사람과 같으니(he is like a man which built an house, and digged deep, and laid the foundation on a rock) (성경 누가복음 6:48)}
비트코인은 변치않고, 움직이지 않으며 새로운 금융 시스템의 토대를 제공한다.

\begin{comment}
	Just like geologists, who know that rock formations are always moving
	and evolving, one can see that Bitcoin is always moving and evolving as
	well. You just have to know where to look and how to look at it.
\end{comment}
암석층이 항상 움직이고 진화하고 있다는 것을 아는 지질학자들과 마찬가지로
비트코인도 항상 움직이고 진화하고 있다는 것을 알 수 있다.
어디를 어떻게 봐야 하는지만 알면 된다.
		
\begin{comment}
	The introduction of pay to script hash\footnote{ Pay to script hash (P2SH)
		transactions were standardised in BIP 16. They allow transactions to be sent to
		a script hash (address starting with 3) instead of a public key hash (addresses
		starting with 1).~\cite{btcwiki:p2sh}} and segregated
	witness\footnote{Segregated Witness (abbreviated as SegWit) is an implemented
		protocol upgrade intended to provide protection from transaction malleability
		and increase block capacity. SegWit separates the \textit{witness} from the list
		of inputs.~\cite{btcwiki:segwit}} are proof that Bitcoin's rules can be changed
	if enough users are convinced that adopting said change is to the benefit of the
	network. The latter enabled the development of the lightning
	network\footnote{\url{https://lightning.network/}}, which is one of the houses
	being built on Bitcoin's solid foundation. Future upgrades like Schnorr
	signatures~\cite{bip:schnorr} will enhance efficiency and privacy, as well as
	scripts (read: smart contracts) which will be indistinguishable from regular
	transactions thanks to Taproot~\cite{taproot}. Wise builders do indeed build on
	solid foundations.
\end{comment}
P2SH\footnote{Pay to script hash (P2SH)
	트랜잭션 표준은 BIP 16에 정의되어 있다. 이 표준은 공개키(1로 시작하는 주소)로 지불하는 것 대신 스크립트 해시(3으로 시작하는 주소)에 지불하는 것을 허용한다.
	.~\cite{btcwiki:p2sh}}와
세그윗(SegWit)\footnote{Segregated Witness 는
	트랜잭션의 유연성으로부터 네트워크를 보호하고 블록의 용량 효율을 늘리기위해 구현된 프로토콜 업그레이드이다.
	SegWit는 입력값에서 검증 데이터를 분리한다.~\cite{btcwiki:segwit}}의 도입은
다수의 네트워크 참여자가 해당 변경을 채택하는 것이 네트워크에 이익이 된다는 확신이 있다면 규칙을 변경할 수 있다는 것을 보여준 증거이다.
세그윗은 비트코인의 단단한 반석 위에 지어진 집 중 하나인 라이트닝 네트워크\footnote{\url{https://lightning.network/}} 개발을 가능하게 했다.
향후 슈노르 서명\cite{bip:schnorr}과 같은 업그레이드를 통해 효율성과 프라이버시가 향상될 것이며, 탭루트 덕분에 일반 트랜잭션과 구별할 수 없는 스마트 컨트랙트가 등장할 것이다.
현명한 건축가는 견고한 반석 위에 집을 짓는다.
		
\begin{comment}
	Satoshi wasn't only a wise builder technologically. He also understood
	that it would be necessary to make wise decisions ideologically.
\end{comment}
사토시는 기술적으로만 현명한 건축가가 아니었다.
그는 이념적으로도 현명한 결정이 필요하다는 것을 이해하고 있었다.

\begin{quotation}\begin{samepage}
	\begin{comment}
		\enquote{Being open source means anyone can independently review the code. If
			it was closed source, nobody could verify the security. I think it's
			essential for a program of this nature to be open source.}
		\begin{flushright} -- Satoshi Nakamoto\footnote{Satoshi Nakamoto, in a reply to SmokeTooMuch \cite{satoshi-open-source}}
		\end{comment}
		\enquote{오픈소스라는 것은 누구나 독립적으로 코드를 검토할 수 있다는 것을 뜻합니다.
			비공개 소스라면 누구도 보안성을 검증할 수 없습니다. 
			나는 이런 성격의 프로그램은 당연히 오픈소스로 공개되어야 한다고 생각합니다.}
		\begin{flushright} -- 사토시 나카모토\footnote{Satoshi Nakamoto, in a reply to SmokeTooMuch \cite{satoshi-open-source}}
\end{flushright}\end{samepage}\end{quotation}
	
\begin{comment}
	Openness is paramount to security and inherent in open source and the
	free software movement. As Satoshi pointed out, secure protocols and the
	code which implements them have to be open --- there is no security
	through obscurity. Another benefit is again related to decentralization:
	code which can be run, studied, modified, copied, and distributed freely
	ensures that it is spread far and wide.
\end{comment}
개방성은 보안에 있어 가장 중요한 요소이며 오픈소스 및 자유 소프트웨어 운동에 개방성이 내재되어 있다.
사토시가 지적했듯이 보안 프로토콜과 이를 구현하는 코드는 공개되어야 하며, 모호함으로는 보안을 확보할 수 없다.
개방성의 또 다른 장점은 탈중앙성과도 관련이 있다.
자유롭게 실행되고, 연구하며, 수정, 복사 및 배포할 수 있는 코드는 널리 확산될 수 있다.
	
	
\begin{comment}
	The radically decentralized nature of Bitcoin is what makes it move
	slowly and deliberately. A network of nodes, each run by a sovereign
	individual, is inherently resistant to change --- malicious or not. With
	no way to force updates upon users the only way to introduce changes is
	by slowly convincing each and every one of those individuals to adopt a
	change. This non-central process of introducing and deploying changes is
	what makes the network incredibly resilient to malicious changes. It is
	also what makes fixing broken things more difficult than in a
	centralized environment, which is why everyone tries not to break things
	in the first place.
\end{comment}
비트코인의 극단적인 탈중앙성으로 인해 비트코인은 느리고 신중하게 움직인다.
주권자 개인의 운영으로 구성된 노드 네트워크는 악의적이든 아니든 본질적으로 변화에 저항한다.
참여자에게 업데이트를 강제할 방법이 없기 때문에 변화시킬 수 있는 유일한 방법은
모든 개인이 변경 사항을 채택하도록 천천히 설득하는 것이다.
변경 사항을 도입하고 배포하는 이 탈중앙화된 프로세스는 악의적인 변경에 대해 놀라울 정도로 탄력적으로 대응할 수 있게 해준다.
탈중앙화된 환경에서는 중앙 집중식 환경에서보다 고장난 것을 고치는 것이 더 어렵기 때문에 모든 사람들이 애초에 비트코인을 고장내지 않으려 노력하는 것도 이유이다. 
	
%\paragraph{Bitcoin taught me that moving slowly is one of its features, not a bug.}
\paragraph{비트코인은 느리게 움직이는 것이 버그가 아닌 비트코인의 특징 중 하나라는 것을 가르쳐주었다.}

% ---
%
% #### Through the Looking-Glass
%
% - [Lesson 1: Immutability and Change][lesson1]
%
% #### Down the Rabbit Hole
%
% - [Unpacking Bitcoin's Social Contract] by Hasu
% - [Schnorr signatures BIP][Schnorr signatures] by Pieter Wuille
% - [Taproot proposal][Taproot] by Gregory Maxwell
% - [P2SH][pay to script hash], [SegWit][segregated witness] on the Bitcoin Wiki
% - [Parable of the Wise and the Foolish Builders][Matthew 7:24--27] on Wikipedia
%
% <!-- Down the Rabbit Hole -->
% [lesson1]: {{ '/bitcoin/lessons/ch1-01-immutability-and-change' | absolute_url }}
%
% [Unpacking Bitcoin's Social Contract]: https://uncommoncore.co/unpacking-bitcoins-social-contract/
% [Matthew 7:24--27]: https://en.wikipedia.org/wiki/Parable_of_the_Wise_and_the_Foolish_Builders
% [pay to script hash]: https://en.bitcoin.it/wiki/Pay_to_script_hash
% [segregated witness]: https://en.bitcoin.it/wiki/Segregated_Witness
% [lightning network]: https://lightning.network/
% [Schnorr signatures]: https://github.com/sipa/bips/blob/bip-schnorr/bip-schnorr.mediawiki#cite_ref-6-0
% [Taproot]: https://lists.linuxfoundation.org/pipermail/bitcoin-dev/2018-January/015614.html
%
% <!-- Wikipedia -->
% [alice]: https://en.wikipedia.org/wiki/Alice%27s_Adventures_in_Wonderland
% [carroll]: https://en.wikipedia.org/wiki/Lewis_Carroll

\chapter{프라이버시는 죽지 않았다.}
\label{les:19}

\begin{chapquote}
	%{Lewis Carroll, \textit{Alice in Wonderland}}
	%The players all played at once without waiting for turns, and quarrelled all
	%the while at the tops of their voices, and in a very few minutes the Queen was
	%in a furious passion, and went stamping about and shouting \enquote{off with his
		%head!} of \enquote{off with her head!} about once in a minute.
	{루이스 캐롤, \textit{이상한 나라의 앨리스}}
	선수들은 차례를 기다리지 않고 한꺼번에 플레이했고, 내내 큰 목소리로 다투었다.
	여왕은 발을 구르며 1분에 한 번씩 맹렬하게 말했다.
	\enquote{그의 목을 쳐라!}
	\enquote{그녀의 목을 쳐라!}
\end{chapquote}

\begin{comment}
	If pundits are to believed, privacy has been dead since the
	80ies\footnote{\url{https://bit.ly/privacy-is-dead}}. The pseudonymous invention
	of Bitcoin and other events in recent history show that this is not the case.
	Privacy is alive, even though it is by no means easy to escape the surveillance
	state.
\end{comment}
전문가들의 말에 의하면 80년대 이후로 프라이버시는 죽었다.\footnote{\url{https://bit.ly/privacy-is-dead}}
하지만, 비트코인의 발명과 최근의 여러 사건들은 이것이 사실이 아님을 보여준다.
감시를 벗어나는 것이 결코 쉽지 않지만, 프라이버시는 살아있다.

\begin{comment}
	Satoshi went through great lengths to cover up his tracks and conceal
	his identity. Ten years later, it is still unknown if Satoshi Nakamoto
	was a single person, a group of people, male, female, or a
	time-traveling AI which bootstrapped itself to take over the world.
	Conspiracy theories aside, Satoshi chose to identify himself to be a
	Japanese male, which is why I don't assume but respect his chosen gender
	and refer to him as \textit{he}.
\end{comment}
사토시는 자신의 흔적을 지우고 신분을 감추기 위해 부단히 노력했다.
십년이 지난 지금도 사토시 나카모토가 한 사람인지 집단인지, 
남성인지 여성인지, 아니면 세상을 정복하기 위해 미래에서 온 인공지능인지 알 수 없다.
음모론은 차치하고, 사토시는 자신을 일본 남성으로 밝히길 선택했기 때문에
그의 선택을 존중하여 나는 그를 'he'라고 지칭한다.
\begin{figure}
	\includegraphics{assets/images/nope.png}
	%\caption{I am not Dorian Nakamoto.}
	\caption{나는 도리안 나카모토가 아닙니다.}
	\label{fig:nope}
\end{figure}

\begin{comment}
	Whatever his real identity might be, Satoshi was successful in hiding
	it. He set an encouraging example for everyone who wishes to remain
	anonymous: it is possible to have privacy online.
\end{comment}
그의 진짜 정체가 무엇이든 간에 사토시는 성공적으로 정체를 숨겼다.
그는 익명을 추구하는 모든 이들에게 힘을 북돋아 줄 만한 모범을 보였다.
온라인에서도 프라이버시 보호가 가능함을 말이다.

\begin{quotation}\begin{samepage}
		%\enquote{Encryption works. Properly implemented strong crypto systems are one
			%of the few things that you can rely on.}
		\enquote{암호학은 잘 작동합니다. 제대로 구현된 강력한 암호 시스템은 우리가 신뢰할 수 있는 몇 안 되는 것 중 하나입니다.}
		\begin{flushright} -- 에드워드 스노든\footnote{Edward Snowden, answers to reader questions\cite{snowden}}
\end{flushright}\end{samepage}\end{quotation}

\begin{comment}
	Satoshi wasn't the first pseudonymous or anonymous inventor, and he won't be the
	last. Some have directly imitated this pseudonymous publication style, like Tom
	Elvis Yedusor of MimbleWimble~\cite{mimblewimble-origin} fame, while others have
	published advanced mathematical proofs while remaining completely
	anonymous~\cite{4chan-math}.
\end{comment}
사토시는 최초의 익명 발명가도 아니고, 마지막 익명 발명가도 아닐 것이다.
밈블윔블(MimbleWimble)~\cite{mimblewimble-origin}로 유명한 톰 엘비스 예두서(Tom Elvis Yedusor)처럼 사토시의 익명 출판 스타일을 모방한 사람도 있고,
완전히 익명으로 고급 수학 증명을 출판한 사람도 있다.~\cite{4chan-math}


\begin{comment}
	It is a strange new world we are living in. A world where identity is
	optional, contributions are accepted based on merit, and people can
	collaborate and transact freely. It will take some adjustment to get
	comfortable with these new paradigms, but I strongly believe that all of
	this has the potential to change the world for the better.
\end{comment}
우리는 낯선 신세계를 살고 있다. 
이 세계에서 신원은 선택 사항이고, 능력에 따라 기여가 인정되며, 사람들은 자유롭게 협력하고 거래할 수 있다.
새로운 패러다임에 익숙해지려면 약간의 적응이 필요하겠지만, 이 모든 것이 세상을 더 나은 방향으로 변화시킬 잠재력이 있다고 굳게 믿고 있다.

\begin{comment}
	We should all remember that privacy is a fundamental human right\footnote{Universal Declaration of Human Rights, \textit{Article 12}.~\cite{article12}}. And as long
	as people exercise and defend these rights the battle for privacy is far from
	over.
\end{comment}
우리 모두는 프라이버시가 기본적인 인권이라는 사실을 기억해야 한다.\cite{article12}
그리고 사람들이 이러한 권리를 행사하고 보호하는 한 프라이버시를 지키기 위한 투쟁은 끝나지 않을 것이다.

%\paragraph{Bitcoin taught me that privacy is not dead.}
\paragraph{비트코인은 프라이버시가 죽지 않았다는 것을 가르쳐주었다.}

% ---
%
% #### Down the Rabbit Hole
%
% - [Universal Declaration of Human Rights][fundamental human right] by the United Nations
% - [A lower bound on the length of the shortest superpattern][anonymous] by Anonymous 4chan Poster, Robin Houston, Jay Pantone, and Vince Vatter
%
% [since the 80ies]: https://books.google.com/ngrams/graph?content=privacy+is+dead&year_start=1970&year_end=2019&corpus=15&smoothing=3&share=&direct_url=t1%3B%2Cprivacy%20is%20dead%3B%2Cc0
% [time-traveling AI]: https://blockchain24-7.com/is-crypto-creator-a-time-travelling-ai/
% ["I am not Dorian Nakamoto."]: http://p2pfoundation.ning.com/forum/topics/bitcoin-open-source?commentId=2003008%3AComment%3A52186
% [MimbleWimble]: https://github.com/mimblewimble/docs/wiki/MimbleWimble-Origin
% [anonymous]: https://oeis.org/A180632/a180632.pdf
% [fundamental human right]: http://www.un.org/en/universal-declaration-human-rights/
%
% <!-- Wikipedia -->
% [alice]: https://en.wikipedia.org/wiki/Alice%27s_Adventures_in_Wonderland
% [carroll]: https://en.wikipedia.org/wiki/Lewis_Carroll

\chapter{사이퍼펑크는 코드를 작성한다.}
\label{les:20}

\begin{chapquote}
	%{Lewis Carroll, \textit{Alice in Wonderland}}
	%\enquote{I see you're trying to invent something.}
	{루이스 캐롤, \textit{이상한 나라의 앨리스}}
	\enquote{네가 무엇을 발명하려고 하는지 보고있어.}
\end{chapquote}

\begin{comment}
	Like many great ideas, Bitcoin didn't come out of nowhere. It was made
	possible by utilizing and combining many innovations and discoveries in
	mathematics, physics, computer science, and other fields. While
	undoubtedly a genius, Satoshi wouldn't have been able to invent Bitcoin
	without the giants on whose shoulders he was standing on.
\end{comment}
많은 훌륭한 아이디어가 그렇듯, 
비트코인은 갑자기 나온 것이 아니다.
비트코인은 수학, 물리학, 컴퓨터 과학 및 기타 분야의 많은 혁신과 발견을 활용하고 결합하여 발명되었다.
사토시는 의심할 여지 없이 천재이지만 위대한 거인들의 도움이 없었다면 비트코인을 발명할 수 없었을 것이다.

\begin{quotation}\begin{samepage}
		%\enquote{He who only wishes and hopes does not interfere actively with the
			%course of events and with the shaping of his own destiny.}
		%\begin{flushright} -- Ludwig von Mises\footnote{Ludwig von Mises, \textit{Human Action} \cite{human-action}}
		\enquote{희망뿐인 사람은 사건의 과정과 운명의 결정에 적극적으로 참여하지 않는다.}
		\begin{flushright} -- 루드비히 폰 미제스\footnote{Ludwig von Mises, \textit{Human Action} \cite{human-action}}
\end{flushright}\end{samepage}\end{quotation}
% > <cite>[Ludwig Von Mises]</cite>

\begin{comment}
	One of these giants is Eric Hughes, one of the founders of the cypherpunk
	movement and author of \textit{A Cypherpunk's Manifesto}. It's hard to imagine
	that Satoshi wasn't influenced by this manifesto. It speaks of many things which
	Bitcoin enables and utilizes, such as direct and private transactions,
	electronic money and cash, anonymous systems, and defending privacy with
	cryptography and digital signatures.
\end{comment}
이 거인 중 한 사람은 사이퍼펑크 운동의 창시자이자 사이퍼펑크 선언문(A Cypherpunk's Manifesto)의 저자인 
에릭 휴즈(Eric Hughes)이다.
사토시가 분명 사이퍼펑크 선언문의 영향을 받지 않았다고 생각하기 어렵다.
이 선언문은 개인 간의 직접거래, 전자화폐 및 현금, 익명 시스템, 암호화 및 디지털 서명에 기반한 프라이버시 등 
비트코인에 적용된 많은 것들을 언급하고 있다.

\begin{quotation}\begin{samepage}
		\begin{comment}
			\enquote{Privacy is necessary for an open society in the electronic age.
				[...] Since we desire privacy, we must ensure that each party to a
				transaction have knowledge only of that which is directly necessary
				for that transaction. [...]
				Therefore, privacy in an open society requires anonymous transaction
				systems. Until now, cash has been the primary such system. An
				anonymous transaction system is not a secret transaction system.
				[...]
				We the Cypherpunks are dedicated to building anonymous systems. We are
				defending our privacy with cryptography, with anonymous mail
				forwarding systems, with digital signatures, and with electronic
				money.
				Cypherpunks write code.}
		\end{comment}
		\enquote{프라이버시는 전자 시대의 열린 사회를 위해 필요하다. [...]
			우리는 프라이버시를 원하기 때문에 거래 당사자가 꼭 필요한 정보만을 갖도록 해야 한다. [...]
			따라서 열린 사회의 프라이버시를 위해 익명화된 거래 시스템이 필요하다.
			지금까지 현금은 이를 가능하게 했다. 익명 거래 시스템은 비밀 거래를 말하는 것은 아니다. [...]
			우리 사이퍼펑크는 익명 시스템 구축에 전념하고 있다.
			우리는 암호화, 익명의 메일 전송 시스템, 디지털 서명 및 전자화폐로 개인정보를 보호하고 있다.
			사이퍼펑크는 코드를 작성한다.}
		\begin{flushright} -- 에릭 휴즈\footnote{Eric Hughes, A Cypherpunk's Manifesto \cite{cypherpunk-manifesto}}
\end{flushright}\end{samepage}\end{quotation}

\begin{comment}
	Cypherpunks do not find comfort in hopes and wishes. They actively
	interfere with the course of events and shape their own destiny.
	Cypherpunks write code.
\end{comment}
사이퍼펑크는 희망만을 바라지 않는다.
사이퍼펑크는 사건의 진행 과정을 적극적으로 개입하고 운명을 결정한다.
사이퍼펑크는 코드를 작성한다.

\begin{comment}
	Thus, in true cypherpunk fashion, Satoshi sat down and started to write
	code. Code which took an abstract idea and proved to the world that it
	actually worked. Code which planted the seed of a new economic reality.
	Thanks to this code, everyone can verify that this novel system actually
	works, and every 10 minutes or so Bitcoin proofs to the world that it is
	still living.
\end{comment}
사토시는 사이퍼펑크의 방식대로 앉아서 코드를 작성하기 시작했다.
추상적인 아이디어를 가져와 실제로 작동한다는 것을 증명하기 위한 코드이다.
새로운 경제적 희망의 씨앗을 심은 코드이다.
이 코드 덕분에 모든 사람이 이 시스템이 실제로 동작한다는 것을 확인할 수 있으며
비트코인은 10분마다 자신이 살아있음을 세상에 증명한다.

\begin{figure}
	\includegraphics{assets/images/bitcoin-code-white.png}
	% \caption{Code excerpts from Bitcoin version 0.1}
	\caption{비트코인 버전 0.1의 일부}
	\label{fig:bitcoin-code-white}
\end{figure}

\begin{comment}
	To make sure that his innovation transcends fantasy and becomes reality, Satoshi
	wrote code to implement his idea before he wrote the whitepaper. He also made
	sure not to delay\footnote{\enquote{We shouldn't delay forever until every possible
			feature is done.} -- Satoshi Nakamoto~\cite{satoshi-delay}} any release forever.
	After all, \enquote{there's always going to be one more thing to do.}
\end{comment}
사토시는 백서를 작성하기 전에 이 혁신이 환상을 넘어 현실로 실현되도록 하기 위해 아이디어를 실현하는 코드를 작성하였다.
그는 어떠한 릴리즈도 지연\footnote{\enquote{우리는 모든 기능이 완성될 때까지 한 차례도 지체되어서는 안된다.} -- 사토시 나카모토~\cite{satoshi-delay}}시키지 않았다.
이렇게 말했을 뿐이다. \enquote{해야 할 일이 하나 더 늘었을 뿐입니다.}


\begin{quotation}\begin{samepage}
		%\enquote{I had to write all the code before I could convince myself that I
			%could solve every problem, then I wrote the paper.}
		\enquote{모든 문제를 해결할 수 있다는 확신을 위해 코드를 먼저 작성해야 했고, 그런 다음 백서를 작성하였다.}
		\begin{flushright} -- 사토시 나카모토 \footnote{Satoshi Nakamoto, Re: Bitcoin P2P e-cash paper \cite{satoshi-mail-code-first}}
\end{flushright}\end{samepage}\end{quotation}

\begin{comment}
	In today's world of endless promises and doubtful execution, an exercise
	in dedicated building was desperately needed. Be deliberate, convince
	yourself that you can actually solve the problems, and implement the
	solutions. We should all aim to be a bit more cypherpunk.
\end{comment}
약속이 넘쳐나고 이 약속을 지키는 것이 의심스러운 오늘날의 세계에서 
헌신적인 업적을 위한 움직임은 절실하게 필요했다.
신중한 생각과 할 수 있다는 확신이 해결책을 만들어 낼 수 있다.
우리는 모두 조금이라도 더 사이퍼펑크가 되어야 한다.

\paragraph{비트코인은 사이퍼펑크는 코드를 작성한다는 것을 가르쳐주었다.}

% ---
%
% #### Down the Rabbit Hole
%
% - [Bitcoin version 0.1.0 announcement][version 0.1.0] by Satoshi Nakamoto
% - [Bitcoin P2P e-cash paper announcement][mail-announcement] by Satoshi Nakamoto
%
% [mail-announcement]: http://www.metzdowd.com/pipermail/cryptography/2008-October/014810.html
% [Ludwig Von Mises]: https://mises.org/library/human-action-0/html/pp/613
% [version 0.1.0]: https://bitcointalk.org/index.php?topic=68121.0
% [not to delay]: https://bitcointalk.org/index.php?topic=199.msg1670#msg1670
% [6]: http://www.metzdowd.com/pipermail/cryptography/2008-November/014832.html
%
% <!-- Wikipedia -->
% [alice]: https://en.wikipedia.org/wiki/Alice%27s_Adventures_in_Wonderland
% [carroll]: https://en.wikipedia.org/wiki/Lewis_Carroll

\chapter{비트코인의 미래에 대한 비유}
\label{les:21}

\begin{chapquote}{루이스 캐롤, \textit{이상한 나라의 앨리스}}
	%\enquote{I know something interesting is sure to happen\ldots}
	\enquote{나는 흥미로운 일이 일어나리라 확신하고 있었다\ldots}
\end{chapquote}

\begin{comment}
	In the last couple of decades, it became apparent that technological
	innovation does not follow a linear trend. Whether you believe in the
	technological singularity or not, it is undeniable that progress is
	exponential in many fields. Not only that, but the rate at which
	technologies are being adopted is accelerating, and before you know it
	the bush in the local schoolyard is gone and your kids are using
	Snapchat instead. Exponential curves have the tendency to slap you in
	the face way before you see them coming.
\end{comment}
지난 수십 년 동안 기술 혁신이 선형적 추세를 따르지 않는다는 것이 명백해졌다.
기술 특이점을 믿든 믿지 않든, 많은 분야에서 기하급수적으로 발전이 이루어지고 있다는 것은 부인할 수 없는 사실이다.
뿐만 아니라, 기술이 채택되는 속도도 빨라지고 있다. 
어느새 동네 학교 운동장에 덤불은 사라졌고 아이들이 스냅챗을 사용하고 있다.
기하급수적인 곡선은 우리가 방심하는 사이 갑자기 존재감을 드러내곤 한다.

\begin{comment}
	Bitcoin is an exponential technology built upon exponential technologies.
	\textit{Our World in Data}\footnote{\url{https://ourworldindata.org/}}
	beautifully shows the rising speed of technological adoption, starting in 1903
	with the introduction of landlines (see Figure~\ref{fig:tech-adoption}).
	Landlines, electricity, computers, the internet, smartphones; all follow
	exponential trends in price-performance and adoption. Bitcoin does
	too~\cite{tech-adoption}.
\end{comment}
비트코인은 기하급수적 기술을 바탕으로 구축된 기하급수적 기술이다.
데이터로 보는 세상(Our World in Data)\footnote{\url{https://ourworldindata.org/}}에서는
1903년 유선 전화 도입을 시작으로 기술 채택의 속도가 증가하는 과정을 아름답게 보여준다.(그림 ~\ref{fig:tech-adoption})
유선전화, 전기, 컴퓨터, 인터넷, 스마트폰 등 모두 가격 대비 성능과 채택률에서 기하급수적 성장 추세를 따른다. 
비트코인도 마찬가지이다.~\cite{tech-adoption}

\begin{figure}
	\includegraphics{assets/images/tech-adoption.png}
	%\caption{Bitcoin is literally off the charts.}
	\caption{비트코인은 말그대로 차트를 벗어났다.}
	\label{fig:tech-adoption}
\end{figure}

\begin{comment}
	Bitcoin has not one but multiple network effects\footnote{Trace Mayer,
		\textit{The Seven Network Effects of Bitcoin}~\cite{7-network-effects}}, all of
	which resulting in exponential growth patterns in their respective area: price,
	users, security, developers, market share, and adoption as global money.
\end{comment}
비트코인은 여러 네트워크 효과\footnote{Trace Mayer,
	\textit{The Seven Network Effects of Bitcoin}~\cite{7-network-effects}}를 가지고 있다.
그 결과 가격, 사용자, 보안, 개발자, 시장 점유율, 글로벌 화폐로서의 채택 등 각 영역에서 기하급수적 성장 패턴을 보인다.

\begin{comment}
	Having survived its infancy, Bitcoin is continuing to grow every day in
	more aspects than one. Granted, the technology has not reached maturity
	yet. It might be in its adolescence. But if the technology is
	exponential, the path from obscurity to ubiquity is short.
\end{comment}
비트코인은 초기 단계를 넘어서 여러 측면에서 매일 성장을 거듭하고 있다.
물론 이 기술은 아직 성숙 단계에 도달하지 않았다. 아직 청소년기일 수도 있다.
하지만 기술이 기하급수적으로 발전한다면, 모호함에서 보편성으로의 전이는 순식간일 것이다.

\begin{figure}
	\includegraphics{assets/images/mobile-phone.png}
	%\caption{Mobile phone, ca 1965 vs 2019.}
	\caption{1965년과 2019년의 모바일 휴대폰}
	\label{fig:mobile-phone}
\end{figure}

\begin{comment}
	In his 2003 TED talk, Jeff Bezos chose to use electricity as a metaphor for the
	web's future.\footnote{\url{http://bit.ly/bezos-web}} All three phenomena ---
	electricity, the internet, Bitcoin --- are \textit{enabling} technologies,
	networks which enable other things. They are infrastructure to be built upon,
	foundational in nature.
\end{comment}
2003년 테드 강연에서 제프 베조스는 웹의 미래를 전기에 비유했다.\footnote{\url{http://bit.ly/bezos-web}}
전기, 인터넷, 비트코인이라는 세 가지 현상은 다른 모든 것을 가능하게하는 네트워크 기술을 실현한다.
이 세 가지 현상은 본질적으로 인프라적 성격을 갖는다.

\begin{comment}
	Electricity has been around for a while now. We take it for granted. The
	internet is quite a bit younger, but most people already take it for
	granted as well. Bitcoin is ten years old and has entered public
	consciousness during the last hype cycle. Only the earliest of adopters
	take it for granted. As more time passes, more and more people will
	recognize Bitcoin as something which simply is.\footnote{This is known as the
		\textit{Lindy Effect}. The Lindy effect is a theory that the future life expectancy
		of some non-perishable things like a technology or an idea is proportional to
		their current age, so that every additional period of survival implies a longer
		remaining life expectancy.~\cite{wiki:lindy}}
\end{comment}
전기는 오래전부터 존재해 왔다. 그래서 우리는 이를 당연하게 여긴다.
인터넷은 이보다 비교적 최근에 등장했지만 대부분의 사람들은 이미 인터넷을 당연하게 여긴다.
비트코인은 출시된 지 이제 10년이며 지난 과대광고 주기\footnote{역주: 기대감이 지나치게 높아진 시점} 동안 대중이 인식하기 시작했다.
아직은 초기 수용자들만이 비트코인을 당연하게 받아들인다.
시간이 지날수록 점점 더 많은 사람이 비트코인이 존재하는 것을 당연하게 여기게 될 것이다.\footnote{이러한 것을
	린디 효과라 한다. 린디 효과는 기술이나 아이디어처럼 썩지 않는 것의 미래 기대 수명이 현재 나이에 비례한다는 이론으로,
	생존 기간이 늘어날 때마다 남은 기대 수명도 길어진다.~\cite{wiki:lindy}}.

\begin{comment}
	In 1994, the internet was still confusing and unintuitive. Watching this old
	recording of the \textit{Today
		Show}\footnote{\url{https://youtu.be/UlJku_CSyNg}} makes it obvious that what
	feels natural and intuitive now actually wasn't back then. Bitcoin is still
	confusing and alien to most, but just like the internet is second nature for
	digital natives, spending and stacking
	sats\footnote{\url{https://twitter.com/hashtag/stackingsats}} will be second
	nature to the bitcoin natives of the future.
\end{comment}
1994년 당시, 인터넷은 여전히 혼란스럽고 직관적이지 않았다. 
투데이 쇼(Today show) 녹화 영상\footnote{\url{https://youtu.be/UlJku_CSyNg}}을 보면 
현재에는 자연스럽고 직관적이라고 느끼는 것들이 당시에는 그렇지 않았다는 것을 알 수 있다.
비트코인은 여전히 많은 사람들에게 혼란스럽고 낯설지만, 
디지털 세대에게 인터넷이 제2의 고향인 것처럼 
미래의 비트코인 세대들에게 사토시를 쌓는 것\footnote{\url{https://twitter.com/hashtag/stackingsats}}이 제2의 고향이 될 것이다.

\begin{quotation}\begin{samepage}
		\enquote{미래가 여기에 있습니다. 단지 대중화되지 않았을 뿐입니다.}
		\begin{flushright} -- 윌리엄 깁슨\footnote{William Gibson, \textit{The Science in Science Fiction(공상 과학 소설 속 과학)} \cite{william-gibson}}
\end{flushright}\end{samepage}\end{quotation}

\begin{comment}
	In 1995, about $15\%$ of American adults used the internet. Historical
	data from the Pew Research Center~\cite{pew-research} shows how the internet has woven
	itself into all our lives. According to a consumer survey by Kaspersky
	Lab~\cite{web:kaspersky}, 13\% of respondents have used Bitcoin and its clones to pay for
	goods in 2018. While payments aren't the only use-case of bitcoin, it is
	some indication of where we are in Internet time: in the early- to
	mid-90s.
\end{comment}
1995년에는 미국 인구의 약 $15\%$가 인터넷을 사용했다.
퓨 리서치 센터\cite{pew-research}의 과거 데이터는 인터넷이 우리 삶에 어떻게 스며들었는지를 보여준다.
카스퍼스키 랩\cite{web:kaspersky}의 소비자 설문 조사에 따르면 
응답자의 13\%가 2018년에 비트코인과 비트코인 유사품을 사용해 상품을 결제한 경험이 있다고 답했다.
결제만이 비트코인의 유일한 사용사례는 아니지만, 
이 지표는 인터넷 발전 단계로 보았을 때 비트코인이 90년대 초중반 어느 시점에 와 있는지를 보여준다.

\begin{comment}
	In 1997, Jeff Bezos stated in a letter to shareholders~\cite{bezos-letter} that
	\enquote{this is Day 1 for the Internet,} recognizing the great untapped
	potential for the internet and, by extension, his company. Whatever day this is
	for Bitcoin, the vast amounts of untapped potential are clear to all but the
	most casual observer.
\end{comment}
1997년 제프 베조스는 주주 서한\cite{bezos-letter}에 \enquote{오늘은 인터넷의 첫 날(Day 1) 입니다.}라고 적었다.
이 서한에 인터넷과 더 나아가 자신의 회사에 대한 엄청난 잠재력을 언급한 것이었다.
비트코인의 날이 언제가 되었든, 아직 개척되지 않은 엄청난 잠재력이 있다는 것은
비트코인을 무심코 바라보는 몇몇을 제외한 모든 사람에게 분명히 인식될 것이다.


\begin{figure}
	\includegraphics{assets/images/internet-evolution-white-dates.png}
	%  \caption{The internet, 1982 vs 2005. Source: cc-by Merit Network, Inc. and Barrett Lyon, Opte Project}
	\caption{1982년의 인터넷과 2005년의 인터넷 (출처: Merit Network)}
	\label{fig:internet-evolution-white-dates}
\end{figure}

\begin{comment}
	Bitcoin's first node went online in 2009 after Satoshi mined the \textit{genesis
		block}\footnote{The genesis block is the first block of the Bitcoin block chain.
		Modern versions of Bitcoin number it as block $0$, though very early versions
		counted it as block $1$. The genesis block is usually hardcoded into the
		software of the applications that utilize the Bitcoin block chain. It is a
		special case in that it does not reference a previous block and produces an
		unspendable subsidy. The \textit{coinbase} parameter contains, along with the
		normal data, the following text: \textit{\enquote{The Times 03/Jan/2009 Chancellor on
				brink of second bailout for banks}} \cite{btcwiki:genesis-block}} and released
	the software into the wild. His node wasn't alone for long. Hal Finney was one
	of the first people to pick up on the idea and join the network. Ten years
	later, as of this writing, more than
	$75.000$\footnote{\url{https://bit.ly/luke-nodecount}} nodes are running
	bitcoin.
\end{comment}
비트코인의 첫 번째 노드는 2009년 사토시가 소프트웨어를 공개한 후 제네시스 블록\footnote{
	제네시스 블록은 비트코인의 첫 번째 블록이다. 최신 버전의 비트코인은
	블록을 $0$으로 시작했지만, 초기 버전에서는 블록을 $1$로 계산했다.
	제네시스 블록은 일반적으로 비트코인 블록체인을 구동하는 응용프로그램의 소프트웨어에
	하드코딩 된다. 이전 블록이 없이 보상을 만들어 내는 데서 다른 블록과는 차별점이 있다.
	제네시스 블록의 코인 베이스 매개변수에는 다음 문구가 포함되어 있다.
	\enquote{The Times 03/Jan/2009 Chancellor on brink of second bailout for banks}\cite{btcwiki:genesis-block}}
을 채굴하며 온라인 상태가 되었다.
사토시 노드는 머지않아 혼자가 아니게 되었다.
할 피니는 사토시의 아이디어를 인정하고 비트코인 네트워크에 참여한 최초의 사람 중 한 명이었다.
10년 후, 이 글을 쓰는 시점에는 $75,000$개 이상\footnote{\url{https://bit.ly/luke-nodecount}}의 노드가 비트코인을 구동하고 있다.

\begin{figure}
	\centering
	\includegraphics[width=8cm]{assets/images/running-bitcoin.png}
	\caption{할피니가 2009년 1월 비트코인을 언급하는 첫 번째 트윗}
	\label{fig:running-bitcoin}
\end{figure}

\begin{comment}
	The protocol's base layer isn't the only thing growing exponentially.
	The lightning network, a second layer technology, is growing at an even
	faster rate.
\end{comment}
비트코인 프로토콜의 기본 레이어만 기하급수적으로 성장하고 있는 것이 아니다.
두 번째 레이어인 라이트닝 네트워크는 훨씬 더 빠른 속도로 성장하고 있다.

\begin{comment}
	In January 2018, the lightning network had $40$ nodes and $60$
	channels~\cite{web:lightning-nodes}. In April 2019, the network grew to more
	than $4000$ nodes and around $40.000$ channels. Keep in mind that this is still
	experimental technology where loss of funds can and does occur. Yet the trend is
	clear: thousands of people are reckless and eager to use it.
\end{comment}
2018년 1월, 라이트닝 네트워크에는 $40$개의 노드와 $60$개의 채널이 있었다.\cite{web:lightning-nodes}
2019년 4월, 네트워크는 $4,000$개 이상의 노드와  $40,000$개의 채널로 성장했다.
라이트닝 네트워크는 자금 손실이 발생할 수 있고, 실제로 발생하기도 하는 실험적인 기술임을 명심해야 한다.
하지만 수천 명의 사람들이 무모하고 열성적으로 라이트닝 네트워크를 사용하려는 추세만은 분명하다. 

\begin{figure}
	\includegraphics{assets/images/lnd-growth-lopp-white.png}
	%\caption{Lightning Network, January 2018 vs December 2018. Source: Jameson Lopp}
	\caption{2018년 1월과 2018년 12월의 라이트닝 네트워크 (출처: Jameson Lopp)}
	\label{fig:lnd-growth-lopp-white.png}
\end{figure}

\begin{comment}
	To me, having lived through the meteoric rise of the web, the parallels
	between the internet and Bitcoin are obvious. Both are networks, both
	are exponential technologies, and both enable new possibilities, new
	industries, new ways of life. Just like electricity was the best
	metaphor to understand where the internet is heading, the internet might
	be the best metaphor to understand where bitcoin is heading. Or, in the
	words of Andreas Antonopoulos, Bitcoin is \textit{The Internet of Money}.
	These metaphors are a great reminder that while history doesn't repeat
	itself, it often rhymes.
\end{comment}
웹의 급격한 성장을 경험한 나에게 인터넷과 비트코인의 유사점은 분명하다.
둘 다 네트워크이고 기하급수적인 기술이며 새로운 가능성, 새로운 산업, 새로운 삶의 방식을 가능하게 한다.
인터넷이 어디로 향하고 있는지 이해하기 위해 전기가 가장 좋은 비유였던 것처럼,
비트코인이 어디로 향하고 있는지 이해하기 위해 인터넷이 가장 좋은 비유가 될 수 있다.
안드레아스 안토노풀로스(Andreas Antonopoulos)의 말을 빌리자면, 비트코인은 돈의 인터넷(The Internet of Money)이다.
역사가 반복되지 않는다 하더라도 종종 이러한 라임(rhymes)이 맞아 떨어질 수도 있다.

\begin{comment}
	Exponential technologies are hard to grasp and often underestimated.
	Even though I have a great interest in such technologies, I am
	constantly surprised by the pace of progress and innovation. Watching
	the Bitcoin ecosystem grow is like watching the rise of the internet in
	fast-forward. It is exhilarating.
\end{comment}
기하급수적으로 발전하는 기술은 파악하기도 어렵고 종종 과소평가되기도 한다.
이런 기술에 큰 관심을 가진 나로써도 그 발전과 혁신의 속도에 끊임없이 놀라고 있다.
비트코인 생태계가 성장하는 것을 지켜보는 것은 마치 인터넷의 부상을 빨리 감기로 보고 있는 것과 같다.
정말 짜릿하다.

\begin{comment}
	My quest of trying to make sense of Bitcoin has led me down the pathways
	of history in more ways than one. Understanding ancient societal
	structures, past monies, and how communication networks evolved were all
	part of the journey. From the handaxe to the smartphone, technology has
	undoubtedly changed our world many times over. Networked technologies
	are especially transformational: writing, roads, electricity, the
	internet. All of them changed the world. Bitcoin has changed mine and
	will continue to change the minds and hearts of those who dare to use
	it.
\end{comment}
비트코인을 이해하고자 하는 나의 탐구는 여러 방면으로 역사를 돌아보는 여정이었다.
고대 사회 구조, 과거의 화폐, 통신 네트워크가 어떻게 진화했는지 이해하는 것도 이 여정의 일부였다.
손도끼에서 스마트폰에 이르기까지 기술은 의심할 여지 없이 우리 세상을 여러 차례 변화시켰다.
특히 네트워크 기술은 문자, 도로, 전기, 인터넷과 같은 혁신을 일으켰다.
이 모든 것이 세상을 변화시켰다. 
비트코인은 나를 변화시켰고, 비트코인을 사용하는 사람들의 생각과 마음을 계속 변화시킬 것이다.

%\paragraph{Bitcoin taught me that understanding the past is essential to
	%understanding its future. A future which is just beginning\ldots}
\paragraph{비트코인은 과거를 이해하는 것이 미래를 이해하는 데 필수적이라는 것을 알려주었다. 미래는 이제 막 시작되었다.\ldots}

% ---
%
% #### Down the Rabbit Hole
%
% - [The Rising Speed of Technological Adoption][the rising speed of technological adoption] by Jeff Desjardins
% - [The 7 Network Effects of Bitcoin][multiple network effects] by Trace Mayer
% - [The Electricity Metaphor for the Web's Future][TED talk] by Jeff Bezos
% - [How the internet has woven itself into American life][data from the Pew Research Center] by Susannah Fox and Lee Rainie
% - [Genesis Block][genesis block] on the Bitcoin Wiki
% - [Lindy Effect][more time] on Wikipedia
%
% [Our World in Data]: https://ourworldindata.org/
% [the rising speed of technological adoption]: https://www.visualcapitalist.com/rising-speed-technological-adoption/
% [multiple network effects]: https://www.thrivenotes.com/the-7-network-effects-of-bitcoin/
% [TED talk]: https://www.ted.com/talks/jeff_bezos_on_the_next_web_innovation
% [recording of the Today Show]: https://www.youtube.com/watch?v=UlJku_CSyNg
% [William Gibson]: https://www.npr.org/2018/10/22/1067220/the-science-in-science-fiction
% [data from the Pew Research Center]: https://www.pewinternet.org/2014/02/27/part-1-how-the-internet-has-woven-itself-into-american-life/
% [consumer survey]: https://www.kaspersky.com/blog/money-report-2018/
% [letter to shareholders]: http://media.corporate-ir.net/media_files/irol/97/97664/reports/Shareholderletter97.pdf
% [running bitcoin]: https://twitter.com/halfin/status/1110302988?lang=en
% [40 nodes]: https://bitcoinist.com/bitcoin-lightning-network-mainnet-nodes/
% [reckless]: https://twitter.com/hashtag/reckless
% [Jameson Lopp]: https://twitter.com/lopp/status/1077200836072296449
% [\textit{The Internet of Money}]: https://theinternetofmoney.info/
% [stacking]: https://twitter.com/hashtag/stackingsats
%
% <!-- Bitcoin Wiki -->
% [genesis block]: https://en.bitcoin.it/wiki/Genesis_block
%
% <!-- Wikipedia -->
% [more time]: https://en.wikipedia.org/wiki/Lindy_effect
% [alice]: https://en.wikipedia.org/wiki/Alice%27s_Adventures_in_Wonderland
% [carroll]: https://en.wikipedia.org/wiki/Lewis_Carroll

\addpart{마지막 생각}
\pdfbookmark{Conclusion}{결론}
\label{ch:conclusion}

\chapter*{결론}

\begin{chapquote}{루이스 캐롤, \textit{이상한 나라의 앨리스}}
	\enquote{처음부터 읽어라.} 왕은 엄중하게 말했다., \enquote{그리고 끝까지 읽고, 끝내라.}
\end{chapquote}

\begin{comment}
	As mentioned in the beginning, I think that any answer to the
	question \textit{“What have you learned from Bitcoin?”} will always be incomplete. The
	symbiosis of what can be seen as multiple living systems -- Bitcoin, the
	technosphere, and economics -- is too intertwined, the topics too numerous, and
	things are moving too fast to ever be fully understood by a single person.
\end{comment}
서두에서도 언급했지만 \enquote{비트코인으로부터 배운 것이 뭐야?}에 대한 대답은 
늘 불완전하다.
비트코인은 기술과 경제학같이 여러 살아있는 시스템들의 공생 관계가 얽혀있어서 다루어야 할 주제가 너무 많고, 
너무 빠르게 변화하고 있기 때문에 한 사람이 완전히 이해하는 것은 불가능하다.

\begin{comment}
	Even without understanding it fully, and even with all its quirks and seeming
	shortcomings, Bitcoin undoubtedly works. It keeps producing blocks roughly every
	ten minutes and does so beautifully. The longer Bitcoin continues to work, the
	more people will opt-in to use it.
\end{comment}
비트코인을 완전히 이해하지 못하고 단점과 결점이 있다 하더라도 
비트코인은 의심할 여지 없이 작동하고 있다.
대략 10분마다 계속해서 블록을 생성하며 매우 훌륭하게 구동되고 있다.
비트코인이 더 긴 시간 동안 작동할 수 있도록 더 많은 사람이 참여할 것이다.

\begin{quotation}\begin{samepage}
		%\enquote{It's true that things are beautiful when they work. Art is function.}
		\enquote{그것이 작동할 때 아름답다는 것은 사실이다. 예술은 기능이다.}
		\begin{flushright} -- 지안니나 브라스키\footnote{Giannina Braschi, \textit{Empire of Dreams} \cite{braschi2011empire}}
\end{flushright}\end{samepage}\end{quotation}

\paragraph{} 
\begin{comment}Bitcoin is a child of the internet. It is growing exponentially,
	blurring the lines between disciplines. It isn’t clear, for example, where the
	realm of pure technology ends and where another realm begins. Even though
	Bitcoin requires computers to function efficiently, computer science is not
	sufficient to understand it. Bitcoin is not only borderless in regards to its
	inner workings but also boundaryless in respect to academic disciplines.
\end{comment}
비트코인은 인터넷의 산물이다. 
비트코인은 기하급수적으로 성장하고 있으며 분야의 경계를 무너뜨리고 있다.
예를 들어 순수 기술은 어딘가에서 끝나지만 또 다른 영역에서 어떤 영향을 끼치기 시작하는지는 확신하기 어렵다.
비트코인이 작동하려면 컴퓨터가 필요하지만 컴퓨터 과학은 비트코인을 이해하기에 충분하지 않다.
비트코인의 내부 작동 측면에서도 경계가 없을 뿐 아니라 학문 분야에서도 경계가 없다.

\begin{comment}
	Economics, politics, game theory, monetary history, network theory, finance,
	cryptography, information theory, censorship, law and regulation, human
	organization, psychology -- all these and more are areas of expertise which might
	help in the quest of understanding how Bitcoin works and what Bitcoin is.
\end{comment}
비트코인이 무엇이고 비트코인이 어떻게 작동하는지 이해하기 위해서는 
경제학, 정치, 게임이론, 화폐의 역사, 네트워크 이론, 금융, 암호학, 정보이론, 검열, 법률과 규제,
인간 조직, 심리학의 도움이 필요하다.


\begin{comment}
	No single invention is responsible for its success. It is the combination of
	multiple, previously unrelated pieces, glued together by game theoretical
	incentives, which make up the revolution that is Bitcoin. The beautiful blend of
	many disciplines is what makes Satoshi a genius.
\end{comment}
어떤 발명품도 단일 기술로는 성공을 장담할 순 없다. 
비트코인은 게임 이론에 기반한 혁신적인 인센티브를 위하여
아무런 관련이 없는 여러 조각들을 잘 조합하여 구성되었다.
사토시가 천재인 이유는 여러 분야의 조화를 아름답게 완성하였기 때문이다.


\paragraph{} 
\begin{comment}Like every complex system, Bitcoin has to make tradeoffs in terms
	of efficiency, cost, security, and many other properties. Just like there is no
	perfect solution to deriving a square from a circle, any solution to the
	problems which Bitcoin tries to solve will always be imperfect as well.
\end{comment}
모든 복잡한 시스템이 그러하듯이 비트코인은 효율성, 비용, 보안 등 여러 측면에서 절충이 필요하다.
원에서 사각형을 만드는 완벽한 방법이 없듯이 
비트코인이 해결하려는 해결책도 항상 불완전하다.

\begin{quotation}\begin{samepage}
		%\enquote{I don’t believe we shall ever have a good money again before we take the
			%thing out of the hands of government, that is, we can’t take it violently
			%out of the hands of government, all we can do is by some sly roundabout way
			%introduce something that they can’t stop.}
		\enquote{나는 우리가 정부로 부터 그것을 빼앗기 전까지 다시 건전화폐를 가질 것이라고 믿지 않는다. 그러나 정부의 손에서 폭력적으로 빼앗을 수는 없다. 우리가 할 수 있는 유일한 방법은 교활하고 우회적으로 그들이 멈출 수 없는 무언가를 도입하는 것이다.}
		\begin{flushright} -- 프리드리히 하이에크\footnote{Friedrich Hayek on Monetary Policy, the Gold Standard, Deficits, Inflation, and John Maynard Keynes \url{https://youtu.be/EYhEDxFwFRU}}
\end{flushright}\end{samepage}\end{quotation}

\begin{comment}
	Bitcoin is the sly, roundabout way to re-introduce good money to the world. It
	does so by placing a sovereign individual behind each node, just like Da Vinci
	tried to solve the intractable problem of squaring a circle by placing the
	Vitruvian Man in its center. Nodes effectively remove any concept of a center,
	creating a system which is astonishingly antifragile and extremely hard to shut
	down. Bitcoin lives, and its heartbeat will probably outlast all of ours.
\end{comment}
비트코인은 건전 화폐를 세상에 다시 도입하는 교활하고 우회적인 방법이다.
디빈치가 비트루비우스 적 인간에서 인간을 중앙에 배치하여 원을 정사각형으로 만드는 방법을 찾으려고 시도했던 것처럼,
비트코인의 각 노드 뒤에 자주적 개인을 배치하여 해결을 시도하여야 한다.
노드는 중앙의 개념을 효과적으로 제거하여 공격에 취약하지 않고 종료하기 어려운 시스템을 만든다.
비트코인은 살아있고, 그 심장 박동은 우리보다 더 오래 지속될 것이다.

\begin{comment}
	I hope you have enjoyed these twenty-one lessons. Maybe the most important
	lesson is that Bitcoin should be examined holistically, from multiple angles, if
	one would like to have something approximating a complete picture. Just like
	removing one part from a complex system destroys the whole, examining parts of
	Bitcoin in isolation seems to taint the understanding of it. If only one person
	strikes \enquote{blockchain} from her vocabulary and replaces it with \enquote{a
		chain of blocks} I will die a happy man.
\end{comment}
스물한 가지의 교훈이 즐거웠기를 바란다. 
아마도 가장 중요한 교훈은 비트코인을 완벽하게 알고 싶다면 여러 각도에서 전체적으로 검토해야 한 다는 것이다. 
복잡한 시스템의 한 부분을 제거하면 전체가 파괴되는 것처럼, 
비트코인의 일부를 따로따로 공부하는 것은 비트코인의 이해에 오히려 방해된다.
누군가가 그녀의 단어장에서 \enquote{blockchain}을 빼고 \enquote{a chain of blocks}으로 바꾸면,
난 정말 행복하게 죽을 수 있을 것 같다.


\begin{comment}
	In any case, my journey continues. I plan to venture further down into the
	depths of this rabbit hole, and I invite you to tag
	along for the ride.\footnote{\url{https://twitter.com/dergigi}}
\end{comment}
어쨌든 내 여정은 계속된다. 나는 이 토끼 굴에 더 깊이 들어갈 계획이다.
함께 갈 수 있도록 당신을 초대한다.\footnote{\url{https://twitter.com/dergigi}}

% <!-- Twitter -->
% [dergigi]: https://twitter.com/dergigi
%
% <!-- Internal -->
% [sly roundabout way]: https://youtu.be/EYhEDxFwFRU?t=1124
% [Giannina Braschi]: https://en.wikipedia.org/wiki/Braschi%27s_Empire_of_Dreams


\cleardoublepage

\chapter*{감사의 글}
\pdfbookmark{Acknowledgments}{acknowledgments}

%Thanks to the countless authors and content producers who influenced my thinking
%on Bitcoin and the topics it touches. There are too many to list them all, but
%I’ll do my best to name a few.
비트코인에 대해 제 생각에 영향을 준 수 많은 저자와 콘텐츠 제작자에게 감사드립니다.
모두 나열하기에는 너무 많지만, 몇 분만 추려서 언급하겠습니다.


\begin{itemize}
	\begin{comment}
		\item Thanks to Arjun Balaji for the tweet which motivated me to write this.
		\item Thanks to Marty Bent for providing endless food for thought and entertainment. If you are not subscribed to Marty’s Bent and Tales From The Crypt, you are missing out. Cheers Matt and Marty for guiding us through the rabbit hole.
		\item Thanks to Michael Goldstein and Pierre Rochard for curating and providing the greatest Bitcoin literature via the Nakamoto Institute. And thank you for creating the Noded Podcast which influenced my philosophical views on Bitcoin substantially.
		\item Thanks to Saifedean Ammous for his convictions, savage tweets, and writing The Bitcoin Standard
		\item Thanks to Francis Pouliot for sharing his excitement about finding out about the timechain.
		\item Thanks to Andreas M. Antonopoulos for all the educational material he has put out over the years.
		\item Thanks to Peter McCormack for his honest tweets and the What Bitcoin Did podcast, which keeps providing great insights from many areas of the space.
		\item Thanks to Jannik, Brandon, Matt, Camilo, Daniel, Michael, and Raphael for providing feedback to early drafts of some lessons. Special thanks to Jannik who proofread multiple drafts multiple times.
		\item Thanks to Dhruv Bansal and Matt Odell for taking the time to discuss some of these ideas with me.
		\item Thanks to Guy Swann for producing an audio version of 21lessons.com.
		\item Thanks to Friar Hass for his spiritual support and guidance, and for taking the time to write a foreword for this book.
		\item Thanks to my wife for putting up with me and my obsessive nature.
		\item Thanks to my family for supporting me during both the good times and the bad.
		\item Last but not least, thanks to all the bitcoin maximalists, shitcoin minimalists, shills, bots, and shitposters which reside in the beautiful garden that is Bitcoin twitter.
	\end{comment}
	\item 이 글을 쓰도록 동기를 제공한 아준 발라지(Arjun Balaji)에게 감사드립니다.
	\item 생각과 여가를 위하여 끝없이 음식을 제공한 마티 벤트 (Marty Bent)에게 감사드립니다. 마티 벤트 (Marty’s Bent)와 크립토 이야기(Tales From The Crypt)를 구독해 주세요. 토끼굴을 통해 우리에게 안내해 준 맷과 마티에게 환호를 보냅니다.
	\item 나카모토 재단을 통해 최고의 비트코인 문헌을 제공한 마이클 골드스타인(Michael Goldstein)과 피에르 로차드(Pierre Rochard)에게 감사드립니다. 그리고 비트코인에 대한 철학적 견해에 상당한 영향을 준 노디드(Noded) 팟캐스트를 만들어주셔서 감사합니다.
	\item 그의 신념, 잔인한 트윗과 비트코인 스탠다드를 집필해 준 사이페딘 아모스(Saifedean Ammous)에게 감사드립니다.
	\item 타임체인에 대해 알게 된 기쁨을 공유해 준 프랑시스 폴리엇(Francis Pouliot)에게 감사드립니다.
	\item 수년에 걸쳐 모든 교육자료를 제공한 안드레아스 안토노풀로스(Andreas M. Antonopoulos)에게 감사드립니다.
	\item 솔직한 트윗과 공간의 여러 영역에서 훌륭한 통찰력을 지속적으로 제공한 What Bitcoin Did 팟캐스트의 피터 맥코맥(Peter McCormack)에게 감사드립니다.
	\item 일부 교훈 초기에 피드백을 준 재닉(Jannik), 브랜든(Brandon), 맷(Matt), 카밀로(Camilo), 다니엘(Daniel), 마이클(Michael), 라파엘(Raphael)에게 감사드립니다. 어려 초안을 여러 번 교정해 준 재닉에게 특별히 감사드립니다.
	\item 시간을 내어 아이디어에 대해 논의해 준 데룹 반살(Dhruv Bansal)과 맷 오델(Matt Odell)에게 감사드립니다.
	\item 21lessons.com의 오디오 버전을 제작해 준 가이 스완(Guy Swann)에게 감사드립니다.
	\item 영적 지원과 인도를 배풀고 시간을 내어 이 책의 서문을 써 준 하스(Hass) 신부에게 감사드립니다.
	\item 나의 강박적인 성격을 참아준 아내에게 감사드립니다.
	\item 좋을 때나 나쁠 때나 저를 지지해 준 가족들에게 감사드립니다.
	\item 마지막으로 비트코인 트위터라는 아름다운 정원에 상주하는 모든 비트코인 맥시멀리스트, 똥코인 미니멀리스트, 사기꾼들, 봇, 똥포스터에게 감사드립니다.
\end{itemize}


%And finally, thank you for reading this. I hope you enjoyed it as much as I did enjoy writing it.
그리고 마지막으로 이 글을 읽어주셔서 감사합니다. 
제가 글을 쓰면서 즐거웠던 만큼 여러분도 즐거우셨기를 바랍니다.


\listoffigures

\chapter*{참고문헌에 관하여}
\pdfbookmark{Bibliography}{bibliography}

%Today, plenty of books have been published about Bitcoin. However, most of the
%conversation -- and thus most of the resources of interest -- happen online.
최근 비트코인에 관련된 많은 책이 출간되었다. 그러나 대부분의 관심이 있을 만한 자료는 온라인에 있다.

\paragraph{}
%The following bibliography lists books, papers, and online resources alike. If
%the resource has a URL associated with it, the URL was alive and kicking in
%October 2019, since I was able to successfully access the resource in question.
%If any of the following URLs leads to a dead page, I'm sorry. Please let me
%know\footnote{\url{https://dergigi.com/contact}} so I can update the link(s).
다음 참고문헌에는 서적, 논문, 온라인 자료들이 모두 나열되어 있다.
자료에 연결된 URL은 2019년 10월에 살아있었고 접근이 확인된 자료들이다. 
만약 하나라도 연결이 끊긴 페이지로 연결되었다면 송구하게 생각한다.
연결이 끊긴 페이지를 업데이트할 수 있도록 나에게 알려달라\footnote{\url{https://dergigi.com/contact}}.

\paragraph{}
P.S: 비트코인과 \href{https://ipfs.io/}{IPFS}는 이 문제를 해결할 수 있다.


\bibliography{main}

\end{document}
